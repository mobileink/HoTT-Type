%%%%%%%%%%%%%%%%%%%%%%%%%%%%%%%%
\chapter{Mathematics}
\label{sect:math}

%%%%%%%%
\section{Traditional}
\label{subs:mathtrad}

%%%%%%%%
\section{Modern: classic}
\label{subs:mathmodclassic}

%%%%%%%%%%%%%%%%%%%%%%%%%%%%%%%%
\section{Modern: Intuitionism}
\label{sect:mathmodintuit}

\begin{ednote}
  Why Brouwer should be deemed a pragmatist.
\end{ednote}


%%%%%%%%%%%%%%%%%%%%%%%%%%%%%%%%
\section{Mathematical Pragmatism}
\label{sect:mathprag}

\begin{ednote}
  \HoTT is (largely) founded on \ML{}'s account of ``judgment''
  (assertion).  I don't know if that's entirely accurate, but it's my
  story and I'm sticking with it for now.  (\ML{} was quite specific
  that his project was motivated by ``purely philosophical''
  considerations.  See his 1972 paper.)  Brandom's account of
  assertion is part of a larger, very ambitious project that aims to
  explain the structure of rationality.  It's a thoroughly pragmatic
  account; everything comes down in the end to ``proprieties of
  practice'': conceptual activity (thinking and talking) is explained
  in terms not of what we know but of what we do (or what we know
  \textit{how} to do).

  Brandom's account of assertion is much more refined and
  sophisticated than \ML{}'s.  If we replace \ML{}'s account with
  Brandom's, then \HoTT comes out as a piece of ``mathematical
  pragmatism'' (or pragmatic mathematics): an account mathematics
  grounded in practice.
\end{ednote}

\begin{ednote}
  TODO: Brandom's philosophy, like most of contemporary pragmatism,
  subverts the dominant representationalist mode of thinking.  It
  turns things upside-down, or inside-out.  So it is with type theory.
  (In one of his papers \ML{} suggests something similar, pointing out
  that his take on judgment etc. reverts (in some sense) back to
  practices that preceded the ways of thinking that have dominated
  modern ``classic'' mathematics and logic.)  The to-do item here is
  to show how the relation of type theoretic to classic thinking in
  mathematics and logic parallels the relation between pragmatist
  (anti-representational, expressivist) thinking and representational
  (cartesian, platonistic) thinking in philosophy, about rationalism,
  conceptual content, etc.  Show how type-theoretic thinking turns
  traditional classic thinking inside-out.
\end{ednote}

