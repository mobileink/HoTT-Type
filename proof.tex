%%%%%%%%%%%%%%%%%%%%%%%%%%%%%%%%
\chapter{Proof: Truth and Consequences}
\label{sect:prooftruth}

\begin{ednote}
  Why Truth is not Important in Computation.  Proof is a mode of
  expression.  The statement of a proposition--e.g. \(1>0\)--expresses
  the propostion but leaves its inferential articulation implicit.  A
  proof of the proposition explicitly articulates its upstream
  inferential structure.  To say that a proof demonstrates the truth
  of the proposition it proves is just to say that it makes (part of)
  its inferential articulation explicit---in whatever proof formalism
  you decide to use (more than one possibility).  Truth plays no
  substantial role in this view; it's just a convenient way to say
  ``explicit inferential articulation''.
\end{ednote}

%%%%%%%%
\section{Truth}
\label{subs:truth}

Why Truth is Not Important in Type Theory (with apologies to
R. Brandom\cite{brandom_why_2009})

Consequence as prior to truth --- \cite{schroeder-heister_pts}

Proof before truth \cite{schroeder-heister_validity_2006}

\begin{ednote}
  Propositions are either true or false in classic math and logic; in
  type theory they are either proven or disproven.  In this respect
  type theory is just like contemporary pragmatism, which (generally
  speaking) treats truth as otiose; what matters is not truth but
  function or expressiveness.

  This suggests a test for learners: until you've grasped why truth is
  not imporant in type theory you haven't really grasped type theory.
\end{ednote}

Traditional (classic) view: a proof is an epistemic device; it
displays, exhibits, makes \textit{visible} (if only to the mind's eye)
a form of \textit{certain knowledge}.\sidenote{The link between
  knowing and seeing runs very deep in Western culture.  Not
  surprisingly it is closely connected with representationalism and
  cartesianism generally.  It has pretty much dominated Western
  thinking since Descartes, but has come under strong attack from
  Pragmatists.  Dewey called it ``the spectator theory of knowledge.''f
  See \citep{rorty_philosophy_2009} etc.}

Alternatives to the spectator theory: pragmatism, know-how over know-that.

\begin{ednote}
  TODO: summary of concepts of proof.  Emphasize contrast between
  representationalism and inferentialism.  Representationalism is
  atomistic: you could have only one concept.  Inferentialism is
  holistic: you have to start out with at least two concepts, since
  every inference involves a premise and a conclusion.  Inferentialism
  is a natural fit for \HoTT.

  Question: can you have only one type?  In other words, is type
  theory essentially holistic or atomistic?
\end{ednote}


For \HoTT{}, as for most varieties of constructivism, it is better to
abandon traditional notions of proof as something you see in favor of a
more pragmatic notion of proof as something you do.

\begin{ednote}
  But proof as explicit articulation of inferential structure also
  comes out as a thing.  It must, if we are to have proof objects as
  mathematical objects; it is not enough just to do a proof, we must
  produce a proof object---which is just the explicit statement of the
  inferential structure implicit in the mere statement of a
  proposition.  So if it is something we do it is also something we
  produce.
\end{ednote}

etc.

Critical point: in \HoTT we have two ``kinds'' of types: propositional
types and non-propositional types.\sidenote{This is not in general
  recognized in \HoTTB, but I think it should be emphasized, if only
  because it reflects intuition.}  If we are to also treat ``proof''
(or witness or whatever) as a fundamental principle of \HoTT, one that
complements the concept of type, then we need to treat both ``type''
and ``proof'' as genuses (genii?) of which propositional and non-propositional
are species.

\begin{ednote}
  General point (to be made elsewhere, maybe in
  \cref{sect:foundations}: the concepts of type and proof go together.
  You cannot have one without the other.  That's very different than
  set theory.  You can have sets and elements without proofs.
\end{ednote}



Long story short: we are in dire need of improved terminology.  My
suggestion is as follows:

\begin{description}
\item [Proof of a proposition] In contrast to the classic spectator
  view, we treat proof not as the exhibition (or: making available for
  inspection) of the form of a bit of certain knowledge, but as the
  \textit{demonstrative expression}\sidenote{Close, but no cookie.
    The notion of demonstrative expression must be made concrete, in
    terms of explicit articulation of inferential structure of
    proposition.} of the proposition.  Alternatively, the expressive
  demonstration of the proposition.  So whereas a classic proof is
  something that must be ``seen'' in order to be grasped, a
  type-theoretic proof is something that must be actively
  \textit{done}, not merely passively observed.  One must be able to
  follow the construction of the proof.


\item [Proof of a non-propositional type] Classically, one only proves
  propositions, not terms.  So the idea of e.g. ``proving'' the
  natural numbers doesn't even make sense; it reflects a category
  mistake.  But in \HoTT, the concept of ``proving'' a type is
  primitive; the problem is that ``proving'' is the wrong
  word.\sidenote{Hmm.  Proof of the type \(\nat\) by producing a
    natural number --- how does this count as explicit articulation of
    inferential structure of \(\nat\)?  What is that structure,
    anyway?  The inductive definition?  In constructing a natural
    number, e.g. \(3\), we \emph{use} of the principle of induction,
    of Succ.  Does this count as making the inferential articulation
    of \(\nat\) explicit?  That would imply that the explicit
    articulation of a proof must include the entire apparatus,
    including inference rules, type formation rules, etc.  But this
    would mean that definitions (type formation rules) form part of
    the proof, which seems circular - you cannot ``prove'' \(\nat\) by
    citing the very definition of \(\nat\).  And while we're at it:
    what are the type formation rules for propositions like \(1 >
    0\)?}
\end{description}

\begin{ednote}
  Can we think of each natural number as a dependent type?
  I.e. something like Succ parameterized by -- what?
\end{ednote}

So here's one way to look at it: we construct (make) proofs; but the
proofs we construct are expressions of the type (the thing we prove).

%%%%%%%%
\section{Proof}
\label{subs:proof}

\begin{ednote}
  Concept of ``proof obligation''
\end{ednote}

\begin{ednote}
  Critical point: we want to draw a sharp contrast between the classic
  and pragmatist conceptions of proof.  Classic: the business of proof
  is to preserve truth; valid inference is defined in terms of
  truth-preservation; and the knowledge dispensed by proof is
  \emph{knowledge that} the conclusion is true of the premises are
  true.  Pragmatist: validity of inference is grounded in proprieties
  (norms) of practice rather than truth; valid inference
  \emph{expresses} good (material) inference; the knowledge expressed
  by proof is \emph{know-how}: a practical skill of doing rather than
  an epistemic state of knowing.  Truth has no substantive role to
  play; it is just a complement we pay to the premises and conclusions
  of good inference.

  In a nutshell: classically, good inference is what preserves truth;
  proof is ``truth-conditional''.  Pragmatism turns this upside down
  and says that truth is what good inference preserves; truth is
  proof-conditional.  The former starts with truth and derives proof;
  the latter starts with good inference and derives truth.

  Relevance to type theory: intuitionistic type theory grew out of
  what can be called a pragmatist tradition in mathematics and logic,
  even if the key players did not think of themselves as pragmatists.
  The ``intuitionistic'' bit is key; the concept of ``type'' is
  neutral with respect to these issues; both classic and pragmatic
  approaches can use it, and neither can claim exclusive rights to it.
  ITT provides a pragmatist interpretation of the concept.  One of the
  key themes of \ML{}'s theory, for example, is an account of
  assertion (judgment); this is fundamental because it explains the
  meanings of propositions and proofs.  Dummett is a key figure here,
  cited explicitly by Brandom as a major influence (I don't know if
  \ML{} cites him.)  Gentzen, too: natural deduction as a
  meaning-is-use model of reasoning.

  Another way to put it: reality does not tell us which inferences are
  good, nor which premises are true.  That's something we decide, by
  settling on normative practices.

  TODO: centrality of concept of meaning-as-use; role of idea of
  theory of meaning as essential -- Dummett, but also Frege, etc.

\end{ednote}

\begin{ednote}
  TODO: concise schematic account of classic concept of proof.  To
  prove a proposition is to...?  Construct a linguistic expression
  that corresponds to the truth?  True premises plus truth-preserving
  inferences.  Writing a concise account is a bit hairy, since we need
  to account for both syntax and semantics, axioms and deductions.
  Basic idea: it's not about construction but about representation.
  How one goes about making a proof is irrelevant; all that matters is
  the truth value of the result, meaning that the result must
  ``mirror'' objective reality.  Even inferences are essentially
  axiomatic; the truth tables \emph{define} the logical constants and
  the inferences they are involved in.  Proof involves checking the
  truth values of propositions and inferences in a structure.  In
  other words, truth tells us whether the proof is good or not.
  Contrast pragmatism: proof tells us whether conclusion is true or
  not.
\end{ednote}

\begin{ednote}
  Another critical point: pragmatism is about language, that is
  discursive practice.  It denies that there is anything interesting
  (read: useful) to say about Truth (capital T), but it has plenty to
  say about the role that the concept (and terminology) of truth plays
  in our practical doings and sayings.  It's an extremely useful
  concept, and without linguistic devices like ``... is true'' we
  would find discourse vastly more difficult (but not impossible in
  principle).  But that does not compel us think that Truth is a
  substantial property, of that something called ``Truth'' exists.  Or
  more specifically, that true propositions are true in virtue of
  their correspond to ``reality''.
\end{ednote}

\newthought{To prove a proposition is to justify it}; in Brandom's
pragmatist model, this means to to vindicate \emph{entitlement} to
\emph{commitment} to the proposition that is \emph{expressed by
  asserting} it.  There are three ways to do this:

\begin{description}
\item [Demonstration] One can \emph{say explicitly} what entitles
  commitment to the proposition.  Informally, this means laying out
  the reasons for it - the chain of inferences in which the
  proposition plays the role of conclusion or premise.\sidenote{Note
    that this goes both ways, upstream and downstream.}  Formally,
  this means stating a proof in the traditional sense of a schedule of
  propositions linked by inferences licensed by the deductive system.
  But validity of such a proof is not to be construed in terms of
  truth-preservation; rather, it is instituted by normative
  practices.\sidenote{Or, the deductive system is not
    truth-conditional, but pragmatic (constructive): it encodes
    know-how rather than knowledge-that.  We don't know \emph{that}
    the truth of the conclusion preserves the truth of the premises;
    rather we grasp the norms governing correct \emph{use} of the
    premises, and the normative practices involving \emph{how} to get
    from the premises to the conclusion---the concept of truth need
    not ever enter the picture.}  What counts as good inference for us
  is determined by what the community has decided to \emph{treat} as
  good inference, not by what Reality has to say about the matter.

\item [Appeal to Authority] Continually articulating explicity proofs
  would be impossible in practice; but the rational structure of
  discursive practice means that we can also cite the authority of
  others who have asserted the proposition.  Formally, this means we
  can simply refer to previously proven propositions instead of
  repeating their proofs.  If the a step in the proof of your
  proposition depends on the Pythagorean Theorem, for example, you can
  appeal to the theorem by name to justify that step, rather than
  proving it explicitly.  In Brandom's idiom, what makes this work
  involves a concept of \emph{intra}personal inheritance of
  entitlement; it is an aspect of the essentially social nature of
  discourse and thus reasoning.

\item [Reliable Disposition] The trickiest of Brandom's three
  techniques of justification involves the appeal to the commitments
  of reliable reporters.  By that he means people whose reports are
  reliable even if they are unable to provide explicit justification
  of them.  He gives the example of an expert in a certain kind of
  pottery, whose (``instinctive'') judgments as to whether a
  particular shard is or is not an example of that kind are generally
  considered reliable.
\end{description}

\begin{ednote}
  I'm not sure where reliable reporting fits into formal reasoning.
  In general I don't think it plays much of a role in mathematical or
  logical proofs; mathematicians and logicians do not as a rule simply
  accept the word of somebody just because his intuitions have proven
  reliable in the past.  They might accept it informally and say that
  he is probabaly right, but they would nonetheless demand that
  eventually the intuition be backed by explicit proof.  Maybe at a
  very basic level, such as the intuition that there is a unity and a
  plurality, is a reliable disposition that we attribute to just about
  everybody.  Brouwer's account of the subjective origin of basic
  mathematical concepts in terms of intuitions about time, etc. should
  be mentioned here.
\end{ednote}

%%%%%%%%
\section{Logical Consequence}
\label{subs:logconseq}

\cite{prawitz_logical_2005}

\cite{prawitz_meaning_2006}

\cite{prawitz_inference_2009}

\cite{prawitz_epistemic_2011}

\cite{prawitz_truth_epistemic}

%%%%%%%%%%%%%%%%%%%%%%%%%%%%%%%%
\section{Inference and Deduction}
\label{sect:infdeduc}

\begin{description}
\item [Gentzen]
\item [etc]
\end{description}

%%%%%%%%
\section{Of the Ambiguity of Of}
\label{subs:ofofof}

``Of'' supports two distinct readings.  Consider ``the conviction of
the defendant''.  If the court did the convicting, then ``of'' acts as
a kind of intermediary between a verbal noun (``conviction'' as act or
action of convicting) and its direct object (e.g. ``The court
convicted the defendant'').  The conviction affects the defendant from
the outside; it does not ``belong'' to the defendant but to the court.
On the other hand, if we take ``the conviction of the defendant'' to
refer to a belief to which the defendant is firmly committed, then the
conviction is ``internal''; it belongs to and comes from the
defendant.

This ambiguity of ``of'' afflicts phrases like ``proof of a
proposition'' as well.  If we can disambiguate it some of the mystery
of the relation between types and proofs will vanish.

%%%%%%%%
\section{Demonstrations and Demonstratives}
\label{subs:}

When we \textit{exhibit} a classic proof of a proposition, the proof
comes out as external to the proposition proved, just as a court's
conviction of a defendant is external to the defendant.  Such a proof
is something added or attached to the proposition.

But when we \textit{demonstrate} a proposition,\sidenote{Note: we
  demonstrate propositions, not proofs; a demonstration of a
  proposition \textit{is} a proof.} the demonstration (that is, proof)
is to be deemed an expression of the proposition in the internal
sense: an expression whose source, so to speak, is the proposition
itself, rather than the writer of the proof.  This may sound odd or
even ridiculously anthropomorphic, but if you think about it a bit it
makes perfect sense.  The mathematical proofs we write down are not
really expressions our our thought, but of mathematical structures,
entities, relations etc.  So they express
mathematics.\sidenote{Actually we should probably think of them as
  having a dual expressivism.  On the one hand they clearly express
  mathematics; but on the other hand, the particular form a proof
  takes is an expression of the writer's style or way of thinking.}

We can think of a demonstration in this sense as expressing a type's
structure, construed as the inferential articulation of the concept of
the type.\sidenote{See \cref{sect:brandom} for more on the inferential
articulation of conceptual content.}

The nice thing about this way of thinking is that it resolves the
tension between propositional and non-propositional types with respect
to proof.  In both cases, what \HoTT{} calls proof or witness is to be
taken as a demonstrative expression, or expressive demonstration, of
the type itself.  In the case of propositional types, favor the term
``demonstration'', with its connotations of progressive unfolding of a
logical structure, or better, rational argument.  In the case of
non-propositional types like \N, favor the term ``demonstrative'',
with its adjectival sense of ``something having a demonstrative
function'', rather than a nominal sense of ``act or action of
demonstrating''.  So an element\sidenote{We really must get rid of
  ``element''; it's too suggestive of set theory.  Maybe
  ``demonstrative'' fits the bill; instead of ``element of a type'' we
  would say ``demonstrative of a type''.  Or maybe ``demonstrator''.}
of a propositional type we would call a demonstration of the type, and
an element of a non-propositional type we would call a demonstrative
of the type.\marginnote{So $2$ is a demonstrative of the natural
  numbers; a proof that ``$2$ is even'' is a demonstration that
  expresses just that ``$2$ is even''.}

\begin{ednote}
  Demonstration qua demonstration of know-how?  Expression as
  expression of a type's structure - that is, its inferential
  articulation?
\end{ednote}

In both cases we have demonstration rather than proof of the type.

\begin{ednote}
  ``Demonstrator'' as the genus of ``demonstration'' and
  ``demonstrative''.  It has the virtue of paralleling
  ``constructor''.
\end{ednote}
