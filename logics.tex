\chapter{Logics}
\label{sect:logics}

\begin{description}
\item [Traditional] terms are primitive; propositions are combinations of terms; judgments apply to proposotions
\item [Modern: classic] LEM, AC, etc.
\item [Modern: intuitionistic]
\item [Expressivism]  Brandom's version: propositions are primitive; relation to inferential semantics; Price's global expressivism
\end{description}

\begin{ednote}
  From schema to type.  E.g. \(A,B\vdash A\land B\) --- traditionally
  viewed as a schema (involving either substitution or denotation), no
  construction involved.  Move from this to viewing it as a rule of
  construction or recipe for making something.
\end{ednote}

\chapter{Proposition}

\begin{ednote}
  BHK interpretation.  How \ML{} got it wrong wrt classic interpretation.
\end{ednote}

