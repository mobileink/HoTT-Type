%%%%  CAVEAT:  xelatex chokes on pkg soul (loaded by tufte); use lualatex

%% \documentclass[12pt,toc]{tufte-handout}
\documentclass[reqno,12pt]{tufte-book}
%% \usepackage{trace}
%% \documentclass[reqno,12pt]{article}

\usepackage{draftwatermark}

% BLACK & WHITE
% Options for black & white book

% BACKGROUND OF COVER PAGE
\def\OPTcovercolor{0,0,0,0}
\def\OPTcovertextcolor{0,0,0,1}

% COVER IMAGE
\def\OPTlofrontimage{cover-lores-front-bw.png}
\def\OPTlobackimage{cover-lores-back-bw.png}
\def\OPThifrontimage{cover-hires-front-bw.png}
\def\OPThibackimage{cover-hires-back-bw.png}

% LINK COLORS
\def\OPTlinkcolor{0,0,0}       % RGB components for clickable links

% PICTURE COLORS
\def\OPTblue{black}
\def\OPTred{black}
\def\OPTpurple{black}

% COLORS FOR TABLE 8.1
\def\OPTcolormodel{gray}
\def\OPTcolxA{0.95}
\def\OPTcolxB{0.75}
\def\OPTcolxC{0.85}
\def\OPTcolxD{0.95}
\def\OPTcolxE{0.75}
\def\OPTcolxF{0.85}
\def\OPTcolxG{0.95}
\def\OPTcolxH{0.75}
\def\OPTcolxI{0.85}
\def\OPTcolxJ{0.95}
\def\OPTcolxK{0.75}
\def\OPTcolxL{0.85}
\def\OPTcolxM{0.95}


% FORMATTING DEPENDENT ON PAPER SIZE
%%% FORMATTING OPTIONS FOR LETTER SIZE
%%% This file gets included by the hott-xxx.tex files.

% MACROS FOR FINE TUNING
\newcommand{\narrowbreak}{}
\newcommand{\narrowamp}{}
\newcommand{\narrowequation}[1]{$#1$}
\newenvironment{narrowmultline}{\csname equation\endcsname}{\csname endequation\endcsname}
\newenvironment{narrowmultline*}{\csname equation*\endcsname}{\csname endequation*\endcsname}

% FONTS
\def\OPTfontsize{11pt}        % Font size

% PAGE FORMAT
%
% These settings are for letter format
\def\OPTpagesize{8.5in,11in}  % Page size
\def\OPTtopmargin{1in}        % Margin at the top of the page
\def\OPTbottommargin{1in}     % Margin at the bottom of the page
\def\OPTinnermargin{0.75in}   % Margin on the inner side of the page
\def\OPTbindingoffset{0.35in} % Extra offset on the inner side
\def\OPToutermargin{1.0in}    % Margin on the outer side of the page

% FORMATTING OF COVER PAGE
\def\OPTcoverwidth{8.45in}    % width of text on cover page
\def\OPTcoverheight{10.95in}  % height of text on cover page
\def\OPTtopskip{43pt}         % Skip at top of cover and back page (with units)
\def\OPTbotskip{43pt}         % Skip at bottom of cover and back page (with units)
\def\OPTcovertitlefont{74.5}  % Size of title font (no unit, pt assumed)
\def\OPTcovertitleskip{24pt}  % Skip between title and subtitle (with units)
\def\OPTcoversubtitlefont{27} % Size of subtitle font (no unit, pt assumed)
\def\OPTcoverauthorfont{22}   % Size of author font (no unit, pt assumed)
\def\OPTcoverauthorskip{8pt}  % Skip betewen first and second line of author (with units)

% FORMATTING OF BACK COVER PAGE
\def\OPTbacktitlefont{\LARGE} % Size of "From the introduction" and "Get this book ..."
\def\OPTbackfont{\large}      % Size of text font

% FORMATTING OF BASTARD TITLE, IF PRESENT
% (see opt-no-bastard.tex and opt-bastard.tex)
\def\OPTbastardtitlefont{22}
\def\OPTbastardsubtitlefont{15}
\def\OPTbastardwidth{0.40\textwidth} % Width of bastard title
\def\OPTbastardtitleskip{8pt}

% FORMATTING OF TITLE PAGE
\def\OPTtitletitlefont{37}    % Size of title font (no unit, pt assumed)
\def\OPTtitletitleskip{10pt}  % Skip between title and subtitle (with units)
\def\OPTtitlesubtitlefont{25} % Size of subtitle font (no unit, pt assumed)
\def\OPTtitlewidth{0.65\textwidth} % Width of title
\def\OPTtitleskip{30pt}       % Skip between title and author (with units)
\def\OPTtitleauthorfont{18}   % Size of author font (no unit, pt assumed)
\def\OPTtitleauthorskip{6pt}  % Skip betewen first and second line of author (with units)

% FORMATTING OF PART PAGE
\def\OPTpartfont{40}          % Size of "Part X" (without unit, pt assumed)
\def\OPTpartskip{20pt}        % Skip between Part and title (with unit)
\def\OPTparttitlefont{60}     % Size of part title (without unit, pt assumed)

% FORMATTING OF CHAPTER TITLE
\def\OPTchapterfont{23}       % Size of "Chapter X"
\def\OPTchapterskip{20pt}     % Skip between Chapter and title
\def\OPTchaptertitlefont{35}  % Size of chapter title

% FORMATTING OF PREFACE
\def\OPTprefacecols{3}        % Number of columns when listing people in Preface

% GETTING RID OF THE WIDOW ON PAGE 2 IN LETTER FORMAT
\def\OPTwidow{\enlargethispage{\baselineskip}}

% FORMATTING OF TABLE 1 IN INTRODUCTION
\def\OPTsmalltable{}

% FORMATTING OF TABLE 8.1 OF HOMOTOPY GROUPS OF SPHERES
\newcommand{\OPTspherescolwidth}{30pt}

% FORMATTING THE BIBLIOGRAPHY
\newcommand{\OPTbibliographyfont}{\small}

% FORMATTING OF INDEX
\newcommand{\OPTindexfont}{\small}      % size of font in the index
\newcommand{\OPTindexcolumnsep}{20.0pt} % column separation in index (default is 10.0pt)


\usepackage{etex}

%%%%%%%%%%%%%%%%%%%%%%%%%%%%%%%%%%%%%%%%%%%%%%%%%%%%%%%%%%%%%%%%
%% packages included by original hott main.tex

%%% For table {tab:theorems}
\usepackage{pifont}

%%% Multi-Columns for long lists of names
\usepackage{multicol}

\usepackage{graphicx}
\usepackage{comment}

\usepackage{fancyhdr} % To set headers and footers

%% \usepackage{nextpage} % So we can jump to odd-numbered pages

\usepackage{amssymb,amsmath,stmaryrd,mathrsfs,wasysym}
\usepackage{enumitem,mathtools,xspace}
%% \numberwithin{equation}{subsection}

\usepackage{xstring} % For generating singluars and plurals in \backref

%% \usepackage{xcolor,mdframed}
\usepackage{xcolor} % For colored cells in tables we need \cellcolor
\usepackage{wallpaper} % For the background image on the cover page

\usepackage{booktabs} % For nice tables
\usepackage{array} % For nice tables

\definecolor{linkcolor}{rgb}{\OPTlinkcolor}
\usepackage{aliascnt}
\usepackage[capitalize]{cleveref}
\usepackage[all,2cell,cmtip]{xy}
\UseAllTwocells
%\usepackage{natbib}
\usepackage{braket} % used for \setof{ ... } macro

\usepackage{tikz}
\usetikzlibrary{decorations.pathmorphing,arrows}

\usepackage{etoolbox}           % hacking commands for TOC

%% \usepackage{mathpartir}         % for formal.tex appendix, section 3

\usepackage[numbered]{bookmark} % add chapter/section numbers to the toc in the pdf metadata
%%%%%%%%%%%%%%%%%%%%%%%%%%%%%%%%%%%%%%%%%%%%%%%%%%%%%%%%%%%%%%%%

%% these two go together!
\usepackage{framed}
\usepackage[standard,framed]{ntheorem}
%% \newtheorem{theorem}{Theorem}
%\newtheorem{cor}{Corollary}
%\newtheorem{lem}{Lemma}
%% \newtheorem*{defn}{Definition}
%% \theoremstyle{remark}
%% \newtheorem{remark}{Remark}
%% \newtheorem*{commentary}{Commentary}

%% \theoremclass{Remark}
%% \theoremstyle{break}
%% \newtheorem{note}{Note}[section]

\theoremstyle{plain}
\theorembodyfont{\upshape}
\theoremsymbol{\ensuremath{\ast}}
\theoremseparator{}
%% \newtheorem{ednote}{Ed. note}[section]
\newframedtheorem{ednote}{Ed. note}[section]

\newtheorem*{todo}{TODO}
%% \newtheorem{eg}{Example}

%%%% MACROS FOR NOTATION %%%%
% Use these for any notation where there are multiple options.

%%% Notes and exercise sections
\makeatletter
\newcommand{\sectionNotes}{\phantomsection\section*{Notes}\addcontentsline{toc}{section}{Notes}\markright{\textsc{\@chapapp{} \thechapter{} Notes}}}
\newcommand{\sectionExercises}[1]{\phantomsection\section*{Exercises}\addcontentsline{toc}{section}{Exercises}\markright{\textsc{\@chapapp{} \thechapter{} Exercises}}}
\makeatother

%%% Definitional equality (used infix) %%%
\newcommand{\jdeq}{\equiv}      % An equality judgment
\let\judgeq\jdeq
%\newcommand{\defeq}{\coloneqq}  % An equality currently being defined
\newcommand{\defeq}{\vcentcolon\equiv}  % A judgmental equality currently being defined

%%% Term being defined
\newcommand{\define}[1]{\textbf{#1}}

%%% Vec (for example)

\newcommand{\Vect}{\ensuremath{\mathsf{Vec}}}
\newcommand{\Fin}{\ensuremath{\mathsf{Fin}}}
\newcommand{\fmax}{\ensuremath{\mathsf{fmax}}}
\newcommand{\seq}[1]{\langle #1\rangle}

%%% Dependent products %%%
\def\prdsym{\textstyle\prod}
%% Call the macro like \prd{x,y:A}{p:x=y} with any number of
%% arguments.  Make sure that whatever comes *after* the call doesn't
%% begin with an open-brace, or it will be parsed as another argument.
\makeatletter
% Currently the macro is configured to produce
%     {\textstyle\prod}(x:A) \; {\textstyle\prod}(y:B),\ 
% in display-math mode, and
%     \prod_{(x:A)} \prod_{y:B}
% in text-math mode.
% \def\prd#1{\@ifnextchar\bgroup{\prd@parens{#1}}{%
%     \@ifnextchar\sm{\prd@parens{#1}\@eatsm}{%
%         \prd@noparens{#1}}}}
\def\prd#1{\@ifnextchar\bgroup{\prd@parens{#1}}{%
    \@ifnextchar\sm{\prd@parens{#1}\@eatsm}{%
    \@ifnextchar\prd{\prd@parens{#1}\@eatprd}{%
    \@ifnextchar\;{\prd@parens{#1}\@eatsemicolonspace}{%
    \@ifnextchar\\{\prd@parens{#1}\@eatlinebreak}{%
    \@ifnextchar\narrowbreak{\prd@parens{#1}\@eatnarrowbreak}{%
      \prd@noparens{#1}}}}}}}}
\def\prd@parens#1{\@ifnextchar\bgroup%
  {\mathchoice{\@dprd{#1}}{\@tprd{#1}}{\@tprd{#1}}{\@tprd{#1}}\prd@parens}%
  {\@ifnextchar\sm%
    {\mathchoice{\@dprd{#1}}{\@tprd{#1}}{\@tprd{#1}}{\@tprd{#1}}\@eatsm}%
    {\mathchoice{\@dprd{#1}}{\@tprd{#1}}{\@tprd{#1}}{\@tprd{#1}}}}}
\def\@eatsm\sm{\sm@parens}
\def\prd@noparens#1{\mathchoice{\@dprd@noparens{#1}}{\@tprd{#1}}{\@tprd{#1}}{\@tprd{#1}}}
% Helper macros for three styles
\def\lprd#1{\@ifnextchar\bgroup{\@lprd{#1}\lprd}{\@@lprd{#1}}}
\def\@lprd#1{\mathchoice{{\textstyle\prod}}{\prod}{\prod}{\prod}({\textstyle #1})\;}
\def\@@lprd#1{\mathchoice{{\textstyle\prod}}{\prod}{\prod}{\prod}({\textstyle #1}),\ }
\def\tprd#1{\@tprd{#1}\@ifnextchar\bgroup{\tprd}{}}
\def\@tprd#1{\mathchoice{{\textstyle\prod_{(#1)}}}{\prod_{(#1)}}{\prod_{(#1)}}{\prod_{(#1)}}}
\def\dprd#1{\@dprd{#1}\@ifnextchar\bgroup{\dprd}{}}
\def\@dprd#1{\prod_{(#1)}\,}
\def\@dprd@noparens#1{\prod_{#1}\,}

% Look through spaces and linebreaks
\def\@eatnarrowbreak\narrowbreak{%
  \@ifnextchar\prd{\narrowbreak\@eatprd}{%
    \@ifnextchar\sm{\narrowbreak\@eatsm}{%
      \narrowbreak}}}
\def\@eatlinebreak\\{%
  \@ifnextchar\prd{\\\@eatprd}{%
    \@ifnextchar\sm{\\\@eatsm}{%
      \\}}}
\def\@eatsemicolonspace\;{%
  \@ifnextchar\prd{\;\@eatprd}{%
    \@ifnextchar\sm{\;\@eatsm}{%
      \;}}}

%%% Lambda abstractions.
% Each variable being abstracted over is a separate argument.  If
% there is more than one such argument, they *must* be enclosed in
% braces.  Arguments can be untyped, as in \lam{x}{y}, or typed with a
% colon, as in \lam{x:A}{y:B}. In the latter case, the colons are
% automatically noticed and (with current implementation) the space
% around the colon is reduced.  You can even give more than one variable
% the same type, as in \lam{x,y:A}.
\def\lam#1{{\lambda}\@lamarg#1:\@endlamarg\@ifnextchar\bgroup{.\,\lam}{.\,}}
\def\@lamarg#1:#2\@endlamarg{\if\relax\detokenize{#2}\relax #1\else\@lamvar{\@lameatcolon#2},#1\@endlamvar\fi}
\def\@lamvar#1,#2\@endlamvar{(#2\,{:}\,#1)}
% \def\@lamvar#1,#2{{#2}^{#1}\@ifnextchar,{.\,{\lambda}\@lamvar{#1}}{\let\@endlamvar\relax}}
\def\@lameatcolon#1:{#1}
\let\lamt\lam
% This version silently eats any typing annotation.
\def\lamu#1{{\lambda}\@lamuarg#1:\@endlamuarg\@ifnextchar\bgroup{.\,\lamu}{.\,}}
\def\@lamuarg#1:#2\@endlamuarg{#1}

%%% Dependent products written with \forall, in the same style
\def\fall#1{\forall (#1)\@ifnextchar\bgroup{.\,\fall}{.\,}}

%%% Existential quantifier %%%
\def\exis#1{\exists (#1)\@ifnextchar\bgroup{.\,\exis}{.\,}}

%%% Dependent sums %%%
\def\smsym{\textstyle\sum}
% Use in the same way as \prd
\def\sm#1{\@ifnextchar\bgroup{\sm@parens{#1}}{%
    \@ifnextchar\prd{\sm@parens{#1}\@eatprd}{%
    \@ifnextchar\sm{\sm@parens{#1}\@eatsm}{%
    \@ifnextchar\;{\sm@parens{#1}\@eatsemicolonspace}{%
    \@ifnextchar\\{\sm@parens{#1}\@eatlinebreak}{%
    \@ifnextchar\narrowbreak{\sm@parens{#1}\@eatnarrowbreak}{%
        \sm@noparens{#1}}}}}}}}
\def\sm@parens#1{\@ifnextchar\bgroup%
  {\mathchoice{\@dsm{#1}}{\@tsm{#1}}{\@tsm{#1}}{\@tsm{#1}}\sm@parens}%
  {\@ifnextchar\prd%
    {\mathchoice{\@dsm{#1}}{\@tsm{#1}}{\@tsm{#1}}{\@tsm{#1}}\@eatprd}%
    {\mathchoice{\@dsm{#1}}{\@tsm{#1}}{\@tsm{#1}}{\@tsm{#1}}}}}
\def\@eatprd\prd{\prd@parens}
\def\sm@noparens#1{\mathchoice{\@dsm@noparens{#1}}{\@tsm{#1}}{\@tsm{#1}}{\@tsm{#1}}}
\def\lsm#1{\@ifnextchar\bgroup{\@lsm{#1}\lsm}{\@@lsm{#1}}}
\def\@lsm#1{\mathchoice{{\textstyle\sum}}{\sum}{\sum}{\sum}({\textstyle #1})\;}
\def\@@lsm#1{\mathchoice{{\textstyle\sum}}{\sum}{\sum}{\sum}({\textstyle #1}),\ }
\def\tsm#1{\@tsm{#1}\@ifnextchar\bgroup{\tsm}{}}
\def\@tsm#1{\mathchoice{{\textstyle\sum_{(#1)}}}{\sum_{(#1)}}{\sum_{(#1)}}{\sum_{(#1)}}}
\def\dsm#1{\@dsm{#1}\@ifnextchar\bgroup{\dsm}{}}
\def\@dsm#1{\sum_{(#1)}\,}
\def\@dsm@noparens#1{\sum_{#1}\,}

%%% W-types
\def\wtypesym{{\mathsf{W}}}
\def\wtype#1{\@ifnextchar\bgroup%
  {\mathchoice{\@twtype{#1}}{\@twtype{#1}}{\@twtype{#1}}{\@twtype{#1}}\wtype}%
  {\mathchoice{\@twtype{#1}}{\@twtype{#1}}{\@twtype{#1}}{\@twtype{#1}}}}
\def\lwtype#1{\@ifnextchar\bgroup{\@lwtype{#1}\lwtype}{\@@lwtype{#1}}}
\def\@lwtype#1{\mathchoice{{\textstyle\mathsf{W}}}{\mathsf{W}}{\mathsf{W}}{\mathsf{W}}({\textstyle #1})\;}
\def\@@lwtype#1{\mathchoice{{\textstyle\mathsf{W}}}{\mathsf{W}}{\mathsf{W}}{\mathsf{W}}({\textstyle #1}),\ }
\def\twtype#1{\@twtype{#1}\@ifnextchar\bgroup{\twtype}{}}
\def\@twtype#1{\mathchoice{{\textstyle\mathsf{W}_{(#1)}}}{\mathsf{W}_{(#1)}}{\mathsf{W}_{(#1)}}{\mathsf{W}_{(#1)}}}
\def\dwtype#1{\@dwtype{#1}\@ifnextchar\bgroup{\dwtype}{}}
\def\@dwtype#1{\mathsf{W}_{(#1)}\,}

\newcommand{\suppsym}{{\mathsf{sup}}}
\newcommand{\supp}{\ensuremath\suppsym\xspace}

\def\wtypeh#1{\@ifnextchar\bgroup%
  {\mathchoice{\@lwtypeh{#1}}{\@twtypeh{#1}}{\@twtypeh{#1}}{\@twtypeh{#1}}\wtypeh}%
  {\mathchoice{\@@lwtypeh{#1}}{\@twtypeh{#1}}{\@twtypeh{#1}}{\@twtypeh{#1}}}}
\def\lwtypeh#1{\@ifnextchar\bgroup{\@lwtypeh{#1}\lwtypeh}{\@@lwtypeh{#1}}}
\def\@lwtypeh#1{\mathchoice{{\textstyle\mathsf{W}^h}}{\mathsf{W}^h}{\mathsf{W}^h}{\mathsf{W}^h}({\textstyle #1})\;}
\def\@@lwtypeh#1{\mathchoice{{\textstyle\mathsf{W}^h}}{\mathsf{W}^h}{\mathsf{W}^h}{\mathsf{W}^h}({\textstyle #1}),\ }
\def\twtypeh#1{\@twtypeh{#1}\@ifnextchar\bgroup{\twtypeh}{}}
\def\@twtypeh#1{\mathchoice{{\textstyle\mathsf{W}^h_{(#1)}}}{\mathsf{W}^h_{(#1)}}{\mathsf{W}^h_{(#1)}}{\mathsf{W}^h_{(#1)}}}
\def\dwtypeh#1{\@dwtypeh{#1}\@ifnextchar\bgroup{\dwtypeh}{}}
\def\@dwtypeh#1{\mathsf{W}^h_{(#1)}\,}


\makeatother

% Other notations related to dependent sums
\let\setof\Set    % from package 'braket', write \setof{ x:A | P(x) }.
\newcommand{\pair}{\ensuremath{\mathsf{pair}}\xspace}
\newcommand{\tup}[2]{(#1,#2)}
\newcommand{\proj}[1]{\ensuremath{\mathsf{pr}_{#1}}\xspace}
\newcommand{\fst}{\ensuremath{\proj1}\xspace}
\newcommand{\snd}{\ensuremath{\proj2}\xspace}
\newcommand{\ac}{\ensuremath{\mathsf{ac}}\xspace} % not needed in symbol index
\newcommand{\un}{\ensuremath{\mathsf{upun}}\xspace} % not needed in symbol index, uniqueness principle for unit type

%%% recursor and induction
\newcommand{\rec}[1]{\mathsf{rec}_{#1}}
\newcommand{\ind}[1]{\mathsf{ind}_{#1}}
\newcommand{\indid}[1]{\ind{=_{#1}}} % (Martin-Lof) path induction principle for identity types
\newcommand{\indidb}[1]{\ind{=_{#1}}'} % (Paulin-Mohring) based path induction principle for identity types 

%%% the uniqueness principle for product types, formerly called surjective pairing and named \spr:
\newcommand{\uppt}{\ensuremath{\mathsf{uppt}}\xspace}

% Paths in pairs
\newcommand{\pairpath}{\ensuremath{\mathsf{pair}^{\mathord{=}}}\xspace}
% \newcommand{\projpath}[1]{\proj{#1}^{\mathord{=}}}
\newcommand{\projpath}[1]{\ensuremath{\apfunc{\proj{#1}}}\xspace}

%%% For quotients %%%
%\newcommand{\pairr}[1]{{\langle #1\rangle}}
\newcommand{\pairr}[1]{{\mathopen{}(#1)\mathclose{}}}
\newcommand{\Pairr}[1]{{\mathopen{}\left(#1\right)\mathclose{}}}

% \newcommand{\type}{\ensuremath{\mathsf{Type}}} % this command is overridden below, so it's commented out
\newcommand{\im}{\ensuremath{\mathsf{im}}} % the image

%%% 2D path operations
\newcommand{\leftwhisker}{\mathbin{{\ct}_{\mathsf{l}}}}  % was \ell
\newcommand{\rightwhisker}{\mathbin{{\ct}_{\mathsf{r}}}} % was r
\newcommand{\hct}{\star}

%%% modalities %%%
\newcommand{\modal}{\ensuremath{\ocircle}}
\let\reflect\modal
\newcommand{\modaltype}{\ensuremath{\type_\modal}}
% \newcommand{\ism}[1]{\ensuremath{\mathsf{is}_{#1}}}
% \newcommand{\ismodal}{\ism{\modal}}
% \newcommand{\existsmodal}{\ensuremath{{\exists}_{\modal}}}
% \newcommand{\existsmodalunique}{\ensuremath{{\exists!}_{\modal}}}
% \newcommand{\modalfunc}{\textsf{\modal-fun}}
% \newcommand{\Ecirc}{\ensuremath{\mathsf{E}_\modal}}
% \newcommand{\Mcirc}{\ensuremath{\mathsf{M}_\modal}}
\newcommand{\mreturn}{\ensuremath{\eta}}
%% \let\project\mreturn
%\newcommand{\mbind}[1]{\ensuremath{\hat{#1}}}
\newcommand{\ext}{\mathsf{ext}}
%\newcommand{\mmap}[1]{\ensuremath{\bar{#1}}}
%\newcommand{\mjoin}{\ensuremath{\mreturn^{-1}}}
% Subuniverse
\renewcommand{\P}{\ensuremath{\type_{P}}\xspace}

%%% Localizations
% \newcommand{\islocal}[1]{\ensuremath{\mathsf{islocal}_{#1}}\xspace}
% \newcommand{\loc}[1]{\ensuremath{\mathcal{L}_{#1}}\xspace}

%%% Identity types %%%
\newcommand{\idsym}{{=}}
\newcommand{\id}[3][]{\ensuremath{#2 =_{#1} #3}\xspace}
\newcommand{\idtype}[3][]{\ensuremath{\mathsf{Id}_{#1}(#2,#3)}\xspace}
\newcommand{\idtypevar}[1]{\ensuremath{\mathsf{Id}_{#1}}\xspace}
% A propositional equality currently being defined
\newcommand{\defid}{\coloneqq}

%%% Dependent paths
\newcommand{\dpath}[4]{#3 =^{#1}_{#2} #4}

%%% singleton
% \newcommand{\sgl}{\ensuremath{\mathsf{sgl}}\xspace}
% \newcommand{\sctr}{\ensuremath{\mathsf{sctr}}\xspace}

%%% Reflexivity terms %%%
% \newcommand{\reflsym}{{\mathsf{refl}}}
\newcommand{\refl}[1]{\ensuremath{\mathsf{refl}_{#1}}\xspace}

%%% Path concatenation (used infix, in diagrammatic order) %%%
\newcommand{\ct}{%
  \mathchoice{\mathbin{\raisebox{0.5ex}{$\displaystyle\centerdot$}}}%
             {\mathbin{\raisebox{0.5ex}{$\centerdot$}}}%
             {\mathbin{\raisebox{0.25ex}{$\scriptstyle\,\centerdot\,$}}}%
             {\mathbin{\raisebox{0.1ex}{$\scriptscriptstyle\,\centerdot\,$}}}
}

%%% Path reversal %%%
\newcommand{\opp}[1]{\mathord{{#1}^{-1}}}
\let\rev\opp

%%% Transport (covariant) %%%
\newcommand{\trans}[2]{\ensuremath{{#1}_{*}\mathopen{}\left({#2}\right)\mathclose{}}\xspace}
\let\Trans\trans
%\newcommand{\Trans}[2]{\ensuremath{{#1}_{*}\left({#2}\right)}\xspace}
\newcommand{\transf}[1]{\ensuremath{{#1}_{*}}\xspace} % Without argument
%\newcommand{\transport}[2]{\ensuremath{\mathsf{transport}_{*} \: {#2}\xspace}}
\newcommand{\transfib}[3]{\ensuremath{\mathsf{transport}^{#1}(#2,#3)\xspace}}
\newcommand{\Transfib}[3]{\ensuremath{\mathsf{transport}^{#1}\Big(#2,\, #3\Big)\xspace}}
\newcommand{\transfibf}[1]{\ensuremath{\mathsf{transport}^{#1}\xspace}}

%%% 2D transport
\newcommand{\transtwo}[2]{\ensuremath{\mathsf{transport}^2\mathopen{}\left({#1},{#2}\right)\mathclose{}}\xspace}

%%% Constant transport
\newcommand{\transconst}[3]{\ensuremath{\mathsf{transportconst}}^{#1}_{#2}(#3)\xspace}
\newcommand{\transconstf}{\ensuremath{\mathsf{transportconst}}\xspace}

%%% Map on paths %%%
\newcommand{\mapfunc}[1]{\ensuremath{\mathsf{ap}_{#1}}\xspace} % Without argument
\newcommand{\map}[2]{\ensuremath{{#1}\mathopen{}\left({#2}\right)\mathclose{}}\xspace}
\let\Ap\map
%\newcommand{\Ap}[2]{\ensuremath{{#1}\left({#2}\right)}\xspace}
\newcommand{\mapdepfunc}[1]{\ensuremath{\mathsf{apd}_{#1}}\xspace} % Without argument
% \newcommand{\mapdep}[2]{\ensuremath{{#1}\llparenthesis{#2}\rrparenthesis}\xspace}
\newcommand{\mapdep}[2]{\ensuremath{\mapdepfunc{#1}\mathopen{}\left(#2\right)\mathclose{}}\xspace}
\let\apfunc\mapfunc
\let\ap\map
\let\apdfunc\mapdepfunc
\let\apd\mapdep

%%% 2D map on paths
\newcommand{\aptwofunc}[1]{\ensuremath{\mathsf{ap}^2_{#1}}\xspace}
\newcommand{\aptwo}[2]{\ensuremath{\aptwofunc{#1}\mathopen{}\left({#2}\right)\mathclose{}}\xspace}
\newcommand{\apdtwofunc}[1]{\ensuremath{\mathsf{apd}^2_{#1}}\xspace}
\newcommand{\apdtwo}[2]{\ensuremath{\apdtwofunc{#1}\mathopen{}\left(#2\right)\mathclose{}}\xspace}

%%% Identity functions %%%
\newcommand{\idfunc}[1][]{\ensuremath{\mathsf{id}_{#1}}\xspace}

%%% Homotopies (written infix) %%%
\newcommand{\htpy}{\sim}

%%% Other meanings of \sim
\newcommand{\bisim}{\sim}       % bisimulation
\newcommand{\eqr}{\sim}         % an equivalence relation

%%% Equivalence types %%%
\newcommand{\eqv}[2]{\ensuremath{#1 \simeq #2}\xspace}
\newcommand{\eqvspaced}[2]{\ensuremath{#1 \;\simeq\; #2}\xspace}
\newcommand{\eqvsym}{\simeq}    % infix symbol
\newcommand{\texteqv}[2]{\ensuremath{\mathsf{Equiv}(#1,#2)}\xspace}
\newcommand{\isequiv}{\ensuremath{\mathsf{isequiv}}}
\newcommand{\qinv}{\ensuremath{\mathsf{qinv}}}
\newcommand{\ishae}{\ensuremath{\mathsf{ishae}}}
\newcommand{\linv}{\ensuremath{\mathsf{linv}}}
\newcommand{\rinv}{\ensuremath{\mathsf{rinv}}}
\newcommand{\biinv}{\ensuremath{\mathsf{biinv}}}
\newcommand{\lcoh}[3]{\mathsf{lcoh}_{#1}(#2,#3)}
\newcommand{\rcoh}[3]{\mathsf{rcoh}_{#1}(#2,#3)}
\newcommand{\hfib}[2]{{\mathsf{fib}}_{#1}(#2)}

%%% Map on total spaces %%%
\newcommand{\total}[1]{\ensuremath{\mathsf{total}(#1)}}

%%% Universe types %%%
%\newcommand{\type}{\ensuremath{\mathsf{Type}}\xspace}
\newcommand{\UU}{\ensuremath{\mathcal{U}}\xspace}
\let\bbU\UU
\let\type\UU
% Universes of truncated types
\newcommand{\typele}[1]{\ensuremath{{#1}\text-\mathsf{Type}}\xspace}
\newcommand{\typeleU}[1]{\ensuremath{{#1}\text-\mathsf{Type}_\UU}\xspace}
\newcommand{\typelep}[1]{\ensuremath{{(#1)}\text-\mathsf{Type}}\xspace}
\newcommand{\typelepU}[1]{\ensuremath{{(#1)}\text-\mathsf{Type}_\UU}\xspace}
\let\ntype\typele
\let\ntypeU\typeleU
\let\ntypep\typelep
\let\ntypepU\typelepU
\renewcommand{\set}{\ensuremath{\mathsf{Set}}\xspace}
\newcommand{\setU}{\ensuremath{\mathsf{Set}_\UU}\xspace}
\newcommand{\prop}{\ensuremath{\mathsf{Prop}}\xspace}
\newcommand{\propU}{\ensuremath{\mathsf{Prop}_\UU}\xspace}
%Pointed types
\newcommand{\pointed}[1]{\ensuremath{#1_\bullet}}

%%% Ordinals and cardinals
\newcommand{\card}{\ensuremath{\mathsf{Card}}\xspace}
\newcommand{\ord}{\ensuremath{\mathsf{Ord}}\xspace}
\newcommand{\ordsl}[2]{{#1}_{/#2}}

%%% Univalence
\newcommand{\ua}{\ensuremath{\mathsf{ua}}\xspace} % the inverse of idtoeqv
\newcommand{\idtoeqv}{\ensuremath{\mathsf{idtoeqv}}\xspace}
\newcommand{\univalence}{\ensuremath{\mathsf{univalence}}\xspace} % the full axiom

%%% Truncation levels
\newcommand{\iscontr}{\ensuremath{\mathsf{isContr}}}
\newcommand{\contr}{\ensuremath{\mathsf{contr}}} % The path to the center of contraction
\newcommand{\isset}{\ensuremath{\mathsf{isSet}}}
\newcommand{\isprop}{\ensuremath{\mathsf{isProp}}}
% h-propositions
% \newcommand{\anhprop}{a mere proposition\xspace}
% \newcommand{\hprops}{mere propositions\xspace}

%%% Homotopy fibers %%%
%\newcommand{\hfiber}[2]{\ensuremath{\mathsf{hFiber}(#1,#2)}\xspace}
\let\hfiber\hfib

%%% Bracket/squash/truncation types %%%
% \newcommand{\brck}[1]{\textsf{mere}(#1)}
% \newcommand{\Brck}[1]{\textsf{mere}\Big(#1\Big)}
% \newcommand{\trunc}[2]{\tau_{#1}(#2)}
% \newcommand{\Trunc}[2]{\tau_{#1}\Big(#2\Big)}
% \newcommand{\truncf}[1]{\tau_{#1}}
%\newcommand{\trunc}[2]{\Vert #2\Vert_{#1}}
\newcommand{\trunc}[2]{\mathopen{}\left\Vert #2\right\Vert_{#1}\mathclose{}}
\newcommand{\ttrunc}[2]{\bigl\Vert #2\bigr\Vert_{#1}}
\newcommand{\Trunc}[2]{\Bigl\Vert #2\Bigr\Vert_{#1}}
\newcommand{\truncf}[1]{\Vert \blank \Vert_{#1}}
\newcommand{\tproj}[3][]{\mathopen{}\left|#3\right|_{#2}^{#1}\mathclose{}}
\newcommand{\tprojf}[2][]{|\blank|_{#2}^{#1}}
\def\pizero{\trunc0}
%\newcommand{\brck}[1]{\trunc{-1}{#1}}
%\newcommand{\Brck}[1]{\Trunc{-1}{#1}}
%\newcommand{\bproj}[1]{\tproj{-1}{#1}}
%\newcommand{\bprojf}{\tprojf{-1}}

\newcommand{\brck}[1]{\trunc{}{#1}}
\newcommand{\bbrck}[1]{\ttrunc{}{#1}}
\newcommand{\Brck}[1]{\Trunc{}{#1}}
\newcommand{\bproj}[1]{\tproj{}{#1}}
\newcommand{\bprojf}{\tprojf{}}

% Big parentheses
\newcommand{\Parens}[1]{\Bigl(#1\Bigr)}

% Projection and extension for truncations
\let\extendsmb\ext
\newcommand{\extend}[1]{\extendsmb(#1)}

%
%%% The empty type
\newcommand{\emptyt}{\ensuremath{\mathbf{0}}\xspace}

%%% The unit type
\newcommand{\unit}{\ensuremath{\mathbf{1}}\xspace}
\newcommand{\ttt}{\ensuremath{\star}\xspace}

%%% The two-element type
\newcommand{\bool}{\ensuremath{\mathbf{2}}\xspace}
\newcommand{\btrue}{{1_{\bool}}}
\newcommand{\bfalse}{{0_{\bool}}}

%%% Injections into binary sums and pushouts
\newcommand{\inlsym}{{\mathsf{inl}}}
\newcommand{\inrsym}{{\mathsf{inr}}}
\newcommand{\inl}{\ensuremath\inlsym\xspace}
\newcommand{\inr}{\ensuremath\inrsym\xspace}

%%% The segment of the interval
\newcommand{\seg}{\ensuremath{\mathsf{seg}}\xspace}

%%% Free groups
\newcommand{\freegroup}[1]{F(#1)}
\newcommand{\freegroupx}[1]{F'(#1)} % the "other" free group

%%% Glue of a pushout
\newcommand{\glue}{\mathsf{glue}}

%%% Circles and spheres
\newcommand{\Sn}{\mathbb{S}}
\newcommand{\base}{\ensuremath{\mathsf{base}}\xspace}
\newcommand{\lloop}{\ensuremath{\mathsf{loop}}\xspace}
\newcommand{\surf}{\ensuremath{\mathsf{surf}}\xspace}

%%% Suspension
\newcommand{\susp}{\Sigma}
\newcommand{\north}{\mathsf{N}}
\newcommand{\south}{\mathsf{S}}
\newcommand{\merid}{\mathsf{merid}}

%%% Blanks (shorthand for lambda abstractions)
\newcommand{\blank}{\mathord{\hspace{1pt}\text{--}\hspace{1pt}}}

%%% Nameless objects
\newcommand{\nameless}{\mathord{\hspace{1pt}\underline{\hspace{1ex}}\hspace{1pt}}}

%%% Some decorations
%\newcommand{\bbU}{\ensuremath{\mathbb{U}}\xspace}
% \newcommand{\bbB}{\ensuremath{\mathbb{B}}\xspace}
\newcommand{\bbP}{\ensuremath{\mathbb{P}}\xspace}

%%% Some categories
\newcommand{\uset}{\ensuremath{\mathcal{S}et}\xspace}
\newcommand{\ucat}{\ensuremath{{\mathcal{C}at}}\xspace}
\newcommand{\urel}{\ensuremath{\mathcal{R}el}\xspace}
\newcommand{\uhilb}{\ensuremath{\mathcal{H}ilb}\xspace}
\newcommand{\utype}{\ensuremath{\mathcal{T}\!ype}\xspace}

% Pullback corner
\newbox\pbbox
\setbox\pbbox=\hbox{\xy \POS(65,0)\ar@{-} (0,0) \ar@{-} (65,65)\endxy}
\def\pb{\save[]+<3.5mm,-3.5mm>*{\copy\pbbox} \restore}

% Macros for the categories chapter
\newcommand{\inv}[1]{{#1}^{-1}}
\newcommand{\idtoiso}{\ensuremath{\mathsf{idtoiso}}\xspace}
\newcommand{\isotoid}{\ensuremath{\mathsf{isotoid}}\xspace}
\newcommand{\op}{^{\mathrm{op}}}
\newcommand{\y}{\ensuremath{\mathbf{y}}\xspace}
\newcommand{\dgr}[1]{{#1}^{\dagger}}
\newcommand{\unitaryiso}{\mathrel{\cong^\dagger}}
\newcommand{\cteqv}[2]{\ensuremath{#1 \simeq #2}\xspace}
\newcommand{\cteqvsym}{\simeq}     % Symbol for equivalence of categories

%%% Natural numbers
\newcommand{\N}{\ensuremath{\mathbb{N}}\xspace}
%\newcommand{\N}{\textbf{N}}
\let\nat\N
\newcommand{\natp}{\ensuremath{\nat'}\xspace} % alternative nat in induction chapter

\newcommand{\zerop}{\ensuremath{0'}\xspace}   % alternative zero in induction chapter
\newcommand{\suc}{\mathsf{succ}}
\newcommand{\sucp}{\ensuremath{\suc'}\xspace} % alternative suc in induction chapter
\newcommand{\add}{\mathsf{add}}
\newcommand{\ack}{\mathsf{ack}}
\newcommand{\ite}{\mathsf{iter}}
\newcommand{\assoc}{\mathsf{assoc}}
\newcommand{\dbl}{\ensuremath{\mathsf{double}}}
\newcommand{\dblp}{\ensuremath{\dbl'}\xspace} % alternative double in induction chapter


%%% Lists
\newcommand{\lst}[1]{\mathsf{List}(#1)}
\newcommand{\nil}{\mathsf{nil}}
\newcommand{\cons}{\mathsf{cons}}
\newcommand{\lost}[1]{\mathsf{Lost}(#1)}

%%% Vectors of given length, used in induction chapter
\newcommand{\vect}[2]{\ensuremath{\mathsf{Vec}_{#1}(#2)}\xspace}

%%% Integers
\newcommand{\Z}{\ensuremath{\mathbb{Z}}\xspace}
\newcommand{\Zsuc}{\mathsf{succ}}
\newcommand{\Zpred}{\mathsf{pred}}

%%% Rationals
\newcommand{\Q}{\ensuremath{\mathbb{Q}}\xspace}

%%% Function extensionality
\newcommand{\funext}{\mathsf{funext}}
\newcommand{\happly}{\mathsf{happly}}

%%% A naturality lemma
\newcommand{\com}[3]{\mathsf{swap}_{#1,#2}(#3)}

%%% Code/encode/decode
\newcommand{\code}{\ensuremath{\mathsf{code}}\xspace}
\newcommand{\encode}{\ensuremath{\mathsf{encode}}\xspace}
\newcommand{\decode}{\ensuremath{\mathsf{decode}}\xspace}

% Function definition with domain and codomain
\newcommand{\function}[4]{\left\{\begin{array}{rcl}#1 &
      \longrightarrow & #2 \\ #3 & \longmapsto & #4 \end{array}\right.}

%%% Cones and cocones
\newcommand{\cone}[2]{\mathsf{cone}_{#1}(#2)}
\newcommand{\cocone}[2]{\mathsf{cocone}_{#1}(#2)}
% Apply a function to a cocone
\newcommand{\composecocone}[2]{#1\circ#2}
\newcommand{\composecone}[2]{#2\circ#1}
%%% Diagrams
\newcommand{\Ddiag}{\mathscr{D}}

%%% (pointed) mapping spaces
\newcommand{\Map}{\mathsf{Map}}

%%% The interval
\newcommand{\interval}{\ensuremath{I}\xspace}
\newcommand{\izero}{\ensuremath{0_{\interval}}\xspace}
\newcommand{\ione}{\ensuremath{1_{\interval}}\xspace}

%%% Arrows
\newcommand{\epi}{\ensuremath{\twoheadrightarrow}}
\newcommand{\mono}{\ensuremath{\rightarrowtail}}

%%% Sets
\newcommand{\bin}{\ensuremath{\mathrel{\widetilde{\in}}}}

%%% Semigroup structure
\newcommand{\semigroupstrsym}{\ensuremath{\mathsf{SemigroupStr}}}
\newcommand{\semigroupstr}[1]{\ensuremath{\mathsf{SemigroupStr}}(#1)}
\newcommand{\semigroup}[0]{\ensuremath{\mathsf{Semigroup}}}

%%% Macros for the formal type theory
\newcommand{\emptyctx}{\ensuremath{\cdot}}
\newcommand{\production}{\vcentcolon\vcentcolon=}
\newcommand{\conv}{\downarrow}
\newcommand{\ctx}{\ensuremath{\mathsf{ctx}}}
\newcommand{\wfctx}[1]{#1\ \ctx}
\newcommand{\oftp}[3]{#1 \vdash #2 : #3}
\newcommand{\jdeqtp}[4]{#1 \vdash #2 \jdeq #3 : #4}
\newcommand{\judg}[2]{#1 \vdash #2}
\newcommand{\tmtp}[2]{#1 \mathord{:} #2}

% rule names
\newcommand{\form}{\textsc{form}}
\newcommand{\intro}{\textsc{intro}}
\newcommand{\elim}{\textsc{elim}}
\newcommand{\comp}{\textsc{comp}}
\newcommand{\uniq}{\textsc{uniq}}
\newcommand{\Weak}{\mathsf{Wkg}}
\newcommand{\Vble}{\mathsf{Vble}}
\newcommand{\Exch}{\mathsf{Exch}}
\newcommand{\Subst}{\mathsf{Subst}}

%%% Macros for HITs
\newcommand{\cc}{\mathsf{c}}
\newcommand{\pp}{\mathsf{p}}
\newcommand{\cct}{\widetilde{\mathsf{c}}}
\newcommand{\ppt}{\widetilde{\mathsf{p}}}
\newcommand{\Wtil}{\ensuremath{\widetilde{W}}\xspace}

%%% Macros for n-types
\newcommand{\istype}[1]{\mathsf{is}\mbox{-}{#1}\mbox{-}\mathsf{type}}
\newcommand{\nplusone}{\ensuremath{(n+1)}}
\newcommand{\nminusone}{\ensuremath{(n-1)}}
\newcommand{\fact}{\mathsf{fact}}

%%% Macros for homotopy
\newcommand{\kbar}{\overline{k}} % Used in van Kampen's theorem

%%% Macros for induction
\newcommand{\natw}{\ensuremath{\mathbf{N^w}}\xspace}
\newcommand{\zerow}{\ensuremath{0^\mathbf{w}}\xspace}
\newcommand{\sucw}{\ensuremath{\mathsf{succ}^{\mathbf{w}}}\xspace}
\newcommand{\nalg}{\nat\mathsf{Alg}}
\newcommand{\nhom}{\nat\mathsf{Hom}}
\newcommand{\ishinitw}{\mathsf{isHinit}_{\mathsf{W}}}
\newcommand{\ishinitn}{\mathsf{isHinit}_\nat}
\newcommand{\w}{\mathsf{W}}
\newcommand{\walg}{\w\mathsf{Alg}}
\newcommand{\whom}{\w\mathsf{Hom}}

%%% Macros for real numbers
\newcommand{\RC}{\ensuremath{\mathbb{R}_\mathsf{c}}\xspace} % Cauchy
\newcommand{\RD}{\ensuremath{\mathbb{R}_\mathsf{d}}\xspace} % Dedekind
\newcommand{\R}{\ensuremath{\mathbb{R}}\xspace}           % Either 
\newcommand{\barRD}{\ensuremath{\bar{\mathbb{R}}_\mathsf{d}}\xspace} % Dedekind completion of Dedekind

\newcommand{\close}[1]{\sim_{#1}} % Relation of closeness
\newcommand{\closesym}{\mathord\sim}
\newcommand{\rclim}{\mathsf{lim}} % HIT constructor for Cauchy reals
\newcommand{\rcrat}{\mathsf{rat}} % Embedding of rationals into Cauchy reals
\newcommand{\rceq}{\mathsf{eq}_{\RC}} % HIT path constructor
\newcommand{\CAP}{\mathcal{C}}    % The type of Cauchy approximations
\newcommand{\Qp}{\Q_{+}}
\newcommand{\apart}{\mathrel{\#}}  % apartness
\newcommand{\dcut}{\mathsf{isCut}}  % Dedekind cut
\newcommand{\cover}{\triangleleft} % inductive cover
\newcommand{\intfam}[3]{(#2, \lam{#1} #3)} % family of rational intervals

% Macros for the Cauchy reals construction
\newcommand{\bsim}{\frown}
\newcommand{\bbsim}{\smile}

\newcommand{\hapx}{\diamondsuit\approx}
\newcommand{\hapname}{\diamondsuit}
\newcommand{\hapxb}{\heartsuit\approx}
\newcommand{\hapbname}{\heartsuit}
\newcommand{\tap}[1]{\bullet\approx_{#1}\triangle}
\newcommand{\tapname}{\triangle}
\newcommand{\tapb}[1]{\bullet\approx_{#1}\square}
\newcommand{\tapbname}{\square}

%%% Macros for surreals
\newcommand{\NO}{\ensuremath{\mathsf{No}}\xspace}
\newcommand{\surr}[2]{\{\,#1\,\big|\,#2\,\}}
\newcommand{\LL}{\mathcal{L}}
\newcommand{\RR}{\mathcal{R}}
\newcommand{\noeq}{\mathsf{eq}_{\NO}} % HIT path constructor

\newcommand{\ble}{\trianglelefteqslant}
\newcommand{\blt}{\vartriangleleft}
\newcommand{\bble}{\sqsubseteq}
\newcommand{\bblt}{\sqsubset}

\newcommand{\hle}{\diamondsuit\preceq}
\newcommand{\hlt}{\diamondsuit\prec}
\newcommand{\hlname}{\diamondsuit}
\newcommand{\hleb}{\heartsuit\preceq}
\newcommand{\hltb}{\heartsuit\prec}
\newcommand{\hlbname}{\heartsuit}
% \newcommand{\tle}{(\bullet\preceq\triangle)}
% \newcommand{\tlt}{(\bullet\prec\triangle)}
\newcommand{\tle}{\triangle\preceq}
\newcommand{\tlt}{\triangle\prec}
\newcommand{\tlname}{\triangle}
% \newcommand{\tleb}{(\bullet\preceq\square)}
% \newcommand{\tltb}{(\bullet\prec\square)}
\newcommand{\tleb}{\square\preceq}
\newcommand{\tltb}{\square\prec}
\newcommand{\tlbname}{\square}

%%% Macros for set theory
\newcommand{\vset}{\mathsf{set}}  % point constructor for cummulative hierarchy V
\def\cd{\tproj0}
\newcommand{\inj}{\ensuremath{\mathsf{inj}}} % type of injections
\newcommand{\acc}{\ensuremath{\mathsf{acc}}} % accessibility

\newcommand{\atMostOne}{\mathsf{atMostOne}}

\newcommand{\power}[1]{\mathcal{P}(#1)} % power set
\newcommand{\powerp}[1]{\mathcal{P}_+(#1)} % inhabited power set

%%%% THEOREM ENVIRONMENTS %%%%

% Hyperref includes the command \autoref{...} which is like \ref{...}
% except that it automatically inserts the type of the thing you're
% referring to, e.g. it produces "Theorem 3.8" instead of just "3.8"
% (and makes the whole thing a hyperlink).  This saves a slight amount
% of typing, but more importantly it means that if you decide later on
% that 3.8 should be a Lemma or a Definition instead of a Theorem, you
% don't have to change the name in all the places you referred to it.

% The following hack improves on this by using the same counter for
% all theorem-type environments, so that after Theorem 1.1 comes
% Corollary 1.2 rather than Corollary 1.1.  This makes it much easier
% for the reader to find a particular theorem when flipping through
% the document.
\makeatletter
\def\defthm#1#2#3{%
  %% Ensure all theorem types are numbered with the same counter
  \newaliascnt{#1}{thm}
  \newtheorem{#1}[#1]{#2}
  \aliascntresetthe{#1}
  %% This command tells cleveref's \cref what to call things
  \crefname{#1}{#2}{#3}}

% Now define a bunch of theorem-type environments.
\newtheorem{thm}{Theorem}[section]
\crefname{thm}{Theorem}{Theorems}
%\defthm{prop}{Proposition}   % Probably we shouldn't use "Proposition" in this way
\defthm{cor}{Corollary}{Corollaries}
\defthm{lem}{Lemma}{Lemmas}
\defthm{axiom}{Axiom}{Axioms}
% Since definitions and theorems in type theory are synonymous, should
% we actually use the same theoremstyle for them?
\theoremstyle{definition}
\defthm{defn}{Definition}{Definitions}
\theoremstyle{remark}
\defthm{rmk}{Remark}{Remarks}
\defthm{eg}{Example}{Examples}
\defthm{egs}{Examples}{Examples}
\defthm{notes}{Notes}{Notes}
% Number exercises within chapters, with their own counter.
%% \newtheorem{ex}{Exercise}[chapter]
\crefname{ex}{Exercise}{Exercises}

% Display format for sections
\crefformat{section}{\S#2#1#3}
\Crefformat{section}{Section~#2#1#3}
\crefrangeformat{section}{\S\S#3#1#4--#5#2#6}
\Crefrangeformat{section}{Sections~#3#1#4--#5#2#6}
\crefmultiformat{section}{\S\S#2#1#3}{ and~#2#1#3}{, #2#1#3}{ and~#2#1#3}
\Crefmultiformat{section}{Sections~#2#1#3}{ and~#2#1#3}{, #2#1#3}{ and~#2#1#3}
\crefrangemultiformat{section}{\S\S#3#1#4--#5#2#6}{ and~#3#1#4--#5#2#6}{, #3#1#4--#5#2#6}{ and~#3#1#4--#5#2#6}
\Crefrangemultiformat{section}{Sections~#3#1#4--#5#2#6}{ and~#3#1#4--#5#2#6}{, #3#1#4--#5#2#6}{ and~#3#1#4--#5#2#6}

% Display format for appendices
\crefformat{appendix}{Appendix~#2#1#3}
\Crefformat{appendix}{Appendix~#2#1#3}
\crefrangeformat{appendix}{Appendices~#3#1#4--#5#2#6}
\Crefrangeformat{appendix}{Appendices~#3#1#4--#5#2#6}
\crefmultiformat{appendix}{Appendices~#2#1#3}{ and~#2#1#3}{, #2#1#3}{ and~#2#1#3}
\Crefmultiformat{appendix}{Appendices~#2#1#3}{ and~#2#1#3}{, #2#1#3}{ and~#2#1#3}
\crefrangemultiformat{appendix}{Appendices~#3#1#4--#5#2#6}{ and~#3#1#4--#5#2#6}{, #3#1#4--#5#2#6}{ and~#3#1#4--#5#2#6}
\Crefrangemultiformat{appendix}{Appendices~#3#1#4--#5#2#6}{ and~#3#1#4--#5#2#6}{, #3#1#4--#5#2#6}{ and~#3#1#4--#5#2#6}

\crefname{part}{Part}{Parts}

% Number subsubsections
\setcounter{secnumdepth}{5}

% Display format for figures
\crefname{figure}{Figure}{Figures}

% Use cleveref instead of hyperref's \autoref
\let\autoref\cref


%%%% EQUATION NUMBERING %%%%

% The following hack uses the single theorem counter to number
% equations as well, so that we don't have both Theorem 1.1 and
% equation (1.1).
\let\c@equation\c@thm
\numberwithin{equation}{section}


%%%% ENUMERATE NUMBERING %%%%

% Number the first level of enumerates as (i), (ii), ...
\renewcommand{\theenumi}{(\roman{enumi})}
\renewcommand{\labelenumi}{\theenumi}


%%%% MARGINS %%%%

% This is a matter of personal preference, but I think the left
% margins on enumerates and itemizes are too wide.
\setitemize[1]{leftmargin=2em}
\setenumerate[1]{leftmargin=*}

% Likewise that they are too spaced out.
\setitemize[1]{itemsep=-0.2em}
\setenumerate[1]{itemsep=-0.2em}

%%% Notes %%%
\def\noteson{%
\gdef\note##1{\mbox{}\marginpar{\color{blue}\textasteriskcentered\ ##1}}}
\gdef\notesoff{\gdef\note##1{\null}}
\noteson

\newcommand{\Coq}{\textsc{Coq}\xspace}
\newcommand{\Agda}{\textsc{Agda}\xspace}
\newcommand{\NuPRL}{\textsc{NuPRL}\xspace}

%%%% CITATIONS %%%%

% \let \cite \citep

%%%% INDEX %%%%

\newcommand{\footstyle}[1]{{\hyperpage{#1}}n} % If you index something that is in a footnote
\newcommand{\defstyle}[1]{\textbf{\hyperpage{#1}}}  % Style for pageref to a definition

\newcommand{\indexdef}[1]{\index{#1|defstyle}}   % Index a definition
\newcommand{\indexfoot}[1]{\index{#1|footstyle}} % Index a term in a footnote
\newcommand{\indexsee}[2]{\index{#1|see{#2}}}    % Index "see also"


%%%% Standard phrasing or spelling of common phrases %%%%

\newcommand{\ZF}{Zermelo--Fraenkel}
\newcommand{\CZF}{Constructive \ZF{} Set Theory}

\newcommand{\LEM}[1]{\ensuremath{\mathsf{LEM}_{#1}}\xspace}
\newcommand{\choice}[1]{\ensuremath{\mathsf{AC}_{#1}}\xspace}

%%%% MISC %%%%

\newcommand{\mentalpause}{\medskip} % Use for "mental" pause, instead of \smallskip or \medskip

%% Use \symlabel instead of \label to mark a pageref that you need in the index of symbols
\newcounter{symindex}
\newcommand{\symlabel}[1]{\refstepcounter{symindex}\label{#1}}

% Local Variables:
% mode: latex
% TeX-master: "hott-online"
% End:


\usepackage{appendix}

%% \usepackage{csquotes}

\usepackage{setspace}

%% broken (doesn't work with tufte-handout):
\usepackage{zed-csp}
%% broken:
%% \usepackage{ltcadiz-fam}

\usepackage{fontspec}
%% \usepackage{xltxtra,xunicode}
\defaultfontfeatures{Scale=MatchLowercase}

%% \defaultfontfeatures{Scale=MatchLowercase}
%% \setmainfont[Mapping=tex-text]{Times New Roman}
%% \setsansfont[Mapping=tex-text]{Arial}
%% \setmonofont{Courier}

\setmainfont[Ligatures=TeX]{TeX Gyre Bonum}
\setromanfont[Ligatures=TeX]{TeX Gyre Bonum}
\setsansfont[Ligatures=TeX]{TeX Gyre Adventor}
\setmonofont[Ligatures=TeX]{TeX Gyre Cursor}


%% \setmainfont[Mapping=tex-text]{Minion Pro}
%% \setromanfont[Mapping=tex-text]{Minion Pro}
%% \setsansfont[Mapping=tex-text]{TeX Gyre Heros}

%% Bugfix: see https://code.google.com/p/tufte-latex/issues/detail?id=64
% Set up the spacing using fontspec features
\renewcommand\allcapsspacing[1]{{\addfontfeature{LetterSpace=15}#1}}
\renewcommand\smallcapsspacing[1]{{\addfontfeature{LetterSpace=0.0}#1}}

\usepackage{epigraph}
\setlength{\epigraphwidth}{.8\textwidth}

%% general symbols - degree, etc.
%% \usepackage{gensymb}

\usepackage [english]{babel}
\usepackage [autostyle, english = american]{csquotes}
%% \usepackage{quoting}

%% nice double-stroke fonts
\usepackage{dsfont}

% Small sections of multiple columns
\usepackage{multicol}

% Provides paragraphs of dummy text
\usepackage{lipsum}

% The units package provides nice, non-stacked fractions and better spacing
% for units.
\usepackage{units}

%\usepackage{geometry}                % See geometry.pdf to learn the layout options. There are lots.
%\geometry{letterpaper}                   % ... or a4paper or a5paper or ...

\usepackage{xfrac}

\usepackage{hyperref}
\hypersetup{
  bookmarks=true,         % show bookmarks bar?
  bookmarksdepth=3,
  unicode=true,          % non-Latin characters in Acrobat’s bookmarks
  pdftoolbar=true,        % show Acrobat’s toolbar?
  pdfmenubar=true,        % show Acrobat’s menu?
  pdffitwindow=false,     % window fit to page when opened
  pdfstartview={FitH},    % fits the width of the page to the window
  pdftitle={Intuition and Exponentiation},    % title
  pdfauthor={G. A. Reynolds},     % author
  pdfsubject={Mathematics},   % subject of the document
  pdfcreator={G. A. Reynolds},   % creator of the document
  pdfproducer={Producer}, % producer of the document
  pdfkeywords={Exponentiation} {Mathematics}
  pdfnewwindow=true,      % links in new window
  colorlinks=true,       % false: boxed links; true: colored links
  linkcolor=blue,          % color of internal links
  citecolor=blue,        % color of links to bibliography
  filecolor=magenta,      % color of file links
  urlcolor=cyan           % color of external links
}

%% \usepackage[
%% bibstyle=numeric,
%% citestyle=authoryear,
%% hyperref,
%% bibencoding=utf8,
%% backref=true,
%% backend=biber]{biblatex}

%% http://tex.stackexchange.com/questions/66778/citation-alias-with-multibib-and-natbib
%% \makeatletter
%% \def\@mb@citenamelist{cite,citep,citet,citealp,citealt,citepalias,citetalias}
%% \makeatother

%% http://stackoverflow.com/questions/2496599/how-do-i-cite-the-title-of-an-article-in-latex
\defcitealias{z-iso-13568}{ISO 13568:2002 Information technology -- Z formal specification notation --
  Syntax, type system and semantics}

\usepackage{tikz}
\usepackage[markings,customcolors]{hf-tikz}
\usetikzlibrary{%
  arrows%
  ,calc%
  ,decorations.text%
  ,decorations.pathreplacing%
  ,fadings%
  ,positioning
  ,shapes.geometric%
}

\usepackage{tikz-3dplot}

\usepackage{pgfplots}
\pgfplotsset{height=7cm,compat=1.9}

\usepackage{tkz-euclide}
\usetkzobj{all}

%% prettier integral syms, but broken on miktex
%% \usepackage{esint}


%% \usepackage{MnSymbol}
%% \usepackage[misc]{ifsym}

%% \usepackage{morefloats}

%%%%%%%%%%%%%%%%%%%%%%%%%%%%%%%%%%%%%%%%%%%%%%%%%%%%%%%%%%%%%%%%
\title{HoTT Types}
%% \\
%% \Large Derived from the HoTT Book}
\author{}
%\date{}                                           % Activate to display a given date or no date

%%%%%%%%%%%%%%%%
%% tufte-latex customizations

\makeatletter
\let\runauthor\@author
\let\runtitle\@title
\makeatother

%% running headers
\newcommand{\changefont}{%
  \fontsize{7}{9.5}\selectfont
}
\fancypagestyle{plain}{
  \fancyhead[LO,LE]{\leftmark }
  \fancyhead[RO,RE]{\rightmark}
  \fancyfoot[CO,CE]{\thepage}
  \fancyfoot[LE]{\textsc{\runtitle}}
  \fancyfoot[RO]{\textsc{\runtitle}}
  \renewcommand{\headrulewidth}{0pt}
  \renewcommand{\footrulewidth}{0pt}
}
\pagestyle{plain}

\def\chpcolor{blue!45}
\def\chpcolortxt{blue!60}
\def\sectionfont{\LARGE}

\setcounter{secnumdepth}{5}
\setcounter{tocdepth}{5}        % sections and subsections for the toc

\makeatletter
%% Section:
\def\@sectionstrut{\vrule\@width\z@\@height12.5\p@}
\def\@makesectionhead#1{%
  {%\par\vspace{20pt}%
    \parindent -10pt\raggedleft\sectionfont
    %% \colorbox{\chpcolor}{%
    %%   \parbox[t]{90pt}{\color{white}\@sectionstrut\@depth4.5\p@\hfill
    %%     \ifnum\c@secnumdepth>\z@\thesection\fi}%
    %% }%
    \vspace{10pt}%
    \begin{minipage}[t]{\textwidth}%{\dimexpr\textwidth-90pt-2\fboxsep\relax}
      \@sectionstrut\hspace{-15pt}\textit{\textbf\Huge #1}
    \end{minipage}\par
    \vspace{5pt}%
  }
}
%% \def\@makesectionhead#1{%
%%   {\par\vspace{20pt}%
%%    \parindent 0pt\raggedleft\sectionfont
%%    \colorbox{\chpcolor}{%
%%      \parbox[t]{90pt}{\color{white}\@sectionstrut\@depth4.5\p@\hfill
%%        \ifnum\c@secnumdepth>\z@\thesection\fi}%
%%    }%
%%    \begin{minipage}[t]{\dimexpr\textwidth-90pt-2\fboxsep\relax}
%%    \color{\chpcolortxt}\@sectionstrut\hspace{5pt}\textbf{#1}
%%    \end{minipage}\par
%%    \vspace{10pt}%
%%   }
%% }
\def\section{\@afterindentfalse\secdef\@section\@ssection}
\def\@section[#1]#2{%
  \ifnum\c@secnumdepth>\m@ne
  \refstepcounter{section}%
  \addcontentsline{toc}{section}{\protect\numberline{\thesection}#1}%
  \else
  \phantomsection
  \addcontentsline{toc}{section}{#1}%
  \fi
  \sectionmark{#1}%
  \if@twocolumn
  \@topnewpage[\@makesectionhead{#2}]%
  \else
  \@makesectionhead{#2}\@afterheading
  \fi
}
\def\@ssection#1{%
  \if@twocolumn
  \@topnewpage[\@makesectionhead{#1}]%
  \else
  \@makesectionhead{#1}\@afterheading
  \fi
}
\makeatother

%%%%%%%%%%%%%%%%
%% macros

\newenvironment{important}[1][]{%
  \begin{mdframed}[%
      backgroundcolor={red!15}, hidealllines=true,
      skipabove=0.7\baselineskip, skipbelow=0.7\baselineskip,
      splitbottomskip=2pt, splittopskip=4pt, #1]%
    \makebox[0pt]{% ignore the withd of !
      \smash{% ignor the height of !
        \fontsize{32pt}{32pt}\selectfont% make the ! bigger
        \hspace*{-19pt}% move ! to the left
        \raisebox{-2pt}{% move ! up a little
          {\color{red!70!black}\sffamily\bfseries !}% type the bold red !
        }%
      }%
    }%
}{\end{mdframed}}

%% reversed integral sign
\makeatletter
\providecommand*{\curv}{%
  \mathrel{%
    \mathpalette\@curv\int
  }%
}
\newcommand*{\@curv}[2]{%
  \reflectbox{$\m@th#1#2$}%
}
\makeatother

%% \def\LaTeX{%
%%   L\kern-.36em
%%   {\setbox0=\hbox{T}%
%%     \vbox to \ht0{\hbox{\the\scriptfont0 A}\vss}}%
%%   \kern-.15em
%%   \TeX
%% }

%%%%%%%%%%%%%%%%

\newcommand\cspace{coordinate space}
\newcommand\Cspace{Coordinate space}
\newcommand\CSpace{Coordinate Space}

\newcommand\dspace{design space}
\newcommand\Dspace{Design space}
\newcommand\DSpace{Design Space}

\newcommand\Omg{\(\Omega\)}
\newcommand\sccs{standard cartesian coordinate space}
\newcommand\origin{\((0,0)\)}
\newcommand\ab{\((a,b)\)}

\newcommand\atypeA{\ensuremath{(a : A)}}

%% \newcommand\N{\(\mathds{N}\)}
%% \newcommand\R{\(\mathds{R}\)}
%% \newcommand\RR{\(\mathds{R}\times\mathds{R}\)}
%% \newcommand\Rtwo{\(\mathds{R}^2\)}
%% \newcommand\Z{\(\mathds{Z}\)}


\def\HoTT{%
  H\kern-.7pt
  {\tiny\raisebox{1pt}{o}}%
  %% {\setbox0=\hbox{T}%
  %%  \vbox to \ht0{\vss\hbox{\the\scriptfont0 o}\vss}}%
  \kern-1.5pt
  TT}

\def\HoTTB{%
  the H\kern-.7pt
  {\tiny\raisebox{1pt}{o}}%
  %% {\setbox0=\hbox{T}%
  %%  \vbox to \ht0{\vss\hbox{\the\scriptfont0 o}\vss}}%
  \kern-1.5pt
  TT Book
}

\newcommand\ML{Martin-L\"{o}f}

\newcommand\ITT{Intuitionisti Type Theory}

\newcommand\TTh{Type Theory}
\newcommand\tth{type theory}

\includeonly{%
%% introduction%
%% ,pragmatism
%% ,proof
%% ,semantics
%% ,math
%% ,foundations
%% ,types
,curry-howard
%% ,assertion
%% ,equality
%% ,brandom
%% ,hotttypes
%% ,misc
%% ,introduction
%% ,preliminaries
,lexicon
,proofassistants
}

%%%%%%%%%%%%%%%%%%%%%%%%%%%%%%%%%%%%%%%%%%%%%%%%%%%%%%%%%%%%%%%%
\begin{document}
%% \ifx\traceon\undefined \tracingall \else \traceon \fi

\maketitle

\begin{ednote}
  Currently this doc contains a (mildly organized) set of notes
  followed by the intro and chapter 1 from the
  \href{http://homotopytypetheory.org/book/}{HoTT Book}.  Eventually
  (maybe) the intro and chapter 1 will contain annotations, comments,
  additional examples, etc., but I have not started that yet, so if
  you are already familiar with the text you need not read them -- I
  haven't (so far as I recall) changed anything.

  The idea is to winnow out some of the strictly mathematical stuff
  leaving the core ``philosophical'' stuff, and annotate the text with
  some comments and quotes from Martin-L\"{o}f, Brandom, etc.  Or
  maybe leave the math stuff in, but annotate it with more detailed
  explanation and examples in programming languages.  In any case the
  purpose is to more fully articulate the link between HoTT's ideas of
  type and judgment (etc.) to the philosophical debates about
  language, assertion, proposition from which they emerged.  Why?
  Because I find those bits of the HoTT a little murky, and
  philosophers like Brandom have a lot to say about the issues.  Also,
  to show more clearly how type theory differs from set theory and
  classic logic.  Another goal is to provide more practical guidance
  to programmers interested in exploring dependent types.
\end{ednote}

\tableofcontents
%% \setcounter{tocdepth}{2}        % chapters, sections, and subsections for the
%%                                 % metadata of the pdf
%% \cleartooddpage[\thispagestyle{empty}]

%% \mainmatter % Turn on roman page numbers and numbered chapters

%%%%%%%%%%%%%%%%%%%%%%%%%%%%%%%%
%%%%%%%%%%%%%%%%%%%%%%%%%%%%%%%%
\chapter{The Pragmatist Enlightenment}
\label{sect:enlightenment}

%%%%%%%%
\section{Liberation}
\label{subs:liberation}



%%%%%%%%
\section{Pluralism}
\label{subs:pluralism}

\begin{ednote}
  Not just propositions-as-types, but types-as-propositions.  Example:
  the type \N can be viewed as a proposition ``there exists a natural
  number''.  This means that there is no authoritative definition of
  what a type is, which means that pluralism is an essential aspect of
  type theory.  Is this a sharp contrast with traditional mathematics?
  For pre-modern mathematics, number was unequivocally quantity or
  magnitude - no pluralism there.  Modern mathematics discarded
  quantitative interpretations of number in favor of structural
  notions.  The issue of pluralism is not so clearly decided there.
  Once you have isomorphisms, you can't really say that one structure
  is \emph{the} structure for a given class.  Groups, for example.  So
  isn't modern math essentially pluralistic?  Well let's look at
  foundations - set theory doesn't seem to be very pluralistic; a set
  is a set is a set, and not something else.  You can come up with
  distinct set \emph{theories}, but they all depend on the primitive
  notion of set, or maybe set membership.  Type theory, by contrast,
  seems to be different.  It doesn't have this kind of unity.  In fact
  there are many distinct type theories, so we should probably always
  use the plural.  The primitive seems to be ``type''; but the concept
  of type is not primitive in all type theories---\HoTT{} being a case
  in point.  ``In fact, no type former is 'primitive' to the game of
  type theory in this sense: you can very well have a type theory with
  no type formers! But it won't be very interesting...'' (M. Shulman,
  \href{https://groups.google.com/d/msg/hott-amateurs/U1X0m4r6G-A/K5eeMSPXE5YJ})
\end{ednote}

``Type theory, formal or informal, is a collection of rules for
manipulating types and their elements.  But when writing mathematics
informally in natural language, we generally use familiar words,
particularly logical connectives such as “and” and “or”, and logical
quantifiers such as “for all” and “there exists”. In contrast to set
theory, type theory offers us more than one way to regard these
English phrases as operations on types. This potential ambiguity needs
to be resolved, by setting out local or global conventions, by
introducing new annotations to informal mathematics, or both.''\HoTTB, p. 101

%%%%%%%%
\section{Normative Pragmatics}
\label{subs:normprag}

  Chapter 1 of \cite{brandom_mie}

%%%%%%%%
\section{Inferential Semantics}
\label{subs:inferentialism}

  Chapter 2 of \cite{brandom_mie}


%%%%%%%%
\section{Expressivism}
\label{subs:expressivism}

See \cite{price_expressivism_2013}


%%%%%%%%%%%%%%%%%%%%%%%%%%%%%%%%
\chapter{Logics}
\label{sect:logics}

\begin{description}
\item [Traditional] terms are primitive; propositions are combinations of terms; judgments apply to proposotions
\item [Modern: classic] LEM, AC, etc.
\item [Modern: intuitionistic]
\item [Expressivism]  Brandom's version: propositions are primitive; relation to inferential semantics; Price's global expressivism
\end{description}

\begin{ednote}
  From schema to type.  E.g. \(A,B\vdash A\land B\) --- traditionally
  viewed as a schema (involving either substitution or denotation), no
  construction involved.  Move from this to viewing it as a rule of
  construction or recipe for making something.
\end{ednote}

\chapter{Proposition}

\begin{ednote}
  BHK interpretation.  How \ML{} got it wrong wrt classic interpretation.
\end{ednote}


%%%%%%%%%%%%%%%%%%%%%%%%%%%%%%%%
\section{Proof}
\label{sect:proof}

Traditional (classic) view: a proof is an epistemic device; it
displays, exhibits, makes \textit{visible} (if only to the mind's eye)
a form of \textit{certain knowledge}.\sidenote{The link between
  knowing and seeing runs very deep in Western culture.  Not
  surprisingly it is closely connected with representationalism and
  cartesianism generally.  It has pretty much dominated Western
  thinking since Descartes, but has come under strong attack from
  Pragmatists.  Dewey called it ``the spectator theory of knowledge.''f
  See \citep{rorty_philosophy_2009} etc.}

Alternatives to the spectator theory: pragmatism, know-how over know-that.

\begin{ednote}
  TODO: summary of concepts of proof.  Emphasize contrast between
  representationalism and inferentialism.  Representationalism is
  atomistic: you could have only one concept.  Inferentialism is
  holistic: you have to start out with at least two concepts, since
  every inference involves a premise and a conclusion.  Inferentialism
  is a natural fit for \HoTT.

  Question: can you have only one type?  In other words, is type
  theory essentially holistic or atomistic?
\end{ednote}


For \HoTT{}, as for most varieties of constructivism, it is better to
abandon traditional notions of proof as something you see in favor of a
more pragmatic notion of proof as something you do.

etc.

Critical point: in \HoTT we have two ``kinds'' of types: propositional
types and non-propositional types.\sidenote{This is not in general
  recognized in \HoTTB, but I think it should be emphasized, if only
  because it reflects intuition.}  If we are to also treat ``proof''
(or witness or whatever) as a fundamental principle of \HoTT, one that
complements the concept of type, then we need to treat both ``type''
and ``proof'' as genuses (genii?) of which propositional and non-propositional
are species.

\begin{ednote}
  General point (to be made elsewhere, maybe in
  \cref{sect:foundations}: the concepts of type and proof go together.
  You cannot have one without the other.  That's very different than
  set theory.  You can have sets and elements without proofs.
\end{ednote}



Long story short: we are in dire need of improved terminology.  My
suggestion is as follows:

\begin{description}
\item [Proof of a proposition] In contrast to the classic spectator
  view, we treat proof not as the exhibition (or: making available for
  inspection) of the form of a bit of certain knowledge, but as the
  \textit{demonstrative expression} of the proposition.
  Alternatively, the expressive demonstration of the proposition.  So
  whereas a classic proof is something that must be ``seen'' in order
  to be grasped, a type-theoretic proof is something that must be
  actively \textit{done}, not merely passively observed.  One must be
  able to follow the construction of the proof.

\item [Proof of a non-propositional type] Classically, one only proves
  propositions, not terms.  So the idea of e.g. ``proving'' the
  natural numbers doesn't even make sense; it reflects a category
  mistake.  But in \HoTT, the concept of ``proving'' a type is
  primitive; the problem is that ``proving'' is the wrong word.
\end{description}

So here's one way to look at it: we construct (make) proofs; but the
proofs we construct are expressions of the type (the thing we prove).

%%%%%%%%
\subsection{Of the Ambiguity of Of}
\label{subs:ofofof}

``Of'' supports two distinct readings.  Consider ``the conviction of
the defendant''.  If the court did the convicting, then ``of'' acts as
a kind of intermediary between a verbal noun (``conviction'' as act or
action of convicting) and its direct object (e.g. ``The court
convicted the defendant'').  The conviction affects the defendant from
the outside; it does not ``belong'' to the defendant but to the court.
On the other hand, if we take ``the conviction of the defendant'' to
refer to a belief to which the defendant is firmly committed, then the
conviction is ``internal''; it belongs to and comes from the
defendant.

This ambiguity of ``of'' afflicts phrases like ``proof of a
proposition'' as well.  If we can disambiguate it some of the mystery
of the relation between types and proofs will vanish.

%%%%%%%%
\subsection{Demonstrations and Demonstratives}
\label{subs:}

When we \textit{exhibit} a classic proof of a proposition, the proof
comes out as external to the proposition proved, just as a court's
conviction of a defendant is external to the defendant.  Such a proof
is something added or attached to the proposition.

But when we \textit{demonstrate} a proposition,\sidenote{Note: we
  demonstrate propositions, not proofs; a demonstration of a
  proposition \textit{is} a proof.} the demonstration (that is, proof)
is to be deemed an expression of the proposition in the internal
sense: an expression whose source, so to speak, is the proposition
itself, rather than the writer of the proof.  This may sound odd or
even ridiculously anthropomorphic, but if you think about it a bit it
makes perfect sense.  The mathematical proofs we write down are not
really expressions our our thought, but of mathematical structures,
entities, relations etc.  So they express
mathematics.\sidenote{Actually we should probably think of them as
  having a dual expressivism.  On the one hand they clearly express
  mathematics; but on the other hand, the particular form a proof
  takes is an expression of the writer's style or way of thinking.}

We can think of a demonstration in this sense as expressing a type's
structure, construed as the inferential articulation of the concept of
the type.\sidenote{See \cref{sect:brandom} for more on the inferential
articulation of conceptual content.}

The nice thing about this way of thinking is that it resolves the
tension between propositional and non-propositional types with respect
to proof.  In both cases, what \HoTT{} calls proof or witness is to be
taken as a demonstrative expression, or expressive demonstration, of
the type itself.  In the case of propositional types, favor the term
``demonstration'', with its connotations of progressive unfolding of a
logical structure, or better, rational argument.  In the case of
non-propositional types like \N, favor the term ``demonstrative'',
with its adjectival sense of ``something having a demonstrative
function'', rather than a nominal sense of ``act or action of
demonstrating''.  So an element\sidenote{We really must get rid of
  ``element''; it's too suggestive of set theory.  Maybe
  ``demonstrative'' fits the bill; instead of ``element of a type'' we
  would say ``demonstrative of a type''.  Or maybe ``demonstrator''.}
of a propositional type we would call a demonstration of the type, and
an element of a non-propositional type we would call a demonstrative
of the type.\marginnote{So $2$ is a demonstrative of the natural
  numbers; a proof that ``$2$ is even'' is a demonstration that
  expresses just that ``$2$ is even''.}

\begin{ednote}
  Demonstration qua demonstration of know-how?  Expression as
  expression of a type's structure - that is, its inferential
  articulation?
\end{ednote}

In both cases we have demonstration rather than proof of the type.

\begin{ednote}
  ``Demonstrator'' as the genus of ``demonstration'' and
  ``demonstrative''.  It has the virtue of paralleling
  ``constructor''.
\end{ednote}

%%%%%%%%%%%%%%%%%%%%%%%%%%%%%%%%
\section{Semantics}
\label{sect:semantics}

%%%%%%%%
\subsection{Meaning}
\label{subs:meaning}

%%%%%%%%
\subsection{Model-theoretic Semantics}
\label{subs:modeltheorysem}

%%%%%%%%
\subsection{Proof-theoretic Semantics}
\label{subs:proofsem}

``Proof-theoretic semantics is an alternative to truth-condition semantics. It is based on the fundamental assumption that the central notion in terms of which meanings are assigned to certain expressions of our language, in particular to logical constants, is that of proof rather than truth. In this sense proof-theoretic semantics is semantics in terms of proof . Proof-theoretic semantics also means the semantics of proofs, i.e., the semantics of entities which describe how we arrive at certain assertions given certain assumptions. Both aspects of proof-theoretic semantics can be intertwined, i.e. the semantics of proofs is itself often given in terms of proofs.''\cite{schroeder-heister_proof-theoretic_sep}


%%%%%%%%
\subsection{Inferential Semantics}
\label{subs:inferensem}


%%%%%%%%%%%%%%%%%%%%%%%%%%%%%%%%
\section{Mathematics}
\label{sect:math}

%%%%%%%%
\subsection{Traditional}
\label{subs:mathtrad}

%%%%%%%%
\subsection{Modern: classic}
\label{subs:mathmodclassic}

%%%%%%%%%%%%%%%%%%%%%%%%%%%%%%%%
\subsection{Modern: Intuitionism}
\label{sect:mathmodintuit}

\begin{ednote}
  Why Brouwer should be deemed a pragmatist.
\end{ednote}


%%%%%%%%%%%%%%%%%%%%%%%%%%%%%%%%
\subsection{Mathematical Pragmatism}
\label{sect:mathprag}

\begin{ednote}
  \HoTT is (largely) founded on \ML{}'s account of ``judgment''
  (assertion).  I don't know if that's entirely accurate, but it's my
  story and I'm sticking with it for now.  (\ML{} was quite specific
  that his project was motivated by ``purely philosophical''
  considerations.  See his 1972 paper.)  Brandom's account of
  assertion is part of a larger, very ambitious project that aims to
  explain the structure of rationality.  It's a thoroughly pragmatic
  account; everything comes down in the end to ``proprieties of
  practice'': conceptual activity (thinking and talking) is explained
  in terms not of what we know but of what we do (or what we know
  \textit{how} to do).

  Brandom's account of assertion is much more refined and
  sophisticated than \ML{}'s.  If we replace \ML{}'s account with
  Brandom's, then \HoTT comes out as a piece of ``mathematical
  pragmatism'' (or pragmatic mathematics): an account mathematics
  grounded in practice.
\end{ednote}

\begin{ednote}
  TODO: Brandom's philosophy, like most of contemporary pragmatism,
  subverts the dominant representationalist mode of thinking.  It
  turns things upside-down, or inside-out.  So it is with type theory.
  (In one of his papers \ML{} suggests something similar, pointing out
  that his take on judgment etc. reverts (in some sense) back to
  practices that preceded the ways of thinking that have dominated
  modern ``classic'' mathematics and logic.)  The to-do item here is
  to show how the relation of type theoretic to classic thinking in
  mathematics and logic parallels the relation between pragmatist
  (anti-representational, expressivist) thinking and representational
  (cartesian, platonistic) thinking in philosophy, about rationalism,
  conceptual content, etc.  Show how type-theoretic thinking turns
  traditional classic thinking inside-out.
\end{ednote}



%%%%%%%%%%%%%%%%%%%%%%%%%%%%%%%%
%%%%%%%%%%%%%%%%%%%%%%%%%%%%%%%%
\section{Foundations}
\label{sect:foundations}

\HoTT purports to offer a new foundational concept for mathematics.  If
we take assertion to be the foundational concept of type theory (I'm
not sure this works, but it seems plausible), then Brandom's account
of assertion can link type theory to a foundational account of
discursive practice (rationality).

Today set theory is the reigning foundational theory of mathematics.
It's fairly easy to present it as such: first you list the axioms,
then you show how to ``construct'' the natural numbers from sets,
using a successor function.  Or you might follow the lead of the Z
specification notation\sidenote{\cite{zed_spec}}, and proceed from
sets to relations and then to functions.  However you do it, it's all
pretty intuitive and relatively easy to explain, even to mathophobes.

What would such a foundational presentation look like for \HoTT?  If
\HoTT turns out to be a genuinely foundational theory, then it must be
grounded in intuition; specifically, we should expect that its basic
notions correspond in some way to some collection of pre-theoretic
mathematical intuitions, just as the axioms of set theory do, or as
the axioms of geometry match our ordinary intuitions about the
organization of space as we experience it pre-theoretically.

Presentations of set theory usually begin by discussing the axioms;
but even though axioms serve as ``unexplained explainers'', such a
presentation inevitably depends on a yet more primitive layer of
concepts.  Specifically, not only the (pre-theoretical) concepts of
set, subset, and membership, but also axiom and perhaps proof.  All of
these ``preliminary'' concepts---let's call them
``principles''---correspond more or less directly to intuitions
available to any concept-user.

In general, an explicit account of the fundamental \textit{principles}
of set theory is either omitted or informally glossed, before the
presentation moves on to the axioms.  But type theory, in the end, is
radically different from set theory at a very fundamental level, as
far as I can see.  ``Set'' and ``type'' are so easily grasped that it
is easy think of them as more-or-less the same sort of thing; but if
you look hard at them, they are very different, even fundamentally
different.  So I think a presentation of \HoTT would be well served by
beginning with an explicit account of principles, even before moving
on to consider primitives of the theory.

What are the pre-theoretical principles and primitives of \HoTT?  The
obvious place to start is ``type''.  The concept of ``type'' obviously
emerges from ordinary experience; indeed, it is arguably more
primitive than the concept of ``set''.  Just look at the vast
literature on the emergence of categorization in developmental and
cognitive psychology; the ability to categorize is undoubtedly one of
the most primitive human intellectual skills, if not the most
primitive.  It may even be a primitive animal capability--bees
categorize flowers, and every member of sexually reproducing species
categorizes possible mates.

What about ``axiom''?  At first glance it would seem that any
foundational account of mathematics (or anything else for that matter)
must rest on an axiomatic foundation.  Which is just another way of
saying that any explanation of anything must eventually bottom out on
a bedrock of unexplained explainers.  You can't explain everything
without entering an infinite regression.

On the other hand, we can view axiomatic explanation as just one
``style'' of explanation, one of many.  When you begin with axioms,
you present them as unequivocally (and unquestionnably) true.  But
this is really a bit of salesmanship; sometimes axioms turn out not to
be quite as axiomatic as they seem.  Reconceptualizations happen,
which may lead us to view axioms in a new light in which they do not
look quite as certain.  Then axiomatic explanations are still
intelligible, but are no longer unquestionnable.  The classic example
of this sort of evolution is to be found in the history of geometry.
Before the development of non-Euclidean geometries in the 19th
century, the axioms of Euclidean geometry were not only unquestioned
but unquestionable:\sidenote{I suspect I'm overstating the case here.
  Mathematicians: is this true?} the idea that parallel lines could
meet was not just wrong, but crazy.  Today, using axioms to define a
geometry is just a way of making clear the assumptions necessary to
make the theory work.  They no longer represent essential
connections to externally available bits of certain knowledge.

In other words, axioms are not a necessary condition of adequate
explanation.  So the question is whether or not the axiomatic style is
most appropriate for a presentation of \HoTT?  On the one hand, it
seems to me that it is not necessary; an adequate explanation of \HoTT
without axioms should be possible.  For example, we can treat the
concept of ``type'' as primitive, even if we cannot find a good way to
express it axiomatically.\sidenote{This is a little fuzzy; maybe it
  doesn't even make sense.  But as long as we're rethinking the
  foundations of mathematics, we might as well rethink everything.}

\begin{remark}
  Leaving full presentation of principles for later.  I think it
  includes at least proposition and judgment, maybe inference and
  proof.
\end{remark}

In any case, we'll have to begin somewhere, by stating some
fundamental principles; then we'll need an account of the primitives,
whether they take the form of axioms or not.  What are the principles
upon which \HoTT depends?  And once we have some principles (which are
external to the theory proper), what are the primitives (which are
``inside'' the theory)?\sidenote{Ok, ``primitive'' sounds a lot like
  ``axiom''.  But I think there's a difference, even if I can't quite
  articulate it.  Let's provisionally say that a primitive is an axiom
  without the concommitant commitment to unquestionned certainty.}

Here are some possibilities, based on my understanding of the material
in \HoTTB.  Please keep in mind this is coming from somebody who
thinks he has a fairly good grasp of what type theory is all about,
but is still grappling with \HoTT.

%%%%%%%%
\subsection{\HoTT Principles}
\label{subs:hottprinciples}

\begin{description}
\item [Type] Obviously a fundamental concept.  What to say about it, though, is
  not so obvious.
\item [Proposition]
\item [Judgment]
\item [Proof]\sidenote{from Latin \textit{probare} "to make good;
  esteem, represent as good; make credible, show, demonstrate; test,
  inspect; judge by trial" (source also of Spanish \textit{probar},
  Italian \textit{probare}), from \textit{probus} "worthy, good,
  upright, virtuous,"} Two kinds, corresponding to the two kinds of
  provables:\sidenote{Remember, we're talking about pre-theoretical
    principles (concepts) here, not about \HoTT per se.}
\begin{description}
\item [Demonstration] - \textit{rational argument} that compels assent
  to a proposition\sidenote{``Demonstration'' is intuitively
    satisfying, but conceptually misleading, insofar as it suggests a
    visual metaphor.  That would be classical; but for type theory we
    want metaphors of construction, not inspection.}
\item [Witness] - evidence that bears witness to the existence of a kind or category
\end{description}
\item [Inference]
\item [Induction] Seems pretty basic to me, and as far as I know,
  nobody has ever been able to explain it in terms of more primitive
  notions.  Without a Principle of Induction (of some kind), we would
  not be able to, for example, form the Natural Numbers in type theory.
\item etc.
\end{description}

\newthought{The concept of proof in type theory} deserves special
attention.  \Cref{sect:proof} examines it in detail; here, suffice it
to say that it extends beyond the traditional and intuitive notion of
proof as something one does to or with propositions.  In type theory,
propositional types represent propositions, so a type-theoretic proof
of a propositional type---call it a ``tt-proof''--- corresponds to an
ordinary proof of a proposition; it essentially involves inference,
for example.  But type theory also has lots of non-propositional
types, like \N.  These do \textit{not} represent propositions:
propositions have truth-values, natural numbers do not.  In set
theory, there is no connection between sets, elements, and proofs.  An
element either is, or is not, a member of a given set.  Period, full
stop.  The notion of proof never enters the set-membership
picture.\sidenote{That need not mean that proving membership is never
  an issue.  But you don't prove membership; rather, you prove that
  the element satisfies some predicate, which is a different concept.}
In particular, the existence of a set is not dependent on particular
members, and the fact that some element is a member of some set has no
significance with respect to the existence of the set.  By contrast,
in type theory construction of an element of a type counts as proof of
the type.  Etc.\sidenote{FIXME: fix this language.}  But this kind of
``proof'' is not like proof of a proposition; it does not involve a
proposition that may be true or false, and it does not involve
inference.  Instead it serves as a kind of evidence that shows the
type.

\begin{remark}
  Is there a significant distinction to be made between proof and
  witness?  I suspect there is, based on the difference between
  propositions and names.  Both count as evidence, but there is a
  difference between an inferential proof of a proposition and a
  ``testimonial'' witness to a kind.  Propositions-as-types unifies
  the two ideas, but does not erase the distinction.
\end{remark}

%%%%%%%%
\subsection{\HoTT Primitives}
\label{subs:hottprimitives}

\HoTT primitives are ....\sidenote[][-48pt]{A proper exposition would list 1) the name of the
  primitive, e.g. ``\(\Pi\)-type''; 2) the ``constructor'' symbol,
  e.g. \(\cross\) for product types; 3) the analogous concept from set
  theory, and then the ``rules'' for defining a type (formation,
  construction, elimination, computation, uniqueness).}

\begin{description}
\item [Function] ``Unlike in set theory, functions are not defined as
  functional relations; rather they are a primitive concept in type
  theory.''\sidenote{Or: set theory
    \textit{defines} a function as a set of ordered pairs whose domain
    has no duplicates; in other words, it treats a function and its
    ``graph'' as the same thing.  Question: what happens to the graph
    of a function in type theory?} \citep[p. 21]{hottbook}

\item [Product] Product types correspond to cartesian products in set
  theory.  The constructor symbol is the same as in set theory:
  \(\cross\).\sidenote{Why isn't this called the ``\(\Huge\cross\)-type''?}
  ``[U]nlike in set theory, where we define ordered pairs to be
  particular sets and then collect them all together into the
  cartesian product, in type theory, ordered pairs are a primitive
  concept, as are functions.''\citep[p. 26]{hottbook}

\item [Coproduct type] Coproduct types correspond to disjoint unions
  in set theory.  ``In type theory, as was the case with functions and
  products, the coproduct must be a fundamental construction, since
  there is no previously defined notion of ``union of
  types''.\citep[p. 33]{hottbook}

\item [Proposition type] Conceptually, at least, this seems primitive.
  Especially if the concept of ``proposition'' counts as a
  pre-theoretic principle.  Which implies that proof is also a
  pre-theoretic principle.  Propositions are fundamentally different
  than the other kinds of type, since they have truth-conditions. Etc.
  It follows that proofs are fundamentally different from other kinds
  of witness.

\end{description}

\begin{remark}
  For consistency, we might want to use symbols to designate all of
  the primitives, just as we do for \(\Pi\) and \(\Sigma\).  This
  would give us: \(\Huge\fun\)-types, \(\Huge\cross\)-types, and
  \(\Huge +\)-types.
\end{remark}


%%%%%%%%
\subsection{\HoTT Quasi-primitives}
\label{subs:quasiprim}

\noindent ``Fundamental''\sidenote{Obviously we need a better bit of
  terminology.  ``Quasi-primitives''?  ``Neo-primitives''?  These
  types are not primitive, strictly speaking, but on the other hand
  they are basic.  I think there is another fundamental principle at
  work here.  In set theory, for example, the concept of function is
  not only not primitive, it isn't necessary.  You could discard it
  and still have set theory.  But my intuition tells me that e.g. the
  concept \(\Pi\)-type is in a sense necessary or essential in type
  theory, even if it is not primitive.  Once you have the primitives,
  you necessarily have these non-primitive basic types.  Dunno if
  that's correct, but it would sure be nice if it were.} (but
non-primitive) types.  These types seem to be on a par with the
primitive types as far as importance goes, but they presuppose the
primitives, so cannot themselves be considered primitive.

\begin{description}

\item [Universe]  Is this a primitive?  Probably not, since it builds on the type concept.

\item [\(\Pi\)-type] Informally, ``dependent function''
  types.\sidenote{As a practical matter, I think it would be useful to
    have an informal term for these types that falls between
    ``dependent function type'' and \(\Pi\)-type.  Something like
    ``p-function type''.  \(\Pi\)-type is admirably concise, but I
    think it should mention ``function'', since it names a kind of
    function.}  The concept of \(\Pi\)-type is a generalization of the
  concept of function type, so it isn't primitive.

\item [\(\Sigma\)-type] Informally, ``dependent pair''
  type.\sidenote{Shouldn't this be called ``dependent
    \textit{product}'' type?  The type is product, not pair; pairs are
    ``elements'' of the type.  Informally, maybe ``sig-prod type?}
  The concept of \(\Sigma\)-types is a generalization of the concept
  of product type.

\end{description}

%%%%%%%%
\subsection{\HoTT Standard Type Library}
\label{subs:hottstdlib}

\begin{remark}
  By analogy to the usual ``standard library'' of programming
  languages.  The idea is to list commonly used types that are neither
  primitive nor quasi-primitive; ``application'' types, in a sense.
\end{remark}


\begin{description}
\item [Boolean] \citep[p. 34]{hottbook}
\item [$\nat$] \citep[p. 36]{hottbook} But there's a problem here;
  actually several.  First of all, the section on the naturals in
  chapter one does not actually show how to construct them; its really
  a section about induction, not the natural numbers.  Second, what it
  does say about the naturals is that they start from zero.  That
  obviously won't do; zero is not a natural number, and there is no
  intuitive notion of zero as a number.  You can't even think of it as
  a number until you've severed the link between the concept of number
  and the concepts of quantity and/or magnitude.  So the natural
  numbers really must start with 1, not 0.
\item [Propsition]
\item [Identity] 
\end{description}



%%%%%%%%%%%%%%%%%%%%%%%%%%%%%%%%
\chapter{Types}
\label{sect:type}

\HoTTB page 27 describes a ``general pattern for introduction of a new
kind of type''.  Martin-L\"{o}f does this too, somewhere.  In \HoTTB,
the list is

\begin{description}
\item [Formation Rules]
\item [Introduction Rules]  or constructors
\item [Elimination Rules] or eliminators
\item [Computation Rules]  ``which express how an eliminator acts on a constructor''
\item [Uniqueness Principle] which ``expresses uniqueness of maps into
  or out of that type.  Optional.
\end{description}


The question is where to place this stuff in the description of \HoTT.
Are these things primitives?  Do they form essential aspects of a
type?  Or in other words, can we have (think of) types without these rules?

\HoTTB introduces them almost as an afterthought, as a Remark in the
third major construction defined in Chapter 1.  But I suspect this is
a mistake or oversight; it looks to me like these rules are indeed
fundamental, essential to the concept of type.  In that case, they
should be presented along with the introduction of the type concept,
rather than in the middle of a description of a particular type.

%%%%%%%%%%%%%%%%%%%%%%%%%%%%%%%%
\section{Terms}
\label{sect:terms}

\begin{ednote}
  ``Terms'' is Awodey's terminology.  More common terminology include:
  witness; inhabitant.  Also proof.
\end{ednote}

``Under the Curry-Howard cor- respondence, one identifies types
with propositions, and terms with proofs...''\cite{awodey_tth}

%%%%%%%%
\section{Witness}
\label{subs:witness}

\begin{ednote}
  In what sense is a proof a witness to a type, or an ``inhabitant''
  of a type?  Intuitively this language does not work very well; we
  don't intuitively think of a proposition as a type ``inhabited'' by
  proofs.  The notion of proof as ``witness'' to a type is a
  substantive epistemological notion; it not only says that the proof
  is related to the type, but also it says something about the nature
  of that relationship.

  The trick is to see it from the perspective of the machine.  A
  proposition like \(1+1=2\) is just a form to the machine.  We can
  see that it is true just by looking, due to some mysterious
  epistemic capability.  But machines do not have epistemic abilities;
  a form is a form is a form to a machine.  Hammer, nail.  So in order
  for the machine to treat \(1+1=2\) as a \emph{true} proposition, we
  have to give it something more: a proof.  But ``proof'', again, is
  an substantive epistemic notion; the machine analog must be purely
  formal.  From the machine perspective, a proof is just another form,
  or rather, collection of forms (including inference rules as complex
  forms): to give the machine a proof of P we must provide it with a
  form or forms that ``lead to'' (produce, result in) P.  To prove a
  proposition to a machine, we give it forms and reduction rules such
  that the formal use of those forms and rules results in the form of
  the proposition to be proved.  (FIXME: a more accurately way of
  putting this would involve reduction of formulae to normal form,
  confluence, etc.)

  So we can think of a proof as a kind of device---just another
  machine (or machine description), but one whose sole output is the
  proposition to be proved.  Since for any given proposition there may
  be many ways of building such a proving device, we can treat these
  devices as forming a kind of equivalence class, which we can
  identify by taking (the form of) the proposition as a symbol
  referring to the class.  Now the connection to types and witness
  becomes clear: the equivalence class of such proving devices forms a
  type, the type of the devices (proofs), and each device (proof)
  ``inhabits'' (or as we would prefer, expresses) the type.
\end{ednote}


%%%%%%%%%%%%%%%%%%%%%%%%%%%%%%%%
\section{Curry-Howard}
\label{sect:curry-howard}

\begin{ednote}
  Usually presented as ``propositions-as-types'', but this suggests an
  asymmetrical relationship; in fact the principle is that
  propositions \emph{are} types, and vice-versa.  This is a major move
  in type theory, introduced by \ML(?) based on work by Curry and
  Howard.  TODO: what exactly are the implications of this principle?
\end{ednote}



%%%%%%%%%%%%%%%%%%%%%%%%%%%%%%%%
\section{Assertion and Judgment}
\label{sect:assertionjudgment}

%%%%%%%%
\subsection{notes}
\label{subs:notes}

This section needs some serious revision.  Here's the straight dope,
in a nutshell.  In his paper ``Truth of a proposition, evidence of a
judgement, validity of a proof''\citep{martin-lof_truth_1987}, which
is specifically about the philosophical basis of \ITT{}, \ML{}
attempts to explain the concepts proposition, truth, evidence,
proof, and validity.  The first part of the paper, which gives some
historical and conceptual background, is just right for the most
part.  He points out, for example, that for the intuitionist proof
comes before truth.  But he makes a major blunder when he claims
that the classic truth-conditional account of the logical
connectives, an account that is based on truth table semantics, and
the BHK accounts, which treat a proposition as an expectation or
task etc., are just different ways of saying the same thing.
``Façons de parler'', as the saying goes.  But I think this is flat
out wrong.  Classic and intutionistic logic may use the same
formulas, but they could not be more different conceptually.
Classic truth-conditional logic presupposes something like what
Wittgenstein called (in his Tractatus days, at least) a picture
theory of meaning. (I may not be getting the exact wording right
here, but the idea should be clear enough.)  Proof in that kind of
logic has nothing to do with construction; it's all about
correspondence, a representational relation between language and
objective reality.  Obviously there's much more to be said about
this, but suffice it to say that \ML's claim that classic logic and
intuitionistic BHK logic are in the same line of business strikes me
as not only wrong but a little bit shocking.  So wrong that I have
to wonder why he made that sort of claim.  Maybe he was unfamiliar
with the pragmatist literature.  And we'll just proceed on the
assumption that I am not wrong, if you don't mind.  I'll provide a
more detailed justification of this claim later.

Another thing that looks wrong to me is his account of the BHK
interpretation of propositions; but in this case, he has an excuse:
R. Brandom's more refined account of assertion and proposition was
not yet available.  Brandom's account makes the problems with \ML's
account quite clear.  The latter follows BHK in treating a
proposition as a problem to be solved, a task to be accomplished, or
an expectation of a proof, etc.  The problem is that propositions
are clearly exactly \emph{not} these things.  Obviously we may
\emph{treat} a proposition as e.g. a task to be accomplished; but
that does not determine what a proposition \emph{is}.  Or to put it
differently \ML{} seems to ignore the significance of \emph{force},
which is distinct from conceptual content.

For now I don't have time to explicate the point in detail, so
here's the short version: Brandom divides assertion into commitment
and entitlement.  And what makes the proposition primitive is that
it is the minimal unit of \emph{responsibiity} - Brandom traces this
notion to Kant.  To assert a proposition is to undertake a
commitment to it, and also to license others to challenge ones
entitlement to that commitment.  Thus it inescapably involves a kind
of responsibiity: the responsibility to justify (``vindicate'', as
Brandom says) one's commitment.  One way to do this is to
demonstrate the entitlement by giving \emph{reasons} for it.

If you're familiar with the \ML{} paper mentioned above, the
connection should be fairly obvious.  Commitment and entitlement are
deontic attitudes, which institute deontic statuses (e.g. being
correct or incorrect).  They are emphatically \emph{not} properties
of propositions.  So it is just a mistake to think that propositions
are or express tasks, problems, or expectations.  On the other hand,
\emph{assertion} of a proposition does give rise to a responsibility
to vindicate commitment.  Talk of ``expectation of a proof'' is
entirely intelligible as a way of saying that assertion licenses
others to challenge one's commitment---to expect that one can or
will prove it.  Talk of task or problem is really a way of talking
of the justification or vindication that one is responsible
for.\sidenote{And note that this is not a mere matter of voluntary
  acceptance of responsibility; it arises because assertion licenses
  others to \emph{hold} one respondible, treat one \emph{as}
  responsible, and therefore sanction speakers who fail to vindicate
  their commitments.}

So in the end, \ML{} is speaking more or less the right vocabulary,
but his explanation is off, and his characterization of propositions
as involving something in addition to propositional content
(e.g. expectation, task, etc.) is not defensible, at least from a
Brandomian perspective.  \HoTT{}, unfortunately, duplicates his
error in its account of judgment.

The remedy is close at hand, though.  All we need do is recognize that
what \HoTTB calls ``judgments'', like ``a : A'' and ``a := b'', are
really \emph{stipulations}, and what it calls propositions,
\emph{assertions}.  A stipulation, unlike an assertion, does not
require justification.  A stipulator does not license listeners to
demand reasons for the stipulation.  Of course they can make such
demands, but almost by definition we can stipulate \emph{ad libitum}.
Listeners who don't like our stipulations can go elsewhere; but no
\emph{rational} challenge can be mounted against a stipulation.  The
reason it makes no sense to ask for a proof of ``a : A'' is not
because it is a ``judgment'' but because it is a stipulation and
therefore needs no justification.  By contrast, ``propositional
equality'' is just a fancy (and rather unfortunate) term for
``asserted equality''.

Another way to look at it: \HoTT, following \ML{}, attributes
special properties to propositions.  But that's hard to defend,
philosophically, and doesn't amount to genuine explanation; a better
way is to explain assertion in terms of deontic attitudes and
responsibilities.

%%%%%%%%
\subsection{Judgment}
\label{subs:judgment}


The account of judgment offered in the HoTT Book doesn't really work.
Ditto for Martin-L\"{o}f's account.  For example, it makes sense to
say ``P is a proposition'', but it doesn't make sense to say ``P is a
judgment''.  That's because judgment is a act, something one does.

On the other hand, ``judgment'', like ``proposition'', can be treated
as a verbal noun or as a ``plain'' noun.  Saying ``P is a
proposition'' is usually taken to mean that P refers to what has been
proposed.  There is no obvious reason not to treat ``P is a judgment''
in a similar manner: P refers to what has been judged.

However, there is a difference.  Judging a proposition (what was
proposed) amounts to \textit{evaluating} what was proposed, as good or
bad, true or false, or whatever.  By contrast, proposing a proposition
amounts to merely exhibiting it for consideration.  This arguably
involves an implicit evaluation - to propose a proposition is to
implicitly claim that it is good, or true, etc.  But proposing does
not involve offering an evaluation that is distinct from what is
proposed, whereas judgment does.  The two are distinct kinds of speech
act, and refering to the content of a speech act is not the same as
referring to the speech act itself.

Furthermore, it is not correct to treat the nominal sense of
``judgment'' as being the content, what has been judged.  The nominal
sense of ``judgment'' refers to the act of judgment itself, and not
the proposition judged.

Actually, by the same reasoning it is not correct to say that the
nominal sense of ``proposition'' is what-is-proposed; rather, it is
the act proposing, nominalized.  This makes perfect sense when you
consider that ``proposing'' can also be nominalized; ``the proposing''
is another way of saying ``the proposition''.

The same goes for all -tion words: suggestion, opposition, etc.  In
each case, the word can refer to the doing, or to what is done, and
what is done is always the act of doing itself -- not the subject or
object of the doing.

This suggests we should make a distinction between, for example, the
content of a proposition and ``proposition''.  But this term seems to
be a special case; it has the usual plain noun sense of
what-was-proposed, the usual verbal sense of ``proposing'', but also
the nominalized verbal sense of ``act of proposing''.

(But then the same considerations apply to ``judgment''.  The
difference must go back to semantics.)

\begin{remark}
  The Arabic grammatical tradition captures this distinction
  beautifully, mainly because the structure of the language makes it
  simple to do so.
\end{remark}

Or put it this way: when we judge a proposition like ``2+2=4'' to be
true, the what-was-judged is not ``2+2=4'' but the truth of ``2+2=4''.

\begin{remark}
  But how is this different from ordinary predication, like ``The
  triangle is red'' as a proposition?  Should we say that what is
  proposed is not that the triangle is red, but the redness of the
  triangle?  No, since we're treating it as a propostion, and the
  whole thing is proposed (exhibited).  If we judge it to be true,
  then again the judgment 
\end{remark}

So saying ``P is a judgment'' is incoherent if P is taken to refer to
nothing more than what is proposed.  If P refers to a claim of the
form ``X is true'' (or good, etc.), then ``P is a judgment'' seems to
make more sense; but it doesn't, really.  P still refers to an
unasserted content; to make sense, we would have to say something like
``P is a judgment when asserted''.  More explicitly, ``'X is true' is
a judgment'' (or better, ``'X is true' expresses a judgment'') only
\textit{exhibits} ``X is true'', which is a proposition, not a
judgment.  As a proposition it expresses a judgment; but when embedded
(equivalently, quoted) it does not express anything.

\begin{remark}
  Compare: ``Snow is white'' iff snow is white.  The quoted bit is a
  name of the sentence; it counts as a \textit{mention} of the
  sentence, which has no force.  The unquoted version of same is the
  sentence itself; it counts as a \textit{use} of the sentence, which
  has assertional force.  Obviously, the occurances of ``P'' in ``P is
  a proposition'' and ``P is a judgment are names of a proposition and
  thus mentions.  So they have no force.
\end{remark}

The key point is Frege's point: the content of a proposition is
distinct from the force of the utterance.  That means that P in ``P is
a proposition'' is unasserted, just as it is when embedded, as in ``If
P then Q''.  The truth of ``P is a proposition'' is independent of the
truth of P.

So even if we take the act of declaring ``P'' to be an act of
judgment, it does not follow that a reference to P is a reference to
the act of judging that P.  Hence there is no way to make ``P is a
judgment'' work.  If we take P to refer to what was judged, that again
is a proposition (or propositional content), so ``P is a judgment'' is
incoherent.

\begin{remark}
  We can assert that P, and we can assert P.  We can judge that P, but
  we cannot judge P.  I don't think this is a mere grammatical
  distintion; I think it reflects a genuine semantic difference.
\end{remark}



%%%%%%%%%%%%%%%%%%%%%%%%%%%%%%%%
\section{What's the Big Deal about Equality?}
\label{sect:equality}

\begin{ednote}
  Equality is arguably the most important concept of \HoTT{}, as far
  as I can tell, because of the ``Univalence Axiom''.
\end{ednote}

``In the intensional version of the theory, with which we are
concerned here, one thus has two different notions of equality:
propositional equality is the notion represented by the identity
types, in that two terms are propositionally equal just if their
identity type IdA(a,b) is inhabited by a term. By contrast,
definitional equality is a primitive relation on terms and is not
represented by a type; it behaves much like equality between terms in
the simply-typed lambda-calculus, or any conventional equational
theory.

If the terms a and b are definitially equal, then (since they can be
freely substituted for each other) they are also propositionally
equal; but the converse is generally not true in the intensional
version of the theory''\cite{awodey_tth}

``The constructive character, computational tractability, and proof-
theoretic clarity of the type theory are owed in part to this rather
subtle treatment of equality between terms, which itself is
expressible within the theory using the identity types IdA(a, b).''\cite{awodey_tth}

%%%%%%%%
\subsection{Substitution}
\label{subs:substitution}

As the quote from Awodey above suggests, the concept of
substitutability plays a basic role.

\begin{ednote}
  Compare substitution in lambda calculus, and in Brandom's model.
  Maybe something about combinatory logic and the elimination of
  variables?
\end{ednote}



%%%%%%%%%%%%%%%%%%%%%%%%%%%%%%%%%%%%%%%%%%%%%%%%%%%%%%%%%%%%%%%%
\chapter{The Language of \HoTT}
\label{sect:hottlang}

\begin{ednote}
  TODO: disentangle commonly used terminology: parameterized types,
  generic types, algebraic types, generalized algebraic types, types
  indexed/parameterized, polymorphism, parametric polymorphism,
  polytypism etc. etc.
\end{ednote}

\begin{ednote}
  Type formers as entries in a \HoTT{} lexicon, serving not as
  definitions but as (normative) rules of usage.  They don't say what
  the terms mean, they set out how to use them.  That means,
  specifically, the rules governing the notation, not rules governing
  denoted entities.

  NB: rules of use \emph{of vocabulary} \(\neq\) rules of construction
  \emph{of objects}.  But the idea is for one set of rules to work
  both ways.  That's pretty much how model-theoretic semantics
  connects vocab to semantic domain (completeness and consistency).
  The difference is that the TT semantic domain here (i.e. objects and
  their rules of construction) is not a passive, platonistic realm of
  ``real'' objects, but a pragmatic ``field'' of action.  So rules of
  vocab use and rules of construction converge while remaining
  conceptually distinct.  IOW the difference is not metaphysical.

  On this view we treat \HoTT{} as truly a vocabulary rather than a
  theory about something.  Or more precisely as a regimented idiom or
  dialect.  The user is free to treat e.g. \N as ``defining'' a true
  model of the natural numbers, but \HoTT{} makes no such claim.
\end{ednote}

One way to think about mathematics and logic is in terms of objects,
structures, relations, and the like.  etc.

But one can also think of it in terms of vocabularies (or idioms,
etc.).  Then mastering a discipline is not just a matter grasping some
content, but also of acquiring practical mastery over a vocabulary.

The vocabulary of set theory has dominated mathematical discourse for
most of the last 100 years or so.  Starting in the late 1940s, a
competing vocabulary based on category theory began to emerge.  Today
it is not uncommon to see both vocabularies deployed in the same
discourse (lecture, paper).

{\todo Type theory as a vocabulary - mostly confined to logic, then
  computer science.  Etc.  \HoTT{} as the latest distinctive vocab. -
  covering both math and compsci, also regions of logic.
  Significantly different that both set theory and classic logic.}

``it is possible to directly formalize the world of homotopy types
using the class of languages called dependent type systems and in
particular Martin-Lof type systems.'' V. Voevodsky
\url{http://www.math.ias.edu/~vladimir/Site3/Univalent\_Foundations\_files/univalent\_foundations\_project.pdf}

Note: ``class of languages called dependent type systems'' -
languages, not theories

``Type systems are syntactic objects which are specified in several
steps. First one chooses a formal language L which allows the use of
variables and substitution. Then one chooses a collection of relations
on the sets of L-expressions with a given set of free variables which
is stable under the substitutions. These relations are called the
reduction rules and the equivalence relation generated by the
reduction rules is called the conversion relation....  A type system
based on L is defined as a pair of subsets BB and BBg in the sets of
pre-contexts and pre-sequents respectively which satisfy a number of
conditions with respect to reduction and substitution. Elements of BB
are called the (valid) contexts of a type system and elements of BBg
the (valid) sequents of the type system.'' same, p. 3



%%%%%%%%
\section{\HoTT{} Types}
\label{subs:hott}

\HoTT{} primitives are ....\sidenote{A proper exposition would
  list 1) the name of the primitive, e.g. ``Dependent Product-type'';
  2) the ``constructor'' symbol, e.g. \(\Pi\); 3) the analogous
  concept from set theory; 4) the ``rules'' for defining a type
  (formation, construction, elimination, computation, uniqueness).}

\begin{ednote}
  The concept of primitives probably isn't going to work for type
  theories proper.  Type theories seem to be inherently pluralistic,
  so there is no way to pick out some things as intrinsically
  primitive in all type theories.  Each theory might \emph{define} a
  set of primitives, but that would define conventions of the theory,
  not the notion of primitive that we're after here.  So if we want to
  talk about primitives it will have to be as above, involving
  principles antecedent to any type-theoretic talk. (?)
\end{ednote}

%%%%%%%%
\subsection{Simple Types}
\label{subs:simpletypes}

\begin{ednote}
  E.g. \N
\end{ednote}

``Proposition type'' Conceptually, at least, this seems primitive.
Especially if the concept of ``proposition'' counts as a pre-theoretic
principle.  Which implies that proof is also a pre-theoretic
principle.  Proposition types are fundamentally different than the
other kinds of type, since they have
truth-conditions.\sidenote{Actually this isn't quite right.
  Propositions have truth-conditions in classic logic, but not not in
  type theory.  In type theory they have proofs; a propositional type
  is not true or false, but proven or disproven.  But the larger point
  stands: the \textit{concept} of proposition is different than the
  concept of, say, natural number.} Etc.  It follows that proofs are
fundamentally different from other kinds of witness.

\begin{ednote}
  But Curry-Howard means there is no distinction between types and
  propositions, so it makes no sense to try to demarcate a
  ``proposition type''.  E.g. the type \N can be viewed as the type of
  ``there exists a natural number''.  This feature demarcates type
  theory; in classic logic and math, and esp. traditional logic, there
  is a fundamental difference between propositions and the terms from
  which they are constructed.  Not so in \tth{}.
\end{ednote}


%%%%%%%%
\subsection{Compound Types}
\label{subs:compountypes}

\begin{description}
\item [Function] ``Unlike in set theory, functions are not defined as
  functional relations; rather they are a primitive concept in type
  theory.''\sidenote{Or: set theory
    \textit{defines} a function as a set of ordered pairs whose domain
    has no duplicates; in other words, it treats a function and its
    ``graph'' as the same thing.  Question: what happens to the graph
    of a function in type theory?} \citep[p. 21]{hottbook}

\item [Product] Product types correspond to cartesian products in set
  theory.  The constructor symbol is the same as in set theory:
  \(\cross\).\sidenote{Why isn't this called the ``\(\Huge\cross\)-type''?}
  ``[U]nlike in set theory, where we define ordered pairs to be
  particular sets and then collect them all together into the
  cartesian product, in type theory, ordered pairs are a primitive
  concept, as are functions.''\citep[p. 26]{hottbook}

\item [Coproduct type] Coproduct types correspond to disjoint unions
  in set theory.  ``In type theory, as was the case with functions and
  products, the coproduct must be a fundamental construction, since
  there is no previously defined notion of ``union of
  types''.\citep[p. 33]{hottbook}

\end{description}

\begin{ednote}
  [Updated] [Update: M. Shulman pulled the scales from my eyes: ``From
    the point of view taken in the book, the difference is only one of
    perspective, and any type can represent a proposition by simply
    shifting our perspective on it. For instance, the type Nat
    represents the proposition "there exists a natural number".'']
  What I've called ``proposition type'' is not \textit{formally}
  distinct; the distinction I'm after is conceptual.  But this
  suggests that we need to add at least one more item to our list of
  primitives: ``ordinary'' or simple types.  Maybe it would be best to
  start with the natural numbers as an example of a simple type,
  rather than function types.  Then an example of a proposition type,
  before proceeding to function type.  The items listed (following the
  \HoTTB{}) are really constructions, or let's say complex types,
  built out of two other types.  So maybe the distinction we want is
  between simple and complex or compound types.  Then the simple types
  would come out as primitive, and the complex types as derived (just
  like dependent types.)  Compare the idea of constructions in
  category theory.  There categories are primitive and e.g. the
  product category is an example of a category constructed from other
  categories.
\end{ednote}

\begin{ednote}
  For consistency, we might want to use symbols to designate all of
  the primitives, just as we do for \(\Pi\) and \(\Sigma\).  This
  would give us: \(\Huge\fun\)-types, \(\Huge\cross\)-types, and
  \(\Huge +\)-types.
\end{ednote}


%%%%%%%%
\subsection{Dependent Types}
\label{subs:quasiprim}

\noindent ``Fundamental''\sidenote[][-28pt]{\begin{ednote}
  Obviously we need a better bit of
  terminology.  ``Quasi-primitives''?  ``Neo-primitives''?  These
  types are not primitive, strictly speaking, but on the other hand
  they are basic.  I think there is another fundamental principle at
  work here.  In set theory, for example, the concept of function is
  not only not primitive, it isn't necessary.  You could discard it
  and still have set theory.  But my intuition tells me that e.g. the
  concept \(\Pi\)-type is in a sense necessary or essential in type
  theory, even if it is not primitive.  Once you have the primitives,
  you necessarily have these non-primitive basic types.  Dunno if
  that's correct, but it would sure be nice if it were.
  \end{ednote}}%
(but non-primitive) concepts and types.  These types seem to be on a par
with the primitive types as far as importance goes, but they
presuppose the primitives, so cannot themselves be considered
primitive.

\begin{ednote}
  We can make a distinction between the concept of dependent type, and
  the two specific dependent types introduced here.  Neither is
  primitive; you can have a type theory without the concept of
  dependent types.  Most programming languages fit this description,
  whether they have an explicit type discipline or not.
\end{ednote}

\begin{ednote}
  Regarding the notion ``quasi-primitive'': not a very satisfactory
  term, but I can't come up with a better one at the moment.  What I'd
  like to show is that the concept of dependent type (maybe also type
  universe) follows ``naturally'' or necessarily or essentially from
  the more primitive concepts.  Maybe the right concept here is
  ``induction'': the primitive concepts (types) ``induce'' the concept
  of dependent type.  That would be nice esp. if induction is a
  primary principle.
\end{ednote}

\begin{theorem}
  theorem test
\end{theorem}

\begin{definition}
  test definition.
\end{definition}

\begin{description}

\item [Universe]  Is this a primitive?  Probably not, since it builds on the type concept.

\item [\(\Pi\)-type] Informally, ``dependent function''
  types.\sidenote{As a practical matter, I think it would be useful to
    have an informal term for these types that falls between
    ``dependent function type'' and \(\Pi\)-type.  Something like
    ``p-function type''.  \(\Pi\)-type is admirably concise, but I
    think it should mention ``function'', since it names a kind of
    function.}  The concept of \(\Pi\)-type is a generalization of the
  concept of function type, so it isn't primitive.

\item [\(\Sigma\)-type] Informally, ``dependent pair''
  type.\sidenote{Shouldn't this be called ``dependent
    \textit{product}'' type?  The type is product, not pair; pairs are
    ``elements'' of the type.  Informally, maybe ``sig-prod type?}
  The concept of \(\Sigma\)-types is a generalization of the concept
  of product type.

\end{description}

%%%%%%%%
\subsection{Standard Type Library}
\label{subs:hottstdlib}

\begin{ednote}
  By analogy to the usual ``standard library'' of programming
  languages.  The idea is to list commonly used types that are neither
  primitive nor quasi-primitive; ``application'' types, in a sense.
\end{ednote}


\begin{description}
\item [Boolean] \citep[p. 34]{hottbook}
\item [$\nat$] \citep[p. 36]{hottbook} But there's a problem here;
  actually several.  First of all, the section on the naturals in
  chapter one does not actually show how to construct them; its really
  a section about induction, not the natural numbers.  Second, what it
  does say about the naturals is that they start from zero.  That
  obviously won't do; zero is not a natural number, and there is no
  intuitive notion of zero as a number.  You can't even think of it as
  a number until you've severed the link between the concept of number
  and the concepts of quantity and/or magnitude.  So the natural
  numbers really must start with 1, not 0.
\item [Proposition]  Moved to Primitives section.
\item [Identity]
\end{description}


\section{Misc. Niceties}

ML Type Theory is centered (more or less) on one of the major
logico-philosophical topics of the 20th century, namely the nature of
assertion and its relation to propositions and inferences.

You don't have to understand the arcana of this debate in order to
understand type theory (or HoTT), but some familiarity with the main
outline is very helpful.  Actually, I think it's essential, if you
want to understand the HoTT Book's account of \textit{judgement},
presented in HoTT Chapter 1 (reproduced below).  Fortunately the
presentation is relatively straightforward.

\begin{remark}
  Stress: this is largely a philosophical issue, or perhaps an issue
  in Philosophy of Language.  It's really about how our utterances
  come to have the significances they do.
\end{remark}

Outline:

\begin{itemize}
\item Frege's elevation of \textit{force} as essential
\item Dealing with embedded (and therefore forceless) propositions
\item Wittgenstein
\item Dummett
\item etc.
\item Brandom's recent innovation: decompose ``assertion'' into ``commitment'' and ``entitlement''
\end{itemize}

What the HoTT Book refers to as judgment (following ML) could also be
called assertion.  Brandom's account of the ``fine structure'' of
assertion is very helpful here.  Among other things, it provides a
very simple explanation of how embedded propositions work.  Embedded
propositions are unasserted; the problem is how to reconcile this with
the fact that they are function as assertions if unembedded.  On
Brandom's account, [todo...]

In other words, we can have commitment with or without entitlement,
and vice-versa.

A set membership statement can be explained in terms of commitments
and entitlements.  A free occurance of e.g. \(a\in A\) is ordinarily
taken as an assertion (judgment).  We can follow Frege and make this
\textit{force} explicit: \(\vdash a\in A\).  The problem with this,
however, is that, in contemporary usage, this would make \(a\in A\)
\textit{logically} true, which is not what we want.  Instead we want a
representation of committment to the proposition, as at least
ordinarily true, without regard to its logical truth.\marginnote{TODO:
  logical v. ordinary truth is pretty hairy for non-logicians so the
  distinction should be explicated.}

In sum:  the implicity sense of \(a\in A\) is something like: 
\[\exists \Gamma, a, A | \Gamma\vdash a\in A\]

Informally: there exists a set of propositions \(\Gamma\), a value (or
object) \(a\), and a set \(A\) such that the propostion \(a\in A\) is
deducible from \(\Gamma\).

So the meaning of \(a\in A\) essentially involves existential
quantification.  It is a statement about the world, that it contains
the relavant entities, not about the entities themselves.

\begin{remark}
  Not quite; \(a\in A\) is surely a statement about \(a\), maybe also
  about \(A\), no?  But still there must be an implicity existential
  quantification over the propositions that entail the statement.
\end{remark}

There is a logical subtlety here.  \(a\in A\) seems to be about a
determinate \(a\) and a determinate \(A\), but it isn't, not if we
take it to be an existentially quantified statement.  That's because
\(\exists a, A | a\in A\) does not pick out determinate individuals;
it just says that \textit{some} such individuals exist in the domain
of interpretation.  True, \(a\) and \(A\) are said to be bound by
\(\exists\), but that's not entirely accurate; quantified variables
are not bound in the way that constant symbols like \(\pi\) or \(0\)
are bound.  Whatever we go on to say about \(a\) and \(A\) --
e.g. \(a\in A\) -- remains within the scope of the quantifier, so it
does not count as a statement about determinate individuals.  It's a
statement about the world, that it contains entities that satisfy the
predicate.

On the other hand, the same seems to be true of \(a : A\): though
these symbols be bound, we don't know what they are bound to.  They
are not bound by an implicit existential quantifier; \(a : A\) does
\textit{not} mean \(\exists a, A | a : A\).

\begin{remark}
  Plus, quantifiers have to be used with a predicate; strictly
  speaking, \(\exists a, A\) is not a complete statement.
\end{remark}

By contrast, the Type Theoretic analogue \(a : A\) is a statement
about a specific value and a specific type, without any
quantification.  It is not directly a statement about the world, but
about part of the world.  Or: it expresses both commitment and
entitlement.  That's why it cannot be embedded in e.g. ``if \(a : A\)
then it is not the case that \(b : B\)''.  Embedded propositions
cannot carry force, but \(a : A\) always carries force, intrinsically,
as it were.

%%%%%%%%
\subsection{a : A}
\label{subs:aA}

Forms go from symbols to terms to sentences; from \(a\) to \(a+b\) to
\(a+b=c\).

The ``judgment'' \(a : A\) is clearly a compound term, so it cannot
merely name something.  But is it a sentence?  Does it denote a
proposition?  Or is it analogous to terms like \(a+b\) which are names
of a sort but involve some additional meaning beyond mere reference.

It seems it must involve a proposition, or let's say propositional
content.  We take \(a : A\) as a statement of fact, rather than a mere
reference to some part of the world.  Then how is it distinct from
\(a\in A\)?

The HoTT Book says it is ``analogous'' to the set-theoretic statement
\(a\in A\), but essentially different, since \(a\in A\) is a
proposition but \(a : A\) is a judgment.  It says that, \textit{when
  working internally in type theory}, \(a : A\) cannot be embedded, as
in `` if \(a : A\) then it is not the case that \(b : B\)'', nor can
the judgment \(a : A\) be disproved.

So let's look closely at what this means.  Earlier, HoTT says that
(some) judgments involving A ``exist at a different level from the
\textit{proposition} \(A\) itself, which is an internal statement of
the theory.''  (p. 18) There's a bit of circularity there; what is an
``internal statement''?

{\todo The nature of ``proposition'' has been a topic of
  considerable debate.  Review some of the alternative accounts on
  offer.}


The basic idea seems to be based on the well-known concept that
propositions by themselves are devoid of force, and must be asserted.
HoTT seems to imply that judgments are asserted propositions -- or
more correctly, assertings of propositions.

This seems a little bit off.  Assertion is something only people do.
An inked form on a page cannot really be construed as an assertion.
So we need to work out the mechanics of how a written form like \(a :
A\) can be viewed as a ``judgment'' in this sense.  I think Brandom's
model of assertion works.  It would say, I think, that \(a : A\)
counts as a judgment (assertion) because by convention we agree to
treat it that way, whereas we treat \(a\in A\) slightly differently,
because of the conventions elaborated by 20th century logic.

When HoTTB refers to ``working internally in type theory'', it seems,
the idea is to consider propositions in isolation from their
assertion.  Assertion, on this view, is something that comes from
outside of the world of propositions.  This is perfectly in tune with
the idea that asserting is something people do, but that what gets
assert\textit{ed} -- the \textit{content} of an assertion -- is
distinct from the assert\textit{ing}.

\begin{remark}
  Sellars called this the notorious -ing/-ed distinction.
\end{remark}

This would seem to make \(a : A\) an assert\textit{ing}.

We can think of \atypeA{} as a \textit{given} proposition: one that,
while unasserted, has the same force as a propositional assertion.  Or
another way to put it would be to say that use of \atypeA{} is
inalienably performative.

In fact \atypeA{} corresponds nicely to a common linguistic practice,
namely combining a proper name and a description, as in ``Joan of
Arc'', ``King George'', or ``Slick Willy''.  Or, more colloquially,
``poor Tom'', ``angry Joe'', or ``Gimpel the fool''.  And the
primitive nature of types can be clearly illustrated by analogy with
the military.  In type theory, every object has a type, just as
everybody in the military has a rank.  You cannot be in the military
unless you have a rank.  Within the military, the proper way refer to
someone in the miliary is to combine rank and name: General Custer,
Sergeant York, Private Bilko.  So the difference between \atypeA{} and
\(a\in A\) is like the difference between ``This is General Custer''
and ``This is Custer; he is a General''.

On the other hand, ``This is General Custer'' doesn't look much like a
\textit{judgment}, although it does look like a \textit{claim}.  But
not that it is not a claim about the meaning of ``General Custer'';
rather it is a claim about the relation between ``This'' and ``General
Custer''.  You could be wrong about the name or the rank of whomever
you mean by ``This'', but you cannot be wrong about ``General
Custer''; that's just a qualified name.  Being wrong in this sense
about ``This is General Custer'' is an empirical matter; in type
theory, the question of whether \atypeA (``this is a-of-A'') is
correct or not never even arises.  It doesn't make an empirical
assertion, it states a \textit{given}.  Or we might say it gives a
fact.  By contrast, \(a\in A\), as a proposition, may be either true
or false; when we say ``let \(a\in A\)'', we implicitly stipulate that
\(a\in A\) is to be \textit{assumed} to be true, but it is not
\textit{given} as true.  In other words, we can gloss it as ``\(a\in
A\) has a truth-value like any proposition, so it could be false, but
please assume that it is true.''

Another critical distinction: in standard set theory and logic,
judgments come from the outside, as it were.  But in HoTT, judgments
of the form \atypeA are internal.  They may be derivable inside the
system (by production of a proof or witness.)  In other words,
inference in set theory comes from outside of the world of sets, but
inference in HoTT is built in to the structure of types.  Inference
(construction) is part of the intrinsic meaning of types.

%%%%%%%%
\subsection{Justification}
\label{subs:just}

The HoTT Book's account of judgments in Chapter 1 section 1 seems to
conflate the distinction\sidenote{This is only to be
expected, since Brandom is the first (so far as I know) to see that
assertion (judgment) has an internal structure involving commitment
and entitlement (and some other stuff like a social dimension.)} Brandom makes between commitment and
entitlement.  ``Informally, a deductive system is a collection of
rules for deriving things called judgments.''

But derivation (proof) starts and ends in propositions; commitment is
something else.  The derivation or proof provide warrant for
entitlement to the commitment - justification of the conclusion.  So
how would Brandom parse ``judgment'' as HoTT uses the term?



\clearpage
\appendix
\begin{appendices}
  %%%%%%%%%%%%%%%%%%%%%%%%%%%%%%%%%%%%%%%%%%%%%%%%%%%%%%%%%%%%%%%%
  \chapter{HoTT Book Excerpts}
  \section{HoTT Introduction}
%% \markboth{\textsc{Introduction}}{}
%% \addcontentsline{toc}{chapter}{Introduction}
%% \setcounter{page}{1}
%% \pagenumbering{arabic}


\emph{Homotopy type theory} is a new branch of mathematics that combines aspects of several different fields in a surprising way. It is based on a recently discovered connection between \emph{homotopy theory} and \emph{type theory}.
Homotopy theory is an outgrowth of algebraic topology and homological algebra, with relationships to higher category theory; while type theory is a branch of mathematical logic and theoretical computer science.
Although the connections between the two are currently the focus of intense investigation, it is increasingly clear that they are just the beginning of a subject that will take more time and more hard work to fully understand.
It touches on topics as seemingly distant as the homotopy groups of spheres, the algorithms for type checking, and the definition of weak $\infty$-groupoids.

Homotopy type theory also brings new ideas into the very foundation of mathematics.
%% \index{foundations, univalent}%
On the one hand, there is Voevodsky's subtle and beautiful \emph{univalence axiom}. 
%% \index{univalence axiom}%
The univalence axiom implies, in particular, that isomorphic structures can be identified, a principle that mathematicians have been happily using on workdays, despite its incompatibility with the ``official'' doctrines of conventional foundations.
On the other hand, we have \emph{higher inductive types}, which provide direct, logical descriptions of some of the basic spaces and constructions of homotopy theory: spheres, cylinders, truncations, localizations, etc.
Both ideas are impossible to capture directly in classical set-theoretic foundations, but when combined in homotopy type theory, they permit an entirely new kind of ``logic of homotopy types''.
%% \index{foundations}%

This suggests a new conception of foundations of mathematics, with intrinsic homotopical content, an ``invariant'' conception of the objects of mathematics --- and convenient machine implementations, which can serve as a practical aid to the working mathematician.
This is the \emph{Univalent Foundations} program.
The present book is intended as a first systematic exposition of the basics of univalent foundations, and a collection of examples of this new style of reasoning --- but without requiring the reader to know or learn any formal logic, or to use any computer proof assistant.

% This enlarges the page by one line in letter format. Use sparringly.
%% \OPTwidow

We emphasize that homotopy type theory is a young field, and univalent foundations is very much a work in progress.
This book should be regarded as a ``snapshot'' of the state of the field at the time it was written, rather than a polished exposition of an established edifice.
As we will discuss briefly later, there are many aspects of homotopy type theory that are not yet fully understood --- but as of this writing, its broad outlines seem clear enough.
The ultimate theory will probably not look exactly like the one described in this book, but it will surely be \emph{at least} as capable and powerful; we therefore believe that univalent foundations will eventually become a viable alternative to set theory as the ``implicit foundation'' for the unformalized mathematics done by most mathematicians.

\subsection{Type theory}

Type theory was originally invented by Bertrand Russell \citep{Russell:1908},\index{Russell, Bertrand} as a device for blocking the paradoxes in the logical foundations of mathematics  that were under investigation at the time. It was later developed as a rigorous formal system in its own right (under the name ``$\lambda$-calculus'') by Alonzo Church \citep{Church:1933cl,Church:1940tu,Church:1941tc}.  Although it is not generally regarded as the foundation for classical mathematics, set theory being more customary, type theory still has numerous applications, especially in computer science and the theory of programming languages~\citep{Pierce-TAPL}.
%% \index{programming}%
%% \index{type theory}%
%% \index{lambda-calculus@$\lambda$-calculus}%
Per Martin-L\"{o}f %% \citep{Martin-Lof-1972},
\citep{Martin-Lof-1972,Martin-Lof-1973,Martin-Lof-1979,martin-lof:bibliopolis},
 among others,
developed a ``predicative'' modification of Church's type system, which is now usually called dependent, constructive, intuitionistic, or simply \emph{Martin\--L\"of type theory}. This is the basis of the system that we consider here; it was originally intended as a rigorous framework for the formalization of constructive mathematics.\marginnote{
  ``[T]he intuitionistic type theory (\citep{Martin-Lof-1973}), which
  I began to develop \emph{solely with the philosophical motive} of
  clarifying the syntax and semantics of intuitionistic mathematics,
  may equally well be viewed as a programming language.'' \citep{Martin-Lof-1979} (Emphasis added.)
}  In what follows, we will often use ``type theory'' to refer specifically to this system and similar ones, although type theory as a subject is much broader (see \citep{somma,kamar})
 for the history of type theory).

In type theory, unlike set theory, objects are classified using a primitive notion of \emph{type}, similar to the data-types used in programming languages.  These elaborately structured types can be used to express detailed specifications of the objects classified, giving rise to principles of reasoning about these objects.  To take a very simple example, the objects of a product type $A\times B$ are known to be of the form $\pairr{a,b}$, and so one automatically knows how to construct them and how to decompose them. Similarly, an object of function type $A\to B$ can be acquired from an object of type $B$ parametrized by objects of type $A$, and can be evaluated at an argument of type $A$.  This rigidly predictable behavior of all objects (as opposed to set theory's more liberal formation principles, allowing inhomogeneous sets) is one aspect of type theory that has led to its extensive use in verifying the correctness of computer programs.  The clear reasoning principles associated with the construction of types also form the basis of modern \emph{computer proof assistants},%
%% \index{proof!assistant}%
%% \indexsee{computer proof assistant}{proof assistant}
%% \index{mathematics!formalized}%
which are used for formalizing mathematics and verifying the correctness of formalized proofs.  We return to this aspect of type theory below.  

One problem in understanding type theory from a mathematical point of view, however, has always been that the basic concept of \emph{type} is unlike that of \emph{set} in ways that have been hard to make precise.  We believe that the new idea of regarding types, not as strange sets (perhaps constructed without using classical logic), but as spaces, viewed from the perspective of homotopy theory, is a significant step forward.  In particular, it solves the problem of understanding how the notion of equality of elements of a type differs from that of elements of a set.

In homotopy theory one is concerned with spaces
%% \index{topological!space}%
and continuous mappings between them, 
%% \index{function!continuous!in classical homotopy theory}%
up to homotopy.  A \emph{homotopy}
%% \index{homotopy!topological}%
between a pair of continuous maps $f : X \to Y$
and  $g : X\to Y$ is 
a continuous map $H : X \times [0, 1] \to Y$ satisfying
$H(x, 0) = f (x)$  and $H(x, 1) = g(x)$. The homotopy $H$ may be thought of as a ``continuous deformation'' of $f$ into $g$. The spaces $X$ and $Y$ are said to be \emph{homotopy equivalent},
%% \index{homotopy!equivalence!topological}%
$\eqv X Y$, if there are continuous maps going back and forth, the composites of which are homotopical to the respective identity mappings, i.e., if they are isomorphic ``up to homotopy''.  Homotopy equivalent spaces have the same algebraic invariants (e.g., homology, or the fundamental group), and are said to have the same \emph{homotopy type}.

\subsection{Homotopy type theory}

Homotopy type theory (HoTT) interprets type theory from a homotopical perspective.
In homotopy type theory, we regard the types as ``spaces'' (as studied in homotopy theory) or higher groupoids, and the logical constructions (such as the product $A\times B$) as homotopy-invariant constructions on these spaces.
In this way, we are able to manipulate spaces directly without first having to develop point-set topology (or any combinatorial replacement for it, such as the theory of simplicial sets).
To briefly explain this perspective, consider first the basic concept of type theory, namely that
the \emph{term} $a$ is of \emph{type} $A$, which is written:
\[ a:A. \]
This expression is traditionally thought of as akin to:
\begin{center}
``$a$ is an element of the set $A$''.
\end{center}
However, in homotopy type theory we think of it instead as:
\begin{center}
``$a$ is a point of the space $A$''.
\end{center}
%% \index{continuity of functions in type theory@``continuity'' of functions in type theory}%
Similarly, every function $f : A\to B$ in type theory is regarded as a continuous map from the space $A$ to the space $B$.

We should stress that these ``spaces'' are treated purely homotopically, not topologically.
For instance, there is no notion of ``open subset'' of a type or of ``convergence'' of a sequence of elements of a type.
We only have ``homotopical'' notions, such as paths between points and homotopies between paths, which also make sense in other models of homotopy theory (such as simplicial sets).
Thus, it would be more accurate to say that we treat types as \emph{$\infty$-groupoids}\index{.infinity-groupoid@$\infty$-groupoid}; this is a name for the ``invariant objects'' of homotopy theory which can be presented by topological spaces,
%% \index{topological!space}%
simplicial sets, or any other model for homotopy theory.
However, it is convenient to sometimes use topological words such as ``space'' and ``path'', as long as we remember that other topological concepts are not applicable.

(It is tempting to also use the phrase \emph{homotopy type}
%% \index{homotopy!type}%
for these objects, suggesting the dual interpretation of ``a type (as in type theory) viewed homotopically'' and ``a space considered from the point of view of homotopy theory''.
The latter is a bit different from the classical meaning of ``homotopy type'' as an \emph{equivalence class} of spaces modulo homotopy equivalence, although it does preserve the meaning of phrases such as ``these two spaces have the same homotopy type''.)

The idea of interpreting types as structured objects, rather than sets, has a long pedigree, and is known to clarify various mysterious aspects of type theory.
For instance, interpreting types as sheaves helps explain the intuitionistic nature of type-theoretic logic, while interpreting them as partial equivalence relations or ``domains'' helps explain its computational aspects.
It also implies that we can use type-theoretic reasoning to study the structured objects, leading to the rich field of categorical logic.
The homotopical interpretation fits this same pattern: it clarifies the nature of \emph{identity} (or equality) in type theory, and allows us to use type-theoretic reasoning in the study of homotopy theory.

The key new idea of the homotopy interpretation is that the logical notion of identity $a = b$ of two objects $a, b: A$ of the same type $A$ can be understood as the existence of a path $p : a \leadsto b$ from point $a$ to point $b$ in the space $A$.
This also means that two functions $f, g: A\to B$ can be identified if they are homotopic, since a homotopy is just a (continuous) family of paths $p_x: f(x) \leadsto g(x)$ in $B$, one for each $x:A$.
In type theory, for every type $A$ there is a (formerly somewhat mysterious) type $\idtypevar{A}$ of identifications of two objects of $A$; in homotopy type theory, this is just the \emph{path space} $A^I$ of all continuous maps $I\to A$ from the unit interval.
%% \index{unit!interval}%
%% \index{interval!topological unit}%
%% \index{path!topological}%
%% \index{topological!path}%
In this way, a term $p : \idtype[A]{a}{b}$ represents a path $p : a \leadsto b$ in $A$. 

The idea of homotopy type theory arose around 2006 in independent work by Awodey and Warren~\citep{AW} and Voevodsky~\citep{VV}, but it was inspired by 
Hofmann and Streicher's earlier groupoid interpretation~\citep{hs:gpd-typethy}.
Indeed, higher-dimensional category theory (particularly the theory of weak $\infty$-groupoids) is now known to be intimately connected to homotopy theory, as proposed by Grothendieck and now being studied intensely by mathematicians of both sorts.
The original semantic models of Awodey--Warren and Voevodsky use well-known notions and techniques from homotopy theory which are now also in use in higher category theory, such as  Quillen model categories and Kan\index{Kan complex} simplicial sets\index{simplicial!sets}.
%% \index{Quillen model category}%
%% \index{model category}%

Voevodsky recognized that the simplicial interpretation of type theory satisfies a further crucial property, dubbed \emph{univalence}, which had not previously been considered in type theory (although Church's principle of extensionality for propositions turns out to be a very special case of it).
Adding univalence to type theory in the form of a new axiom has far-reaching consequences, many of which are natural, simplifying and compelling.
The univalence axiom also further strengthens the homotopical view of type theory, since it holds in the simplicial model and other related models, while failing under the view of types as sets.

\subsection{Univalent foundations}

Very briefly, the basic idea of the univalence axiom can be explained as follows.
In type theory, one can have a universe $\UU$, the terms of which are themselves types, $A : \UU$, etc.
Those types that are terms of $\UU$ are commonly called \emph{small} types.
%% \index{type!small}%
%% \index{small!type}%
Like any type, $\UU$ has an identity type $\idtypevar{\UU}$, which expresses the identity relation $A = B$ between small types.
Thinking of types as spaces, $\UU$ is a space, the points of which are spaces; to understand its identity type, we must ask, what is a path $p : A \leadsto B$ between spaces in $\UU$?
The univalence axiom says that such paths correspond to homotopy equivalences $\eqv A B$, (roughly) as explained above.
A bit more precisely, given any (small) types $A$ and $B$, in addition to the primitive type $\idtype[\UU]AB$ of identifications of $A$ with $B$, there is the defined type $\texteqv AB$ of equivalences from $A$ to $B$.
Since the identity map on any object is an equivalence, there is a canonical map,
\[\idtype[\UU]AB\to\texteqv AB.\]
The univalence axiom states that this map is itself an equivalence.
At the risk of oversimplifying, we can state this succinctly as follows:

\begin{description}\index{univalence axiom}%
\item[Univalence Axiom:]  $\eqvspaced{(A = B)}{(\eqv A B)}$.
\end{description}
%
In other words, identity is equivalent to equivalence. \index{identity}% 
In particular, one may say that ``equivalent types are identical''.
However, this phrase is somewhat misleading, since it may sound like a sort of ``skeletality'' condition which \emph{collapses} the notion of equivalence to coincide with identity, whereas in fact univalence is about \emph{expanding} the notion of identity so as to coincide with the (unchanged) notion of equivalence.

From the homotopical point of view, univalence implies that spaces of the same homotopy type are connected by a path in the universe $\UU$, in accord with the intuition of a classifying space for (small) spaces.
From the logical point of view, however, it is a radically new idea: it says that isomorphic things can be identified!  Mathematicians are of course used to identifying isomorphic structures in practice, but they generally do so by ``abuse of notation''\index{abuse!of notation}, or some other informal device, knowing that the objects involved are not ``really'' identical.  But in this new foundational scheme, such structures can be formally identified, in the logical sense that every property or construction involving one also applies to the other. Indeed, the identification is now made explicit, and properties and constructions can be systematically transported along it.  Moreover, the different ways in which such identifications may be made themselves form a structure that one can (and should!)\ take into account.

Thus in sum, for points $A$ and $B$ of the universe $\UU$ (i.e., small types), the univalence axiom identifies the following three notions:
\begin{itemize}
\item (logical) an identification $p:A=B$ of $A$ and $B$
\item (topological) a path $p:A \leadsto B$ from $A$ to $B$ in $\UU$
\item (homotopical) an equivalence $p:\eqv A B$ between $A$ and $B$.
\end{itemize}

\subsection{Higher inductive types}\index{type!higher inductive}%

One of the classical advantages of type theory is its simple and effective techniques for working with inductively defined structures.
The simplest nontrivial inductively defined structure is the natural numbers, which is inductively generated by zero and the successor function.
From this statement one can algorithmically\index{algorithm} extract the principle of mathematical induction, which characterizes the natural numbers.
More general inductive definitions encompass lists and well-founded trees of all sorts, each of which is characterized by a corresponding ``induction principle''.
This includes most data structures used in certain programming languages; hence the usefulness of type theory in formal reasoning about the latter.
If conceived in a very general sense, inductive definitions also include examples such as a disjoint union $A+B$, which may be regarded as ``inductively'' generated by the two injections $A\to A+B$ and $B\to A+B$.
The ``induction principle'' in this case is ``proof by case analysis'', which characterizes the disjoint union.

In homotopy theory, it is natural to consider also ``inductively defined spaces'' which are generated not merely by a collection of \emph{points}, but also by collections of \emph{paths} and higher paths.
Classically, such spaces are called \emph{CW complexes}.
%% \index{CW complex}%
For instance, the circle $S^1$ is generated by a single point and a single path from that point to itself.
Similarly, the 2-sphere $S^2$ is generated by a single point $b$ and a single two-dimensional path from the constant path at $b$ to itself, while the torus $T^2$ is generated by a single point, two paths $p$ and $q$ from that point to itself, and a two-dimensional path from $p\ct q$ to $q\ct p$.

By using the identification of paths with identities in homotopy type theory, these sort of ``inductively defined spaces'' can be characterized in type theory by ``induction principles'', entirely analogously to classical examples such as the natural numbers and the disjoint union.
The resulting \emph{higher inductive types}
%% \index{type!higher inductive}%
give a direct ``logical'' way to reason about familiar spaces such as spheres, which (in combination with univalence) can be used to perform familiar arguments from homotopy theory, such as calculating homotopy groups of spheres, in a purely formal way.
The resulting proofs are a marriage of classical homotopy-theoretic ideas with classical type-theoretic ones, yielding new insight into both disciplines.

Moreover, this is only the tip of the iceberg: many abstract constructions from homotopy theory, such as homotopy colimits, suspensions, Postnikov towers, localization, completion, and spectrification, can also be expressed as higher inductive types.
Many of these are classically constructed using Quillen's ``small object argument'', which can be regarded as a finite way of algorithmically describing an infinite CW complex presentation\index{presentation!of a space as a CW complex} of a space, just as ``zero and successor'' is a finite algorithmic\index{algorithm} description of the infinite set of natural numbers.
Spaces produced by the small object argument are infamously complicated and difficult to understand; the type-theoretic approach is potentially much simpler, bypassing the need for any explicit construction by giving direct access to the appropriate ``induction principle''.
Thus, the combination of univalence and higher inductive types suggests the possibility of a revolution, of sorts, in the practice of homotopy theory.


\subsection{Sets in univalent foundations}

%% \index{set|(}%

We have claimed that univalent foundations can eventually serve as a foundation for ``all'' of mathematics, but so far we have discussed 
only homotopy theory.  Of course, there are many specific examples of the use of type theory without the new homotopy type theory features to formalize mathematics,
%% \index{mathematics!formalized}%
%% \index{theorem!Feit--Thompson}%
%% \index{theorem!odd-order}%
%% \index{Feit--Thompson theorem}%
%% \index{odd-order theorem}%
such as the recent formalization of the Feit--Thompson odd-order theorem in \Coq~\citep{gonthier}.

But the traditional view is that mathematics is founded on set theory, in the sense that all mathematical objects and constructions can be coded into a theory such as Zermelo--Fraenkel set theory (ZF).
%% \index{set theory!Zermelo--Fraenkel}%
%% \indexsee{Zermelo-Fraenkel set theory}{set theory}%
%% \indexsee{ZF}{set theory}%
%% \indexsee{ZFC}{set theory}%
However, it is well-established by now that for most mathematics outside of set theory proper, the intricate hierarchical membership structure of sets in ZF is really unnecessary: a more ``structural'' theory, such as Lawvere's\index{Lawvere} Elementary Theory of the Category of Sets~\citep{lawvere:etcs-long}, suffices.
\index{Elementary Theory of the Category of Sets}%

In univalent foundations, the basic objects are ``homotopy types'' rather than sets, but we can \emph{define} a class of types which behave like sets.
Homotopically, these can be thought of as spaces in which every connected component is contractible, i.e.\ those which are homotopy equivalent to a discrete space.
%% \index{discrete!space}%
It is a theorem  that the category of such ``sets'' satisfies Lawvere's\index{Lawvere} axioms (or related ones, depending on the details of the theory).
Thus, any sort of mathematics that can be represented in an ETCS-like theory (which, experience suggests, is essentially all of mathematics) can equally well be represented in univalent foundations.  

This supports the claim that univalent foundations is at least as good as existing foundations of mathematics.
A mathematician working in univalent foundations can build structures out of sets in a familiar way, with more general homotopy types waiting in the foundational background until there is need of them.
For this reason, most of the applications in this book have been chosen to be areas where univalent foundations has something \emph{new} to contribute that distinguishes it from existing foundational systems.

Unsurprisingly, homotopy theory and category theory are two of these, but perhaps less obvious is that univalent foundations has something new and interesting to offer even in subjects such as set theory and real analysis.
For instance, the univalence axiom allows us to identify isomorphic structures, while higher inductive types allow direct descriptions of objects by their universal properties.
Thus we can generally avoid resorting to arbitrarily chosen representatives or transfinite iterative constructions.
In fact, even the objects of study in ZF set theory can be characterized, inside the sets of univalent foundations, by such an inductive universal property.

\index{set|)}%


\subsection{Informal type theory}

\index{mathematics!formalized|(defstyle}%
\index{informal type theory|(defstyle}%
\index{type theory!informal|(defstyle}%
\index{type theory!formal|(}%
One difficulty often encountered by the classical mathematician when faced with learning about type theory is that it is usually presented as a fully or partially formalized deductive system.
This style, which is very useful for proof-theoretic investigations, is not particularly convenient for use in applied, informal reasoning.
Nor is it even familiar to most working mathematicians, even those who might be interested in foundations of mathematics.
One objective of the present work is to develop an informal style of doing mathematics in univalent foundations that is at once rigorous and precise, but is also closer to the language and style of presentation of everyday mathematics.

In present-day mathematics, one usually constructs and reasons about mathematical objects in a way that could in principle, one presumes, be formalized in a system of elementary set theory, such as ZFC --- at least given enough ingenuity and patience.
For the most part, one does not even need to be aware of this possibility, since it largely coincides with the condition that a proof be ``fully rigorous'' (in the sense that all mathematicians have come to understand intuitively through education and experience).
But one does need to learn to be careful about a few aspects of ``informal set theory'': the use of collections too large or inchoate to be sets; the axiom of choice and its equivalents; even (for undergraduates) the method of proof by contradiction; and so on.
Adopting a new foundational system such as homotopy type theory as the \emph{implicit formal basis} of informal reasoning will require adjusting some of one's instincts and practices.
The present text is intended to serve as an example of this ``new kind of mathematics'', which is still informal, but could now in principle be formalized in homotopy type theory, rather than ZFC, again given enough ingenuity and patience.

It is worth emphasizing that, in this new system, such formalization can have real practical benefits.
The formal system of type theory is suited to computer systems and has been implemented in existing proof assistants.
\index{proof!assistant}%
A proof assistant is a computer program which guides the user in construction of a fully formal proof, only allowing valid steps of reasoning.
It also provides some degree of automation, can search libraries for existing theorems, and can even extract numerical algorithms\index{algorithm} \index{extraction of algorithms} from the resulting (constructive) proofs.

We believe that this aspect of the univalent foundations program distinguishes it from other approaches to foundations, potentially providing a new practical utility for the working mathematician.
Indeed, proof assistants based on older type theories have already been used to formalize substantial mathematical proofs, such as the four-color theorem\index{theorem!four-color} \index{four-color theorem} and the Feit--Thompson theorem.
Computer implementations of univalent foundations are presently works in progress (like the theory itself).
\index{proof!assistant}%
However, even its currently available implementations (which are mostly small modifications to existing proof assistants such as \Coq and 
\Agda) have already demonstrated their worth, not only in the formalization of known proofs, but in the discovery of new ones.
Indeed, many of the proofs described in this book were actually \emph{first} done in a fully formalized form in a proof assistant, and are only now being ``unformalized'' for the first time --- a reversal of the usual relation between formal and informal mathematics.

One can imagine a not-too-distant future when it will be possible for mathematicians to verify the correctness of their own papers by working within the system of univalent foundations, formalized in a proof assistant, and that doing so will become as natural as typesetting their own papers in \TeX.
%(Whether this proves to be the publishers' dream or their nightmare remains to be seen.) 
In principle, this could be equally true for any other foundational system, but we believe it to be more practically attainable using univalent foundations, as witnessed by the present work and its formal counterpart.

\index{type theory!formal|)}%
\index{informal type theory|)}%
\index{type theory!informal|)}%
\index{mathematics!formalized|)}%

\subsection{Constructivity} 

\index{mathematics!constructive|(}%

One of the most striking differences between
classical\index{mathematics!classical} foundations and type theory is
the idea of \emph{proof relevance}, according to which mathematical
statements, and even their proofs, become first-class mathematical
objects.  In type theory, we represent mathematical statements by
types, which can be regarded simultaneously as both mathematical
constructions and mathematical assertions, a conception also known as
\emph{propositions as types}.  \index{proposition!as types}%
Accordingly, we can regard a term $a : A$ as both an element of the
type $A$ (or in homotopy type theory, a point of the space $A$), and
at the same time, a proof of the proposition $A$.\sidenote{This
  suggests (to me, anyway) that ``element of type $A$'' and ``proof of
  the proposition $A$'' are different things, and $a : A$ can be used
  to name both.  But I don't think that's right; $a : A$ is one thing.
  Better to say that they are distinct ways of looking at one thing.}
To take an example, suppose we have sets $A$ and $B$ (discrete
spaces), and consider the statement ``$A$ is
isomorphic to $B$''.  In type theory, this can be rendered as:
\begin{narrowmultline*}
  \mathsf{Iso}(A,B) \defeq \narrowbreak
  \sm{f : A\to B}{g : B\to A}\Big(\big(\tprd{x:A} g(f(x)) = x\big) \times \big(\tprd{y:B}\, f(g(y)) = y\big)\Big).
\end{narrowmultline*}
%
Reading the type constructors $\Sigma, \Pi, \times$  here  as ``there exists'', ``for all'', and ``and'' respectively yields the usual formulation of ``$A$ and $B$ are isomorphic''; on the other hand, reading them as sums and products yields the \emph{type of all isomorphisms} between $A$ and $B$!  To prove that $A$ and $B$ are isomorphic, one  constructs a proof $p : \mathsf{Iso}(A,B)$, which is therefore the same  as constructing an isomorphism between $A$ and $B$, i.e., exhibiting a pair of functions $f, g$ together with \emph{proofs} that their composites are the respective identity maps.  The latter proofs, in turn, are nothing but homotopies of the appropriate sorts.  In this way, \emph{proving a proposition is the same as constructing an element of some particular type.}
In particular, to prove a statement of the form ``$A$ and $B$'' is just to prove $A$ and to prove $B$, i.e., to give an element of the type $A\times B$.
And to prove that $A$ implies $B$ is just to find an element of $A\to B$, i.e.\ a function from $A$ to $B$ (determining a mapping of proofs of $A$ to proofs of $B$).

The logic of propositions-as-types is flexible and supports many variations, such as using only a subclass of types to represent propositions.
In homotopy type theory, there are natural such subclasses arising from the fact that the system of all types, just like spaces in classical homotopy theory, is ``stratified'' according to the dimensions in which their higher homotopy structure exists or collapses.
In particular, Voevodsky has found a purely type-theoretic definition of \emph{homotopy $n$-types}, corresponding to spaces with no nontrivial homotopy information above dimension $n$.
(The $0$-types are the ``sets'' mentioned previously as satisfying Lawvere's axioms\index{Lawvere}.)
Moreover, with higher inductive types, we can universally ``truncate'' a type into an $n$-type; in classical homotopy theory this would be its $n^{\mathrm{th}}$ Postnikov\index{Postnikov tower} section.\index{n-type@$n$-type}
Particularly important for logic is the case of homotopy $(-1)$-types, which we call \emph{mere propositions}.
Classically, every $(-1)$-type is empty or contractible; we interpret these possibilities as the truth values ``false'' and ``true'' respectively.

Using all types as propositions yields a ``constructive'' conception of logic (for more on which, see~\citep{kolmogorov,TroelstraI,TroelstraII}), which gives type theory its good 
computational character.
For instance, every proof that something exists carries with it enough information to actually find such an object; and from a proof that  ``$A$ or $B$'' holds, one can extract either a proof that $A$ holds or one that $B$ holds.
Thus, from every proof we can automatically extract an algorithm;\index{algorithm} \index{extraction of algorithms} this can be very useful in applications to computer programming.

However, this logic does not faithfully represent certain important classical principles of reasoning, such as the axiom of choice and the law of excluded middle.
For these we need to use the ``$(-1)$-truncated'' logic, in which only the homotopy $(-1)$-types represent propositions; and under this interpretation, the system is fully compatible with classical mathematics.
Homotopy type theory is thus compatible with both constructive and classical conceptions of logic, and many more besides.

\index{axiom!of choice}%
More specifically, consider on one hand the \emph{axiom of choice}: ``if for every $x: A$ there exists a $y:B$ such that $R(x,y)$, there is a function $f : A\to B$ such that for all $x:A$ we have $R(x, f(x))$.''
The pure propositions-as-types notion of ``there exists'' is strong enough to make this statement simply provable --- yet it does not have all the consequences of the usual axiom of choice.
However, in $(-1)$-truncated logic, this statement is not automatically true, but is a strong assumption with the same sorts of consequences as its counterpart in classical\index{mathematics!classical} set theory.

\index{excluded middle}%
\index{univalence axiom}%
On the other hand, consider the \emph{law of excluded middle}: ``for all $A$, either $A$ or not $A$.''
Interpreting this in the pure propositions-as-types logic yields a statement that is inconsistent with the univalence axiom.
For since proving ``$A$'' means exhibiting an element of it, this assumption would give a uniform way of selecting an element from every nonempty type --- a sort of Hilbertian choice operator.
Univalence implies that the element of $A$ selected by such a choice operator must be invariant under all self-equivalences of $A$, since these are identified with self-identities and every operation must respect identity; but clearly some types have automorphisms with no fixed points, e.g.\ we can swap the elements of a two-element type.
\index{automorphism!fixed-point-free}%
However, the ``$(-1)$-truncated law of excluded middle'', though also not automatically true, may consistently be assumed with most of the same consequences as in classical mathematics.

In other words, while the pure propositions-as-types logic is ``constructive'' in the strong algorithmic sense mentioned above, the default $(-1)$-truncated logic is ``constructive'' in a different sense (namely, that of the logic formalized by Heyting under the name ``intuitionistic''); and to the latter we may freely add the axioms of choice and excluded middle to obtain a logic that may be called ``classical''.
\index{logic!constructive vs classical}%
Thus, the homotopical perspective reveals that classical and constructive logic can coexist, as endpoints of a spectrum of different systems, with an infinite number of possibilities in between (the homotopy $n$-types for $-1 < n < \infty$).
We may speak of ``\LEM{n}'' and ``\choice{n}'', with $\choice{\infty}$ being provable and \LEM{\infty} inconsistent with univalence, while $\choice{-1}$ and $\LEM{-1}$ are the versions familiar to classical mathematicians (hence in most cases it is appropriate to assume the subscript $(-1)$ when none is given).  Indeed, one can even have useful systems in which only \emph{certain} types satisfy such further ``classical'' principles, while types in general remain ``constructive''.\index{excluded middle}\index{axiom!of choice}%%

It is worth emphasizing that univalent foundations does not \emph{require} the use of constructive or intuitionistic logic.\index{logic!intuitionistic}\index{logic!constructive} %
Most of classical mathematics which depends on the law of excluded middle and the axiom of choice can be performed in univalent foundations, simply by assuming that these two principles hold (in their proper, $(-1)$-truncated, form).
However, type theory does encourage avoiding these principles when they are unnecessary, for several reasons.

First of all, every mathematician knows that a theorem is more powerful when proven using fewer assumptions, since it applies to more examples.
The situation with \choice{} and \LEM{} is no different:
type theory admits many interesting ``nonstandard'' models, such as in sheaf toposes,\index{topos} where classicality principles such as \choice{} and \LEM{} tend to fail.
Homotopy type theory admits similar models in higher toposes, such as are studied in~\citep{ToenVezzosi02,Rezk05,lurie:higher-topoi}.
Thus, if we avoid using these principles, the theorems we prove will be valid internally to all such models.

Secondly, one of the additional virtues of type theory is its computable character.
In addition to being a foundation for mathematics, type theory is a formal theory of computation, and can be treated as a powerful programming language.
\index{programming}%
From this perspective, the rules of the system cannot be chosen arbitrarily the way set-theoretic axioms can: there must be a harmony between them which allows all proofs to be ``executed'' as programs.
We do not yet fully understand the new principles introduced by homotopy type theory, such as univalence and higher inductive types, from
this point of view, but the basic outlines are emerging; see, for example,~\citep{lh:canonicity}.
It has been known for a long time, however, that principles such as \choice{} and \LEM{} are fundamentally antithetical to computability, since they assert baldly that certain things exist without giving any way to compute them.
Thus, avoiding them is necessary to maintain the character of type theory as a theory of computation.

Fortunately, constructive reasoning is not as hard as it may seem.
In some cases, simply by rephrasing some definitions, a theorem can be made constructive and its proof more elegant.
Moreover, in univalent foundations this seems to happen more often.
For instance:
\begin{enumerate}
\item In set-theoretic foundations, at various points in homotopy theory and category theory one needs the axiom of choice to perform transfinite constructions.
  But with higher inductive types, we can encode these constructions directly and constructively.
  In particular, none of the ``synthetic'' homotopy theory in \autoref{cha:homotopy} requires \LEM{} or \choice{}.
\item In set-theoretic foundations, the statement ``every fully faithful and essentially surjective functor is an equivalence of categories'' is equiv\-a\-lent to the axiom of choice.
  But with the univalence axiom, it is just \emph{true}; see \autoref{cha:category-theory}.
\item In set theory, various circumlocutions are required to obtain notions of ``cardinal number'' and ``ordinal number'' which canonically represent isomorphism classes of sets and well-ordered sets, respectively --- possibly involving the axiom of choice or the axiom of foundation.
  But with univalence and higher inductive types, we can obtain such representatives directly by truncating the universe; see \autoref{cha:set-math}.
\item In set-theoretic foundations, the definition of the real numbers as equivalence classes of Cauchy sequences requires either the law of excluded middle or the axiom of (countable) choice to be well-behaved.
  But with higher inductive types, we can give a version of this definition which is well-behaved and avoids any choice principles; see \autoref{cha:real-numbers}.
\end{enumerate}
Of course, these simplifications could as well be taken as evidence that the new methods will not, ultimately, prove to be really constructive.  However, we emphasize again that the reader does not have to care, or worry, about constructivity in order to read this book.  The point is that in all of the above examples, the version of the theory we give has independent advantages, whether or not \LEM{} and \choice{} are assumed to be available.  Constructivity, if attained, will be an added bonus.\index{constructivity}%

Given this discussion of adding new principles such as univalence, higher inductive types, \choice{}, and \LEM{}, one may wonder whether the resulting system remains consistent.
(One of the original virtues of type theory, relative to set theory, was that it can be seen to be consistent by proof-theoretic means).
As with any foundational system, consistency\index{consistency} is a relative question: ``consistent with respect to what?''
The short answer is that all of the constructions and axioms considered in this book have a model in the category of Kan\index{Kan complex} complexes, due to Voevodsky~\citep{klv:ssetmodel} (see~\citep{ls:hits} for higher inductive types).
Thus, they are known to be consistent relative to ZFC (with as many inaccessible cardinals
\index{inaccessible cardinal}\index{consistency}%
as we need nested univalent universes).
Giving a more traditionally type-theoretic account of this consistency is work in progress (see,
e.g.,~\citep{lh:canonicity,coquand2012constructive}).

We summarize the different points of view of the type-theoretic operations in \autoref{tab:pov}.

\begin{table}[htb]
  \centering
  \OPTsmalltable
 \begin{tabular}{lllll}
    \toprule
       Types && Logic & Sets & Homotopy\\ \addlinespace[2pt]
    \midrule
       $A$ && proposition & set & space\\ \addlinespace[2pt]
       $a:A$ && proof & element & point \\ \addlinespace[2pt]
       $B(x)$ && predicate & family of sets & fibration \\ \addlinespace[2pt]
       $b(x) : B(x)$ && conditional proof & family of elements & section\\ \addlinespace[2pt]
       \(\emptyt, \unit{}\) && \(\bot, \top\) & \(\emptyset, \{ \emptyset \}\) & \(\emptyset, *\)\\ \addlinespace[2pt]
       $A + B$ && $A\vee B$ & disjoint union & coproduct\\ \addlinespace[2pt]
       $A\times B$ && $A\wedge B$ & set of pairs & product space\\ \addlinespace[2pt]
       $A\to B$ && $A\Rightarrow B$ & set of functions & function space\\ \addlinespace[2pt]
       $\sm{x:A}B(x)$ &&  $\exists_{x:A}B(x)$ & disjoint sum & total space\\ \addlinespace[2pt]
       $\prd{x:A}B(x)$ &&  $\forall_{x:A}B(x)$ & product & space of sections\\ \addlinespace[2pt]
       $\mathsf{Id}_{A}$ && equality $=$ & $\setof{\pairr{x,x} | x\in A}$ & path space $A^I$ \\ \addlinespace[2pt]
    \bottomrule
  \end{tabular}
  \caption{Comparing points of view on type-theoretic operations}\label{tab:pov}
\end{table}

\index{mathematics!constructive|)}%

\subsection{Open problems} 

\index{open!problem|(}%

For those interested in contributing to this new branch of mathematics, it may be encouraging to know that there are many interesting open questions.

\index{univalence axiom!constructivity of}%
Perhaps the most pressing of them is the ``constructivity'' of the Univalence Axiom, posed by Voevodsky in \citep{Universe-poly}.
The basic system of type theory follows the structure of Gentzen's natural deduction. Logical connectives are defined by their introduction rules, and have elimination rules justified by computation rules. Following this pattern, and using Tait's computability method, originally designed to analyse G\"odel's Dialectica interpretation, one can show the property of \emph{normalization} for type theory. This in turn implies important properties such as decidability of type-checking (a crucial property since type-checking corresponds to proof-checking, and one can argue that we should be able to ``recognize a proof when we see one''), and the so-called ``canonicity\index{canonicity} property'' that any closed term of the type of natural numbers reduces to a numeral. This last property, and the uniform structure of introduction/elimination rules, are lost when one extends type theory with an axiom, such as the axiom of function extensionality, or the univalence axiom. Voevodsky has formulated a precise mathematical conjecture connected to this question of canonicity for type theory extended with the axiom of Univalence: given a closed term of the type of natural numbers, is it always possible to find a numeral and a proof that this term is equal to this numeral, where this proof of equality may itself use the univalence axiom? More generally, an important issue is whether it is possible to provide a constructive justification of the univalence axiom.
What about if one adds other homotopically motivated constructions, like higher inductive types?
These questions remain open at the present time, although methods are currently being developed to try to find answers.

Another basic issue is the difficulty of working with types, such as the natural numbers, that are essentially sets (i.e., discrete spaces),
%% \index{discrete!space}%
containing only trivial paths.
At present, homotopy type theory can really only characterize spaces up to homotopy equivalence, which means that these ``discrete spaces'' may only be \emph{homotopy equivalent} to discrete spaces.
Type-theoretically, this means there are many paths that are equal to reflexivity, but not \emph{judgmentally} equal to it (see \cref{sec:types-vs-sets} for the meaning of ``judgmentally'').
While this homotopy-invariance has advantages, these ``meaningless'' identity terms do introduce needless complications into arguments and constructions, so it would be convenient to have a systematic way of eliminating or collapsing them.
% In some cases, the proliferation of such superfluous identity terms makes it very difficult or impossible to formulate what should be a straightforward concept, such as the definition of a (semi-)simplicial type.

A more specialized, but no less important, problem is the relation between homotopy type theory and the research on \emph{higher toposes}%
\index{.infinity1-topos@$(\infty,1)$-topos}
currently happening at the intersection of higher category theory and homotopy theory.
There is a growing conviction among those familiar with both subjects that they are intimately connected.
For instance, the notion of a univalent universe should coincide with that of an object classifier, while higher inductive types should be an ``elementary'' reflection of local presentability.
More generally, homotopy type theory should be the ``internal language'' of $(\infty,1)$-toposes, just as intuitionistic higher-order logic is the internal language of ordinary 1-toposes.
Despite this general consensus, however, details remain to be worked out --- in particular, questions of coherence and strictness remain to be addressed  --- and doing so will undoubtedly lead to further insights into both concepts.

\index{mathematics!formalized}%
But by far the largest field of work to be done is in the ongoing formalization of everyday mathematics in this new system.
Recent successes in formalizing some facts from basic homotopy theory and category theory have been encouraging; some of these are described in \cref{cha:homotopy,cha:category-theory}.
Obviously, however, much work remains to be done.

\index{open!problem|)}%

The homotopy type theory community maintains a web site and group blog at \url{http://homotopytypetheory.org}, as well as a discussion email list.
Newcomers are always welcome!


\subsection{How to read this book}

This book is divided into two parts.
\autoref{part:foundations}, ``Foundations'', develops the fundamental concepts of homotopy type theory.
This is the mathematical foundation on which the development of specific subjects is built, and which is required for the understanding of the univalent foundations approach. To a programmer, this is ``library code''.
Since univalent foundations is a new and different kind of mathematics, its basic notions take some getting used to; thus \autoref{part:foundations} is fairly extensive.

\autoref{part:mathematics}, ``Mathematics'', consists of four chapters that build on the basic notions of \autoref{part:foundations} to exhibit some of the new things we can do with univalent foundations in four different areas of mathematics: homotopy theory (\autoref{cha:homotopy}), category theory (\autoref{cha:category-theory}), set theory (\autoref{cha:set-math}), and real analysis (\autoref{cha:real-numbers}).
The chapters in \autoref{part:mathematics} are more or less independent of each other, although occasionally one will use a lemma proven in another.

A reader who wants to seriously understand univalent foundations, and be able to work in it, will eventually have to read and understand most of \autoref{part:foundations}.
However, a reader who just wants to get a taste of univalent foundations and what it can do may understandably balk at having to work through over 200 pages before getting to the ``meat'' in \autoref{part:mathematics}.
Fortunately, not all of \autoref{part:foundations} is necessary in order to read the chapters in \autoref{part:mathematics}.
Each chapter in \autoref{part:mathematics} begins with a brief overview of its subject, what univalent foundations has to contribute to it, and the necessary background from \autoref{part:foundations}, so the courageous reader can turn immediately to the appropriate chapter for their favorite subject.
For those who want to understand one or more chapters in \autoref{part:mathematics} more deeply than this, but are not ready to read all of \autoref{part:foundations}, we provide here a brief summary of \autoref{part:foundations}, with remarks about which parts are necessary for which chapters in \autoref{part:mathematics}.

\autoref{cha:typetheory} is about the basic notions of type theory, prior to any homotopical interpretation.
A reader who is familiar with Martin-L\"of type theory can quickly skim it to pick up the particulars of the theory we are using.
However, readers without experience in type theory will need to read \autoref{cha:typetheory}, as there are many subtle differences between type theory and other foundations such as set theory.

\autoref{cha:basics} introduces the homotopical viewpoint on type theory, along with the basic notions supporting this view, and describes the homotopical behavior of each component of the type theory from \autoref{cha:typetheory}.
It also introduces the \emph{univalence axiom} (\autoref{sec:compute-universe}) --- the first of the two basic innovations of homotopy type theory.
Thus, it is quite basic and we encourage everyone to read it, especially \crefrange{sec:equality}{sec:basics-equivalences}.

\autoref{cha:logic} describes how we represent logic in homotopy type theory, and its connection to classical logic as well as to constructive and intuitionistic logic.
Here we define the law of excluded middle, the axiom of choice, and the axiom of propositional resizing (although, for the most part, we do not need to assume any of these in the rest of the book), as well as the \emph{propositional truncation} which is essential for representing traditional logic.
This chapter is essential background for \autoref{cha:set-math,cha:real-numbers}, less important for \autoref{cha:category-theory}, and not so necessary for \autoref{cha:homotopy}.

\autoref{cha:equivalences,cha:induction} study two special topics in detail: equivalences (and related notions) and generalized inductive definitions.
While these are important subjects in their own rights and provide a deeper understanding of homotopy type theory, for the most part they are not necessary for \autoref{part:mathematics}.
Only a few lemmas from \autoref{cha:equivalences} are used here and there, while the general discussions in \autoref{sec:bool-nat,sec:strictly-positive,sec:generalizations} are helpful for providing the intuition required for \autoref{cha:hits}.
The generalized sorts of inductive definition discussed in \autoref{sec:generalizations} are also used in a few places in \autoref{cha:set-math,cha:real-numbers}.

\autoref{cha:hits} introduces the second basic innovation of homotopy type theory --- \emph{higher inductive types} --- with many examples.
Higher inductive types are the primary object of study in \autoref{cha:homotopy}, and some particular ones play important roles in \autoref{cha:set-math,cha:real-numbers}.
They are not so necessary for \autoref{cha:category-theory}, although one example is used in \autoref{sec:rezk}.

Finally, \autoref{cha:hlevels} discusses homotopy $n$-types and related notions such as $n$-connected types.
These notions are important for \autoref{cha:homotopy}, but not so important in the rest of \autoref{part:mathematics}, although the case $n=-1$ of some of the lemmas are used in \autoref{sec:piw-pretopos}.

This completes \autoref{part:foundations}.
As mentioned above, \autoref{part:mathematics} consists of four largely unrelated chapters, each describing what univalent foundations has to offer to a particular subject.

Of the chapters in \autoref{part:mathematics}, \autoref{cha:homotopy} (Homotopy theory) is perhaps the most radical.
Univalent foundations has a very different ``synthetic'' approach to homotopy theory in which homotopy types are the basic objects (namely, the types) rather than being constructed using topological spaces or some other set-theoretic model.
This enables new styles of proof for classical theorems in algebraic topology, of which we present a sampling, from $\pi_1(\Sn^1)=\Z$ to the Freudenthal suspension theorem.

In \autoref{cha:category-theory} (Category theory), we develop some basic (1-)category theory, adhering to the principle of the univalence axiom that \emph{equality is isomorphism}.
This has the pleasant effect of ensuring that all definitions and constructions are automatically invariant under equivalence of categories: indeed, equivalent categories are equal just as equivalent types are equal.
(It also has connections to higher category theory and higher topos theory.)

\autoref{cha:set-math} (Set theory) studies sets in univalent foundations.
The category of sets has its usual properties, hence provides a foundation for any mathematics that doesn't need homotopical or higher-categorical structures.
We also observe that univalence makes cardinal and ordinal numbers a bit more pleasant, and that higher inductive types yield a cumulative hierarchy satisfying the usual axioms of Zermelo--Fraenkel set theory.

In \autoref{cha:real-numbers} (Real numbers), we summarize the construction of Dedekind real numbers, and then observe that higher inductive types allow a definition of Cauchy real numbers that avoids some associated problems in constructive mathematics.
Then we sketch a similar approach to Conway's surreal numbers.

Each chapter in this book ends with a Notes section, which collects historical comments, references to the literature, and attributions of results, to the extent possible.
We have also included Exercises at the end of each chapter, to assist the reader in gaining familiarity with doing mathematics in univalent foundations.

Finally, recall that this book was written as a massively collaborative effort by a large number of people.
We have done our best to achieve consistency in terminology and notation, and to put the mathematics in a linear sequence that flows logically, but it is very likely that some imperfections remain.
We ask the reader's forgiveness for any such infelicities, and welcome suggestions for improvement of the next edition.




  \section{Type theory}
\label{subsect:typetheory}

\subsection{Type theory versus set theory}
\label{subsec:types-vs-sets}
\label{subsec:axioms}

\index{type theory}
Homotopy type theory is (among other things) a foundational language for mathematics, i.e., an alternative to Zermelo--Fraenkel\index{set theory!Zermelo--Fraenkel} set theory.
However, it behaves differently from set theory in several important ways, and that can take some getting used to.
Explaining these differences carefully requires us to be more formal here than we will be in the rest of the book.
As stated in the introduction, our goal is to write type theory \emph{informally}; but for a mathematician accustomed to set theory, more precision at the beginning can help avoid some common misconceptions and mistakes.

We note that a set-theoretic foundation has two ``layers'': the deductive system of first-order logic,\index{first-order!logic} and, formulated inside this system, the axioms of a particular theory, such as ZFC.
Thus, set theory is not only about sets, but rather about the interplay between sets (the objects of the second layer) and propositions (the objects of the first layer).

By contrast, type theory is its own deductive system: it need not be formulated inside any superstructure, such as first-order logic.
Instead of the two basic notions of set theory, sets and propositions, type theory has one basic notion: \emph{types}.
Propositions (statements which we can prove, disprove, assume, negate, and so on\footnote{Confusingly, it is also a common practice (dating 
back to Euclid) to use the word ``proposition'' synonymously with ``theorem''.
  We will confine ourselves to the logician's usage, according to which a \emph{proposition} is a statement \emph{susceptible to} proof, whereas a \emph{theorem}\indexfoot{theorem} (or ``lemma''\indexfoot{lemma} or ``corollary''\indexfoot{corollary}) is such a statement that \emph{has been} proven.
Thus ``$0=1$'' and its negation ``$\neg(0=1)$'' are both propositions, but only the latter is a theorem.}) are identified with particular types, via the correspondence shown in \autoref{tab:pov} on page~\pageref{tab:pov}.
Thus, the mathematical activity of \emph{proving a theorem} is identified with a special case of the mathematical activity of \emph{constructing an object}---in this case, an inhabitant of a type that represents a proposition.

\index{deductive system}%
This leads us to another difference between type theory and set theory, but to explain it we must say a little about deductive systems in general.
Informally, a deductive system is a collection of \define{rules}
\indexdef{rule}%
for deriving things called \define{judgments}.
\indexdef{judgment}%
If we think of a deductive system as a formal game,
\index{game!deductive system as}%
then the judgments are the ``positions'' in the game which we reach by following the game rules.
We can also think of a deductive system as a sort of algebraic theory, in which case the judgments are the elements (like the elements of a group) and the deductive rules are the operations (like the group multiplication).
From a logical point of view, the judgments can be considered to be the ``external'' statements, living in the metatheory, as opposed to the ``internal'' statements of the theory itself.

In the deductive system of first-order logic (on which set theory is based), there is only one kind of judgment: that a given proposition has a proof.
That is, each proposition $A$ gives rise to a judgment ``$A$ has a proof'', and all judgments are of this form.
A rule of first-order logic such as ``from $A$ and $B$ infer $A\wedge B$'' is actually a rule of ``proof construction'' which says that given the judgments ``$A$ has a proof'' and ``$B$ has a proof'', we may deduce that ``$A\wedge B$ has a proof''.
Note that the judgment ``$A$ has a proof'' exists at a different level from the \emph{proposition} $A$ itself, which is an internal statement of the theory.
% In particular, we cannot manipulate it to construct propositions such as ``if $A$ has a proof, then $B$ does not have a proof''---unless we are using our set-theoretic foundation as a meta-theory with which to talk about some other axiomatic system.

The basic judgment of type theory, analogous to ``$A$ has a proof'', is written ``$a:A$'' and pronounced as ``the term $a$ has type $A$'', or more loosely ``$a$ is an element of $A$'' (or, in homotopy type theory, ``$a$ is a point of $A$'').
\indexdef{term}%
\indexdef{element}%
\indexdef{point!of a type}%
When $A$ is a type representing a proposition, then $a$ may be called a \emph{witness}\index{witness!to the truth of a proposition} to the provability of $A$, or \emph{evidence}\index{evidence, of the truth of a proposition} of the truth of $A$ (or even a \emph{proof}\index{proof} of $A$, but we will try to avoid this confusing terminology).
In this case, the judgment $a:A$ is derivable in type theory (for some $a$) precisely when the analogous judgment ``$A$ has a proof'' is derivable in first-order logic (modulo differences in the axioms assumed and in the encoding of mathematics, as we will discuss throughout the book).
 
On the other hand, if the type $A$ is being treated more like a set than like a proposition (although as we will see, the distinction can become blurry), then ``$a:A$'' may be regarded as analogous to the set-theoretic statement ``$a\in A$''.
However, there is an essential difference in that ``$a:A$'' is a \emph{judgment} whereas ``$a\in A$'' is a \emph{proposition}.
In particular, when working internally in type theory, we cannot make statements such as ``if $a:A$ then it is not the case that $b:B$'', nor can we ``disprove'' the judgment ``$a:A$''.

A good way to think about this is that in set theory, ``membership'' is a relation which may or may not hold between two pre-existing objects ``$a$'' and ``$A$'', while in type theory we cannot talk about an element ``$a$'' in isolation: every element \emph{by its very nature} is an element of some type, and that type is (generally speaking) uniquely determined.
Thus, when we say informally ``let $x$ be a natural number'', in set theory this is shorthand for ``let $x$ be a thing and assume that $x\in\nat$'', whereas in type theory ``let $x:\nat$'' is an atomic statement: we cannot introduce a variable without specifying its type.\index{membership}


At first glance, this may seem an uncomfortable restriction, but it is arguably closer to the intuitive mathematical meaning of ``let $x$ be a natural number''.
In practice, it seems that whenever we actually \emph{need} ``$a\in A$'' to be a proposition rather than a judgment, there is always an ambient set $B$ of which $a$ is known to be an element and $A$ is known to be a subset.
This situation is also easy to represent in type theory, by taking $a$ to be an element of the type $B$, and $A$ to be a predicate on $B$; see \autoref{subsec:prop-subsets}.

A last difference between type theory and set theory is the treatment of equality.
The familiar notion of equality in mathematics is a proposition: e.g.\ we can disprove an equality or assume an equality as a hypothesis.
Since in type theory, propositions are types, this means that equality is a type: for elements $a,b:A$ (that is, both $a:A$ and $b:A$) we have a type ``$\id[A]ab$''.
(In \emph{homotopy} type theory, of course, this equality proposition can behave in unfamiliar ways: see \autoref{sec:identity-types,cha:basics}, and the rest of the book).
When $\id[A]ab$ is inhabited, we say that $a$ and $b$ are \define{(propositionally) equal}.
\index{propositional!equality}%
\index{equality!propositional}%

However, in type theory there is also a need for an equality \emph{judgment}, existing at the same level as the judgment ``$x:A$''.\index{judgment}
\symlabel{defn:judgmental-equality}%
This is called \define{judgmental equality}
\indexdef{equality!judgmental}%
\indexdef{judgmental equality}%
or \define{definitional equality},
\indexdef{equality!definitional}%
\indexsee{definitional equality}{equality, definitional}%
and we write it as $a\jdeq b : A$ or simply $a \jdeq b$.
It is helpful to think of this as meaning ``equal by definition''.
For instance, if we define a function $f:\nat\to\nat$ by the equation $f(x)=x^2$, then the expression $f(3)$ is equal to $3^2$ \emph{by definition}.
Inside the theory, it does not make sense to negate or assume an equality-by-definition; we cannot say ``if $x$ is equal to $y$ by definition, then $z$ is not equal to $w$ by definition''.
Whether or not two expressions are equal by definition is just a matter of expanding out the definitions; in particular, it is algorithmically\index{algorithm} decidable (though the algorithm is necessarily meta-theoretic, not internal to the theory).\index{decidable!definitional equality}

As type theory becomes more complicated, judgmental equality can get more subtle than this, but it is a good intuition to start from.
Alternatively, if we regard a deductive system as an algebraic theory, then judgmental equality is simply the equality in that theory, analogous to the equality between elements of a group---the only potential for confusion is that there is \emph{also} an object \emph{inside} the deductive system of type theory (namely the type ``$a=b$'') which behaves internally as a notion of ``equality''.

The reason we \emph{want} a judgmental notion of equality is so that it can control the other form of judgment, ``$a:A$''.
For instance, suppose we have given a proof that $3^2=9$, i.e.\ we have derived the judgment $p:(3^2=9)$ for some $p$.
Then the same witness $p$ ought to count as a proof that $f(3)=9$, since $f(3)$ is $3^2$ \emph{by definition}.
The best way to represent this is with a rule saying that given the judgments $a:A$ and $A\jdeq B$, we may derive the judgment $a:B$.

Thus, for us, type theory will be a deductive system based on two forms of judgment:
\begin{center}
\medskip
\begin{tabular}{cl}
  \toprule
  Judgment & Meaning\\
  \midrule
  $a : A$       & ``$a$ is an object of type $A$''\\
  $a \jdeq b : A$ & ``$a$ and $b$ are definitionally equal objects of type $A$''\\
  \bottomrule
\end{tabular}
\medskip
\end{center}
%
\symlabel{defn:defeq}%
When introducing a definitional equality, i.e., defining one thing to be equal to another, we will use the symbol ``$\defeq$''.
Thus, the above definition of the function $f$ would be written as $f(x)\defeq x^2$.

Because judgments cannot be put together into more complicated statements, the symbols ``$:$'' and ``$\jdeq$'' bind more loosely than anything else.%
\footnote{In formalized\indexfoot{mathematics!formalized} type theory, commas and turnstiles can bind even more loosely.
  For instance, $x:A,y:B\vdash c:C$ is parsed as $((x:A),(y:B))\vdash (c:C)$.
  However, in this book we refrain from such notation until \autoref{cha:rules}.}
Thus, for instance, ``$p:\id{x}{y}$'' should be parsed as ``$p:(\id{x}{y})$'', which makes sense since ``$\id{x}{y}$'' is a type, and not as ``$\id{(p:x)}{y}$'', which is senseless since ``$p:x$'' is a judgment and cannot be equal to anything.
Similarly, ``$A\jdeq \id{x}{y}$'' can only be parsed as ``$A\jdeq(\id{x}{y})$'', although in extreme cases such as this, one ought to add parentheses anyway to aid reading comprehension.
Moreover, later on we will fall into the common notation of chaining together equalities --- e.g.\ writing $a=b=c=d$ to mean ``$a=b$ and $b=c$ and $c=d$, hence $a=d$'' --- and we will also include judgmental equalities in such chains.
Context usually suffices to make the intent clear.

This is perhaps also an appropriate place to mention that the common mathematical notation ``$f:A\to B$'', expressing the fact that $f$ is a function from $A$ to $B$, can be regarded as a typing judgment, since we use ``$A\to B$'' as notation for the type of functions from $A$ to $B$ (as is standard practice in type theory; see \autoref{sec:pi-types}).

\index{assumption|(defstyle}%
Judgments may depend on \emph{assumptions} of the form $x:A$, where $x$ is a variable
\indexdef{variable}%
and $A$ is a type.
For example, we may construct an object $m + n : \nat$ under the assumptions that $m,n : \nat$.
Another example is that assuming $A$ is a type, $x,y : A$, and $p : \id[A]{x}{y}$, we may construct an element $p^{-1} : \id[A]{y}{x}$.
The collection of all such assumptions is called the \define{context};%
\index{context}
from a topological point of view it may be thought of as a ``parameter\index{parameter!space} space''.
In fact, technically the context must be an ordered list of assumptions, since later assumptions may depend on previous ones: the assumption $x:A$ can only be made \emph{after} the assumptions of any variables appearing in the type $A$.

If the type $A$ in an assumption $x:A$ represents a proposition, then the assumption is a type-theoretic version of a \emph{hypothesis}:
\indexdef{hypothesis}%
we assume that the proposition $A$ holds.
When types are regarded as propositions, we may omit the names of their proofs.
Thus, in the second example above we may instead say that assuming $\id[A]{x}{y}$, we can prove $\id[A]{y}{x}$.
However, since we are doing ``proof-relevant'' mathematics,
\index{mathematics!proof-relevant}%
we will frequently refer back to proofs as objects.
In the example above, for instance, we may want to establish that $p^{-1}$ together with the proofs of transitivity and reflexivity behave like a groupoid; see \autoref{cha:basics}.

Note that under this meaning of the word \emph{assumption}, we can assume a propositional equality (by assuming a variable $p:x=y$), but we cannot assume a judgmental equality $x\jdeq y$, since it is not a type that can have an element.
However, we can do something else which looks kind of like assuming a judgmental equality: if we have a type or an element which involves a variable $x:A$, then we can \emph{substitute} any particular element $a:A$ for $x$ to obtain a more specific type or element.
We will sometimes use language like ``now assume $x\jdeq a$'' to refer to this process of substitution, even though it is not an \emph{assumption} in the technical sense introduced above.
\index{assumption|)}%

By the same token, we cannot \emph{prove} a judgmental equality either, since it is not a type in which we can exhibit a witness.
Nevertheless, we will sometimes state judgmental equalities as part of a theorem, e.g.\ ``there exists $f:A\to B$ such that $f(x)\jdeq y$''.
This should be regarded as the making of two separate judgments: first we make the judgment $f:A\to B$ for some element $f$, then we make the additional judgment that $f(x)\jdeq y$.

In the rest of this chapter, we attempt to give an informal presentation of type theory, sufficient for the purposes of this book; we give a more formal account in \autoref{cha:rules}.
Aside from some fairly obvious rules (such as the fact that judgmentally equal things can always be substituted\index{substitution} for each other), the rules of type theory can be grouped into \emph{type formers}.
Each type former consists of a way to construct types (possibly making use of previously constructed types), together with rules for the construction and behavior of elements of that type.
In most cases, these rules follow a fairly predictable pattern, but we will not attempt to make this precise here; see however the beginning of \autoref{sec:finite-product-types} and also \autoref{cha:induction}.\index{type theory!informal}


\index{axiom!versus rules}%
\index{rule!versus axioms}%
An important aspect of the type theory presented in this chapter is that it consists entirely of \emph{rules}, without any \emph{axioms}.
In the description of deductive systems in terms of judgments, the \emph{rules} are what allow us to conclude one judgment from a collection of others, while the \emph{axioms} are the judgments we are given at the outset.
If we think of a deductive system as a formal game, then the rules are the rules of the game, while the axioms are the starting position.
And if we think of a deductive system as an algebraic theory, then the rules are the operations of the theory, while the axioms are the \emph{generators} for some particular free model of that theory.

In set theory, the only rules are the rules of first-order logic (such as the rule allowing us to deduce ``$A\wedge B$ has a proof'' from ``$A$ has a proof'' and ``$B$ has a proof''): all the information about the behavior of sets is contained in the axioms.
By contrast, in type theory, it is usually the \emph{rules} which contain all the information, with no axioms being necessary.
For instance, in \autoref{sec:finite-product-types} we will see that there is a rule allowing us to deduce the judgment ``$(a,b):A\times B$'' from ``$a:A$'' and ``$b:B$'', whereas in set theory the analogous statement would be (a consequence of) the pairing axiom.

The advantage of formulating type theory using only rules is that rules are ``procedural''.
In particular, this property is what makes possible (though it does not automatically ensure) the good computational properties of type theory, such as ``canonicity''.\index{canonicity}
However, while this style works for traditional type theories, we do not yet understand how to formulate everything we need for \emph{homotopy} type theory in this way.
In particular, in \autoref{sec:compute-pi,sec:compute-universe,cha:hits} we will have to augment the rules of type theory presented in this chapter by introducing additional axioms, notably the \emph{univalence axiom}.
In this chapter, however, we confine ourselves to a traditional rule-based type theory.


\subsection{Function types}
\label{subsec:function-types}

\index{type!function|(defstyle}%
\indexsee{function type}{type, function}%
Given types $A$ and $B$, we can construct the type $A \to B$ of \define{functions}
\index{function|(defstyle}%
\indexsee{map}{function}%
\indexsee{mapping}{function}%
with domain $A$ and codomain $B$.
We also sometimes refer to functions as \define{maps}.
\index{domain!of a function}%
\index{codomain, of a function}%
\index{function!domain of}%
\index{function!codomain of}%
\index{functional relation}%
Unlike in set theory, functions are not defined as
functional relations; rather they are a primitive concept in type theory.
We explain the function type by prescribing what we can do with functions, 
how to construct them and what equalities they induce.

Given a function $f : A \to B$ and an element of the domain $a : A$, we
can \define{apply}
\indexdef{application!of function}%
\indexdef{function!application}%
\indexsee{evaluation}{application, of a function}
the function to obtain an element of the codomain $B$,
denoted $f(a)$ and called the \define{value} of $f$ at $a$.
\indexdef{value!of a function}%
It is common in type theory to omit the parentheses\index{parentheses} and denote $f(a)$ simply by $f\,a$, and we will sometimes do this as well.

But how can we construct elements of $A \to B$? There are two equivalent ways:
either by direct definition or by using
$\lambda$-abstraction. Introducing a function by definition
\indexdef{definition!of function, direct}%
means that
we introduce a function by giving it a name --- let's say, $f$ --- and saying
we define $f : A \to B$ by giving an equation
\begin{equation}
  \label{eq:expldef}
  f(x) \defeq \Phi
\end{equation}
where $x$ is a variable
\index{variable}%
and $\Phi$ is an expression which may use $x$.
In order for this to be valid, we have to check that $\Phi : B$ assuming $x:A$.

Now we can compute $f(a)$ by replacing the variable $x$ in $\Phi$ with
$a$. As an example, consider the function $f : \nat \to \nat$ which is
defined by $f(x) \defeq x+x$.  (We will define $\nat$ and $+$ in \autoref{sec:inductive-types}.)
Then $f(2)$ is judgmentally equal to $2+2$.

If we don't want to introduce a name for the function, we can use
\define{$\lambda$-abstraction}.
\index{lambda abstraction@$\lambda$-abstraction|defstyle}%
\indexsee{function!lambda abstraction@$\lambda$-abstraction}{$\lambda$-abstraction}%
\indexsee{abstraction!lambda-@$\lambda$-}{$\lambda$-abstraction}%
Given an expression $\Phi$ of type $B$ which may use $x:A$, as above, we write $\lam{x:A} \Phi$ to indicate the same function defined by~\eqref{eq:expldef}.
Thus, we have
\[ (\lamt{x:A}\Phi) : A \to B. \]
For the example in the previous paragraph, we have the typing judgment
\[ (\lam{x:\nat}x+x) : \nat \to \nat. \]
As another example, for any types $A$ and $B$ and any element $y:B$, we have a \define{constant function}
\indexdef{constant!function}%
\indexdef{function!constant}%
$(\lam{x:A} y): A\to B$.

We generally omit the type of the variable $x$ in a $\lambda$-abstraction and write $\lam{x}\Phi$, since the typing $x:A$ is inferable from the judgment that the function $\lam x \Phi$ has type $A\to B$.
By convention, the ``scope''
\indexdef{variable!scope of}%
\indexdef{scope}%
of the variable binding ``$\lam{x}$'' is the entire rest of the expression, unless delimited with parentheses\index{parentheses}.
Thus, for instance, $\lam{x} x+x$ should be parsed as $\lam{x} (x+x)$, not as $(\lam{x}x)+x$ (which would, in this case, be ill-typed anyway).

Another equivalent notation is
\symlabel{mapsto}%
\[ (x \mapsto \Phi) : A \to B. \]
\symlabel{blank}%
We may also sometimes use a blank ``$\blank$'' in the expression $\Phi$ in place of a variable, to denote an implicit $\lambda$-abstraction.
For instance, $g(x,\blank)$ is another way to write $\lam{y} g(x,y)$.

Now a $\lambda$-abstraction is a function, so we can apply it to an argument $a:A$.
We then have the following \define{computation rule}\indexdef{computation rule!for function types}\footnote{Use of this equality is often referred to as \define{$\beta$-conversion}
\indexsee{beta-conversion@$\beta $-conversion}{$\beta$-reduction}%
\indexsee{conversion!beta@$\beta $-}{$\beta$-reduction}%
or \define{$\beta$-reduction}.%
\index{beta-reduction@$\beta $-reduction|footstyle}%
\indexsee{reduction!beta@$\beta $-}{$\beta$-reduction}%
}, which is a definitional equality:
\[(\lamu{x:A}\Phi)(a) \jdeq \Phi'\]
where $\Phi'$ is the
expression $\Phi$ in which all occurrences of $x$ have been replaced by $a$.
Continuing the above example, we have
%
\[ (\lamu{x:\nat}x+x)(2) \jdeq 2+2. \]
%
Note that from any function $f:A\to B$, we can construct a lambda abstraction function $\lam{x} f(x)$.
Since this is by definition ``the function that applies $f$ to its argument'' we consider it to be definitionally equal to $f$:\footnote{Use of this equality is often referred to as \define{$\eta$-conversion}
\indexsee{eta-conversion@$\eta $-conversion}{$\eta$-expansion}%
\indexsee{conversion!eta@$\eta $-}{$\eta$-expansion}%
or \define{$\eta$-expansion.
\index{eta-expansion@$\eta $-expansion|footstyle}%
\indexsee{expansion, eta-@expansion, $\eta $-}{$\eta$-expansion}%
}}
\[ f \jdeq (\lam{x} f(x)). \]
This equality is the \define{uniqueness principle for function types}\indexdef{uniqueness!principle!for function types}, because it shows that $f$ is uniquely determined by its values.

The introduction of functions by definitions with explicit parameters can be reduced
to simple definitions by using $\lambda$-abstraction: i.e., we can read 
a definition of $f: A\to B$ by
\[ f(x) \defeq \Phi \]
as 
\[ f \defeq \lamu{x:A}\Phi.\]

When doing calculations involving variables, we have to be 
careful when replacing a variable with an expression that also involves
variables, because we want to preserve the binding structure of
expressions. By the \emph{binding structure}\indexdef{binding structure} we mean the
invisible link generated by binders such as $\lambda$, $\Pi$ and
$\Sigma$ (the latter we are going to meet soon) between the place where the variable is introduced and where it is used. As an example, consider $f : \nat \to (\nat \to \nat)$
defined as 
\[ f(x) \defeq \lamu{y:\nat} x + y. \] 
Now if we have assumed somewhere that $y : \nat$, then what is $f(y)$? It would be wrong to just naively replace $x$ by $y$ everywhere in the expression ``$\lam{y}x+y$'' defining $f(x)$, obtaining $\lamu{y:\nat} y + y$, because this means that $y$ gets \define{captured}.
\indexdef{capture, of a variable}%
\indexdef{variable!captured}%
Previously, the substituted\index{substitution} $y$ was referring to our assumption, but now it is referring to the argument of the $\lambda$-abstraction. Hence, this naive substitution would destroy the binding structure, allowing us to perform calculations which are semantically unsound.

But what \emph{is} $f(y)$ in this example? Note that bound (or ``dummy'')
variables
\indexdef{variable!bound}%
\indexdef{variable!dummy}%
\indexsee{bound variable}{variable, bound}%
\indexsee{dummy variable}{variable, bound}%
such as $y$ in the expression $\lamu{y:\nat} x + y$
have only a local meaning, and can be consistently replaced by any
other variable, preserving the binding structure. Indeed, $\lamu{y:\nat} x + y$ is declared to be judgmentally equal\footnote{Use of this equality is often referred to as \define{$\alpha$-conversion.
\indexfoot{alpha-conversion@$\alpha $-conversion}
\indexsee{conversion!alpha@$\alpha$-}{$\alpha$-conversion}
}} to
$\lamu{z:\nat} x + z$.  It follows that 
$f(y)$ is judgmentally equal to  $\lamu{z:\nat} y + z$, and that answers our question.  (Instead of $z$,
any variable distinct from $y$ could have been used, yielding an equal result.)

Of course, this should all be familiar to any mathematician: it is the same phenomenon as the fact that if $f(x) \defeq \int_1^2 \frac{dt}{x-t}$, then $f(t)$ is not $\int_1^2 \frac{dt}{t-t}$ but rather $\int_1^2 \frac{ds}{t-s}$.
A $\lambda$-abstraction binds a dummy variable in exactly the same way that an integral does.

We have seen how to define functions in one variable. One
way to define functions in several variables would be to use the
cartesian product, which will be introduced later; a function with
parameters $A$ and $B$ and results in $C$ would be given the type 
$f : A \times B \to C$. However, there is another choice that avoids
using product types, which is called \define{currying}
\indexdef{currying}%
\indexdef{function!currying of}%
(after the mathematician Haskell Curry).
\index{programming}%

The idea of currying is to represent a function of two inputs $a:A$ and $b:B$ as a function which takes \emph{one} input $a:A$ and returns \emph{another function}, which then takes a second input $b:B$ and returns the result.
That is, we consider two-variable functions to belong to an iterated function type, $f : A \to (B \to C)$.
We may also write this without the parentheses\index{parentheses}, as $f : A \to B \to C$, with
associativity\index{associativity!of function types} to the right as the default convention.  Then given $a : A$ and $b : B$,
we can apply $f$ to $a$ and then apply the result to $b$, obtaining
$f(a)(b) : C$. To avoid the proliferation of parentheses, we allow ourselves to
write $f(a)(b)$ as $f(a,b)$ even though there are no products
involved.
When omitting parentheses around function arguments entirely, we write $f\,a\,b$ for $(f\,a)\,b$, with the default associativity now being to the left so that $f$ is applied to its arguments in the correct order.

Our notation for definitions with explicit parameters extends to
this situation: we can define a named function $f : A \to B \to C$ by
giving an equation
\[ f(x,y) \defeq \Phi\]
where $\Phi:C$ assuming $x:A$ and $y:B$. Using $\lambda$-abstraction\index{lambda abstraction@$\lambda$-abstraction} this
corresponds to
\[ f \defeq \lamu{x:A}{y:B} \Phi, \]
which may also be written as 
\[ f \defeq x \mapsto y \mapsto \Phi. \] 
We can also implicitly abstract over multiple variables by writing multiple blanks, e.g.\ $g(\blank,\blank)$ means $\lam{x}{y} g(x,y)$.
Currying a function of three or more arguments is a straightforward extension of what we have just described.
 
\index{type!function|)}%
\index{function|)}%


\subsection{Universes and families}
\label{subsec:universes}

So far, we have been using the expression ``$A$ is a type'' informally. We
are going to make this more precise by introducing \define{universes}.
\index{type!universe|(defstyle}%
\indexsee{universe}{type, universe}%
A universe is a type whose elements are types. As in naive set theory,
we might wish for a universe of all types $\UU_\infty$ including itself
(that is, with $\UU_\infty : \UU_\infty$).
However, as in set
theory, this is unsound, i.e.\ we can deduce from it that every type,
including the empty type representing the proposition False (see \autoref{sec:coproduct-types}), is inhabited.
For instance, using a
representation of sets as trees, we can directly encode Russell's
paradox\index{paradox} \citep{coquand:paradox}.
%  or alternatively, in order to avoid the use of
% inductive types to define trees, we can follow Girard \citep{girard:paradox} and encode the Burali-Forti paradox,
% which shows that the collection of all ordinals cannot be an ordinal.

To avoid the paradox we introduce a hierarchy of universes
\indexsee{hierarchy!of universes}{type, universe}%
\[ \UU_0 : \UU_1 : \UU_2 : \cdots \]
where every universe $\UU_i$ is an element of the next universe
$\UU_{i+1}$. Moreover, we assume that our universes are
\define{cumulative},
\indexdef{type!universe!cumulative}%
\indexdef{cumulative!universes}%
that is that all the elements of the $i^{\mathrm{th}}$
universe are also elements of the $(i+1)^{\mathrm{st}}$ universe, i.e.\ if
$A:\UU_i$ then also $A:\UU_{i+1}$.
This is convenient, but has the slightly unpleasant consequence that elements no longer have unique types, and is a bit tricky in other ways that need not concern us here; see the Notes.

When we say that $A$ is a type, we mean that it inhabits some universe
$\UU_i$. We usually want to avoid mentioning the level
\indexdef{universe level}%
\indexsee{level}{universe level or $n$-type}%
\indexsee{type!universe!level}{universe level}%
$i$ explicitly,
and just assume that levels can be assigned in a consistent way; thus we
may write $A:\UU$ omitting the level. This way we can even write
$\UU:\UU$, which can be read as $\UU_i:\UU_{i+1}$, having left the
indices implicit.  Writing universes in this style is referred to as
\define{typical ambiguity}.
\indexdef{typical ambiguity}%
It is convenient but a bit dangerous, since it allows us to write valid-looking proofs that reproduce the paradoxes of self-reference.
If there is any doubt about whether an argument is correct, the way to check it is to try to assign levels consistently to all universes appearing in it.
When some universe \UU is assumed, we may refer to types belonging to \UU as \define{small types}.
\indexdef{small!type}%
\indexdef{type!small}%

To model a collection of types varying over a given type $A$, we use functions $B : A \to \UU$  whose
codomain is a universe. These functions are called
\define{families of types} (or sometimes \emph{dependent types});
\indexsee{family!of types}{type, family of}%
\indexdef{type!family of}%
\indexsee{type!dependent}{type, family of}%
\indexsee{dependent!type}{type, family of}%
they correspond to families of sets as used in
set theory.

\symlabel{fin}%
An example of a type family is the family of finite sets $\Fin
: \nat \to \UU$, where $\Fin(n)$ is a type with exactly $n$ elements.
(We cannot \emph{define} the family $\Fin$ yet --- indeed, we have not even introduced its domain $\nat$ yet --- but we will be able to soon; see \autoref{ex:fin}.)
We may denote the elements of $\Fin(n)$ by $0_n,1_n,\dots,(n-1)_n$, with subscripts to emphasize that the elements of $\Fin(n)$ are different from those of $\Fin(m)$ if $n$ is different from $m$, and all are different from the ordinary natural numbers (which we will introduce in \autoref{sec:inductive-types}).
\index{finite!sets, family of}%

A more trivial (but very important) example of a type family is the \define{constant} type family
\indexdef{constant!type family}%
\indexdef{type!family of!constant}%
at a type $B:\UU$, which is of course the constant function $(\lam{x:A} B):A\to\UU$.

As a \emph{non}-example, in our version of type theory there is no type family ``$\lam{i:\nat} \UU_i$''.
Indeed, there is no universe large enough to be its codomain.
Moreover, we do not even identify the indices $i$ of the universes $\UU_i$ with the natural numbers $\nat$ of type theory (the latter to be introduced in \autoref{sec:inductive-types}).

\index{type!universe|)}%

\subsection{Dependent function types (\texorpdfstring{$\Pi$}{Π}-types)}
\label{sec:pi-types}

\index{type!dependent function|(defstyle}%
\index{function!dependent|(defstyle}%
\indexsee{dependent!function}{function, dependent}%
\indexsee{type!Pi-@$\Pi$-}{type, dependent function}%
\indexsee{Pi-type@$\Pi$-type}{type, dependent function}%
In type theory we often use a more general version of function
types, called a \define{$\Pi$-type} or \define{dependent function type}. The elements of
a $\Pi$-type are functions
whose codomain type can vary depending on the
element of the domain to which the function is applied, called \define{dependent functions}. The name ``$\Pi$-type''
is used because this type can also be regarded as the  cartesian
product over a given type.

Given a type $A:\UU$ and a family $B:A \to \UU$, we may construct
the type of dependent functions $\prd{x:A}B(x) : \UU$.
There are many alternative notations for this type, such as
\[ \tprd{x:A} B(x) \qquad \dprd{x:A}B(x) \qquad \lprd{x:A} B(x). \]
If $B$ is a constant family, then the dependent product type is the ordinary function type:
\[\tprd{x:A} B \jdeq (A \to B).\]
Indeed, all the constructions of $\Pi$-types are generalizations of the corresponding constructions on ordinary function types.

\indexdef{definition!of function, direct}%
We can introduce dependent functions by explicit definitions: to
define $f : \prd{x:A}B(x)$, where $f$ is the name of a dependent function to be
defined, we need an expression $\Phi : B(x)$ possibly involving the variable $x:A$,
\index{variable}%
and we write
\[ f(x) \defeq \Phi \qquad \mbox{for $x:A$}.\]
Alternatively, we can use \define{$\lambda$-abstraction}%
\index{lambda abstraction@$\lambda$-abstraction|defstyle}%
\begin{equation}
  \label{eq:lambda-abstraction}
  \lamu{x:A} \Phi \ :\ \prd{x:A} B(x).
\end{equation}
\indexdef{application!of dependent function}%
\indexdef{function!dependent!application}%
As with non-dependent functions, we can \define{apply} a dependent function $f : \prd{x:A}B(x)$ to an argument $a:A$ to obtain an element $f(a):B(a)$.
The equalities are the same as for the ordinary function type, i.e.\ we have the computation rule
\index{computation rule!for dependent function types}%
given $a:A$ we have $f(a) \jdeq \Phi'$ and  
$(\lamu{x:A} \Phi)(a) \jdeq \Phi'$, where $\Phi' $ is obtained by replacing all
occurrences of $x$ in $\Phi$ by $a$ (avoiding variable capture, as always).
Similarly, we have the uniqueness principle $f\jdeq (\lam{x} f(x))$ for any $f:\prd{x:A} B(x)$.
\index{uniqueness!principle!for dependent function types}%

As an example, recall from \autoref{sec:universes} that there is a type family $\Fin:\nat\to\UU$ whose values are the standard finite sets, with elements $0_n,1_n,\dots,(n-1)_n : \Fin(n)$.
There is then a dependent function $\fmax : \prd{n:\nat} \Fin(n+1)$
which returns the ``largest'' element of each nonempty finite type, $\fmax(n) \defeq n_{n+1}$.
\index{finite!sets, family of}%
As was the case for $\Fin$ itself, we cannot define $\fmax$ yet, but we will be able to soon; see \autoref{ex:fin}.

Another important class of dependent function types, which we can define now, are functions which are \define{polymorphic}
\indexdef{function!polymorphic}%
\indexdef{polymorphic function}%
over a given universe.
A polymorphic function is one which takes a type as one of its arguments, and then acts on elements of that type (or other types constructed from it).
\symlabel{idfunc}%
\indexdef{function!identity}%
\indexdef{identity!function}%
An example is the polymorphic identity function $\idfunc : \prd{A:\UU} A \to A$, which we define by $\idfunc{} \defeq \lam{A:\type}{x:A} x$.

We sometimes write some arguments of a dependent function as subscripts.
For instance, we might equivalently define the polymorphic identity function by $\idfunc[A](x) \defeq x$.
Moreover, if an argument can be inferred from context, we may omit it altogether.
For instance, if $a:A$, then writing $\idfunc(a)$ is unambiguous, since $\idfunc$ must mean $\idfunc[A]$ in order for it to be applicable to $a$.

Another, less trivial, example of a polymorphic function is the ``swap'' operation that switches the order of the arguments of a (curried) two-argument function:
\[ \mathsf{swap} : \prd{A:\UU}{B:\UU}{C:\UU} (A\to B\to C) \to (B\to A \to C)
\]
We can define this by
\[ \mathsf{swap}(A,B,C,g) \defeq \lam{b}{a} g(a)(b). \]
We might also equivalently write the type arguments as subscripts:
\[ \mathsf{swap}_{A,B,C}(g)(b,a) \defeq g(a,b). \]

Note that as we did for ordinary functions, we use currying to define dependent functions with
several arguments (such as $\mathsf{swap}$). However, in the dependent case the second domain may
depend on the first one, and the codomain may depend on both. That is,
given $A:\UU$ and type families $B : A \to \UU$ and $C : \prd{x:A}B(x) \to \UU$, we may construct
the type $\prd{x:A}{y : B(x)} C(x,y)$ of functions with two
arguments.
(Like $\lambda$-abstractions, $\Pi$s automatically scope\index{scope} over the rest of the expression unless delimited; thus $C : \prd{x:A}B(x) \to \UU$ means $C : \prd{x:A}(B(x) \to \UU)$.)
In the case when $B$ is constant and equal to $A$, we may condense the notation and write $\prd{x,y:A}$; for instance, the type of $\mathsf{swap}$ could also be written as
\[ \mathsf{swap} : \prd{A,B,C:\UU} (A\to B\to C) \to (B\to A \to C).
\]
Finally, given $f:\prd{x:A}{y : B(x)} C(x,y)$ and arguments $a:A$ and $b:B(a)$, we have $f(a)(b) : C(a,b)$, which,
as before, we write as $f(a,b) : C(a,b)$.

\index{type!dependent function|)}%
\index{function!dependent|)}%


\subsection{Product types}
\label{subsec:finite-product-types}

Given types $A,B:\UU$ we introduce the type $A\times B:\UU$, which we call their \define{cartesian product}.
\indexsee{cartesian product}{type, product}%
\indexsee{type!cartesian product}{type, product}%
\index{type!product|(defstyle}%
\indexsee{product!of types}{type, product}%
We also introduce a nullary product type, called the \define{unit type} $\unit : \UU$.
\indexsee{nullary!product}{type, unit}%
\indexsee{unit!type}{type, unit}%
\index{type!unit|(defstyle}%
We intend the elements of $A\times B$ to be pairs $\tup{a}{b} : A \times B$, where $a:A$ and $b:B$, and the only element of $\unit{}$ to be some particular object $\ttt : \unit{}$.
\indexdef{pair!ordered}%
However, unlike in set theory, where we define ordered pairs to be particular sets and then collect them all together into the cartesian product, in type theory, ordered pairs are a primitive concept, as are functions.   

\begin{rmk}\label{rmk:introducing-new-concepts}
  There is a general pattern for introduction of a new kind of type in type theory, and because products are our second example following this pattern,\footnote{The description of universes above is an exception.} it is worth emphasizing the general form:
  To specify a type, we specify:
  \begin{enumerate}
  \item how to form new types of this kind, via \define{formation rules}.
    \indexdef{formation rule}%
    \index{rule!formation}%  
(For example, we can form the function type $A \to B$ when $A$ is a type and when $B$ is a type. We can form the dependent function type $\prd{x:A} B(x)$ when $A$ is a type and $B(x)$ is a type for $x:A$.)

  \item how to construct elements of that type.  
    These are called the type's \define{constructors} or \define{introduction rules}.
    \indexdef{constructor}%
    \indexdef{rule!introduction}%
    \indexdef{introduction rule}%
    (For example, a function type has one constructor, $\lambda$-abstraction.
    Recall that a direct definition like $f(x)\defeq 2x$ can equivalently be phrased
    as a $\lambda$-abstraction $f\defeq \lam{x} 2x$.)

  \item how to use elements of that type.  
    These are called the type's \define{eliminators} or \define{elimination rules}.
    \indexsee{rule!elimination}{eliminator}%
    \indexsee{elimination rule}{eliminator}%
    \indexdef{eliminator}%
    (For example, the function type has one eliminator, namely function application.)

  \item 
    a \define{computation rule}\indexdef{computation rule}\footnote{also referred to as \define{$\beta$-reduction}\index{beta-reduction@$\beta $-reduction|footstyle}}, which expresses how an eliminator acts on a constructor.
(For example, for functions, the computation rule states that $(\lamu{x:A}\Phi)(a)$ is judgmentally equal to the substitution of $a$ for $x$ in $\Phi$.)

  \item 
    an optional \define{uniqueness principle}\indexdef{uniqueness!principle}\footnote{also referred to as \define{$\eta$-expansion}\index{eta-expansion@$\eta $-expansion|footstyle}}, which expresses
uniqueness of maps into or out of that type.  
For some types, the uniqueness principle characterizes maps into the type, by stating that 
every element of the type is uniquely determined by the results of applying eliminators to it, and can be reconstructed from those results by applying a constructor---thus expressing how constructors act on eliminators, dually to the computation rule.  
(For example, for functions, the uniqueness principle says that any function $f$ is judgmentally equal to the ``expanded'' function $\lamu{x} f(x)$, and thus is uniquely determined by its values.)
For other types, the uniqueness principle says that every map (function) \emph{from} that type is uniquely determined by some data. (An example is the coproduct type introduced in \cref{sec:coproduct-types}, whose uniqueness principle is mentioned in \cref{sec:universal-properties}.)  
    
    When the uniqueness principle is not taken as a rule of judgmental equality, it is often nevertheless provable as a \emph{propositional} equality from the other rules for the type.
    In this case we call it a \define{propositional uniqueness principle}.
    \indexdef{uniqueness!principle, propositional}%
    \indexsee{propositional!uniqueness principle}{uniqueness principle, propositional}%
    (In later chapters we will also occasionally encounter \emph{propositional computation rules}.)
    \indexdef{computation rule!propositional}%
  \end{enumerate}
The inference rules in \autoref{sec:syntax-more-formally} are organized and named accordingly; see, for example, \autoref{sec:more-formal-pi}, where each possibility is realized.
\end{rmk}

The way to construct pairs is obvious: given $a:A$ and $b:B$, we may form $(a,b):A\times B$.
Similarly, there is a unique way to construct elements of $\unit{}$, namely we have $\ttt:\unit{}$.
We expect that ``every element of $A\times B$ is a pair'', which is the uniqueness principle for products; we do not assert this as a rule of type theory, but we will prove it later on as a propositional equality.

Now, how can we \emph{use} pairs, i.e.\ how can we define functions out of a product type?
Let us first consider the definition of a non-dependent function $f : A\times B \to C$.
Since we intend the only elements of $A\times B$ to be pairs, we expect to be able to define such a function by prescribing the result
when $f$ is applied to a pair $\tup{a}{b}$.
We can prescribe these results by providing a function $g : A \to B \to C$.
Thus, we introduce a new rule (the elimination rule for products), which says that for any such $g$, we can define a function $f : A\times B \to C$ by
\[ f(\tup{a}{b}) \defeq g(a)(b). \]
We avoid writing $g(a,b)$ here, in order to emphasize that $g$ is not a function on a product.
(However, later on in the book we will often write $g(a,b)$ both for functions on a product and for curried functions of two variables.)
This defining equation is the computation rule for product types\index{computation rule!for product types}.

Note that in set theory, we would justify the above definition of $f$ by the fact that every element of $A\times B$ is a pair, so that it suffices to define $f$ on pairs.
By contrast, type theory reverses the situation: we assume that a function on $A\times B$ is well-defined as soon as we specify its values on tuples, and from this (or more precisely, from its more general version for dependent functions, below) we will be able to \emph{prove} that every element of $A\times B$ is a pair.
From a category-theoretic perspective, we can say that we define the product $A\times B$ to be left adjoint to the ``exponential'' $B\to C$, which we have already introduced.

As an example, we can derive the \define{projection}
\indexsee{function!projection}{projection}%
\indexsee{component, of a pair}{projection}%
\indexdef{projection!from cartesian product type}%
functions
\symlabel{defn:proj}%
\begin{align*}
  \fst & :  A \times B \to A \\
  \snd & :  A \times B \to B
\end{align*}
with the defining equations 
\begin{align*}
  \fst(\tup{a}{b}) & \defeq  a \\
  \snd(\tup{a}{b}) & \defeq  b.
\end{align*}
%
\symlabel{defn:recursor-times}%
Rather than invoking this principle of function definition every time we want to define a function, an alternative approach is to invoke it once, in a universal case, and then simply apply the resulting function in all other cases.
That is, we may define a function of type
\begin{equation}
  \rec{A\times B} : \prd{C:\UU}(A \to B \to C) \to A \times B \to C
\end{equation}
with the defining equation
\[\rec{A\times B}(C,g,\tup{a}{b}) \defeq g(a)(b). \]
Then instead of defining functions such as $\fst$ and $\snd$ directly by a defining equation, we could  define
\begin{align*}
  \fst &\defeq \rec{A\times B}(A, \lam{a}{b} a)\\
  \snd &\defeq \rec{A\times B}(B, \lam{a}{b} b).
\end{align*}
We refer to the function $\rec{A\times B}$ as the \define{recursor}
\indexsee{recursor}{recursion principle}%
for product types.  The name ``recursor'' is a bit unfortunate here, since no recursion is taking place.  It comes from the fact that product types are a degenerate example of a general framework for inductive types, and for types such as the natural numbers, the recursor will actually be recursive.  We may also speak of the \define{recursion principle} for cartesian products, meaning the fact that we can define a function $f:A\times B\to C$ as above by giving its value on pairs.
\index{recursion principle!for cartesian product}%

We leave it as a simple exercise to show that the recursor can be
derived from the projections and vice versa.
% Ex: Derive from projections

\symlabel{defn:recursor-unit}%
We also have a recursor for the unit type:
\[\rec{\unit{}} : \prd{C:\UU}C \to \unit{} \to C\]
with the defining equation
\[ \rec{\unit{}}(C,c,\ttt) \defeq c. \]
Although we include it to maintain the pattern of type definitions, the recursor for $\unit{}$ is completely useless,
because we could have defined such a function directly
by simply ignoring the argument of type $\unit{}$.

To be able to define \emph{dependent} functions over the product type, we have
to generalize the recursor. Given $C: A \times B \to \UU$, we may
define a function $f : \prd{x : A \times B} C(x)$ by providing a
function
\narrowequation{
 g : \prd{x:A}\prd{y:B} C(\tup{x}{y})
}
with defining equation
\[ f(\tup x y) \defeq g(x)(y). \] 
For example, in this way we can prove the propositional uniqueness principle, which says that every element of $A\times B$ is equal to a pair.
\index{uniqueness!principle, propositional!for product types}%
Specifically, we can construct a function
\[ \uppt : \prd{x:A \times B} (\id[A\times B]{\tup{\fst {(x)}}{\snd {(x)}}}{x}). \]
Here we are using the identity type, which we are going to introduce below in \autoref{sec:identity-types}.
However, all we need to know now is that there is a reflexivity element $\refl{x} : \id[A]{x}{x}$ for any $x:A$.
Given this, we can define
\[ \uppt(\tup{a}{b}) \defeq \refl{\tup{a}{b}}. \]
This construction works, because in the case that $x \defeq \tup{a}{b}$ we can 
calculate 
\[ \tup{\fst(\tup{a}{b})}{\snd{(\tup{a}{b})}} \jdeq \tup{a}{b} \]
using the defining equations for the projections. Therefore,
\[ \refl{\tup{a}{b}} : \id{\tup{\fst(\tup{a}{b})}{\snd{(\tup{a}{b})}}}{\tup{a}{b}} \]
is well-typed, since both sides of the equality are judgmentally equal.

More generally, the ability to define dependent functions in this way means that to prove a property for all elements of a product, it is enough 
to prove it for its canonical elements, the tuples.
When we come to inductive types such as the natural numbers, the analogous property will be the ability to write proofs by induction.
Thus, if we do as we did above and apply this principle once in the universal case, we call the resulting function \define{induction} for product types: given $A,B : \UU$ we have
\symlabel{defn:induction-times}%
\[ \ind{A\times B} : \prd{C:A \times B \to \UU}
\Parens{\prd{x:A}{y:B} C(\tup{x}{y})} \to \prd{x:A \times B} C(x) \]
with the defining equation 
\[ \ind{A\times B}(C,g,\tup{a}{b}) \defeq g(a)(b). \]
Similarly, we may speak of a dependent function defined on pairs being obtained from the \define{induction principle}
\index{induction principle}%
\index{induction principle!for product}%
of the cartesian product.
It is easy to see that the recursor is just the special case of induction
in the case that the family $C$ is constant.
Because induction describes how to use an element of the product type, induction is also called the \define{(dependent) eliminator},
\indexsee{eliminator!of inductive type!dependent}{induction principle}%
and recursion the \define{non-dependent eliminator}.
\indexsee{eliminator!of inductive type!non-dependent}{recursion principle}%
\indexsee{non-dependent eliminator}{recursion principle}%
\indexsee{dependent eliminator}{induction principle}%

% We can read induction propositionally as saying that a property which
% is true for all pairs holds for all elements of the product type.

Induction for the unit type turns out to be more useful than the
recursor: 
\symlabel{defn:induction-unit}%
\[ \ind{\unit{}} : \prd{C:\unit \to \UU} C(\ttt) \to \prd{x:\unit{}}C(x)\]
with the defining equation
\[ \ind{\unit{}}(C,c,\ttt) \defeq c. \]
Induction enables us to prove the propositional uniqueness principle for $\unit{}$, which asserts that its only inhabitant is $\ttt$.
That is, we can construct
\[\un : \prd{x:\unit{}} \id{x}{\ttt} \]
by using the defining equations
\[\un(\ttt) \defeq \refl{\ttt} \]
or equivalently by using induction:
\[\un \defeq \ind{\unit{}}(\lamu{x:\unit{}} \id{x}{\ttt},\refl{\ttt}). \]

\index{type!product|)}%
\index{type!unit|)}%

\subsection{Dependent pair types (\texorpdfstring{$\Sigma$}{Σ}-types)}
\label{subsec:sigma-types}

\index{type!dependent pair|(defstyle}%
\indexsee{type!dependent sum}{type, dependent pair}%
\indexsee{type!Sigma-@$\Sigma$-}{type, dependent pair}%
\indexsee{Sigma-type@$\Sigma$-type}{type, dependent pair}%
\indexsee{sum!dependent}{type, dependent pair}%

Just as we generalized function types (\autoref{sec:function-types}) to dependent function types (\autoref{sec:pi-types}), it is often useful to generalize the product types from \autoref{sec:finite-product-types} to allow the type of
the second component of a pair to vary depending on the choice of the first
component. This is called a \define{dependent pair type}, or \define{$\Sigma$-type}, because in set theory it
corresponds to an indexed sum (in the sense of coproduct or
disjoint union) over a given type.

Given a type $A:\UU$ and a family $B : A \to \UU$, the dependent
pair type is written as $\sm{x:A} B(x) : \UU$.
Alternative notations are 
\[ \tsm{x:A} B(x) \hspace{2cm} \dsm{x:A}B(x) \hspace{2cm} \lsm{x:A} B(x). \]
Like other binding constructs such as $\lambda$-abstractions and $\Pi$s, $\Sigma$s automatically scope\index{scope} over the rest of the expression unless delimited, so e.g.\ $\sm{x:A} B(x) \times C(x)$ means $\sm{x:A} (B(x) \times C(x))$.

\symlabel{defn:dependent-pair}%
\indexdef{pair!dependent}%
The way to construct elements of a dependent pair type is by pairing: we have
$\tup{a}{b} : \sm{x:A} B(x)$ given $a:A$ and $b:B(a)$.
If $B$ is constant, then the dependent pair type is the
ordinary cartesian product type:
\[ \Parens{\sm{x:A} B} \jdeq (A \times B).\]
All the constructions on $\Sigma$-types arise as straightforward generalizations of the ones for product types, with dependent functions often replacing non-dependent ones.

For instance, the recursion principle%
\index{recursion principle!for dependent pair type}
says that to define a non-dependent function out of a $\Sigma$-type
$f : (\sm{x:A} B(x)) \to C$, we provide a function 
$g : \prd{x:A} B(x) \to C$, and then we can define $f$ via the defining
equation
\[ f(\tup{a}{b}) \defeq g(a)(b). \]
\indexdef{projection!from dependent pair type}%
For instance, we can derive the first projection from a $\Sigma$-type:
\symlabel{defn:dependent-proj1}%
\begin{equation*}
  \fst : \Parens{\sm{x : A}B(x)} \to A.
\end{equation*}
by the defining equation
\begin{equation*}
  \fst(\tup{a}{b}) \defeq a.
\end{equation*}
However, since the type of the second component of a pair
\narrowequation{
  (a,b):\sm{x:A} B(x)
}
is $B(a)$, the second projection must be a \emph{dependent} function, whose type involves the first projection function:
\symlabel{defn:dependent-proj2}%
\[ \snd : \prd{p:\sm{x : A}B(x)}B(\fst(p)). \]
Thus we need the \emph{induction} principle%
\index{induction principle!for dependent pair type}
for $\Sigma$-types (the ``dependent eliminator'').
This says that to construct a dependent function out of a $\Sigma$-type into a family $C : (\sm{x:A} B(x)) \to \UU$, we need a function
\[ g : \prd{a:A}{b:B(a)} C(\tup{a}{b}). \]
We can then derive a function 
\[ f : \prd{p : \sm{x:A}B(x)} C(p) \]
with  defining equation\index{computation rule!for dependent pair type}
\[ f(\tup{a}{b}) \defeq g(a)(b).\]
Applying this with $C(p)\defeq B(\fst(p))$, we can define
\narrowequation{
\snd : \prd{p:\sm{x : A}B(x)}B(\fst(p))
}
with the obvious equation
\[ \snd(\tup{a}{b})  \defeq  b. \]
To convince ourselves that this is correct, we note that $B (\fst(\tup{a}{b})) \jdeq B(a)$, using the defining equation for $\fst$, and
indeed $b : B(a)$.

We can package the recursion and induction principles into the recursor for $\Sigma$:
\symlabel{defn:recursor-sm}%
\[ \rec{\sm{x:A}B(x)} : \dprd{C:\UU}\Parens{\tprd{x:A} B(x) \to C} \to
\Parens{\tsm{x:A}B(x)} \to C \]
with the defining equation
\[ \rec{\sm{x:A}B(x)}(C,g,\tup{a}{b}) \defeq g(a)(b) \]
and the corresponding induction operator:
\symlabel{defn:induction-sm}%
\begin{narrowmultline*}
  \ind{\sm{x:A}B(x)} : \narrowbreak
    \dprd{C:(\sm{x:A} B(x)) \to \UU}
    \Parens{\tprd{a:A}{b:B(a)} C(\tup{a}{b})}
    \to \dprd{p : \sm{x:A}B(x)} C(p)
\end{narrowmultline*}
with the defining equation 
\[ \ind{\sm{x:A}B(x)}(C,g,\tup{a}{b}) \defeq g(a)(b). \]
As before, the recursor is the special case of induction
when the family $C$ is constant.

As a further example, consider the following principle, where $A$ and $B$ are types and $R:A\to B\to \UU$.
\[ \ac : \Parens{\tprd{x:A} \tsm{y :B} R(x,y)} \to
\Parens{\tsm{f:A\to B} \tprd{x:A} R(x,f(x))}
\]
We may regard $R$ as a ``proof-relevant relation''
\index{mathematics!proof-relevant}%
between $A$ and $B$, with $R(a,b)$ the type of witnesses for relatedness of $a:A$ and $b:B$.
Then $\ac$ says intuitively that if we have a dependent function $g$ assigning to every $a:A$ a dependent pair $(b,r)$ where $b:B$ and $r:R(a,b)$, then we have a function $f:A\to B$ and a dependent function assigning to every $a:A$ a witness that $R(a,f(a))$.
Our intuition tells us that we can just split up the values of $g$ into their components.
Indeed, using the projections we have just defined, we can define:
\[ \ac(g) \defeq \Parens{\lamu{x:A} \fst(g(x)),\, \lamu{x:A} \snd(g(x))}. \]
To verify that this is well-typed, note that if $g:\prd{x:A} \sm{y :B} R(x,y)$, we have
\begin{align*}
\lamu{x:A} \fst(g(x)) &: A \to  B, \\
\lamu{x:A} \snd(g(x)) &: \tprd{x:A} R(x,\fst(g(x))).
\end{align*}
Moreover, the type $\prd{x:A} R(x,\fst(g(x)))$ is the result of substituting the function $\lamu{x:A} \fst(g(x))$ for $f$ in the family being summed over in the co\-do\-main of \ac:
\[ \tprd{x:A} R(x,\fst(g(x))) \jdeq
\Parens{\lamu{f:A\to B} \tprd{x:A} R(x,f(x))}\big(\lamu{x:A} \fst(g(x))\big). \]
Thus, we have
\[ \Parens{\lamu{x:A} \fst(g(x)),\, \lamu{x:A} \snd(g(x))} : \tsm{f:A\to B} \tprd{x:A} R(x,f(x))\]
as required.

If we read $\Pi$ as ``for all'' and $\Sigma$ as ``there exists'', then the type of the function $\ac$ expresses:
\emph{if for all $x:A$ there is a $y:B$ such that $R(x,y)$, then there is a function $f : A \to B$ such that for all $x:A$ we have $R(x,f(x))$}.
Since this sounds like a version of the axiom of choice, the function \ac has traditionally been called the \define{type-theoretic axiom of choice}, and as we have just shown, it can be proved directly from the rules of type theory, rather than having to be taken as an axiom.
\index{axiom!of choice!type-theoretic}%
However, note that no choice is actually involved, since the choices have already been given to us in the premise: all we have to do is take it apart into two functions: one representing the choice and the other its correctness.
In \autoref{sec:axiom-choice} we will give another formulation of an ``axiom of choice'' which is closer to the usual one.

Dependent pair types are often used to define types of mathematical structures, which commonly consist of several dependent pieces of data.
To take a simple example, suppose we want to define a \define{magma}\indexdef{magma} to be a type $A$ together with a binary operation $m:A\to A\to A$.
The precise meaning of the phrase ``together with''\index{together with} (and the synonymous ``equipped with''\index{equipped with}) is that ``a magma'' is a \emph{pair} $(A,m)$ consisting of a type $A:\UU$ and an operation $m:A\to A\to A$.
Since the type $A\to A\to A$ of the second component $m$ of this pair depends on its first component $A$, such pairs belong to a dependent pair type.
Thus, the definition ``a magma is a type $A$ together with a binary operation $m:A\to A\to A$'' should be read as defining \emph{the type of magmas} to be
\[ \mathsf{Magma} \defeq \sm{A:\UU} (A\to A\to A). \]
Given a magma, we extract its underlying type (its ``carrier''\index{carrier}) with the first projection $\proj1$, and its operation with the second projection $\proj2$.
Of course, structures built from more than two pieces of data require iterated pair types, which may be only partially dependent; for instance the type of pointed magmas (magmas $(A,m)$ equipped with a basepoint $e:A$) is
\[ \mathsf{PointedMagma} \defeq \sm{A:\UU} (A\to A\to A) \times A. \]
We generally also want to impose axioms on such a structure, e.g.\ to make a pointed magma into a monoid or a group.
This can also be done using $\Sigma$-types; see \autoref{sec:pat}.

In the rest of the book, we will sometimes make definitions of this sort explicit, but eventually we trust the reader to translate them from English into $\Sigma$-types.
We also generally follow the common mathematical practice of using the same letter for a structure of this sort and for its carrier (which amounts to leaving the appropriate projection function implicit in the notation): that is, we will speak of a magma $A$ with its operation $m:A\to A\to A$.

Note that the canonical elements of $\mathsf{PointedMagma}$ are of the form $(A,(m,e))$ where $A:\UU$, $m:A\to A\to A$, and $e:A$.
Because of the frequency with which iterated $\Sigma$-types of this sort arise, we use the usual notation of ordered triples, quadruples and so on to stand for nested pairs (possibly dependent) associating to the right.
That is, we have $(x,y,z) \defeq (x,(y,z))$ and $(x,y,z,w)\defeq (x,(y,(z,w)))$, etc.

\index{type!dependent pair|)}%

\subsection{Coproduct types}
\label{subsec:coproduct-types}

Given $A,B:\UU$, we introduce their \define{coproduct} type $A+B:\UU$.
\indexsee{coproduct}{type, coproduct}%
\index{type!coproduct|(defstyle}%
\indexsee{disjoint!sum}{type, coproduct}%
\indexsee{disjoint!union}{type, coproduct}%
\indexsee{sum!disjoint}{type, coproduct}%
\indexsee{union!disjoint}{type, coproduct}%
This corresponds to the \emph{disjoint union} in set theory, and we may also use that name for it.
In type theory, as was the case with functions and products, the coproduct must be a fundamental construction, since there is no previously given notion of ``union of types''.
We also introduce a
nullary version: the \define{empty type $\emptyt:\UU$}.
\indexsee{nullary!coproduct}{type, empty}%
\indexsee{empty type}{type, empty}%
\index{type!empty|(defstyle}%

There are two ways to construct elements of $A+B$, either as $\inl(a) : A+B$ for $a:A$, or as
$\inr(b):A+B$ for $b:B$.
(The names $\inl$ and $\inr$ are short for ``left injection'' and ``right injection''.)
There are no ways to construct elements of the empty type. 

\index{recursion principle!for coproduct}
To construct a non-dependent function $f : A+B \to C$, we need 
functions $g_0 : A \to C$ and $g_1 : B \to C$. Then $f$ is defined
via the defining equations
\begin{align*}
  f(\inl(a)) &\defeq g_0(a), \\
  f(\inr(b)) &\defeq g_1(b).
\end{align*}
That is, the function $f$ is defined by \define{case analysis}.
\indexdef{case analysis}%
As before, we can derive the recursor:
\symlabel{defn:recursor-plus}%
\[ \rec{A+B} : \dprd{C:\UU}(A \to C) \to (B\to C) \to A+B \to C\]
with the defining equations
\begin{align*}
\rec{A+B}(C,g_0,g_1,\inl(a)) &\defeq g_0(a), \\
\rec{A+B}(C,g_0,g_1,\inr(b)) &\defeq g_1(b).
\end{align*}

\index{recursion principle!for empty type}
We can always construct a function $f : \emptyt \to C$ without
having to give any defining equations, because there are no elements of \emptyt on which to define $f$.
Thus, the recursor for $\emptyt$ is
\symlabel{defn:recursor-emptyt}%
\[\rec{\emptyt} : \tprd{C:\UU} \emptyt \to C,\]
which constructs the canonical function from the empty type to any other type.
Logically, it corresponds to the principle \textit{ex falso quodlibet}.
\index{ex falso quodlibet@\textit{ex falso quodlibet}}

\index{induction principle!for coproduct}
To construct a dependent function $f:\prd{x:A+B}C(x)$ out of a coproduct, we assume as given the family 
$C: (A + B) \to \UU$, and 
require 
\begin{align*}
  g_0 &: \prd{a:A} C(\inl(a)), \\
  g_1 &: \prd{b:B} C(\inr(b)).
\end{align*}
This yields $f$ with the defining equations:\index{computation rule!for coproduct type}
\begin{align*}
  f(\inl(a)) &\defeq g_0(a), \\
  f(\inr(b)) &\defeq g_1(b).
\end{align*}
We package this scheme into the induction principle for coproducts:
\symlabel{defn:induction-plus}%
\begin{narrowmultline*}
  \ind{A+B} :
  \dprd{C: (A + B) \to \UU}
  \Parens{\tprd{a:A} C(\inl(a))} \to \narrowbreak
  \Parens{\tprd{b:B} C(\inr(b))} \to \tprd{x:A+B}C(x). 
\end{narrowmultline*}
As before, the recursor arises in the case that the family $C$ is
constant. 

\index{induction principle!for empty type}
The induction principle for the empty type
\symlabel{defn:induction-emptyt}%
\[ \ind{\emptyt} : \prd{C:\emptyt \to \UU}{z:\emptyt} C(z) \]
gives us a way to define a trivial dependent function out of the
empty type. % In the presence of $\eta$-equality it is derivable
% from the recursor.
% ex

\index{type!coproduct|)}%
\index{type!empty|)}%


\subsection{The type of booleans}
\label{subsec:type-booleans}

\indexsee{boolean!type of}{type of booleans}%
\index{type!of booleans|(defstyle}%
The type of booleans $\bool:\UU$ is intended to have exactly two elements 
$\bfalse,\btrue : \bool$. It is clear that we could construct this
type out of coproduct
% this one results in a warning message just because it's on the same page as the previous entry 
% for {type!coproduct}, so it's not our fault
\index{type!coproduct}%
and unit\index{type!unit} types as $\unit + \unit{}$. However,
since it is used frequently, we give the explicit rules here.
Indeed, we are going to observe that we can also go the other way
and derive binary coproducts from $\Sigma$-types and $\bool$.

\index{recursion principle!for type of booleans}
To derive a function $f : \bool \to C$ we need $c_0,c_1 : C$ and
add the defining equations
\begin{align*}
  f(\bfalse) &\defeq c_0, \\
  f(\btrue)  &\defeq c_1.
\end{align*}
The recursor corresponds to the if-then-else construct in
functional programming:
\symlabel{defn:recursor-bool}%
\[ \rec{\bool} : \prd{C:\UU}  C \to C \to \bool \to C \]
with the defining equations
\begin{align*}
  \rec{\bool}(C,c_0,c_1,\bfalse) &\defeq c_0, \\
  \rec{\bool}(C,c_0,c_1,\btrue)  &\defeq c_1.
\end{align*}

\index{induction principle!for type of booleans}
Given $C : \bool \to \UU$, to derive a dependent function 
$f : \prd{x:\bool}C(x)$ we need $c_0:C(\bfalse)$ and $c_1 : C(\btrue)$, in which case we can give the defining equations
\begin{align*}
  f(\bfalse) &\defeq c_0, \\
  f(\btrue)  &\defeq c_1.
\end{align*}
We package this up into the induction principle
\symlabel{defn:induction-bool}%
\[ \ind{\bool} : \dprd{C:\bool \to \UU}  C(\bfalse) \to C(\btrue)
\to \tprd{x:\bool} C(x) \]
with the defining equations
\begin{align*}
  \ind{\bool}(C,c_0,c_1,\bfalse) &\defeq c_0, \\
  \ind{\bool}(C,c_0,c_1,\btrue)  &\defeq c_1.
\end{align*}

As an example, using the induction principle we can deduce that, as we expect, every element of $\bool$ is either $\btrue$ or $\bfalse$.
As before, in order to state this we use the equality types which we have not yet introduced, but we need only the fact that everything is equal to itself by $\refl{x}:x=x$.
Thus, we construct an element of
\begin{equation}\label{thm:allbool-trueorfalse}
  \prd{x:\bool}(x=\bfalse)+(x=\btrue),
\end{equation}
i.e.\ a function assigning to each $x:\bool$ either an equality $x=\bfalse$ or an equality $x=\btrue$.
We define this element using the induction principle for \bool, with $C(x) \defeq (x=\bfalse)+(x=\btrue)$;
the two inputs are $\inl(\refl{\bfalse}) : C(\bfalse)$ and $\inr(\refl{\btrue}):C(\btrue)$.
In other words, our element of~\eqref{thm:allbool-trueorfalse} is
\[ \ind{\bool}\big(\lam{x}(x=\bfalse)+(x=\btrue),\, \inl(\refl{\bfalse}),\, \inr(\refl{\btrue})\big). \]

We have remarked that $\Sigma$-types can be regarded as analogous to indexed disjoint unions, while coproducts are binary disjoint unions.
It is natural to expect that a binary disjoint union $A+B$ could be constructed as an indexed one over the two-element type \bool.
For this we need a type family $P:\bool\to\type$ such that $P(\bfalse)\jdeq A$ and $P(\btrue)\jdeq B$.
Indeed, we can obtain such a family precisely by the recursion principle for $\bool$.
\index{type!family of}%
(The ability to define \emph{type families} by induction and recursion, using the fact that the universe $\UU$ is itself a type, is a subtle and important aspect of type theory.)
Thus, we could have defined
\index{type!coproduct}%
\[ A + B \defeq \sm{x:\bool} \rec{\bool}(\UU,A,B,x). \]
with
\begin{align*}
  \inl(a) &\defeq \tup{\bfalse}{a}, \\
  \inr(b) &\defeq \tup{\btrue}{b}.
\end{align*}
We leave it as an exercise to derive the induction principle of a coproduct type from this definition.
(See also \autoref{ex:sum-via-bool,sec:appetizer-univalence}.)

We can apply the same idea to products and $\Pi$-types: we could have defined
\[ A \times B \defeq \prd{x:\bool}\rec{\bool}(\UU,A,B,x) \]
Pairs could then be constructed using induction for \bool:
\[ \tup{a}{b} \defeq \ind{\bool}(\rec{\bool}(\UU,A,B),a,b) \]
while the projections are straightforward applications
\begin{align*}
  \fst(p) &\defeq p(\bfalse), \\
  \snd(p) &\defeq p(\btrue).
\end{align*}
The derivation of the induction principle for binary products defined in this way is a bit more involved, and requires function extensionality, which we will introduce in \autoref{sec:compute-pi}.
Moreover, we do not get the same judgmental equalities; see \autoref{ex:prod-via-bool}.
This is a recurrent issue when encoding one type as another; we will return to it in \autoref{sec:htpy-inductive}. 

We may occasionally refer to the elements $\bfalse$ and $\btrue$ of $\bool$ as ``false'' and ``true'' respectively.
However, note that unlike in classical\index{mathematics!classical} mathematics, we do not use elements of $\bool$ as truth values
\index{value!truth}%
or as propositions.
(Instead we identify propositions with types; see \autoref{sec:pat}.)
In particular, the type $A \to \bool$ is not generally the power set\index{power set} of $A$; it represents only the ``decidable'' subsets of $A$ (see \autoref{cha:logic}).
\index{decidable!subset}%

\index{type!of booleans|)}%


\subsection{The natural numbers}
\label{subsec:inductive-types}

\indexsee{type!of natural numbers}{natural numbers}%
\index{natural numbers|(defstyle}%
\indexsee{number!natural}{natural numbers}%
The rules we have introduced so far do not allow us to construct any infinite types.
The simplest infinite type we can think of (and one which is of course also extremely useful) is the type $\nat : \UU$ of natural numbers.
The elements of $\nat$ are constructed using $0 : \nat$\indexdef{zero} and the successor\indexdef{successor} operation $\suc : \nat \to \nat$.
When denoting natural numbers, we adopt the usual decimal notation $1 \defeq \suc(0)$, $2 \defeq \suc(1)$, $3 \defeq \suc(2)$, \dots.

The essential property of the natural numbers is that we can define functions by recursion and perform proofs by induction --- where now the words ``recursion'' and ``induction'' have a more familiar meaning.
\index{recursion principle!for natural numbers}%
To construct a non-dependent function $f : \nat \to C$ out of the natural numbers by recursion, it is enough to provide a starting point $c_0 : C$ and a ``next step'' function $c_s : \nat \to C \to C$.
This gives rise to $f$ with the defining equations\index{computation rule!for natural numbers}
\begin{align*}
  f(0) &\defeq c_0, \\
  f(\suc(n)) &\defeq c_s(n,f(n)).
\end{align*}
We say that $f$ is defined by \define{primitive recursion}.
\indexdef{primitive!recursion}%
\indexdef{recursion!primitive}%

As an example, we look at how to define a function on natural numbers which doubles its argument.
In this case we have $C\defeq \nat$.
We first need to supply the value of $\dbl(0)$, which is easy: we put $c_0 \defeq 0$.
Next, to compute the value of $\dbl(\suc(n))$ for a natural number $n$, we first compute the value of $\dbl(n)$ and then perform the successor operation twice.
This is captured by the recurrence\index{recurrence} $c_s(n,y) \defeq \suc(\suc(y))$.
Note that the second argument $y$ of $c_s$ stands for the result of the \emph{recursive call}\index{recursive call} $\dbl(n)$.

Defining $\dbl:\nat\to\nat$ by primitive recursion in this way, therefore, we obtain the defining equations:
\begin{align*}
  \dbl(0) &\defeq 0\\
  \dbl(\suc(n)) &\defeq \suc(\suc(\dbl(n))).
\end{align*}
This indeed has the correct computational behavior: for example, we have 
\begin{align*}
  \dbl(2) &\jdeq \dbl(\suc(\suc(0)))\\
  & \jdeq c_s(\suc(0), \dbl(\suc(0))) \\
                 & \jdeq \suc(\suc(\dbl(\suc(0)))) \\
                 & \jdeq \suc(\suc(c_s(0,\dbl(0)))) \\
                 & \jdeq \suc(\suc(\suc(\suc(\dbl(0))))) \\
                 & \jdeq \suc(\suc(\suc(\suc(c_0)))) \\
                 & \jdeq \suc(\suc(\suc(\suc(0))))\\
                 &\jdeq 4.
\end{align*}
We can define multi-variable functions by primitive recursion as well, by currying and allowing $C$ to be a function type.
\indexdef{addition!of natural numbers}
For example, we define addition $\add : \nat \to \nat \to \nat$ with $C \defeq \nat \to \nat$ and the following ``starting point'' and ``next step'' data:
\begin{align*}
  c_0 & : \nat \to \nat \\
  c_0 (n) & \defeq n \\
  c_s & : \nat \to (\nat \to \nat) \to (\nat \to \nat) \\
  c_s(m,g)(n) & \defeq \suc(g(n)).
\end{align*}
We thus obtain $\add : \nat \to \nat \to \nat$ satisfying the definitional equalities
\begin{align*}
  \add(0,n) &\jdeq n \\
  \add(\suc(m),n) &\jdeq \suc(\add(m,n)). 
\end{align*}
As usual, we write $\add(m,n)$ as $m+n$.
The reader is invited to verify that $2+2\jdeq 4$.
% ex: define multiplication and exponentiation.

As in previous cases, we can package the principle of primitive recursion into a recursor:
\[\rec{\nat}  : \dprd{C:\UU} C \to (\nat \to C \to C) \to \nat \to C \]
with the defining equations
\symlabel{defn:recursor-nat}%
\begin{align*}
\rec{\nat}(C,c_0,c_s,0)  &\defeq c_0, \\
\rec{\nat}(C,c_0,c_s,\suc(n)) &\defeq c_s(n,\rec{\nat}(C,c_0,c_s,n)).
\end{align*}
%ex derive rec from it
Using $\rec{\nat}$ we can present $\dbl$ and $\add$ as follows:
\begin{align}
\dbl &\defeq \rec\nat\big(\nat,\, 0,\, \lamu{n:\nat}{y:\nat} \suc(\suc(y))\big) \label{eq:dbl-as-rec}\\
\add &\defeq \rec{\nat}\big(\nat \to \nat,\, \lamu{n:\nat} n,\, \lamu{n:\nat}{g:\nat \to \nat}{m :\nat} \suc(g(m))\big).
\end{align}
Of course, all functions definable only using the primitive recursion principle will be \emph{computable}.
(The presence of higher function types --- that is, functions with other functions as arguments --- does, however, mean we can define more than the usual primitive recursive functions; see e.g.~\autoref{ex:ackermann}.)
This is appropriate in constructive mathematics;
\index{mathematics!constructive}%
in \autoref{sec:intuitionism,sec:axiom-choice} we will see how to augment type theory so that we can define more general mathematical functions.

\index{induction principle!for natural numbers}
We now follow the same approach as for other types, generalizing primitive recursion to dependent functions to obtain an \emph{induction principle}.
Thus, assume as given a family $C : \nat \to \UU$, an element $c_0 : C(0)$, and a function $c_s : \prd{n:\nat} C(n) \to C(\suc(n))$; then we can construct $f : \prd{n:\nat} C(n)$ with the defining equations:\index{computation rule!for natural numbers}
\begin{align*}
  f(0) &\defeq c_0, \\
  f(\suc(n)) &\defeq c_s(n,f(n)).
\end{align*}
We can also package this into a single function
\symlabel{defn:induction-nat}%
\[\ind{\nat}  : \dprd{C:\nat\to \UU} C(0) \to \Parens{\tprd{n : \nat} C(n) \to C(\suc(n))} \to \tprd{n : \nat} C(n) \]
with the defining equations
\begin{align*}
\ind{\nat}(C,c_0,c_s,0)  &\defeq c_0, \\
\ind{\nat}(C,c_0,c_s,\suc(n)) &\defeq c_s(n,\ind{\nat}(C,c_0,c_s,n)).
\end{align*}
Here we finally see the connection to the classical notion of proof by induction.
Recall that in type theory we represent propositions by types, and proving a proposition by inhabiting the corresponding type.
In particular, a \emph{property} of natural numbers is represented by a family of types $P:\nat\to\type$.
From this point of view, the above induction principle says that if we can prove $P(0)$, and if for any $n$ we can prove $P(\suc(n))$ assuming $P(n)$, then we have $P(n)$ for all $n$.
This is, of course, exactly the usual principle of proof by induction on natural numbers.

\index{associativity!of addition!of natural numbers}
As an example, consider how we might represent an explicit proof that $+$ is associative.
(We will not actually write out proofs in this style, but it serves as a useful example for understanding how induction is represented formally in type theory.)
To derive
\[\assoc : \prd{i,j,k:\nat} \id{i + (j + k)}{(i + j) + k}, \]
it is sufficient to supply
\[ \assoc_0 :  \prd{j,k:\nat} \id{0 + (j + k)}{(0+ j) + k} \]
and
\begin{narrowmultline*}
  \assoc_s  : \prd{i:\nat} \left(\prd{j,k:\nat} \id{i + (j + k)}{(i + j) + k}\right)
   \narrowbreak
   \to \prd{j,k:\nat} \id{\suc(i) + (j + k)}{(\suc(i) + j) + k}.
\end{narrowmultline*}
To derive $\assoc_0$, recall that $0+n \jdeq n$, and hence  $0 + (j + k) \jdeq j+k \jdeq (0+ j) + k$.
Hence we can just set
\[ \assoc_0(j,k) \defeq \refl{j+k}. \]
For $\assoc_s$, recall that the definition of $+$ gives $\suc(m)+n \jdeq \suc(m+n)$, and hence 
\begin{align*}
   \suc(i) + (j + k)  &\jdeq \suc(i+(j+k)) \qquad\text{and}\\
   (\suc(i)+j)+k &\jdeq \suc((i+j)+k).
\end{align*}
Thus, the output type of $\assoc_s$ is equivalently $\id{\suc(i+(j+k))}{\suc((i+j)+k)}$.
But its input (the ``inductive hypothesis'')
\index{hypothesis!inductive}%
\index{inductive!hypothesis}%
yields $\id{i+(j+k)}{(i+j)+k}$, so it suffices to invoke the fact that if two natural numbers are equal, then so are their successors.
(We will prove this obvious fact in \autoref{lem:map}, using the induction principle of identity types.)
We call this latter fact
$\apfunc{\suc} : %\prd{m,n:\nat}
(\id[\nat]{m}{n}) \to (\id[\nat]{\suc(m)}{\suc(n)})$, so we can define
\[\assoc_s(i,h,j,k) \defeq \apfunc{\suc}( %n+(j+k),(n+j)+k,
h(j,k)). \]
Putting these together with $\ind{\nat}$, we obtain a proof of associativity.

\index{natural numbers|)}%


\subsection{Pattern matching and recursion}
\label{subsec:pattern-matching}

\index{pattern matching|(defstyle}%
\indexsee{matching}{pattern matching}%
\index{definition!by pattern matching|(}%
The natural numbers introduce an additional subtlety over the types considered up until now.
In the case of coproducts, for instance, we could define a function $f:A+B\to C$ either with the recursor:
\[ f \defeq \rec{A+B}(C, g_0, g_1) \]
or by giving the defining equations:
\begin{align*}
  f(\inl(a)) &\defeq g_0(a)\\
  f(\inr(b)) &\defeq g_1(b).
\end{align*}
To go from the former expression of $f$ to the latter, we simply use the computation rules for the recursor.
Conversely, given any defining equations
\begin{align*}
  f(\inl(a)) &\defeq \Phi_0\\
  f(\inr(b)) &\defeq \Phi_1
\end{align*}
where $\Phi_0$ and $\Phi_1$ are expressions that may involve the variables
\index{variable}%
$a$ and $b$ respectively, we can express these equations equivalently in terms of the recursor by using $\lambda$-abstraction\index{lambda abstraction@$\lambda$-abstraction}:
\[ f\defeq \rec{A+B}(C, \lam{a} \Phi_0, \lam{b} \Phi_1).\]
In the case of the natural numbers, however, the ``defining equations'' of a function such as $\dbl$:
\begin{align}
  \dbl(0) &\defeq 0 \label{eq:dbl0}\\
  \dbl(\suc(n)) &\defeq \suc(\suc(\dbl(n)))\label{eq:dblsuc}
\end{align}
involve \emph{the function $\dbl$ itself} on the right-hand side.
However, we would still like to be able to give these equations, rather than~\eqref{eq:dbl-as-rec}, as the definition of \dbl, since they are much more convenient and readable.
The solution is to read the expression ``$\dbl(n)$'' on the right-hand side of~\eqref{eq:dblsuc} as standing in for the result of the recursive call, which in a definition of the form $\dbl\defeq \rec{\nat}(\nat,c_0,c_s)$ would be the second argument of $c_s$.

More generally, if we have a ``definition'' of a function $f:\nat\to C$ such as
\begin{align*}
  f(0) &\defeq \Phi_0\\
  f(\suc(n)) &\defeq \Phi_s
\end{align*}
where $\Phi_0$ is an expression of type $C$, and $\Phi_s$ is an expression of type $C$ which may involve the variable $n$ and also the symbol ``$f(n)$'', we may translate it to a definition
\[ f \defeq \rec{\nat}(C,\,\Phi_0,\,\lam{n}{r} \Phi_s') \]
where $\Phi_s'$ is obtained from $\Phi_s$ by replacing all occurrences of ``$f(n)$'' by the new variable $r$.

This style of defining functions by recursion (or, more generally, dependent functions by induction) is so convenient that we frequently adopt it.
It is called definition by \define{pattern matching}.
Of course, it is very similar to how a computer programmer may define a recursive function with a body that literally contains recursive calls to itself.
However, unlike the programmer, we are restricted in what sort of recursive calls we can make: in order for such a definition to be re-expressible using the recursion principle, the function $f$ being defined can only appear in the body of $f(\suc(n))$ as part of the composite symbol ``$f(n)$''.
Otherwise, we could write nonsense functions such as
\begin{align*}
  f(0)&\defeq 0\\
  f(\suc(n)) &\defeq f(\suc(\suc(n))).
\end{align*}
If a programmer wrote such a function, it would simply call itself forever on any positive input, going into an infinite loop and never returning a value.
In mathematics, however, to be worthy of the name, a \emph{function} must always associate a unique output value to every input value, so this would be unacceptable.

This point will be even more important when we introduce more complicated inductive types in \autoref{cha:induction,cha:hits,cha:real-numbers}.
Whenever we introduce a new kind of inductive definition, we always begin by deriving its induction principle.
Only then do we introduce an appropriate sort of ``pattern matching'' which can be justified as a shorthand for the induction principle.

\index{pattern matching|)}%
\index{definition!by pattern matching|)}%

\subsection{Propositions as types}
\label{subsec:pat}

\index{proposition!as types|(defstyle}%
\index{logic!propositions as types|(}%
As mentioned in the introduction, to show that a proposition is true in type theory corresponds to exhibiting an element of the type corresponding to that proposition.
\index{evidence, of the truth of a proposition}%
\index{witness!to the truth of a proposition}%
\index{proof|(}
We regard the elements of this type as \emph{evidence} or \emph{witnesses} that the proposition is true. (They are sometimes even called \emph{proofs}, but this terminology can be misleading, so we generally avoid it.)
In general, however, we will not construct witnesses explicitly; instead we present the proofs in ordinary mathematical prose, in such a way that they could be translated into an element of a type.
This is no different from reasoning in classical set theory, where we don't expect to see an explicit derivation using the rules of predicate logic and the axioms of set theory.

However, the type-theoretic perspective on proofs is nevertheless different in important ways.
The basic principle of the logic of type theory is that a proposition is not merely true or false, but rather can be seen as the collection of all possible witnesses of its truth.
Under this conception, proofs are not just the means by which mathematics is communicated, but rather are mathematical objects in their own right, on a par with more familiar objects such as numbers, mappings, groups, and so on.
Thus, since types classify the available mathematical objects and govern how they interact, propositions are nothing but special  types --- namely, types whose elements are proofs.

\index{propositional!logic}%
\index{logic!propositional}%
The basic observation which makes this identification feasible is that we have the following natural correspondence between \emph{logical} operations on propositions, expressed in English, and \emph{type-theoretic} operations on their corresponding types of witnesses.
\index{false}%
\index{true}%
\index{conjunction}%
\index{disjunction}%
\index{implication}%
\begin{center}
\medskip
\begin{tabular}{ll}
  \toprule
  English & Type Theory\\
  \midrule
  True & $\unit{}$ \\
  False & $\emptyt$ \\
  $A$ and $B$ & $A \times B$ \\
  $A$ or $B$ & $A + B$ \\
  If $A$ then $B$ & $A \to B$ \\
  $A$ if and only if $B$ & $(A \to B) \times (B \to A)$ \\
  Not $A$ &  $A \to \emptyt$ \\
  \bottomrule
\end{tabular}
\medskip
\end{center}

The point of the correspondence is that in each case, the rules for constructing and using elements of the type on the right correspond to the rules for reasoning about the proposition on the left.
For instance, the basic way to prove a statement of the form ``$A$ and $B$'' is to prove $A$ and also prove $B$, while the basic way to construct an element of $A\times B$ is as a pair $(a,b)$, where $a$ is an element (or witness) of $A$ and $b$ is an element (or witness) of $B$.
And if we want to use ``$A$ and $B$'' to prove something else, we are free to use both $A$ and $B$ in doing so, analogously to how the induction principle for $A\times B$ allows us to construct a function out of it by using elements of $A$ and of $B$.

Similarly, the basic way to prove an implication\index{implication} ``if $A$ then $B$'' is to assume $A$ and prove $B$, while the basic way to construct an element of $A\to B$ is to give an expression which denotes an element (witness) of $B$ which may involve an unspecified variable element (witness) of type $A$.
And the basic way to use an implication ``if $A$ then $B$'' is deduce $B$ if we know $A$, analogously to how we can apply a function $f:A\to B$ to an element of $A$ to produce an element of $B$.
We strongly encourage the reader to do the exercise of verifying that the rules governing the other type constructors translate sensibly into logic.

Of special note is that the empty type $\emptyt$ corresponds to falsity.\index{false}
When speaking logically, we refer to an inhabitant of $\emptyt$ as a \define{contradiction}:
\indexdef{contradiction}%
thus there is no way to prove a contradiction,%
\footnote{More precisely, there is no \emph{basic} way to prove a contradiction, i.e.\ \emptyt has no constructors.
If our type theory were inconsistent, then there would be some more complicated way to construct an element of \emptyt.}
while from a contradiction anything can be derived.
We also define the \define{negation}
\indexdef{negation}%
of a type $A$ as
%
\begin{equation*}
  \neg A \ \defeq\ A \to \emptyt.
\end{equation*}
%
Thus, a witness of $\neg A$ is a function $A \to \emptyt$, which we may construct by assuming $x : A$ and deriving an element of~$\emptyt$.
\index{proof!by contradiction}%
\index{logic!constructive vs classical}
Note that although the logic we obtain is ``constructive'', as discussed in the introduction, this sort of ``proof by contradiction'' (assume $A$ and derive a contradiction, concluding $\neg A$) is perfectly valid constructively: it is simply invoking the \emph{meaning} of ``negation''.
The sort of ``proof by contradiction'' which is disallowed is to assume $\neg A$ and derive a contradiction as a way of proving $A$.
Constructively, such an argument would only allow us to conclude $\neg\neg A$, and the reader can verify that there is no obvious way to get from $\neg\neg A$ (that is, from $(A\to \emptyt)\to\emptyt$) to $A$.

\mentalpause

The above translation of logical connectives into type-forming operations is referred to as \define{propositions as types}: it gives us a way to translate propositions and their proofs, written in English, into types and their elements.
For example, suppose we want to prove the following tautology (one of ``de Morgan's laws''):
\index{law!de Morgan's|(}%
\index{de Morgan's laws|(}%
\begin{equation}\label{eq:tautology1}
  \text{\emph{``If not $A$ and not $B$, then not ($A$ or $B$)''}.}
\end{equation}
An ordinary English proof of this fact might go as follows.
\begin{quote}
  Suppose not $A$ and not $B$, and also suppose $A$ or $B$; we will derive a contradiction.
  There are two cases.
  If $A$ holds, then since not $A$, we have a contradiction.
  Similarly, if $B$ holds, then since not $B$, we also have a contradiction.
  Thus we have a contradiction in either case, so not ($A$ or $B$).
\end{quote}
Now, the type corresponding to our tautology~\eqref{eq:tautology1}, according to the rules given above, is
\begin{equation}\label{eq:tautology2}
  (A\to \emptyt) \times (B\to\emptyt) \to (A+B\to\emptyt)
\end{equation}
so we should be able to translate the above proof into an element of this type.

As an example of how such a translation works, let us describe how a mathematician reading the above English proof might simultaneously construct, in his or her head, an element of~\eqref{eq:tautology2}.
The introductory phrase ``Suppose not $A$ and not $B$'' translates into defining a function, with an implicit application of the recursion principle for the cartesian product in its domain $(A\to\emptyt)\times (B\to\emptyt)$.
This introduces unnamed variables
\index{variable}%
(hypotheses)
\index{hypothesis}%
of types $A\to\emptyt$ and $B\to\emptyt$.
When translating into type theory, we have to give these variables names; let us call them $x$ and $y$.
At this point our partial definition of an element of~\eqref{eq:tautology2} can be written as
\[ f((x,y)) \defeq\; \Box\;:A+B\to\emptyt \]
with a ``hole'' $\Box$ of type $A+B\to\emptyt$ indicating what remains to be done.
(We could equivalently write $f \defeq \rec{(A\to\emptyt)\times (B\to\emptyt)}(A+B\to\emptyt,\lam{x}{y} \Box)$, using the recursor instead of pattern matching.)
The next phrase ``also suppose $A$ or $B$; we will derive a contradiction'' indicates filling this hole by a function definition, introducing another unnamed hypothesis $z:A+B$, leading to the proof state:
\[ f((x,y))(z) \defeq \;\Box\; :\emptyt \]
Now saying ``there are two cases'' indicates a case split, i.e.\ an application of the recursion principle for the coproduct $A+B$.
If we write this using the recursor, it would be
\[ f((x,y))(z) \defeq \rec{A+B}(\emptyt,\lam{a} \Box,\lam{b}\Box,z) \]
while if we write it using pattern matching, it would be
\begin{align*}
  f((x,y))(\inl(a)) &\defeq \;\Box\;:\emptyt\\
  f((x,y))(\inr(b)) &\defeq \;\Box\;:\emptyt.
\end{align*}
Note that in both cases we now have two ``holes'' of type $\emptyt$ to fill in, corresponding to the two cases where we have to derive a contradiction.
Finally, the conclusion of a contradiction from $a:A$ and $x:A\to\emptyt$ is simply application of the function $x$ to $a$, and similarly in the other case.
\index{application!of hypothesis or theorem}%
(Note the convenient coincidence of the phrase ``applying a function'' with that of ``applying a hypothesis'' or theorem.)
Thus our eventual definition is
\begin{align*}
  f((x,y))(\inl(a)) &\defeq x(a)\\
  f((x,y))(\inr(b)) &\defeq y(b).
\end{align*}

As an exercise, you should verify 
the converse tautology \emph{``If not ($A$ or $B$), then  (not $A$) and (not $B$)}'' by exhibiting an element of 
\[ ((A + B) \to \emptyt) \to (A \to \emptyt) \times (B \to \emptyt), \]
for any types $A$ and $B$, using the rules we have just introduced.

\index{logic!classical vs constructive|(}
However, not all classical\index{mathematics!classical} tautologies hold under this interpretation.
For example, the rule 
\emph{``If not ($A$ and $B$), then (not $A$) or (not $B$)''} is not valid: we cannot, in general, construct an element of the corresponding type
\[ ((A \times B) \to \emptyt) \to (A \to \emptyt) + (B \to \emptyt).\]
This reflects the fact that the ``natural'' propositions-as-types logic of type theory is \emph{constructive}.
This means that it does not include certain classical principles, such as the law of excluded middle (\LEM{})\index{excluded middle}
or proof by contradiction,\index{proof!by contradiction}
and others which depend on them, such as this instance of de Morgan's law.
\index{law!de Morgan's|)}%
\index{de Morgan's laws|)}%

Philosophically, constructive logic is so-called because it confines itself to constructions that can be carried out \emph{effectively}, which is to say those with a computational meaning.
Without being too precise, this means there is some sort of algorithm\index{algorithm} specifying, step-by-step, how to build an object (and, as a special case, how to see that a theorem is true).
This requires omission of \LEM{}, since there is no \emph{effective}\index{effective!procedure} procedure for deciding whether a proposition is true or false.

The constructivity of type-theoretic logic means it has an intrinsic computational meaning, which is of interest to computer scientists.
It also means that type theory provides \emph{axiomatic freedom}.\index{axiomatic freedom}
For example, while by default there is no construction witnessing \LEM{}, the logic is still compatible with the existence of one (see \autoref{sec:intuitionism}).
Thus, because type theory does not \emph{deny} \LEM{}, we may consistently add it as an assumption, and work conventionally without restriction.
In this respect, type theory enriches, rather than constrains, conventional mathematical practice.

We encourage the reader who is unfamiliar with constructive logic to work through some more examples as a means of getting familiar with it.
See \autoref{ex:tautologies,ex:not-not-lem} for some suggestions.
\index{logic!classical vs constructive|)}

\mentalpause

So far we have discussed only propositional logic.
\index{quantifier}%
\index{quantifier!existential}%
\index{quantifier!universal}%
\index{predicate!logic}%
\index{logic!predicate}%
Now we consider \emph{predicate} logic, where in addition to logical connectives like ``and'' and ``or'' we have quantifiers ``there exists'' and ``for all''.
In this case, types play a dual role: they serve as propositions and also as types in the conventional sense, i.e., domains we quantify over.
A predicate over a type $A$ is represented as a family $P : A \to \UU$, assigning to every element $a : A$ a type $P(a)$ corresponding to the proposition that $P$ holds for $a$. We now extend the above translation with an explanation of the quantifiers:
\begin{center}
  \medskip
  \begin{tabular}{ll}
    \toprule
    English & Type Theory\\
    \midrule
    For all $x:A$, $P(x)$ holds & $\prd{x:A} P(x)$ \\
    There exists $x:A$ such that $P(x)$ & $\sm{x:A}$ $P(x)$ \\
    \bottomrule
  \end{tabular}
  \medskip
\end{center}
As before, we can show that tautologies of (constructive) predicate logic translate into inhabited types.
For example, \emph{If for all $x:A$, $P(x)$ and $Q(x)$ then (for all $x:A$, $P(x)$) and (for all $x:A$, $Q(x)$)} translates to
\[ (\tprd{x:A} P(x) \times Q(x)) \to (\tprd{x:A} P(x)) \times (\tprd{x:A} Q(x)). \]
An informal proof of this tautology might go as follows:
\begin{quote}
  Suppose for all $x$, $P(x)$ and $Q(x)$.
  First, we suppose given $x$ and prove $P(x)$.
  By assumption, we have $P(x)$ and $Q(x)$, and hence we have $P(x)$.
  Second, we suppose given $x$ and prove $Q(x)$.
  Again by assumption, we have $P(x)$ and $Q(x)$, and hence we have $Q(x)$.
\end{quote}
The first sentence begins defining an implication as a function, by introducing a witness for its hypothesis:\index{hypothesis}
\[ f(p) \defeq \;\Box\; : (\tprd{x:A} P(x)) \times (\tprd{x:A} Q(x)). \]
At this point there is an implicit use of the pairing constructor to produce an element of a product type, which is somewhat signposted in this example by the words ``first'' and ``second'':
\[ f(p) \defeq \Big( \;\Box\; : \tprd{x:A} P(x) \;,\; \Box\; : \tprd{x:A}Q(x) \;\Big). \]
The phrase ``we suppose given $x$ and prove $P(x)$'' now indicates defining a \emph{dependent} function in the usual way, introducing a variable
\index{variable}%
for its input.
Since this is inside a pairing constructor, it is natural to write it as a $\lambda$-abstraction\index{lambda abstraction@$\lambda$-abstraction}:
\[ f(p) \defeq \Big( \; \lam{x} \;\big(\Box\; : P(x)\big) \;,\; \Box\; : \tprd{x:A}Q(x) \;\Big). \]
Now ``we have $P(x)$ and $Q(x)$'' invokes the hypothesis, obtaining $p(x) : P(x)\times Q(x)$, and ``hence we have $P(x)$'' implicitly applies the appropriate projection:
\[ f(p) \defeq \Big( \; \lam{x} \proj1(p(x))  \;,\; \Box\; : \tprd{x:A}Q(x) \;\Big). \]
The next two sentences fill the other hole in the obvious way:
\[ f(p) \defeq \Big( \; \lam{x} \proj1(p(x))  \;,\; \lam{x} \proj2(p(x)) \; \Big). \]
Of course, the English proofs we have been using as examples are much more verbose than those that mathematicians usually use in practice; they are more like the sort of language one uses in an ``introduction to proofs'' class.
The practicing mathematician has learned to fill in the gaps, so in practice we can omit plenty of details, and we will generally do so.
The criterion of validity for proofs, however, is always that they can be translated back into the construction of an element of the corresponding type.

\symlabel{leq-nat}%
As a more concrete example, consider how to define inequalities of natural numbers.
One natural definition is that $n\le m$ if there exists a $k:\nat$ such that $n+k=m$.
(This uses again the identity types that we will introduce in the next section, but we will not need very much about them.)
Under the propositions-as-types translation, this would yield:
\[ (n\le m) \defeq \sm{k:\nat} (\id{n+k}{m}). \]
The reader is invited to prove the familiar properties of $\le$ from this definition.
For strict inequality, there are a couple of natural choices, such as
\[ (n<m) \defeq \sm{k:\nat} (\id{n+\suc(k)}{m}) \]
or
\[ (n<m) \defeq (n\le m) \times \neg(\id{n}{m}). \]
The former is more natural in constructive mathematics, but in this case it is actually equivalent to the latter, since $\nat$ has ``decidable equality'' (see \autoref{sec:intuitionism,prop:nat-is-set}).
\index{decidable!equality}%

There is also another interpretation of the type $\sm{x:A} P(x)$.
Since an inhabitant of it is an element $x:A$ together with a witness that $P(x)$ holds, instead of regarding $\sm{x:A} P(x)$ as the proposition ``there exists an $x:A$ such that $P(x)$'', we can regard it as ``the type of all elements $x:A$ such that $P(x)$'', i.e.\ as a ``subtype'' of $A$.
\index{subtype}%

We will return to this interpretation in \autoref{subsec:prop-subsets}.
For now, we note that it allows us to incorporate axioms into the definition of types as mathematical structures which we discussed in \autoref{sec:sigma-types}.
For example, suppose we want to define a \define{semigroup}\index{semigroup} to be a type $A$ equipped with a binary operation $m:A\to A\to A$ (that is, a magma\index{magma}) and such that for all $x,y,z:A$ we have $m(x,m(y,z)) = m(m(x,y),z)$.
This latter proposition is represented by the type
\[\prd{x,y,z:A} m(x,m(y,z)) = m(m(x,y),z),\]
so the type of semigroups is
\[ \semigroup \defeq \sm{A:\UU}{m:A\to A\to A} \prd{x,y,z:A} m(x,m(y,z)) = m(m(x,y),z), \]
i.e.\ the subtype of $\mathsf{Magma}$ consisting of the semigroups.
From an inhabitant of $\semigroup$ we can extract the carrier $A$, the operation $m$, and a witness of the axiom, by applying appropriate projections.
We will return to this example in \autoref{sec:equality-of-structures}.

Note also that we can use the universes in type theory to represent ``higher order logic'' --- that is, we can quantify over all propositions or over all predicates.
For example, we can represent the proposition \emph{for all properties $P : A \to \UU$, if $P(a)$ then $P(b)$} as
\[ \prd{P : A \to \UU} P(a) \to P(b) \]
where $A : \UU$ and $a,b : A$.
However, \emph{a priori} this proposition lives in a different, higher, universe than the
propositions we are quantifying over; that is
\[ \Parens{\prd{P : A \to \UU_i} P(a) \to P(b)} : \UU_{i+1}. \]
We will return to this issue in \autoref{subsec:prop-subsets}.

\mentalpause

We have described here a ``proof-relevant''
\index{mathematics!proof-relevant}%
translation of propositions, where the proofs of disjunctions and existential statements carry some information.
For instance, if we have an inhabitant of $A+B$, regarded as a witness of ``$A$ or $B$'', then we know whether it came from $A$ or from $B$.
Similarly, if we have an inhabitant of $\sm{x:A} P(x)$, regarded as a witness of ``there exists $x:A$ such that $P(x)$'', then we know what the element $x$ is (it is the first projection of the given inhabitant).

As a consequence of the proof-relevant nature of this logic, we may have ``$A$ if and only if $B$'' (which, recall, means $(A\to B)\times (B\to A)$), and yet the types $A$ and $B$ exhibit different behavior.
For instance, it is easy to verify that ``$\mathbb{N}$ if and only if $\unit{}$'', and yet clearly $\mathbb{N}$ and $\unit{}$ differ in important ways.
The statement ``$\mathbb{N}$ if and only if $\unit{}$'' tells us only that \emph{when regarded as a mere proposition}, the type $\mathbb{N}$ represents the same proposition as $\unit{}$ (in this case, the true proposition).
We sometimes express ``$A$ if and only if $B$'' by saying that $A$ and $B$ are \define{logically equivalent}.
\indexdef{logical equivalence}%
\indexdef{equivalence!logical}%
This is to be distinguished from the stronger notion of \emph{equivalence of types} to be introduced in \autoref{sec:basics-equivalences,cha:equivalences}:
although $\mathbb{N}$ and $\unit{}$ are logically equivalent, they are not equivalent types.

In \autoref{cha:logic} we will introduce a class of types called ``mere propositions'' for which equivalence and logical equivalence coincide.
Using these types, we will introduce a modification to the above-described logic that is sometimes appropriate, in which the additional information contained in disjunctions and existentials is discarded.

Finally, we note that the propositions-as-types correspondence can be viewed in reverse, allowing us to regard any type $A$ as a proposition, which we prove by exhibiting an element of $A$.
Sometimes we will state this proposition as ``$A$ is \define{inhabited}''.
\indexdef{inhabited type}%
\indexsee{type!inhabited}{inhabited type}%
That is, when we say that $A$ is inhabited, we mean that we have given a (particular) element of $A$, but that we are choosing not to give a name to that element.
Similarly, to say that $A$ is \emph{not inhabited} is the same as to give an element of $\neg A$.
In particular, the empty type $\emptyt$ is obviously not inhabited, since $\neg \emptyt \jdeq (\emptyt \to \emptyt)$ is inhabited by $\idfunc[\emptyt]$.\footnote{This should not be confused with the statement that type theory is consistent, which is the \emph{meta-theoretic} claim that it is not possible to obtain an element of $\emptyt$ by following the rules of type theory.\indexfoot{consistency}}

\index{proof|)}%
\index{proposition!as types|)}%
\index{logic!propositions as types|)}%

\subsection{Identity types}
\label{subsec:identity-types}

\index{type!identity|(defstyle}%
\indexsee{identity!type}{type, identity}%
\indexsee{type!equality}{type, identity}%
\indexsee{equality!type}{type, identity}%
While the previous constructions can be seen as generalizations of
standard set theoretic constructions, our way of handling identity  seems to be
specific to type theory.
According to the propositions-as-types conception, the \emph{proposition} that two elements of the same type $a,b:A$ are equal must correspond to some \emph{type}.
Since this proposition depends on what $a$ and $b$ are, these \define{equality types} or \define{identity types} must be type families dependent on two copies of $A$.

We may write the family as $\idtypevar{A}:A\to A\to\type$, so that $\idtype[A]ab$ is the type representing the proposition of equality between $a$ and $b$.
Once we are familiar with propositions-as-types, however, it is convenient to also use the standard equality symbol for this; thus ``$\id{a}{b}$'' will also be a notation for the \emph{type} $\idtype[A]ab$ corresponding to the proposition that $a$ equals $b$.
For clarity, we may also write ``$\id[A]{a}{b}$'' to specify the type $A$.
If we have an element of $\id[A]{a}{b}$, we may say that $a$ and $b$ are \define{equal}, or sometimes \define{propositionally equal} if we want to emphasize that this is different from the judgmental equality $a\jdeq b$ discussed in \autoref{sec:types-vs-sets}.
\indexdef{equality!propositional}%
\indexdef{propositional!equality}%

Just as we remarked in \autoref{sec:pat} that the propositions-as-types versions of ``or'' and ``there exists'' can include more information than just the fact that the proposition is true, nothing prevents  the type $\id{a}{b}$ from also including more information.
Indeed, this is the cornerstone of the homotopical interpretation, where we regard witnesses of $\id{a}{b}$ as \emph{paths}\indexdef{path} or \emph{equivalences} between $a$ and $b$ in the space $A$.  Just as there can be more than one path between two points of a space, there can be more than one witness that two objects are equal.  Put differently, we may regard $\id{a}{b}$ as the type of \emph{identifications}\indexdef{identification} of $a$ and $b$, and there may be many different ways in which $a$ and $b$ can be identified.
We will return to the interpretation in \autoref{cha:basics}; for now we focus on the basic rules for the identity type.
Just like all the other types considered in this chapter, it will have rules for formation, introduction, elimination, and computation, which behave formally in exactly the same way.

The formation rule says that given a type $A:\UU$ and two elements $a,b:A$, we can form the type $(\id[A]{a}{b}):\UU$ in the same universe.
The basic way to construct an element of $\id{a}{b}$ is to know that $a$ and $b$ are the same.
Thus, the introduction rule is a dependent function
\[\refl{} : \prd{a:A} (\id[A]{a}{a})\]
called \define{reflexivity},
\indexdef{reflexivity!of equality}%
which says that every element of $A$ is equal to itself (in a specified way).  We regard $\refl{a}$ as being the
constant path\indexdef{path!constant}\indexsee{loop!constant}{path, constant}
at the point $a$.

In particular, this means that if $a$ and $b$ are \emph{judgmentally} equal, $a\jdeq b$, then we also have an element $\refl{a} : \id[A]{a}{b}$.
This is well-typed because $a\jdeq b$ means that also the type $\id[A]{a}{b}$ is judgmentally equal to $\id[A]{a}{a}$, which is the type of $\refl{a}$.

The induction principle (i.e.\ the elimination rule) for the identity types is one of the most subtle parts of type theory, and crucial to the homotopy interpretation.
We begin by considering an important consequence of it, the principle that ``equals may be substituted for equals'', as expressed by the following:
\index{indiscernability of identicals}%
\index{equals may be substituted for equals}%
\begin{description}
\item[Indiscernability of identicals:]
For every family 
\[
C : A \to \UU
\]
there is a function
\[
f : \prd{x,y:A}{p:\id[A] x y} C(x) \to C(y)
\]
such that
\[
f(x,x,\refl{x}) \defeq \idfunc[C(x)].
\]
\end{description}
This says that every family of types $C$ respects equality, in the sense that applying $C$ to \emph{equal} elements of $A$ also results in a function between the resulting types. The displayed equality states that the function associated to reflexivity is the identity function (and we shall see that, in general, the function $f(x,y,p): C(x) \to C(y)$ is always an equivalence of types).

Indiscernability of identicals can be regarded as a recursion principle for the identity type, analogous to those given for booleans and natural numbers above.
Just as $\rec{\nat}$ gives a specified map $\nat\to C$ for any other type $C$ of a certain sort, indiscernability of identicals gives a specified map from $\id[A] x y$ to certain other reflexive, binary relations on $A$, namely those of the form $C(x) \to C(y)$ for some unary predicate $C(x)$.
We could also formulate a more general recursion principle with respect to reflexive relations of the more general form $C(x,y)$.
However, in order to fully characterize the identity type, we must generalize this recursion principle to an induction principle, which not only considers maps out of $\id[A] x y$ but also families over it.
Put differently, we consider not only allowing equals to be substituted for equals, but also taking into account the evidence $p$ for the equality.
    
\subsection{Path induction}

\index{generation!of a type, inductive|(}
The induction principle for the identity type is called \define{path induction},
\index{path!induction|(}%
\index{induction principle!for identity type|(}%
in view of the homotopical interpretation to be explained in  the introduction to \autoref{cha:basics}.  It can be seen as stating that the family of identity types is freely generated by the elements of the form $\refl{x}: \id{x}{x}$.

\begin{description}
\item[Path induction:] 
  Given a family 
  \[ C : \prd{x,y:A} (\id[A]{x}{y}) \to \UU \]
  and a function
  \[ c :  \prd{x:A} C(x,x,\refl{x}),\]
  there is a function
  \[ f : \prd{x,y:A}{p:\id[A]{x}{y}} C(x,y,p) \]
  such that 
  \[ f(x,x,\refl{x}) \defeq c(x). \]
\end{description}

Note that just like the induction principles for products, coproducts, natural numbers, and so on, path induction allows us to define \emph{specified} functions which exhibit appropriate computational behavior.
Just as we have \emph{the} function $f:\nat\to C$ defined by recursion from $c_0:C$ and $c_s:\nat \to C \to C$, which moreover satisfies $f(0)\jdeq c_0$ and $f(\suc(n))\jdeq c_s(n,f(n))$, we have \emph{the} function $f : \dprd{x,y:A}{p:\id[A]{x}{y}} C(x,y,p)$ defined by path induction from $c :  \prd{x:A} C(x,x,\refl{x})$, which moreover satisfies $f(x,x,\refl{x}) \jdeq c(x)$.

To understand the meaning of this principle, consider first the simpler case when $C$
does not depend on $p$.  Then we have $C:A\to A\to \UU$, which we may
regard as a predicate depending on two elements of $A$.  We are
interested in knowing when the proposition $C(x,y)$ holds for some pair
of elements $x,y:A$.  In this case, the hypothesis of path induction
says that we know $C(x,x)$ holds for all $x:A$, i.e.\ that if we
evaluate $C$ at the pair $x, x$, we get a true proposition --- so $C$ is
a reflexive relation.  The conclusion then tells us that $C(x,y)$ holds
whenever $\id{x}{y}$.  This is exactly the more general recursion principle
for reflexive relations mentioned above.

The general, inductive form of the rule allows $C$ to also depend on the witness $p:\id{x}{y}$ to the identity between $x$ and $y$.  In the premise, we not only replace $x, y$ by $x,x$, but also simultaneously replace $p$ by reflexivity: to prove a property for all elements $x,y$ and paths $p : \id{x}{y}$ between them, it suffices to consider all the cases where the elements are $x,x$ and the path is $\refl{x}: \id{x}{x}$.  If we were viewing types just as sets, it would be unclear what this buys us, but since there may be many different identifications $p : \id{x}{y}$ between $x$ and $y$, it makes sense to keep track of them in considering families over the type $\id[A]{x}{y}$.
In \autoref{cha:basics} we will see that this is very important to the homotopy interpretation.

If we package up path induction into a single function, it takes the form:
\symlabel{defn:induction-ML-id}%
\begin{narrowmultline*}
  \indid{A} :  \dprd{C : \prd{x,y:A} (\id[A]{x}{y}) \to \UU}
  \Parens{\tprd{x:A} C(x,x,\refl{x})} \to 
  \narrowbreak
  \dprd{x,y:A}{p:\id[A]{x}{y}}   C(x,y,p)
\end{narrowmultline*}
with the equality\index{computation rule!for identity types}
\[ \indid{A}(C,c,x,x,\refl{x}) \defeq c(x). \]
The function $ \indid{A}$ is traditionally called $J$.
\indexsee{J@$J$}{induction principle for identity type}%
We leave it as an easy exercise to show that indiscernability of identicals follows from path induction.  

\mentalpause

Given a proof $p : \id{a}{b}$,
path induction requires us to replace \emph{both} $a$ and $b$ with the same unknown element $x$; thus in order to define an element of a family
$C$, for all pairs of elements of $A$, it suffices to define it on the diagonal.
In some proofs, however, it is simpler to make use of an equation $p : \id{a}{b}$ by replacing all occurrences of $b$ with $a$ (or vice versa), because it is sometimes easier to do the remainder of the proof for the specific element $a$ mentioned in the equality than for a general unknown $x$.  This motivates a second induction principle for identity types, which says that the family of types $\id[A]{a}{x}$ is generated by the element $\refl{a} : \id{a}{a}$.  As we show below, this second principle is equivalent to the first; it is just sometimes a more convenient formulation.

\index{path!induction based}%
\index{induction principle!for identity type!based}%
\begin{description}
\item[Based path induction:] 
  Fix an element $a:A$, and suppose given a family
  \[ C : \prd{x:A} (\id[A]{a}{x}) \to \UU \]
  and an element
  \[ c : C(a,\refl{a}). \]
  Then we obtain a function
  \[ f : \prd{x:A}{p:\id{a}{x}} C(x,p) \]
  such that
  \[ f(a,\refl{a}) \defeq c.\]
\end{description}

Here, $C(x,p)$ is a family of types, where $x$ is an element of $A$ and $p$ is an element of the identity type $\id[A]{a}{x}$, for fixed $a$ in $A$. The based path induction principle says that to define an element of this family for all $x$ and $p$, it suffices to consider
just the case where $x$ is $a$ and $p$ is $\refl{a} : \id{a}{a}$.

Packaged as a function, based path induction becomes:
\symlabel{defn:induction-PM-id}%
\begin{align*}
  \indidb{A} :  \dprd{a:A}{C : \prd{x:A} (\id[A]{a}{x}) \to \UU}
  C(a,\refl{a}) \to \dprd{x:A}{p : \id[A]{a}{x}} C(x,p) 
\end{align*}
with the equality
\[ \indidb{A}(a,C,c,a,\refl{a}) \defeq c. \]
%\[ g(x)(x,\refl{x}) \defeq d(x) \]

Below, we show that path induction and based path induction are equivalent.  Because of this, we will sometimes be sloppy and also refer to based path induction simply as ``path induction'', relying on the reader to infer which principle is meant from the form of the proof.

\begin{rmk}\label{rmk:the-only-path-is-refl}
Intuitively, the induction principle for the natural numbers expresses the fact that the only natural numbers are $0$ and $\suc(n)$, so if we prove a property for these cases, then we have proved it for all natural numbers.  Applying this same reading to path induction, we might loosely say that path induction expresses the fact that the only path is \refl{}, so if we prove a property for reflexivity, then we have proved it for all paths.  However, this reading is quite confusing in the context of the homotopy interpretation of paths, where there may be many different ways in which two elements $a$ and $b$ can be identified, and therefore many different elements of the identity type!  How can there be many different paths, but at the same time we have an induction principle asserting that the only path is reflexivity?

The key observation is that it is not the identity \emph{type} that is inductively defined, but the identity \emph{family}.
In particular, path induction says that the \emph{family} of types $(\id[A]{x}{y})$, as $x,y$ vary over all elements of $A$, is inductively defined by the elements of the form $\refl{x}$.
This means that to give an element of any other family $C(x,y,p)$ dependent on a \emph{generic} element $(x,y,p)$ of the identity family, it suffices to consider the cases of the form $(x,x,\refl{x})$.
In the homotopy interpretation, this says that the type of triples $(x,y,p)$, where $x$ and $y$ are the endpoints of the path $p$ (in other words, the $\Sigma$-type $\sm{x,y:A}(\id{x}{y})$), is inductively generated by the constant loops\index{path!constant} at each point $x$.
As we will see in \autoref{cha:basics}, in homotopy theory the space corresponding to $\sm{x,y:A}(\id{x}{y})$ is the \emph{free path space} --- the space of paths in $A$ whose endpoints may vary --- and it is in fact the case that any point of this space is homotopic to the constant loop at some point, since we can simply retract one of its endpoints along the given path.
The analogous fact is also true in type theory: we can prove by path induction on $p:x=y$ that $\id[\sm{x,y:A}(\id{x}{y})]{(x,y,p)}{(x,x,\refl{x})}$.

Similarly, based path induction says that for a fixed $a:A$, the \emph{family} of types $(\id[A]{a}{y})$, as $y$ varies over all elements of $A$, is inductively defined by the element $\refl{a}$.
Thus, to give an element of any other family $C(y,p)$ dependent on a generic element $(y,p)$ of this family, it suffices to consider the case $(a,\refl{a})$.
Homotopically, this expresses the fact that the space of paths starting at some chosen point (the \emph{based path space} at that point, which type-theoretically is $\sm{y:A} (\id{a}{y})$) is contractible to the constant loop on the chosen point.
Again, the corresponding fact is also true in type theory: we can prove by based path induction on $p:a=y$ that $\id[\sm{y:A}(\id{a}{y})]{(y,p)}{(a,\refl{a})}$.
Note also that according to the interpretation of $\Sigma$-types as subtypes mentioned in \autoref{sec:pat}, the type $\sm{y:A}(\id{a}{y})$ can be regarded as ``the type of all elements of $A$ which are equal to $a$'', a type-theoretic version of the ``singleton\index{type!singleton} subset'' $\{a\}$.

Neither path induction nor based path induction provides a way to give an element of a family $C(p)$ where $p$ has \emph{two fixed endpoints} $a$ and $b$.
In particular, for a family $C: (\id[A]{a}{a}) \to \UU$ dependent on a loop, we \emph{cannot} apply path induction and consider only the case for $C(\refl{a})$, and consequently, we cannot prove that all loops are reflexivity.
Thus, inductively defining the identity family does not prohibit non-reflexivity paths in specific instances of the identity type.
In other words, a path $p:\id{x}{x}$ may  be not equal to reflexivity as an element of $(\id{x}{x})$, but the pair $(x,p)$ will nevertheless be equal to the pair $(x,\refl{x})$ as elements of $\sm{y:A}(\id{x}{y})$.
\end{rmk}

\index{path!induction|)}%
\index{induction principle!for identity type|)}%
\index{generation!of a type, inductive|)}

\subsection{Equivalence of path induction and based path induction}

The two induction principles for the identity type introduced above are equivalent.
It is easy to see that path induction follows from the based path induction principle.
Indeed, let us assume the premises of path induction:
\begin{align*}
C &: \prd{x,y:A}(\id[A]{x}{y}) \to \UU,\\
c &: \prd{x:A} C(x,x,\refl{x}).
\end{align*}
Now, given an element $x:A$, we can instantiate both of the above, obtaining
\begin{align*}
C' &: \prd{y:A} (\id[A]{x}{y}) \to \UU,  \\
C' &\defeq C(x), \\
c' &: C'(x,\refl{x}), \\
c' &\defeq c(x).
\end{align*}
Clearly, $C'$ and $c'$ match the premises of based path induction and hence we can construct 
\begin{equation*}
  g : \prd{y:A}{p : \id{x}{y}} C'(y,p)
\end{equation*}
with the defining equality
\[ g(x,\refl{x}) \defeq c'.\]
Now we observe that $g$'s codomain is equal to $C(x,y,p)$.
Thus, discharging our assumption $x:A$, we can derive a function 
\[ f : \prd{x,y:A}{p : \id[A]{x}{y}} C(x,y,p) \]
with the required judgmental equality $f(x,x,\refl{x}) \judgeq g(x,\refl{x}) \defeq c' \defeq c(x)$.

Another proof of this fact is to observe that any such $f$ can be obtained as an instance of $\indidb{A}$
so it suffices to define $\indid{A}$ in terms of $\indidb{A}$ as
\[ \indid{A}(C,c,x,y,p) \defeq \indidb{A}(x,C(x),c(x),y,p). \]

The other direction is a bit trickier; it is not clear how we can use a particular instance of path induction to derive a particular instance of
based path induction. What we can do instead is to construct one instance of path induction which shows 
all possible instantiations of based path induction at once.
Define
\begin{align*}
D &: \prd{x,y:A} (\id[A]{x}{y}) \to \UU, \\
D(x,y,p) &\defeq \prd{C : \prd{z:A} (\id[A]{x}{z}) \to \UU} C(x,\refl{x}) \to C(y,p).
\end{align*}
Then we can construct the function
\begin{align*}
d &: \prd{x : A} D(x,x,\refl{x}), \\
d &\defeq \lamu{x:A}\lamu{C:\prd{z:A}{p : \id[A]{x}{z}} \UU}\lam{c:C(x,\refl{x})} c
\end{align*}
and hence using path induction obtain
\[ f : \prd{x,y:A}{p:\id[A]{x}{y}} D(x,y,p) \]
with $f(x,x,\refl{x}) \defeq d(x)$. Unfolding the definition of $D$, we can expand the type of $f$:
\[ f : \prd{x,y:A}{p:\id[A]{x}{y}}{C : \prd{z:A} (\id[A]{x}{z}) \to \UU} C(x,\refl{x}) \to C(y,p). \]
Now given $x:A$ and $p:\id[A]{a}{x}$, we can derive the conclusion of based path induction:
\[ f(a,x,p,C,c) : C(x,p). \]
Notice that we also obtain the correct definitional equality.

Another proof is to observe that any use of based path induction is an instance of $\indidb{A}$  and to define
\begin{narrowmultline*}
\indidb{A}(a,C,c,x,p) \defeq \narrowbreak
\indid{A}
  \begin{aligned}[t]
    \big(
    &\big(\lamu{x,y:A}{p:\id[A]{x}{y}} \tprd{C : \prd{z:A} (\id[A]{x}{z}) \to \UU} C(x,\refl{x}) \to C(y,p) \big),\\
    &(\lamu{x:A}{C:\prd{z:A} (\id[A]{x}{z}) \to \UU}{d:C(x,\refl{x})} d),
     a, x, p, C, c \big) 
   \end{aligned}
\end{narrowmultline*}


Note that the construction given above uses universes. That is, if we want to
model $\indidb{A}$ with $C : \prd{x:A} (\id[A]{a}{x}) \to \UU_i$, we need
to use $\indid{A}$ with 
%
\[ D:\prd{x,y:A} (\id[A]{x}{y}) \to \UU_{i+1} \]
%
since $D$ quantifies over all $C$ of the given type. While this is
compatible with our definition of universes, it is also possible to
derive $\indidb{A}$ without using universes: we can show that $\indid{A}$ entails \autoref{lem:transport,thm:contr-paths}, and that these two principles imply $\indidb{A}$ directly.
We leave the details to the reader as \autoref{ex:pm-to-ml}.

We can use either of the foregoing formulations of identity types
to establish that equality is an equivalence relation, that every function preserves equality and that every family respects equality. We leave the details to the next chapter, where this will be derived  and explained in the context of homotopy type theory.

\begin{rmk}\label{rmk:propeq-vs-jdeq}
  We emphasize that despite having some unfamiliar features, propositional equality is \emph{the} equality of mathematics in homotopy type theory.
  This distinction does not belong to judgmental equality, which is rather a metatheoretic feature of the rules of type theory.
  For instance, the associativity of addition for natural numbers proven in \autoref{sec:inductive-types} is a \emph{propositional} equality, not a judgmental one.
  The same is true of the commutative law (\autoref{ex:add-nat-commutative}).
  Even the very simple commutativity $n+1=1+n$ is not a judgmental equality for a generic $n$ (though it is judgmental for any specific $n$, e.g. $3+1\jdeq 1+3$, since both are judgmentally equal to $4$ by the computation rules defining $+$).
  We can only prove such facts by using the identity type, since we can only apply the induction principle for $\nat$ with a type as output (not a judgment).
\end{rmk}

\subsection{Disequality}
\label{sec:disequality}

Finally, let us also say something about \define{disequality},
\indexdef{disequality}%
which is negation of equality:%
\footnote{We use ``inequality''
  to refer to $<$ and $\leq$. Also, note that this is negation of the \emph{propositional} identity type.
Of course, it makes no sense to negate judgmental equality $\jdeq$, because judgments are not subject to logical operations.}
%
\begin{equation*}
  (x \neq_A y) \ \defeq\ \lnot (\id[A]{x}{y}).
\end{equation*}
If $x\neq y$, we say that $x$ and $y$ are \define{unequal}
\indexdef{unequal}%
or \define{not equal}.
%
Just like negation, disequality plays a less important role here than it does in classical\index{mathematics!classical}
mathematics. For example, we cannot prove that two things are equal by proving that they
are not unequal: that would be an application of the classical law of double negation, see \autoref{sec:intuitionism}.

Sometimes it is useful to phrase disequality in a positive way. For example,
in~\autoref{RD-inverse-apart-0} we shall prove that a real number $x$ has an inverse if,
and only if, its distance from~$0$ is positive, which is a stronger requirement than $x
\neq 0$.

\index{type!identity|)}%

\subsection{Notes}

The type theory presented here is a version of Martin-L\"{o}f's intuitionistic type 
theory~\citep{Martin-Lof-1972,Martin-Lof-1973,Martin-Lof-1979,martin-lof:bibliopolis}, which itself is based on and influenced 
by the foundational work of Brouwer \citep{beeson},
 Heyting~\citep{heyting1966intuitionism},
 Scott~\citep{scott70},
de Bruijn~\citep{deBruijn-1973},%
 Howard~\citep{howard:pat},
 Tait~\citep{Tait-1966,Tait-1968},
 and Lawvere~\citep{lawvere:adjinfound}\index{Lawvere}.
\index{proof!assistant}%
Three principal variants of Martin-L\"{o}f's type theory underlie the \NuPRL \citep{constable+86nuprl-book}, \Coq~\citep{Coq}, and 
\Agda \citep{norell2007towards} computer implementations of type theory.  The theory given here differs from these formulations in a number 
of respects, some of which are critical to the homotopy interpretation, while others are technical conveniences or involve concepts that 
have not yet been studied in the homotopical setting.

\index{type theory!intensional}%
\index{type theory!extensional}%
\index{intensional type theory}%
\index{extensional!type theory}%
Most significantly, the type theory described here is derived from the \emph{intensional} version of Martin-L\"{o}f's type 
theory~\citep{Martin-Lof-1973}, rather than the \emph{extensional} version~\citep{Martin-Lof-1979}.  Whereas the extensional theory makes no 
distinction between judgmental and propositional equality, the intensional theory regards judgmental equality as purely definitional, and 
admits a much broader, proof-relevant interpretation of the identity type that is central to the homotopy interpretation.  From the 
homotopical perspective, extensional type theory confines itself to homotopically discrete sets (see \autoref{sec:basics-sets}), whereas the 
intensional theory admits types with higher-dimensional structure.  The \NuPRL system~\citep{constable+86nuprl-book} is extensional, whereas 
both \Coq~\citep{Coq} and \Agda~\citep{norell2007towards} are intensional.  Among intensional type theories, there are a number of variants 
that differ in the structure of identity proofs.  The most liberal interpretation, on which we rely here, admits a \emph{proof-relevant} 
interpretation of equality, whereas more restricted variants impose restrictions such as \emph{uniqueness of identity proofs 
  (UIP)}~\citep{Streicher93},
\indexsee{UIP}{uniqueness of identity proofs}%
\index{uniqueness!of identity proofs}%
stating that any two proofs of equality are judgmentally equal, and \emph{Axiom K}~\citep{Streicher93},
\index{axiom!Streicher's Axiom K}
stating that 
the only proof of equality is reflexivity (up to judgmental equality).  These additional requirements may be selectively imposed in the \Coq 
and \Agda\ systems.

%(In the terminology of \autoref{cha:hlevels} such a type theory is about $0$-truncated types.)

Another point of variation among intensional theories is the strength of judgmental equality, particularly as regards objects of function type.  Here we include the uniqueness principle\index{uniqueness!principle} ($\eta$-conversion) $f \jdeq \lam{x} f(x)$, as a principle of judgmental equality.  This principle is used, for example, in \autoref{sec:univalence-implies-funext}, to show that univalence implies propositional function extensionality.  Uniqueness principles are sometimes considered for other types.
For instance, the uniqueness principle\index{uniqueness!principle!for product types} for cartesian products would be a judgmental version of the propositional equality $\uppt$ which we constructed in \autoref{sec:finite-product-types}, saying that $u \jdeq (\proj1(u),\proj2(u))$.
This and the corresponding version for dependent pairs would be reasonable choices (which we did not make), but we cannot include all such rules, because the corresponding uniqueness principle for identity types would trivialize all the higher homotopical structure.  So we are \emph{forced} to leave it out, and the question then becomes where to draw the line. With regards to inductive types, we discuss these points further in~\autoref{sec:htpy-inductive}.

It is important for our purposes that (propositional) equality of functions is taken to be \emph{extensional} (in a different sense than that used above!).
This is not a consequence of the rules in this chapter; it will be expressed by \autoref{axiom:funext}.
\index{function extensionality}%
This decision is significant for our purposes, because it specifies that equality of functions is as expected in mathematics.  Although we include \autoref{axiom:funext} as an axiom, it may be derived from the univalence axiom and the uniqueness principle for functions\index{uniqueness!principle!for function types} (see \autoref{sec:univalence-implies-funext}), as well as from the existence of an interval type (see \autoref{thm:interval-funext}).

Regarding inductive types such as products, $\Sigma$-types, coproducts, natural numbers, and so on (see \autoref{cha:induction}), there are additional choices regarding precisely how to  formulate induction and recursion.
\index{pattern matching}%
Formally, one may describe type theory by taking either \emph{pattern matching} or \emph{induction principles} as basic and deriving the other; see \autoref{cha:rules}.
However, pattern matching in general is not yet well understood from the homotopical perspective (in particular, ``nested'' or ``deep'' pattern matching is difficult to make general sense of for higher inductive types).
Moreover, it can be dangerous unless sufficient care is taken: for instance, the form of pattern matching implemented by default in \Agda
\index{proof!assistant!Agda@\textsc{Agda}}%
allows proving Axiom K.
\index{axiom!Streicher's Axiom K}%
For these reasons, we have chosen to regard the induction principle as the basic property of an inductive definition, with pattern matching justified in terms of induction.

\index{proof!assistant!Coq@\textsc{Coq}}%
Unlike the type theory of \Coq, we do not include a primitive type of propositions.  Instead, as discussed in \autoref{sec:pat}, we embrace 
the \emph{propositions-as-types (PAT)} principle, identifying propositions with types.
This was suggested originally by de Bruijn~\citep{deBruijn-1973}, Howard~\citep{howard:pat}, Tait~\citep{Tait-1968}, and Martin-L\"{o}f~\citep{Martin-Lof-1972}.
(Our decision is explained more fully in \autoref{subsec:pat?,subsec:hprops}.)

We do, however, include a full cumulative hierarchy of universes, so that the type formation and equality judgments become instances of the membership and equality judgments for a universe.
As a convenience, we regard objects of a universe as types, rather than as codes for types; in the terminology of \citep{martin-lof:bibliopolis}, this means we use ``Russell-style universes'' rather than ``Tarski-style universes''.
\index{type!universe!Tarski-style}%
\index{type!universe!Russell-style}%
An alternative would be to use Tarski-style universes, with an explicit coercion\index{coercion, universe-raising} function required to make an element $A:\UU$ of a universe into a type $\mathsf{El}(A)$, and just say that the coercion is omitted when working informally.

We also treat the universe hierarchy as cumulative, in that every type in $\UU_i$ is also in $\UU_j$ for each $j\geq i$.
There are different ways to implement cumulativity formally: the simplest is just to include a rule that if $A:\UU_i$ then $A:\UU_j$.
However, this has the annoying consequence that for a type family $B:A\to \UU_i$ we cannot conclude $B:A\to\UU_j$, although we can conclude $\lam{a} B(a) : A\to\UU_j$.
A more sophisticated approach that solves this problem is to introduce a judgmental subtyping relation $<:$ generated by $\UU_i<:\UU_j$, but this makes the type theory more complicated to study.
Another alternative would be to include an explicit coercion function $\uparrow : \UU_i \to \UU_j$, which could be omitted when working informally.

It is also not necessary that the universes be indexed by natural numbers and linearly ordered.
For some purposes, it is more appropriate to assume only that every universe is an element of some larger universe, together with a ``directedness'' property that any two universes are jointly contained in some larger one.
There are many other possible variations, such as including a universe ``$\UU_{\omega}$'' that contains all $\UU_i$ (or even higher ``large cardinal'' type universes), or by internalizing the hierarchy into a type family $\lam{i} \UU_i$.
The latter is in fact done in \Agda.

The path induction principle for identity types was formulated by Martin-L\"{o}f~\citep{Martin-Lof-1972}.
The based path induction rule in the setting of Martin-L\"of type theory is due to Paulin-Mohring \citep{Moh93}; it can be seen as an intensional generalization of the concept of ``pointwise functionality''\index{pointwise!functionality} for hypothetical judgments from \NuPRL~\citep[Section~8.1]{constable+86nuprl-book}.
The fact that Martin-L\"of's rule implies Paulin-Mohring's was proved by Streicher using Axiom K (see~\autoref{sec:hedberg}), by Altenkirch and Goguen as in \autoref{sec:identity-types}, and finally by Hofmann without universes (as in \autoref{ex:pm-to-ml}); see~\citep[\S1.3 and Addendum]{Streicher93}.



  %%%%%%%%%%%%%%%%%%%%%%%%%%%%%%%%%%%%%%%%%%%%%%%%%%%%%%%%%%%%%%%%
  \chapter{Lexicon}

\begin{description}
\item [BHK]
\item [Witness]
\item [Curry-Howard]
\item [Combinator]
\item [Proof obligation]
\item [Induction]
\item [Recursion]
\item [Type]
\item [Token]
\item [Term]
\end{description}



  %%%% Bibliography
  %% \bibliographystyle{halpha}
  %% \phantomsection % black magic to get TOC to point to correct page
  %% \addcontentsline{toc}{part}{\bibname}
  %% \markboth{}{\textsc{Bibliography}}
  %% {\renewcommand{\markboth}[2]{} % Prevent bibliography from resetting the header to something silly
  %% \OPTbibliographyfont

  \chapter{Proof Assistants}

The \textit{specification} language of Coq is Gallina.

The ``vernacular'' is Coq's command language - it allows you to talk
to the Coq system itself.

\section{Coq}

\subsection{Coq'Art}

\enquote{The relation between a program and its type is the same as
  the relation between a proof and that statement it proves.  Thus
  verifying a proof is done by a type verification algorithm.} p. 4

\enquote{An important characteristic of the Caculus of Constructions is that
every type is also a term and also has a type.  The type of a
proposition is called \texttt{Prop}.  For instance, they proposition
\(3\leq 7\) is at the same time the type of all proofs that 3 is
smaller than 7 and a term of type \texttt{Prop}.} p. 4

\enquote{In the same spirit, a \textit{predicate} makes it possible to
  build a parametric proposition...[E]xamples of predicates are the
  predicate `to be a sorted list' with type \texttt{(list Z)->Prop}
  and the binary relation \(\leq\), with type \texttt{Z->Z->Prop}.} p. 4


  %% \bibliography{references}
  %% \bibliographystyle{plainnat}

\end{appendices}

\end{document}
