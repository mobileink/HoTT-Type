%%%%%%%%%%%%%%%%%%%%%%%%%%%%%%%%
\section{Types}
\label{sect:type}

\HoTTB page 27 describes a ``general pattern for introduction of a new
kind of type''.  Martin-L\"{o}f does this too, somewhere.  In \HoTTB,
the list is

\begin{description}
\item [Formation Rules]
\item [Introduction Rules]  or constructors
\item [Elimination Rules] or eliminators
\item [Computation Rules]  ``which express how an eliminator acts on a constructor''
\item [Uniqueness Principle] which ``expresses uniqueness of maps into
  or out of that type.  Optional.
\end{description}


The question is where to place this stuff in the description of \HoTT.
Are these things primitives?  Do they form essential aspects of a
type?  Or in other words, can we have (think of) types without these rules?

\HoTTB introduces them almost as an afterthought, as a Remark in the
third major construction defined in Chapter 1.  But I suspect this is
a mistake or oversight; it looks to me like these rules are indeed
fundamental, essential to the concept of type.  In that case, they
should be presented along with the introduction of the type concept,
rather than in the middle of a description of a particular type.

%%%%%%%%%%%%%%%%%%%%%%%%%%%%%%%%
\section{Terms}
\label{sect:terms}

\begin{ednote}
  ``Terms'' is Awodey's terminology.  More common terminology include:
  witness; inhabitant.  Also proof.
\end{ednote}

``Under the Curry-Howard cor- respondence, one identifies types
with propositions, and terms with proofs...''\cite{awodey_tth}
