\documentclass{article}
\usepackage[fleqn]{amsmath}
\usepackage{amssymb}
\usepackage{hyperref}
\usepackage{fontspec,xltxtra,xunicode}
\usepackage{fontspec}
\defaultfontfeatures{Scale=MatchLowercase}

%% \setmainfont[Mapping=tex-text]{Times New Roman}
%% \setromanfont[Mapping=tex-text]{Times New Roman}
%% \setsansfont[Mapping=tex-text]{Arial}

\setmainfont[Mapping=tex-text]{TeX Gyre Pagella}
%% \setromanfont[Mapping=tex-text]{TeX Gyre Pagella}
%% \setsansfont[Mapping=tex-text]{TeX Gyre Heros}

\usepackage{tikz}
%% \usepackage{pgfplots}
%% \usepackage{pgfplotstable}
%% %% Preamble:
%% \pgfplotsset{width=7cm,compat=1.9}
%% %% \pgfplotsset{xticklabel={\tick},scaled x ticks=false}
%% %% \pgfplotsset{plot coordinates/math parser=false}

\usetikzlibrary{arrows,shapes,patterns,backgrounds,spy}
\usepackage{pgffor}

%% \usepackage{animate}

%% \usepackage{arrayjobx}
%% \usepackage{multido}

%% \usepackage{layouts}

%% \usepackage{etoolbox}

%% \newcounter{mylistcounter}

\hypersetup{%
  colorlinks=true,% hyperlinks will be coloured
  urlcolor=blue,% hyperlink text will be green
}

%%%%%%%%%%%%%%%%%%%%%%%%%%%%%%%%%%%%%%%%%%%%%%%%%%%%%%%%%%%%%%%%
\begin{document}
\title{HoTT Without Vars}
\author{G. Reynolds}
\maketitle
\large

\begin{abstract}
  There are a few notational conveniences of the lambda calculus that
  seem to be missing in the HoTT calculus. Here are some ideas about
  how to add them.  First, a kind of "reverse lambda" operator that
  allows us to refer to arbitrary elements of a type without using a
  name, just like lambda does with functions.  Lambda abstracts;
  reverse lambda specializes.  Second, an explicit type former for
  enumerated types, that allows us to express such types without using
  a name.  Then, a pipeline application operator that allows us to
  string expressions together.  Finally, a ``universe operator'' that
  allows us to refer to a type drawn from a universe in type family
  expressions.

  The online version of this is at
  \href{http://blog.mobileink.com/2015/06/hott-without-vars.html}{HoTT
    Without Vars}.

  Revision 2.
\end{abstract}

\section{Type Specialization}

The type formers of HoTT (\(\times, \to, \prod, \sum\) and the rest)
are very similar to the lambda operator \(\lambda\).  Just as
\(\lambda\) turns an open formula into the name of the associated
function (\(\lambda x.x+1\) names the function associated with
\(x+1\)), the type formers turn expressions into the names of types.
For example, \(\times\) turns \(A, B\) into the type name \(A\times
B\); \(\to\) turns \(A, B\) into the type name \(A\to B\), and so
forth.

Lambda expressions like \(\lambda x.x+1\) are called \emph{lambda
  abstractions}; by extension, we can call type formation expressions
\emph{type abstractions}.  To ``undo'' a lambda abstraction, we use
\emph{application}: \(\lambda x.x+1(2) = 3\).  Since the opposite of
abstraction is \emph{specialization}, it is tempting to think of
application as a specialization; but application involves more than
mere specialization.  If we think of a function (that is, a lambda
abstraction) as a set of ordered pairs, then its specialization should
be a particular ordered pair; but application does not deliver a pair,
it projects the second element of a pair.  So application can be
thought of as a combination of specialization and projection.

Lambda abstractions are treated not only as the names of functions,
but as function definitions.  That's what allows us to write, e.g. \[f
= \lambda x.x+1\]

There is no application operator for type abstractions.  We cannot
undo \(A\to B\) by writing \((A\to B)(x)\), for example.  But we can
\emph{specialize} type abstractions; that is just what expressions
like \(a:A\) do.  One (classical) way to interpret such an expression
is to say that \(a\) denotes an ``element'' of \(A\).  Such an
interpretation presupposes a denotational semantics and thus
implicates a notion of choice; a more detailed gloss would be to say
that ``:'' chooses an element of \(A\) and assigns it to \(a\).  Under
a pragmatic interpretation that eschews denotational semantics, we
might simply stipulate that \(a:A\) means that \(a\) is a token of
type \(A\) and leave it at that.  Either way, we have introduced a
variable, \(a\), in the service of specializing a type abstraction.

Now one of the great advantages of the lambda notation is that it
allows us to dispense with function names.  In fact this notational
convenience was Church's original motivation in introducing
\(\lambda\); he did not realize until later that it formed the basis
of a powerful computation/logical calculus.  Without the lambda
operator we would have to define and name every function we need to
use, so that we can later refer to it by name.  With the lambda
calculus we can just write out the function definition wherever we
need it.

In principle we should be able to do the same thing -- that is,
dispense with intermediate names -- with type abstractions and
specializations.  But the HoTT notation does not currently support
this; if you want to work with an arbitrary element of a type, you
must name it using the ``:'' operator.  And enumerated types (e.g. the
boolean type) cannot be specified without naming.

To address this notational deficiency we propose a specialization
operator for type abstractions, \(\reflectbox{\lambda}\) (reversed
lambda).  Its meaning is exactly analogous to the lambda operator, but
instead of abstracting it specializes.  \(\lambda\) turns a particular
formula into an abstraction; \(\reflectbox{\lambda}\) turns a type
abstraction into a particular.  Under a classical interpretation, the
operator ``selects'' an element of the type to which it is applied;
under a pragmatic interpretation, the combination of the operator
symbol and a type symbol is a symbol of that type.

Examples:

\begin{align}
&f:A\to B & \textit{f names an arbitrary function of type}\ A\to B \\
&\reflectbox{\lambda}A\to B & \textit{\reflectbox{\lambda}A\to B : A\to B} \\
&\reflectbox{\lambda}\mathbb{N} & \textit{an arbitrary natural number}
\end{align}

The usefulness of this becomes more evident with \(\prod\) and
\(\sum\) types.  From the definition of path induction, p. 49:

\begin{align}
&C:\prod\limits_{x,y:A}(x=_Ay)\to \mathcal{U} & \textit{same as:}\ C:\equiv\reflectbox{\lambda}\prod\limits_{x,y:A}\reflectbox{\lambda}\big((x=_Ay)\to \mathcal{U}\big) \label{exp:pi1}\\
&c:\prod\limits_{x:A}C(x,x,refl_x) & \textit{same as:}\ c:\equiv\reflectbox{\lambda}\prod\limits_{x:A}\reflectbox{\lambda}C(x,x,refl_x) \label{exp:pi2}
\end{align}

These may be rewritten as follows using the pipeline application operator
``|'' described below; briefly, \(x|f = f(x)\) :

%% &C :\equiv\reflectbox{\lambda}\prod\limits_{x,y:A}(x=_Ay)\to \mathcal{U} & \textit{} \\
\begin{align}
&c :\equiv\reflectbox{\lambda}\prod\limits_{x:A}x\to\reflectbox{\lambda}\bigg(\reflectbox{\lambda}\sum\limits_{x,x:A}(x=_Ax)\rvert\reflectbox{\lambda}\prod\limits_{x,y:A}(x=_Ay)\to \mathcal{U}\bigg) & \textit{} \label{exp:c1} \\
&c :\equiv\reflectbox{\lambda}\prod\limits_{x:A}x\to\reflectbox{\lambda}\bigg(\reflectbox{\lambda}\sum\limits_{x:A}\big(\sum\limits_{x:A}(x=_Ax)\Big)\rvert\reflectbox{\lambda}\prod\limits_{x,y:A}(x=_Ay)\to \mathcal{U}\bigg) & \textit{} \label{exp:c2}
\end{align}

These are equivalent (I think); (\ref{exp:c2}) is just more explicit than
(\ref{exp:c1}).  They are slightly different than (\ref{exp:pi1}) and
(\ref{exp:pi2}) in that they do not use \(refl_x\).  In both cases,
the entire meaning can be read off the one expression, just as in the
case of a lambda definition; it can be glossed in order: \(c\) is an
element of a function type that takes any \(x:A\), produces a
dependent triple of the form \((x,x,p)\), where \(p\) is \emph{any}
proof that \(x=x\), and feeds that into a function that maps such
triples to a new type.

This notation seems to make it possible to render structure more
explicitly; whether or not this is a better way to write this is beside
the point; the \(\reflectbox{\lambda}\) operator at least makes it
\emph{possible}.

\bigskip
(Of course, the symbol is arbitrary but motivated - it's a reverse
lambda.  We may find a better symbol than this; e.g. upside down
lambda: \(\raisebox{8pt}{\rotatebox{180}{\lambda}}\)  or the like).

\textbf{Caveat:} You cannot just replace named elements with
\(\reflectbox{\lambda}\) expressions willy-nilly, since the latter pick out
\emph{arbitrary} elements.  Where a name occurs multiple times in a
context it must mean the same thing each time; e.g. in the formula for
\(\textsf{ind}=_A\) on p. 49. (reproduced below).

\section{Definition: implicit v. explicit}

A specialization like \(a:A\) says that \(a\) is an \emph{arbitrary}
token of type \(A\).  But arbitrary does not mean indefinite; if we
adopt the classical perspective and treat \(a:A\) as meaning that
\(a\) denotes an arbitrary element of \(A\) we have not said that
\(a\) is \emph{undefined}.  On the contrary, if it is to denote at
all, it must denote something definite.  So we might say that \(a\)
denotes a definite but \emph{unknown} element of \(A\).  Since it is
definite, we can say that \(a\) is \emph{implicitly defined}.  If we
then offer a ``defining equation'' for \(a\), it becomes
\emph{explicitly defined}.

In other words, typing judgments of the form \(a:A\) implicitly define
their token component.

The difference is clear in the case of function types:

\begin{align}
  &f : \mathbb{N}\to \mathbb{N} \label{exp:fimplicit} \\
  &f :\equiv\lambda x:\mathbb{N}.x+1 \label{exp:fexplicit}
\end{align}

Expression (\ref{exp:fimplicit}) implicitly defines \(f\) by giving its
type explicitly and ``judging'' that f has that type; that is, by
assigning \(f\) to a specialization of its type.

Expression (\ref{exp:fexplicit}) then makes that definition explicit.

Since the term ``definition'' thus has two meanings, we will generally
use the term ``specification'' and leave it to the reader to determine
from context which kind of definition is involved.  For example, we
might say ``Given the specification \(a:A\)'', meaning that \(a:A\)
specifies that \(a\) names an implicitly defined element of \(A\).

\section{An Explicit Operator for Enumerated Types}

Enumerated types like \(\mathbf{2}\) do not have an explicit
abstraction operator.

We propose \(\Lambda\).  This allows us to specify ``anonymous''
enumerated types, e.g.

\[\Lambda\{\textit{Mon, Tues, Wed, Thu, Fri, Sat, Sun\}}\]

names a ``Day'' type without naming it.  To name it we can write:

\[\textit{Day} :\equiv\Lambda\{\textit{Mon, Tues, Wed, Thu, Fri, Sat, Sun\}}\]

\section{Pipeline Application}

\begin{align}
\reflectbox{\lambda}\sum\limits_{x,y:A}(x=_Ay)  \raisebox{-4pt}{\(\biggr\rvert\)} \reflectbox{\lambda}\prod\limits_{x,y:A} (x =_A y) \to \mathcal{U}
\end{align}

meaning, apply an element of the \(\Pi\) type to an element of the
\(\Sigma\) type, or feed the latter into the former.  In this case we
can safely omit the specialization operators since they can be inferred
from the context:

\begin{align}
  \sum\limits_{x,y:A}(x=_Ay) \raisebox{-4pt}{\(\biggr\rvert\)} \prod\limits_{x,y:A} (x =_A y) \to \mathcal{U} \label{exp:feed}
\end{align}

Using names, these two expressions are equivalent to the pair of expressions:

\begin{align}
  &C:\prod\limits_{x,y:A} (x =_A y) \to \mathcal{U} &\\
  &C(a,b,p) & \textit{where}\ a,b:A\ \textit{and}\ p:(a=_Ab)
\end{align}

\section{Universe Operator}

The standard way to express a family is \(B:A\to U\).  This is fine if
B and A are relatively simple; \(B(a)\) is a type.  But if B and/or A
are complex this can be awkward.  Sometimes we may want a name to
refer to members of a type family, if only for expository purposes;
e.g. \(B(a) :\equiv \mathcal{P}\).  The prosposal here is that we
introduce a new operator that allows us to name an element of
\(\mathcal{U}\): \(\mathcal{P}^{\mathcal{U}}\).  Read this as
``\(\mathcal{P}\) is an element drawn from \(\mathcal{U}\)''.

This allows us to say e.g. \[B:A\to \mathcal{P}^{\mathcal{U}}\]

This is to be interpreted as ``\(B\) takes each element of \(A\) to an
element of \(\mathcal{U}\), which we name \(\mathcal{P}\)''; it does
\emph{not} mean ``\(B\) takes each element of \(A\) to and element of
\(\mathcal{P}\)''.

\section{Examples}

Path induction (p. 49):

\begin{align}
  &C:\prod\limits_{x,y:A} (x =_A y) \to \mathcal{U} &\\
  &ind=_A: \prod_{C:\prod_{x,y:A}(x=_Ay)\to\mathcal{U}}\Big(\Pi_{(x:A)}C(x,x,refl_x)\Big)\to \prod_{(x,y:A)}\prod_{(p:x=_Ay)} C(x,y,p)
\end{align}

Task: rewrite this without C.  Without any names.



\end{document}
