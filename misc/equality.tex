\documentclass{article}

\usepackage{hottmacros}

\usepackage{fontspec,xltxtra,xunicode}
\usepackage{fontspec}
\defaultfontfeatures{Scale=MatchLowercase}

%% \setmainfont[Mapping=tex-text]{Times New Roman}
%% \setromanfont[Mapping=tex-text]{Times New Roman}
%% \setsansfont[Mapping=tex-text]{Arial}

\setmainfont[Mapping=tex-text]{TeX Gyre Pagella}
%% \setromanfont[Mapping=tex-text]{TeX Gyre Pagella}
%% \setsansfont[Mapping=tex-text]{TeX Gyre Heros}

\usepackage{tikz}
%% \usetikzlibrary{arrows,shapes,patterns,backgrounds,spy}
%% \usepackage{pgffor}

%% \usepackage{animate}

%%%%%%%%%%%%%%%%%%%%%%%%%%%%%%%%%%%%%%%%%%%%%%%%%%%%%%%%%%%%%%%%
\begin{document}
\title{HoTT without Equality}
\maketitle
\tableofcontents
\vfill
\large

Often said that what's distinctive about HoTT is its treatment of
equality.  But equality is an optional concept.  We offer two other
ways to get the same thing, without mentioning equality: unit
families, and successor families.

Actually we could do three: units, successors, and relations.  All
have the same structure but distinct symbols and intuitive
interpretations.  None depend on a concept of equality.

Equality emerges as a special case in the ordinary economy of types
and tokens.  Or rather one case; there's nothing special about it.

The real innovation is the introduction of a novel concept of
``induction'', which is motivated by problems that arise from the
possibility of non-canonical tokens.

Whether ``induction'' is the appropriate term is another matter.

\section{Unit Types}

This perspective is suggested by the observation that the equality
type in HoTT looks a lot like a Unit type, in that it has a single
constructor.

\subsection{Atomic Units}

\begin{align}
  & \UnitA{:}\,\SuccT & \textit{family of unit types from}\ A \\
  & a{:}A,\ \UnitA(a):\Univ & \textit{unit type at a from A} \\
  & p{:}\,\UnitA(a) & \textit{proof of unit type at a} \\
  & S_{\UnitA(a)}\prod\limits_{x{:}S_{A(a)}}\Univ & \textit{family of unit types at elts of unit type at a} \\
  & q{:}\,\UnitA(a) & \textit{proof of unit at a} \\
  & S_{\UnitA(a)}(q):\Univ & \textit{unit at q from unit at a from A}
\end{align}

\subsection{Dependent Unit Types}

\begin{align}
  & \prod\limits_{X:\Univ}\prod\limits_{(x,y:X)}\Univ & \textit{type of \(3^{\circ}\) family of unit-types} \\
  %% && \prod\limits_{X:\Univ}\prod\limits_{(x,y:X)}\Univ:\equiv\ \prod\limits_{X:\Univ}\prod\limits_{(x:X)}\prod\limits_{(y:X)}\Univ \notag \\
  & \textsf{Unit}{:}\prod\limits_{X:\Univ}\prod\limits_{(x,y:X)}\Univ & \textit{arbitrary \(3^{\circ}\) family of unit-types} \\
  & \UnitA{:}\DsuccT & \textit{arbitrary \(2^{\circ}\) family of unit-types from A}\\
  & \UnitA(a){:}\prod\limits_{(y:A)}\Univ & \textit{\(1^{\circ}\) family of unit-types at a from A} \\
  & \UnitA(a)(b){:}\,\Univ & \textit{unit type at b for a from A} \\
\end{align}

We can think of \(\UnitA(a,b)\) as expressing the unification of \(a\)
and \(b\), rather than their equality: together they form a binary
unit, or something like that.

As usual, we will drop the type symbol when the context makes it
clear, writing \(\Unit(a,b)\) instead of \(\UnitA(a,b)\).

\subsection{Unit Proofs}

What's the intuition for unit proofs?  Reflexivity is out-of-bounds.


\subsubsection{Canonical Constructors}

Since the concept of reflexivity depends essentially on a concept of
equality, we need a different concept here.  Actually we do not need a
concept, we just need a constructor.  Users can conceptualize it in
whichever way they please:

\begin{align}
  &\star{:}\prod\limits_{X{:}\Univ}\prod\limits_{(a:X)}\Unit^X(a,a) \\
  &\star^A{:}\,\prod\limits_{(a:A)}\UnitA(a,a) \\
  &\star^A(a){:}\,\UnitA(a,a) & \star a{:}\,\Unit(a,a)
\end{align}

So \(\star a\) has type \(\Unit(a,a)\); that is all.  No further
meaning is to be read into it; in particular, it has nothing to do
with reflexivity or equality.

\subsection{Unit Induction}

Following the definition of path induction: the principle of unit
induction says that to prove something for all \(p{:}\,\Unit(x,y)\),
it is sufficient to prove \(\star{:}\,\Unit(x,x)\).

The intuitive justification for this principle is that ...?


\section{Successor Types}

For any type \(A\) we may form \(\succt\); this is the type of
\emph{families of successor types from \(A\) at \(a\)}.  Given
\(a{:}A\), we can specify a successor family \(S^A\) for \(A\) by
writing, \(S^A{:}\succt\); then \(S^A(a){:}\Univ\) is a
\emph{successor type at \(a\) from \(A\)} or just \emph{successor type
  at \(a\)} if the base type is evident.  There are as many successor
families for \(A\) as there are elements of \(\succt\), and for each
one, as many successor types at \(a\) as there are elements of \(A\).

The motivation for calling these successor families and types is that
\(\succt\) takes an element from one level to the next in the type
hierarchy.  Given \aA, if \(A\) is level \(n\), then \(a\) is at level
\(n-1\), so \(S^A(a)\) is at \(n\), the same level as \(A\).

We can make this explicit and general by treating each type \(A\) as a
universe, writing \(\Univ_{n+}\) instead of \(A\).  Then to designate a
family of successor types for \(A\) we will write \(S^A\), which we
can express this formally using a \Pi\ type:

\begin{remark}
  We can write \(a{:}A\succ S^A(a)\) or the like to express
  successorship quasi-formally.
\end{remark}

\begin{align}
  & S{:}\prod\limits_{A{:}\Univ_{n+1}}\prod\limits_{(a:A)}\Univ_{n+1}
\end{align}

Then we can expose the inductive structure:

\begin{align}
  & S^A{:}\prod\limits_{(a:U_{n})}\Univ_{n+1} \\
  & a{:}U_{n} \\
  & S^A(a){:}U_{n+1} & (a\succ S^A(a))
\end{align}

Thus for any element \(a\) of any universe \(\Univ_n\) we can find a
family of successor types \(S^{\Univ_n}\) such that
\(S^{\Univ_n}(a){:}\Univ_{n+1}\).

Since every type may be treated as a proposition, this means that
every element of every type is associated with a proposition.

\subsection{Successor Iteration}

We can iterate successors for atomic types.

\begin{align}
  & S_A{:}\,\SuccT & \textit{family of successor types from}\ A \\
  & a{:}A,\ S_A(a):\Univ & \textit{successor type at a from A} \\
  & p{:}\,S_A(a) & \textit{proof of successor type at a} \\
  & S_{S_A(a)}\prod\limits_{x{:}S_{A(a)}}\Univ & \textit{family of succ types at elts of succ type at a} \\
  & q{:}\,S_A(a) & \textit{proof of proposition at a} \\
  & S_{S_A(a)}(q):\Univ & \textit{successor at q from successor at a from A}
\end{align}

and so on, \emph{ad infinitum}.


\subsection{Dependent Successor Types}

Same as relation types?  Conceptually, not so much?

We can also form ``complex'' successor types.  For example, we can
express relation types as dependent successor types.  Note that in the
following each \(\Pi\) operator corresponds to a family; e.g. a
\(3^{\circ}\) (third order) family is a family of families of families.

NB: in what follows we will try to follow the following terminological
regime: successor type \emph{at} element a \emph{for} element b
(meaning dependent on element b) \emph{from} type.

\begin{remark}
  FIXME: for successor types use \(S^A\) instead of \(\circ^A\)?
  Reserve the latter for relation types?  Do we then need distinct
  \(S\) symbols for unary, binary etc. successor families, or can we
  overload \(S\)?
\end{remark}

\begin{align}
  & \prod\limits_{X:\Univ}\prod\limits_{(x,y:X)}\Univ & \textit{type of \(3^{\circ}\) family of successor-types} \\
  && \prod\limits_{X:\Univ}\prod\limits_{(x,y:X)}\Univ:\equiv\ \prod\limits_{X:\Univ}\prod\limits_{(x:X)}\prod\limits_{(y:X)}\Univ \notag \\
  & S{:}\prod\limits_{X:\Univ}\prod\limits_{(x,y:X)}\Univ & \textit{arbitrary \(3^{\circ}\) family of successor-types} \\
  & S^A{:}\DsuccT & \textit{arbitrary \(2^{\circ}\) family of successor-types from A}\\
  & S^A(a){:}\prod\limits_{(y:A)}\Univ & \textit{\(1^{\circ}\) family of successor-types at a from A} \\
  & S^A(a, b){:}\,\Univ & \textit{successor type at b for a from A} \\
  %% & &  (a\circ_A b) :\equiv \circ_A(a)(b) :\equiv ((a\circ_A)b) \notag
\end{align}

Note the regimented vocabulary: \(S^A(a,b)\) is \emph{a} successor
(type) \emph{at} b \emph{for} a \emph{from} A, where \emph{for a}
means dependent on \(a\), and \emph{from A} means that \(A\) is the
base type from which the elements are drawn.

\begin{remark}
  Two kinds of ``forward motion'': from element \(a\) to ``next''
  element \(b\), and from element \(b\) to ``next'' type \(S^A(a,b)\).
\end{remark}

\begin{remark}
  Since \(S^A(a,b)\) goes from element to type we can think of it as a
  kind of ``next'' that crosses type levels.  But since \(S^A(a,b)\)
  is at the same level as \(A\), we can also think of it as a kind of
  next after \(A\) at that level, i.e. inductively generated by \(A\).
\end{remark}

Structurally this presentation is exactly the same as the Unit
presentation; the sole difference is in how we choose to interpret
things intuitively.

\subsection{Successor Proofs}

What's the intuition for successor proofs?  Reflexivity is out-of-bounds.

\subsubsection{Canonical Constructors}
Canonical constructor for dependent successor types?  Why not Latex'
``succ'' symbol: \(\succ\)?

For that matter, why not write \((b\succ a)\) instead of \(S(a,b)\)?
Because that symbol is for succession within a domain, whereas our
\(S\) is for crossing domain boundary.  Besides, it would be
intuitively incompatible with equality.

But we could write \(b\succ S(a,b)\) or the like.

\subsection{Dependent Successor Iteration}

From \(p,q{:}\,S(a,b)\) to \(S^{S(a,b)}(p,q)\), to \(S^{S^{S(a,b)}(p,q)}(r,s)\), etc.
etc.

\subsection{Successor Induction}

To prove something for all \(p{:}\,S(a,b)\) it suffices to prove it for \(\succ{:}\,S(a,a)\).

%%%%%%%%%%%%%%%%%%%%%%%%%%%%%%%%%%%%%%%%%%%%%%%%%%%%%%%%%%%%%%%%
\section{Relation Types}

\begin{remark}
  Here again we have the same structure with different symbols and a different interpretation.
\end{remark}

\begin{align}
  & \circ{:}\prod\limits_{X:\Univ}\prod\limits_{(x,y:X)}\Univ & \textit{arbitrary \(3^{\circ}\) family of relation-types} \\
  & \circ^A{:}\DsuccT & \textit{arbitrary \(2^{\circ}\) family of relation-types from A}\\
  %% & \circ^A(a){:}\prod\limits_{(y:A)}\Univ & \textit{\(1^{\circ}\) family of relation-types at a from A} \\
  & (a\circ^A b){:}\,\Univ & \textit{type of relation (pair) (a, b)} \\
  %% & &  (a\circ^A b) :\equiv \circ^A(a)(b) :\equiv ((a\circ^A)b) \notag
\end{align}

\begin{remark}
  Think of \(\circ\) as a relation on pairs (i.e. \(A\times B\))?
\end{remark}

Classically, a binary relation is a relation between two elements of a
set; in type-theoretic terms we can think of this as a relation
between two elements of the same type.  But if we treat binary
relation types as dependent successor types the nature of the relation
changes.  Instead of a ``horizontal'' relation between two elements at
the same level (so to speak), a binary relation is a ``vertical''
relation between elements of different types, one constructed from the
other.  The critical point is that \(a\circ b\) is a \emph{type}; not
as a ``horizontal'' relation between \(a\) and \(b\) within \(A\), but
as a successor type that is ``vertically' related to \(b\) (for
``previous'' element \(a\)).  At the same time, \((a\circ b)\) is
``horizontally'' related to \(A\) itself, since \((a\circ b)\) is a type at
the same level as \(A\).

We can iterate successor types for relations just as we did for atoms.
Given \((a\circ_A b)\), then for any element \(p\) of that type --
i.e. proof of the relation -- we can form a successor family \(\circ_{(a\circ b)}\):

%%  is a successor type from \(\circ_A\) at \(b\) for
%% (dependent on) \(a\),

\begin{align}
  & \relAdef &  2^{\circ}\textit{family of relation-types from A} \\
  & \relrelAdef & \\
\intertext{\hfill\(\relrelA\) \textit{= family of reln types for successor type at y for x of}\ \(A\)} \\
  %% & a{:}A, b{:}A \\
  %% & \circ_{(x\circ_A y)}(a,b){:}\prod\limits_{p,q{:}(x\circ_Ay)}\Univ \\
  %% & p{:}(a\circ_A b), q{:}\,(b\circ_A a) \\
  %% & \relrelA(a,b,p,q):\Univ \\
  %% & (p\relrelA q) :\equiv \relrelA(a,b,p,q).
  & (p\relrelA q){:}\,\Univ & \textit{successor type at q for p, from type}\ (x\circ_A y)
  %% \intertext{\textit{\hfill ---successor type for dependent successor type \((a\circ_A b)\) at \(p{:}(a\circ_Ab)\)}}
\end{align}

Normally we will write \((p\relrelA q)\) instead of
\(\relrelA(a,b,p,q)\), or just \((p\circ q)\) if the parameters are
evident from the context.  Then we will say that \((p\relrelA q)\) is
``a successor type of \(\relA\) at \(q\) for \(p\)''.

NB: if we really wanted to, we could also iterate \(\circ_A\) itself,
since it is an element of a function type.  Giving
\(\circ_{\circ_A}\); I'm not sure what it means.

%% and we
%% can go from \(p{:}(a\circ b)\) to a successor type \(S_{(a\circ b)}(p):\Univ\).



\subsection{Reflexivity Types}

The types of relation type families can be partitioned into two classes:

\begin{itemize}
\item \textit{reflexivity relation types} of the form \((a\circ_A a)\)
\item non-reflexiivty types of the form \((a\circ_A b)\)
\end{itemize}

\begin{remark}
Every relation type family has reflexivity types; we'll call these the
reflexivities of the family.  Not all reflexivities are inhabited
(e.g. \(a < a\)), but for a given family, if one is they all are(?).
We'll call any family whose reflexivities are inhabited a reflexivity
family.

All equality families are reflexivity families, but not all
reflexivity families are equality families (e.g. \(a \leq a\)).  To
get equality types, we need to add \(\reflA\) constructors to
relation types, to make sure their reflexivity types are inhabited.
In addition, ...
\end{remark}

Reflexivity successor types are self-dependent: successor at x for x.

\subsubsection{Constructors for Reflexivities}

Instead of \(\reflA\) we use the more general \(\rA\):

\begin{align}
  &\bullet{:}\,\rfamU \\
  &\rA{:}\, \rfama \\
  &\rA(x){:}\, \reflTAx & \bullet x{:}\,\reflTx
\end{align}

\(\rA\) works for both canonical and non-canonical tokens of the base type.

We the type is clear from the context we will drop the annotation and
write just \(\bullet x{:}\,\reflTx\).

\subsection{Symmetry Types}

The HoTT book ``proves'' symmetry by working with a function type
\((x=_Ay)\to (y=_Ax)\).

Here, we stick to successor and relation types and show how to
construct symmetry.  In any relation family, some will be symmetric
and some won't.  The task is to explain what it is to be symmetric,
i.e. what are the \emph{defining constructions} for symmetry.

The types of every successor type family \(\relA\) may be partitioned
into classes:

\begin{itemize}
\item Reflexivity types: types of the form \(a\circ_A a\)
\item Symmetry types: pairs of types of the form \((a\circ_A b), (b\circ_A a)\)
\end{itemize}

These types may or may not be inhabited for any given dependent
successor type family.  If the reflexivity types are inhabited
\emph{for all a}, we say the (relation) family is reflexive.

The rule for symmetry is a little bit different.  For any given
family, some symmetric pairs may be inhabited and some may not be
inhabited.  Or one member of the pair may be inhabited and the
other not (so we have four possibilities.)  To count as symmetric,
either both members of a symmetric pair must be inhabited or neither
member can be inhabited.  In other words, if we can find a symmetric
pair only one of whose members is inhabited then it is not a symmetric
relation family.

For example, the relation \(\neq\) is symmetric for \(\Nat\), but
\(\leq\) is not.

\begin{remark}
  TODO: Give example from HoTT of relation where some symmetric pairs
  are inhabited and some are not.  Equality works since we have
  \(a\neq b\) and \(b\neq a\) for lots of pairs, but we also have \(a=
  b\) and \(b=a\) for lots of pairs, just because there are multiple
  ways to have equality etc. But it would be nice to have an example
  other than equality.
\end{remark}

For every \(a,b{:}A\) and dependent successor type family \(\circ_A\)
we have the dependent successor type \(a\circ_A b\); and for every
proof of that type \(p{:}(a\circ_A b)\) we can form a family of
(\(2^{nd}\)-order) successor types, which we can call \(\circ_{(a\circ_A b)}\):

\begin{align}
  %% &\prod\limits_{(p,p^{-1}:x\circ_Ay)}\Univ & \prod\limits_{(p:x\circ_Ay)}\prod\limits_{(p^{-1}:y\circ_Ax)}\Univ \\
  &\circ_{(a\circ_A b)}{:}\prod\limits_{(p,q:a\circ_Ab)}\Univ & \textit{for all a,b}
%% \intertext{\textit{\hfill ---family of successor types for type \((a\circ_A b)\)}}
  %% &\circ_{(a\circ_A b)}{:}\prod\limits_{(p,q:x\circ_Ay)}\Univ
\end{align}


Then given \(p{:}(x\circ_A y)\) and \(q{:}(y\circ_A x)\), we have
\((p\circ_{x\circ_A y} q){:}\,\Univ\) --- the dependent successor type
at \(q\) for (i.e. that depends on) \(p\) (from type \(\relrelA\)).

Note:

\begin{itemize}
  \item \xrely\ is the successor type at \(y\)\ dependent on \(x\)
  \item \yrelx\ is the successor type at \(x\)\ dependent on \(y\)
\end{itemize}

So not only are they different types, they are successor types of
different elements.  Proof of one cannot count as proof of the other,
so we need a proof for each.  But if we do have both proofs, we can
iterate, constructing successors of these successors.  So proving that
either \((p\circ_{x\circ_A y} q)\) or \((q\circ_{x\circ_A y} p)\) is a
proposition shows that the components have proofs, which in turn is
defined as what it means for the base family of relation types
\(\relA\) to be symmetric.

 \begin{align}
   (a\ &\circ_A b){:}\,\Univ & \textit{---given \(a,b:A\)} \\
   (p\ &\circ_{(x\circ_Ay)} p^{-1}){:}\,\Univ & \textit{---given \(p{:}(a\circ_A b),\  p^{-1}{:}(b\circ_A a)\)}
\end{align}

The problem here is that \(p\) and \(p^{-1}\) prove different types, and
\(\circ_{(x\circ_Ay)}\) is only defined for types \(x\circ_A y\). ???

We should view \(\relrelA\) as the successor to \(\relA\).

\(\relA\) applies only to elements of \(A\).  Its successor
\(\relrelA\) applies to elements of different types.  But each such
type is the successor to an element of \(A\), so in a sense
\(\relrelA\) does indirectly apply to elements of \(A\).

\begin{remark}
All of which suggests we should view a type family as a genuine type,
whose elements are the types in the family.  Then \(\relrelA\) will
apply to elements of the same (family) type.  This suggests in turn
that we should make this explicit, so that we can refer to the
elements of a family (rather than the function that generates them).
E.g. \(\Lambda(\relA)\) names the family of types generated by
\(\relA\), giving e.g. \((a\relA b){:}\Lambda(\relA)\).  At the very
least this would make exposition easier.
\end{remark}


\begin{remark}
The goal is to explain that symmetric relations are those whose
symmetries are inhabited.  And they are ``natural'', just as
reflexivities are natural.  To wit, we picked out types \(a\circ a\)
to talk about reflexivities; to talk about symmetries, we pick out
symmetric pairs of types \((a\circ b)\) and \((b\circ a)\).  Every
non-trivial relation type will have such pairs.  If both types in the
pair are inhabited, then we have a symmetry type, full stop.  What it
means for both to be inhabited is that the higher order relation
\((p\circ p^{-1})\) between inhabitants of the types can be
constructed.  Not inhabited; you can only construct it if you have
both proofs, so that's enough.  So to show symmetry we only need to
show that \(p\circ p^{-1}\) is a proposition, not that it is true.  To
construct an element of such symmetry pair, we need a constructor,
just as we needed refl to construct an element of\(a\circ a\).
Contrast this with the functional approach of the HoTT book.
\end{remark}

\subsubsection{Summary}

For symmetry we need to do what we did with reflexivity, that is for
particular families provide a constructive means.  For reflexivity we
provided a constructor refl that can be associated with any relation
family.  For symmetry we cannot do this so directly since it may be
the case that some symmetry pairs are uninhabited; we cannot have a
symm constructor like refl.

Or can we?  Why not something like \(sym{:}(a=_A b)\)?  Because it
would be useless for e.g. \(\Nat\) since it would prove e.g. \(2=3\).

The only thing we have to go on is induction: the notion that proof
for refl:a=a is as good as proof for all p:a=b.



We form successors, then we use path induction.

\vfill

\subsection{Transitivity Types}

\subsection{Equality Types}

\section{Inductions}

The Hott book calls this path induction, in acknowledgement of the
role of homotopy theory in HoTT.  But what is at issue is purely a
matter of type theory, for which homotopy theory is irrelevant, so we
use ``Equality Induction'' instead.

The HoTT book also gives the impression that equality is a kind of
primitive in HoTT, and is one of the major ideas that sets it apart
from other approaches.  I think this is not quite right.  In HoTT, the
Id/= type (actually family) is not primitive, nor is it the only
possible such family; it is one of many.  The critical notion is not
equality but equality (``path'') induction.

What's the point of induction?  Or rather, what problem does a
principle of induction solve?

The basic problem is infinity.  To prove something over an infinity we
cannot inspect each element; instead we need a (finite) \emph{method}
of proof that makes it \emph{possible} for us to prove an arbitrary
case.  Contrast this with the classical approach, where proof is
always \emph{alethic} and does not involve (formal) proof methods.
Classical proof for infinities depends on quantification over their
elements, and offers an alethic conclusion: either the proposition is
true for all elements, or not.  There is no notion of possibility, or
of method of proof.  (Of course there are informal \emph{strategies}
for discovering a proof, but they are external to mathematics proper.)

  Where the infinity is inductively constructed, as in the case
of \(\Nat\), we can rely on the inductive structure of the type to
support proof by (mathematical, structural, etc.) induction.  That's
intuitively satisfying, since we can see that the structure of proof
corresponds to the structure of the type; we know (or are convinced)
that we can ``reach'' any element of the infinity in a finite number
of steps, so we can apply our proof method in a finite number of
steps.

The secondary problem is that a type may ``contain'' elements that are
not canonically (that is, inductively) constructed.  This is obviously
the case for equality types, but it may also be the case for types
like \(\Nat\).  The problem here is that such ``exotic'' elements are
not reachable by construction in the way that canonical (inductively
defined) elements are reachable from the base case.

There are two basic strategies to solve this problem.  One is to show
that a canonical element can be inferred from every non-canonical
element; the other is to show that, given propositions dependent on
elements of the type, if we can prove those dependent canonical
elements, then we can (construct a method to) prove those dependent on
non-canonical elements.  Note that this is very different from
standard induction.

Both forms of induction involve an unjustifiable but intuitively
obvious assumption.  In the case of traditional mathematical
induction, the assumption is that proving the base case and the
inductive hypothesis justifies inference to the general case that P is
true for all elements.  But this assumption cannot be proven.  This is
almost always expressed in the form of a deductive inference, but in
fact it is not a deduction but an induction.  We are \emph{confident}
that P is true for all n if we are \emph{certain} that it is true for
the base case and inductive hypothesis.  But we do not have
\emph{certain} knowledge that P is true for all n; that would require
inspection of every case, which we cannot do.

Compare Church-Turing: we are all very confident that the various
models of computation are equivalent, but we cannot (so far) prove it,
so we cannot be certain.  Proof by induction is similar: we cannot
prove that such proofs are valid, but we are confident that they are,
so we call them proofs.

In the case of finite types...

In the case of constructors, we do have certainty, just because they
are constructors.  There is no question of proving that a constructor
constructs.  Compare e.g. the introduction rule for product types: in
this case, we can be absolutely certain that \((a,b){:}\,A\times B\)
if we are certain that \(a{:}A\) and that \(b{:}B\), because that is
just what the rule for pairs means.  And we are certain that this is
the case for \emph{every} element of \(A\) and element of \(B\),
because it is in effect a definition.

So the first kind of induction involves inference to all
canonically-formed elements of an infinity.  The second kind involves
inference to all elements, including non-canonical elements.  The
legitimacy of this inference is much more tenuous.  For the first kind
of induction, the inference to infinity is justified by the inductive
structure of the elements; for the second kind, there is no such
inductive structure for the non-canonical elements, so it is harder to
see why we should be confident that proving the base case
etc. justifies inference to the general case.

Now HoTT says that the equality types \((a=_Ab)\) are inductively
defined by their canonical constructors \(\reflA\).  This makes some
sense, since it is always the case that types are (in a sense) defined
by their constructors.  (They are also defined by eliminators, etc.)
On the other hand, the constructors only serve to define reflexivity
types \((a=_Aa)\); non-reflexive types \((a=_Ab)\) have \emph{no}
constructors, and so no canonical elements.  But then why should we
believe that \emph{these} types are inductively defined by the
constructors of \emph{other} types?  There seems to be some magic
going on here.

Notice also that this says nothing about the structure of the elements
of equality types.  In particular, it does not say that the elements
of an equality type have inductive structure.  The canonical elements
of \((a=_Aa)\) are obviously inductively defined by \(\reflA\); but we
should we be convinced that the non-canonical elements of \((a=_Aa)\)
are thereby defined, let alone the elements of \((a=_Ab)\)?  There is
no intuitive support for this concept, as far as I can see.  But
that's okay, since the HoTT book does no claim this.

So the problem is twofold: first, how do we get to \((a=_Ab)\) when
all we have to go on is \(\reflA(a){:}(a=_Aa)\)?  We might express
this by asking how we can get from canonical equality types
(i.e. reflexivities) to non-canonical equality types
(i.e. non-reflexivities).  Second, how do we get from canonical to
non-canonical elements of equality types?

On the other hand, we do have the obvious intuition that it is always
the case that (\(a=_Aa)\), so that if it is also the case that
\((a=_Ab)\), those two propositions are in some sense ``the same'', so
proof of \((a=_Aa)\) should count as proof of \((a=_Ab)\), and
vice-versa.  We also have the intuition that if \(a\) and \(b\) are in
fact equal, then any one proof of that fact is as good as any other.
These intuitions are based directly in the very idea of equality.

Etymonline: ``As a term in logic (early 15c.) it [induction] is from
Cicero's use of inductio to translate Greek epagoge "leading to" in
Aristotle. Induction starts with known instances and arrives at
generalizations; deduction starts from the general principle and
arrives at some individual fact.''

This sense of ``leading to'' is what is missing in path induction.
This also shows why mathematical induction cannot be deemed deductive,
because it moves from specific to general.

A better term might be \emph{transduction}.  Etymonline: from Latin
transducere/traducere "lead across, transfer, carry over," from trans-
"across" + ducere "to lead"

\begin{remark}
  Note that there is a big difference between definition by induction
  and proof by induction.
\end{remark}

\subsection{Varieties of Induction}

There are three well-known and widely accepted principles of induction:

\begin{itemize}
\item Mathematical induction (a/k/a \(\Nat\)-induction)
\item Structural induction
\item Transfinite induction
\item Well-founded induction (http://planetmath.org/wellfoundedinduction)
  ``As an example of application of this principle, we mention the proof of the fundamental theorem of arithmetic : every natural number has a unique factorization into prime numbers. The proof goes by well-founded induction in the set ℕ ordered by division.''
  a/k/a Noetherian induction?
  WFI is basic; the others are specializations.
\item Rule induction (computer science?)
\item Epsilon induction https://en.wikipedia.org/wiki/Epsilon-induction
\end{itemize}

What all these have in common is some notion of ``forward progress'':
a ``successor'' operation that moves from one place or element to the
``next'' place or element.

Equality induction does not involve this kind of movement.  For any
equality type we may have multiple distinct inhabitants, but they are
all ``at the same level'', so to speak; they cannot be arranged in a
sequence involving next or preceding elements.

\subsection{Well-Founded Induction}

 Most accounts of WFI are classical.  You start with a well-founded
 set \(X\) and relation \(\preceq\).  Here's one way to state WFI (from
 Andreas Klappenecker's notes on Noetherian Induction):

 \begin{theorem}
   (The Principle of Noetherian Induction). Let \((A,\preceq)\) be a well-
   founded set. To prove that a property \(P(x)\) is true for all
   elements \(x\) in \(A\) it is sufficient to prove the following two
   properties:
   \begin{enumerate}
   \item Induction basis: \(P(x)\) is true for all minimal elements of \(A\).
   \item Induction step: For each non-minimal \(x\) in \(A\), if
     \(P(y)\) is true for all \(y\prec x\), then \(P(x)\) is true.
   \end{enumerate}
\end{theorem}

The conclusion is then that \(\forall x P(x)\).

This is fine, but it's non-constructive.  In order for it to work
constructively, the relation must be constructive; that is, \(y\prec
x\) must mean that \(x\) can be constructed from \(y\).

We don't have that in path induction.  There is no relation of
construction between the base case \(\reflAx{:}\,\reflTa\) and any
other \(p\) where \(\preflTa\).  So WFI (classical) cannot underwrite
path induction.

\bigskip

The HoTT book says this:

``The basic way to construct an element of \(a = b\) is to know that
\(a\) and \(b\) are the same. Thus, the introduction rule is a
dependent function

\begin{align}
  \textsf{refl}{:}\,\reflfam
\end{align}

called \textbf{reflexivity}, which says that every element of A is
equal to itself (in a specified way).'' (p. 47-8)

The formal definition given in appendix A.2.10 says:

\medskip

\begin{tikzpicture}[
    edge from parent path={
        (\tikzparentnode\tikzparentanchor)
        +(0pt,.5\tikzleveldistance)
        -- (0pt,-.5\tikzleveldistance -| \tikzchildnode\tikzchildanchor)
        -- +(0.75cm,0pt)
        -- +(-0.75cm,0pt)
    },
    grow'=up,
    level distance=4ex,
    level/.style={sibling distance=5em/#1}]
  \draw (2.75cm,.35cm) node {=-\textsc{intro}};
  \node (Concl) {\(\Gamma\vdash \textsf{refl}_a{:}\,a=_Aa\)}
    child { node (Major) {\(\Gamma\vdash A{:}\Univ\)} }
    child { node (Minor) {\(\Gamma\vdash a{:}A\)} } ;
\end{tikzpicture}

(Note that this is called the =-\textsc{intro} rule rather than
the \(\reflA\)-\textsc{intro} rule.)

\medskip

This means that we can always infer \(\reflAa{:}\,\reflAaa\) from just
\(a{:}A\); but it says nothing about any other proof of \(\reflAaa\),
nor about proof of \(\reflAxy\) for any other \(x,y{:}A\).  So even if
we have \(\reflAa\prec p\) for all \(\preflAxy\), we cannot invoke WFI,
since there is no constructive path from the former to the latter.

There is another problem with the informal explanation offered by the
HoTT book: it relies on a notion of induction as explanatory.  This
violates the general principle that types are explained by their
inference rules.  ``Path induction'' is explained by the ind-intro
(aka =-elim) rule, which has nothing to do with any antecedent notion
of induction.  It just lays down what it means to be an equality
proof.  We can call this induction, but then we're just labelling an
elimination rule -- misleadingly, since the elimination rule does not
involve and is not justified by any concept of induction.

On the other hand, one could argue that the formal rule captures some
intuitive notion of induction, just as the \&-intro rule captures our
informal notion of conjunction.  But the problem with that is, as
explained above, we do not seem to have an intuitive notion of
induction that corresponds to the =-elim rule.  On the other hand, we
do have intuitions about equality, but they do not depend on
intuitions about induction; they seem to depend mostly on practices of
exchange.

What we do with equalities seems to be fundamentally different from
what we do with inductive stuff.  Induction takes us from the
particular to the general; equality seems to be already general.  We
don't need to prove that one pebble is as good as any other for
tallying; it's built-in to the notion of tallying.  No induction is
needed; it doesn't make sense to try to ``test'' new pebbles to see if
they really work, or to prove that they do.  Using induction to
justify our ways with equality makes no sense.

\bigskip

\noindent
A possible way out is to reinterpret reflexivity as something that
follows from equality generally.  Then we can define a different
introduction rule for \(\reflA\).

\medskip

\begin{tikzpicture}[
    edge from parent path={
        (\tikzparentnode\tikzparentanchor)
        +(0pt,.5\tikzleveldistance)
        -- (0pt,-.5\tikzleveldistance -| \tikzchildnode\tikzchildanchor)
        -- +(1.5cm,0pt)
        -- +(-1cm,0pt)
    },
    grow'=up,
    level distance=4ex,
    level/.style={sibling distance=6em/#1}]
  \draw (5.25cm,.5cm) node {\(\textsf{refl}^1-\)\textsc{intro}};
  \node (Concl) {\(\Gamma\vdash \textsf{refl}^Aa\,{:}\,\reflAaa\)}
    child { node (Major) {\(\Gamma\vdash A{:}\Univ\)} }
    child { node (Minor) {\(\Gamma\vdash a,b{:}A\)} }
    child { node (Minor) {\(\Gamma,p\vdash \reflAab\)} } ;
\end{tikzpicture}

\medskip

\begin{tikzpicture}[
    edge from parent path={
        (\tikzparentnode\tikzparentanchor)
        +(0pt,.5\tikzleveldistance)
        -- (0pt,-.5\tikzleveldistance -| \tikzchildnode\tikzchildanchor)
        -- +(1.5cm,0pt)
        -- +(-1cm,0pt)
    },
    grow'=up,
    level distance=4ex,
    level/.style={sibling distance=6em/#1}]
  \draw (5.25cm,.5cm) node {\(\textsf{refl}^2-\)\textsc{intro}};
  \node (Concl) {\(\Gamma\vdash \textsf{refl}^Ab\,{:}\,\reflAbb\)}
    child { node (Major) {\(\Gamma\vdash A{:}\Univ\)} }
    child { node (Minor) {\(\Gamma\vdash a,b{:}A\)} }
    child { node (Minor) {\(\Gamma,p\vdash \reflAab\)} } ;
\end{tikzpicture}

\medskip

The idea is that, instead of inferring \textsf{refl} directly from
\(a{:}A\), we can infer it from \emph{any} proof of \(\reflAab\).
This corresponds more closely to treating reflexivity as a binary
relation, rather than a unary predicate.  Reflexivity itself does not
\emph{introduce} equality, as the HoTT definition suggests; after all,
we can specify equality types without also introducing their proofs.

Rather, reflexivity is just the canonical form of proof of reflexivity
types, not of equality types generally.

This suggests another revision to the HoTT definition, this one a refinement:

\medskip

\begin{tikzpicture}[
    edge from parent path={
        (\tikzparentnode\tikzparentanchor)
        +(0pt,.5\tikzleveldistance)
        -- (0pt,-.5\tikzleveldistance -| \tikzchildnode\tikzchildanchor)
        -- +(0.75cm,0pt)
        -- +(-0.75cm,0pt)
    },
    grow'=up,
    level distance=4ex,
    level/.style={sibling distance=5em/#1}]
  \draw (2.75cm,.35cm) node {=-\textsc{intro}};
  \node (Concl) {\(\Gamma\vdash (a=_Ab){:}\,\Univ\)}
    child { node (Major) {\(\Gamma\vdash A{:}\Univ\)} }
    child { node (Minor) {\(\Gamma\vdash a,b{:}A\)} } ;
\end{tikzpicture}

\noindent which just says that given \(a,b{:}A\) we can form the
\emph{type} \(\reflAab\), independently of whether we can find a proof
of that type.  The introduction rules \(\refl{1}\) and \(\refl{2}\)
then just say that, given any proof \(p\) of \(\reln{A}{a}{b}\), we
can introduce the canonical proofs \(\refl{A}a{:}\reln{A}{a}{a}\) and
\(\refl{A}b{:}\reln{A}{b}{b}\).

\medskip

What would this mean for constructive WFI?  The core idea of WFI for
proofs on equalities is that to prove something for any equality proof
for any pair (of type \(A\)), it suffices to prove it for the
\(\refl{A}\) proof for reflexivity pairs; the justification is based
on well-founded structure.

With our revisions, any equality proof can be reduced to \(\refl{A}\),
which is more in line with the notion of canonical form as normal
form.  The intro rules do not mean that other proofs are constructed
inductively from the canonical proofs; but as they are indeed
introduction rules they set the meaning of \(\refl{A}\).  Instead of
proof by induction, we prove by normalization -- which in turn is a
kind of induction.  Normalization has well-founded structure; it
always bottoms out at canonical forms.

So it still suffices to prove the base case \(\refl{A}\).  Not
directly because of inductive construction, nor because of (classical)
well-founded structure, but because of an (axiomatic) principle of
normalization.  (?)

\begin{remark}
  In other words, if our intro rules allow us to infer canonical form
  from non-canonical form -- by fiat or axiom, as it were -- this is
  just as good as having an inductive constructor that builds other
  canonical forms from the base case.  For example, for \(\Nat\) we
  use \(S\) to build from \(Z\); but for =, we reverse this, saying in
  effect that non-canonical forms can be ``deconstructed'' and the
  canonical form pulled out.  We might be able to think of our
  \(\refl{A}\) intro rules as implicit elimination rules or
  co-constructors: they say that for any proof of equality we can
  extract the base case that must have been used to construct the
  proof, even if we do not have an explicit \(S\)-like constructor
  showing exactly how the proof was built up from the base case.
\end{remark}

\begin{remark}
  Or, there are two ways for WFI to work constructively, one based on
  inductive construction, and one based on inductive co-construction,
  so to speak.  The former method establishes a well-founded relation
  by construction; the latter by co-construction.  In the former case,
  you can always construct the ``next'' element; in the latter, you
  can always co-construct the base element.  Is this co-induction?  So
  equality proofs are codata?
\end{remark}

\medskip

Consider what the idea would look like for \(\Nat\):

\medskip

\begin{tikzpicture}[
    edge from parent path={
      (\tikzparentnode\tikzparentanchor)
      +(0pt,.5\tikzleveldistance)
      -- (0pt,-.5\tikzleveldistance -| \tikzchildnode\tikzchildanchor)
      -- +(1.5cm,0pt)
      -- +(-1cm,0pt)
    },
    grow'=up,
    level distance=4ex,
    level/.style={sibling distance=6em/#1}]
  %% \draw (5.25cm,.5cm) node {\(0{:}\,\Nat} ;
  \node (Concl) {\(\Gamma\vdash S(n){:}\,\Nat\)}
  child { node (Major) {\(\Gamma\vdash n{:}\Nat\)} } ;
\end{tikzpicture}
\begin{tikzpicture}[
    edge from parent path={
      (\tikzparentnode\tikzparentanchor)
      +(0pt,.5\tikzleveldistance)
      -- (0pt,-.5\tikzleveldistance -| \tikzchildnode\tikzchildanchor)
      -- +(1.5cm,0pt)
      -- +(-1cm,0pt)
    },
    grow'=up,
    level distance=4ex,
    level/.style={sibling distance=6em/#1}]
  %% \draw (5.25cm,.5cm) node {\(0{:}\,\Nat} ;
  \node (Concl) {\(\Gamma\vdash n{:}\,\Nat\)}
  child { node (Major) {\(\Gamma\vdash S(n){:}\Nat\)} } ;
\end{tikzpicture}
\begin{tikzpicture}[
    edge from parent path={
      (\tikzparentnode\tikzparentanchor)
      +(0pt,.5\tikzleveldistance)
      -- (0pt,-.5\tikzleveldistance -| \tikzchildnode\tikzchildanchor)
      -- +(1.5cm,0pt)
      -- +(-1cm,0pt)
    },
    grow'=up,
    level distance=4ex,
    level/.style={sibling distance=6em/#1}]
  %% \draw (5.25cm,.5cm) node {\(0{:}\,\Nat} ;
  \node (Concl) {\(\Gamma\vdash 0{:}\,\Nat\)}
  child { node (Major) {\(\Gamma\vdash n{:}\Nat\)} } ;
\end{tikzpicture}

\medskip

\noindent Here the rule on the left just expresses the definition of
\(S\) for \(\Nat\).  The rule on the right expresses a kind of
elimination rule - we know any \(n{:}\Nat\) must have the form
\(S(S(...S(0)))\), so it expresses the idea that all of the \(S\)s
could be eliminated.  We could prove this, since any \(n{:}\,\Nat\) is
inductively constructed.  But were that not the case, we could simply
stipulate it as part of the definition of what it is to be of type
\(\Nat\).  And that's what we do for equality types and \(\refl{A}\).

\subsection{Coinduction}



\subsection{Reflexivity Induction}

For any reflexivity type \(\reflAa\) we may have infinitely many proofs
\(\preflAa\).  The HoTT book singles out just one such proof per type
and designates it as the canonical constructor for the type:

\begin{align}
  &\textsf{refl}{:}\,\reflfam &  \textsf{HoTT p. 48} \\
  \intertext{(which should be:)}
  &\textsf{refl}{:}\,\reflfamU \\
  \intertext{giving}
  &\reflA(x):\,(x\circ_A x)
\end{align}

Which makes \(\reflA\) a family of ``provers'' of reflexivity types:
give it an \(x\) and you get a (canonical) proof of \(\reflTAx\).

Since we may have non-canonical proofs \(x{:}\reflAa\), we are
confronted with the task of showing that if a predicate on proofs of
\(\reflAa\) can be satisfied by one such proof it can be satisfied by
any such proof.  This is what path induction is for (although it
extends beyond reflexivity types to all \(x\circ_Ay\) types).

A predicate on proofs of \(\reflAa\) looks like this in the HoTT book:

\begin{align}
  & \Cxyp & \textsf{HoTT p. 49} \\
  \intertext{giving}
  & C(x,x,\reflx){:}\,\Univ
\end{align}

\noindent We rewrite this as:

\begin{align}
  & \SXYPT \\
  \intertext{giving:}
  & \SXX \\
  & \SXXP
  \intertext{and we set}
  & S^{x\circ_Ax}(p) :\equiv \reflTAx
\end{align}

We use \(\circ\) instead of \(=\) for generality, and we use
\(S^{\circ_A}\) intead of \(C\) in order to highlight that fact that
we treat it as a (not ``the''!) successor type family drawn from
\(\circ_A\) types, so \(\SXXP\) is \emph{a} successor type at \(p\)
from type \(\reflTAx\) -- which in turn we treat as \emph{a} successor
type at \(x\) for \(x\) from \(A\).

Note that for any \(\preflTx\) we have \(p{:}\,S^{x\circ_Ax}(p)\):
every proof of \(\reflTAx\) is proof of its own successor type.  Or: if
p proves a successor type at x for x, the it also proves its own
successor drawn from that successor type at x for x.  Reflexivity all the way
up!


\medskip

Now the HoTT book bases path induction on \(\reflAx\), which is an
element of \(\reflTAx\); informally, the path induction principle for
reflexive types can be expressed by saying that a proof of a predicate
on \(\reflAx{:}\,\reflTAx\) suffices as proof on all \(\preflTx\).

\bigskip

The approach adopted here is a little different.  First, we don't
single out a canonical constructor.  Intuitively, the idea is that
since all proofs of \(\reflTAx\) are ``equivalent'', it should be
enough to prove the predication for any such proof; there is no call
to single out a canonical one.

\begin{remark}
I suspect that introducing a canonical element for reflexivity
types was motivated by the example of the inductive definition of
\(\Nat\), which has two canonical constructors.  And it is the
inductive structure of those constructors that makes proof by
mathematical induction work.  But the elements/proofs of equality
types are not constructed in this way; there is no sequential
structure to them.  But this suggests that we do not really need a
canonical constructor.
\end{remark}

\begin{remark}
  Note that mathematical induction is a little ambiguous.  Once you've
  proven for base case and inductive hypothesis, you drawn the
  conclusion to ``all n''.  But the justification for this inference
  is the intuition that ``all n'' are in fact constructed using the
  canonical constructors, so ``all n'' really means ``all canonical
  n''.  There is no good intuition (for me, at least) leading from
  ``all canonical n'' to ``all n including non-canonical/exotic n''.
  But that reading is essential to HoTT induction, and I suspect it is
  the real innovation.
\end{remark}

\bigskip

The basic idea is that we replace the notion of a canonical element
\(\reflAx\) with the notion of a canonical successor function.  This
gives us a genuine kind of induction: starting with any \(\preflTx\)
we get a ``next'' object, namely the successor type at \(p\).  There
are infinitely many successor families, and thus successor types at
\(p\); the intuition behind ``reflexivity induction'' is that proving
the canonical successor type for some \(\preflTx\) suffices as proof
for all such \(p\), for all successor types.  In other words, if you
can prove the canonical successor proposition \emph{at p} for some
\(\preflTx\), you can prove any successor proposition \emph{at p} for
any such \(p\).

So what our approach boils down to is moving from elements/proofs of
reflexivities to their successor types/propositions, which are what we
want to prove.  This gives us an inductive structure, analogous to the
structure of \(\Nat\), that the HoTT approach, based on canonical
elements, lacks.  Well, it's not so much that it goes missing as that
it goes unremarked.  What I sketch here is implicit in the HoTT book.

In other words, it is not the case that all elements of \(\reflTAx\)
are inductively constructed from canonical constructors, but it is the
case that the successor type of every such element is inductively
constructed by a successor function.  In contrast to e.g. \(\Nat\),
whose structure can be thought of as ``vertical'', the inductive
structure of reflexivity proofs combines ``horizontal'' and
``vertical'' aspects.  The successor types are all at the same level,
one level ``up''from the elements to which they are successors;
neither can be ordered.  But nonetheless each pair of element and
successor type has inductive structure, from n to n+1.

A critical point is that we do not need a canonical element to do
this; we need not define \emph{any} constructors for reflexivity
types.

Once this structure is clear it also becomes clear that the inductive
principle required to make this work is clearly not like the inductive
principles involved in the three basic kinds of induction.  The
proposal here is that it is this induction principle that is
distinctive, rather than the Id/= type family specified in the HoTT
book.

The implicit reasoning: we prove that if something is true for
arbitrary \(n\) (i.e. for any \(\preflTx\)) then it is true for
\(n+1\) (i.e. \(p{:}S^{x\circ_Ax}(p)\)).  We then infer that it is
true, not for all \(n\) (there are only two levels, \(n\) and \(n+1\),
and we've already shown the equivalent of \(P(n)\to P(n+1)\) for
arbitrary n) but for all successor types (propositions) in the family
\(\prod\limits_{(x,y:A)}\prod\limits_{p{:}(x\circ_Ay)}{:}\,\Univ\).

In other words, mathematical induction goes from \(P(n)\to P(n+1)\)
for arbitary n to \(P(n)\) for all n; we go from \(P(n)\to P(n+1)\)
for arbitary n to \(Q(n)\) for all \(Q, n\).

Hmm, that can't be right.  Some successor props may not be
satisfiable.  We have to do the inductive proof for each successor
type.

Since we don't use a canonical element, there is no move from ``true
for canonical'' to ``true for everything''.  We don't even need to
move to ``true for everything'', since we prove the inductive step for
arbitrary n?  No, proof for arbitrary n is not same as proof for all;
you have to make the inductive inference to get from the former to the
latter.  So we need it too.

What's our base case?  Refl?  Hmm, that would suggest that path
induction is wrong, its missing the inductive step that we have
articulated.

Writers on HoTT (Ladyman?) say that the inductive step goes missing in
path induction; we go straight from base case to generalization.  That
makes no sense to me.

The formal definition of HoTT weasels out of this be defining ind= as
a primitive elimination rule.  But that sticks in my craw too; it just
seems wrong to encode induction as an inference rule the way HoTT does
it.  It should be structured just like the traditional inductions:
base case, then inductive hypothesis, then generalization:

\begin{itemize}
\item Base case:  \(\reflAx{:}\,\reflTAx\to \reflAx{:}\,S^{x\circ_Ax}(\reflAx)\)
\item Inductive hypothesis:  \(\preflTx\to p{:}\,R^{x\circ_Ax}(p)\) for arbitrary \(p\)
\item Generalization:  \(\preflTx\to p{:}\,S^{x\circ_Ax}(p)\) for arbitrary \(p\)
  %% \(S^{x\circ_Ax}(p)\) for all \(\preflTx\) and for all \(S\)
\end{itemize}

Critical move: defining \(S^{x\circ_Ax}(p) :\equiv (x\circ_Ax)\).  Is
that kosher for all successors?  No, the successor type at p can be
anything.  We need canonical reflexive successor:  \(\RXXP\).

The problem is we have two variables: proofs of x=x, and successor
types at those proofs.  To generalize over the latter we must use
conditionals.

Goal is to show that for arbitrary successor S, if S(p) for some
p:x=x, then S(p) for all p:x=x.

\begin{itemize}
\item Base case:  \(\preflTx\to p{:}\,R^{x\circ_Ax}(p)\) for arbitrary \(p\)
\item Inductive hypothesis: function from S(p) to ?? for some p?
\item Generalization: function from S(p) to ?? for all p?
  %% \(S^{x\circ_Ax}(p)\) for all \(\preflTx\) and for all \(S\)
\end{itemize}


\subsection{Recursion}

%% https://en.wikipedia.org/wiki/Course-of-values_recursion

\section{Construction and Co-construction}

The standard presentation of intuisionistic/constructive mathematics
says, in so many words, that you must be able to construct stuff.
This means that if we think we have something but no means of
constructing it, we're in trouble.  Case in point: equality proofs in
HoTT.

But the concepts of coinduction and bisimilarity suggest a more
refined notion.  Construction builds up; co-construction builds down.
If you can start at the base case and build your object using
canonical constructors, you have a construction.  If you start at your
object and ``build'' - that is, co-construct - a base case, then you
have a co-construction.

In other words, it is sufficient for constructivity to be able to get
(co-) constructively from your object to a base case.  If you can do
that, then you are working (co-)constructively, even if you have no
method of constructing your object from the base case.

This corresponds to the idea of ``hidden'' stuff addressed by
bisimilarity.  In this case, the constructive form of the object is
hidden; but since we can co-construct downwards we can infer that the
object is indeed a construction even though we cannot ``see'' the
construction.

\begin{remark}
  What I'm calling co-construction seems to be called ``observation''
  by some authors.  Suggesting that we might think of elimination
  rules as observational rules.  This makes sense; if we have
  arbitrary \(x{:}\,A\), the elim rules tell us how it behaves, not
  how it is constructed.
\end{remark}

\subsection{State Space v. Construction Space}

Can we think of vars like \(x{:}\,A\) in terms of a (hidden) state
space, by analogy with bisimulation?  We don't know how \(x\) was
constructed, but we can ``observe'' it by using eliminators, etc.
Instead of hidden state, hidden construction (structure).

\section{Grounding Equality: Pragmatics of Fair Exchange}

The task is to capture our ordinary intuitions about equality, just
like Turing captured our intuitions about effective procedures.
Similar: logical consequence.  So far nobody has managed to do for
logical consequence what Turing did for effective procedure:provide
something formal that captures our ordinary intuitions.  Ditto for
equality.  (TODO: quotes about problems with equality)

Paradigm: fiat currency.  Each dollar bill is "equal" to each other.
But each is also different, as a physical particular.  What makes them
equal is exchange value, which is socially instituted.  So equality is
a social institution.

Equality as exchange value depends on token recognition, so the
ability to recognize that a particular is a token of a type - this
thing is a \$1 bill - is more primitive than equality, but also enables
equality.  This dependency can be made explicit by writing a =\$ b
instead of a = b, meaning that a and b are equal under the
\$-denominated currency regime - that is, under a particular
perspective.  So equality is always perspectival, which translates
into equality-as-type in HoTT.

Proof of equality: type theories like HoTT need a "proof" for each
equality type; usually this is defined as refl.  But the formal
structure of such definitions in a type theory does not capture our
ordinary intuitions about equality.  We can replace the (arbitrary)
symbols "=" and "refl" with any others without changing the meaning;
in particular we could use "Unit" and "*", respectively, giving "*a :
Unit a a instead of "refl(a): a = a" (leaving type A implicit).  So
the formal definitions have no conceptual content except what is
instituted by the introduction and elimination rules.  We cannot rely
on antecedent notions of equality and reflexivity; the latter in any
case is a specifically mathematical concept, relatively remote from
our ordinary intuitions, and so cannot be counted as primitive.

What counts as "proof" for our ordinary notion of equality, the one
involving exchange value?  We cannot count on merely "knowing that a
and b are the same" (HoTT p. 47 "The basic way to construct an element
of a = b is to know that a and b are the same.")  That approach runs
straight into Hume's problem.  Recall that Hume pointed out that we
cannot observe causality.  We can observe event a, and event b, and we
can observe that b invariably follows a; but we cannot observe a
causal connection between the two, and so we have no iron-clad
guaranteed that b will always follow a in the future; we cannot "know"
that a causes b.  Similarly, we can observe that a is a \$1 bill, and
that b is a \$1 bill, but we cannot observe a relation of equality
between them; we cannot "know" by observation that they are equal.  Of
course we can know that they are both \$1 bills - tokens of the same
type - but that does not count as knowing that they are equal.  (Or
does it?  We might also say that we can know that they are equally
tokens of the same type but that gives us no means of demonstrating
that knowledge - there is no "type of knowing that two things are
equal".  Then again, knowing that a and b are tokens of a type is
knowledge of a "vertical" relation between token and type; the
"horizontal" relation of equality between such tokens is a different
thing.) As mentioned above, equality is a social institution, not an
observable, "objective" property.  The only way to demonstrate
("prove") that a and b are equal is to actually exchange them and have
both parties to the exchange come away thinking the exchange was fair.
So we could simply define "refl" to mean "act of exchange"; but that
doesn't seem very mathematical, since actions are dynamic and
mathematical objects are not.  What we need is an actual physical
token that counts as proof of the equality in the same way that a \$1
bill (or coin) counts as "proof" of \$1.

Demonstration as proof: note that the notion of proof in HoTT and
similar extends our ordinary notion of proof.  We don't ordinarily go
around saying that, e.g. 2 proves Nat, but in type theory that is
exactly what "2:Nat" means (or at least it is one accepted gloss of
the notation.)  But if we change "proves" to "demonstrates" we have
something a little less odd: SS0 (=2) demonstrates Nat just because on
inspection we can see that it is composed solely of the constructors
that define Nat, assembled according to the rules of the type theory;
it "shows" Nat in action, so to speak.  Note the close connection
between compositionality and demonstration (proof); it's not
accidental.  Similarly it makes sense to say that a \$1 bill or coins
demonstrates the concept of \$1, and that fair exchange of dollar bills
demonstrates their equality.

So what physical token would count as demonstrating the equality of
two \$1 bills?  Here again we rely on a social institution: a contract.
Instead of actually exchanging dollar bills, the parties can agree to
such an exchange, and certify their agreement by writing and signing a
contract that obliges them to make the actual exchange some time in
the future.  The contract then serves as a physical demonstration of
the equality of the goods to be exchanged.

Notice that this does not really count as "proving" anything in the
traditional sense of exhibiting justification.  Indeed, in light of
the fact that equality is a social institution, it doesn't even make
sense to try to prove that one \$1 bill is equal to another.  That
would make it sound like there is some kind of equality property or
relation that we can discover and exhibit, or that there is some fact
of the matter that we can expose.  But there is no fact of the matter,
nor is there any "objective" equality relation to be "proved":
equality is social and perpectival.  Things are equal only because we
agree to treat them as equal, and this applies as much to mathematics
as to medium sized dry goods.  To social instutions of mathematics may
be complex, but they remain social institutions.

To put it another way: proof of equality /institutes/ equality; fair
exchange of a pair of 1\$ bills is just what makes them equal.

(But: isn't that always the case?  Proof of P institutes P?  Not
classically; there, propositions and proofs are ontologically
distinct.  Classical proofs justify (assertion of) propositions;
intuitionistic/inferential proofs institute propositions.  A
proposition demonstrated cannot be ontologically severed from its
demonstration.)

So we can view "refl" as a kind of contractual certification of
equality.  On this view, it has nothing to do conceptually with the
notion of reflexivity - it does not mean that a thing stands in a
binary relation of reflexivity to itself; it means that two distinct
things have equal exchange value.

The critical observation here is that the "reflexive" formula "a = a"
contains not one but two distinct 'a' tokens.  They are not "the
same", any more than two distinct \$1 bills are the same.  But they do
have the same exchange value, which is the same exchange value that
all such 'a' tokens have.  So we can take "a = a" as an inference
rule, one that licenses exchange of 'a' tokens.  Then "refl(a) : a = a"
reifies that license, so to speak, in the form of token "refl" whose
role is to certify that such exchanges are fair.

Key point being that this avoids the notion of reflexivity as a binary
relation of a thing to itself.  We could make this more perspicuous by
replacing "refl" with something like "exch", or even some variation on
the "=" symbol, e.g.  "a=a! : a = a", meaning "a=a!" certifies that
all 'a' tokens have equal exchange value.

Another critical point: we derived "exchange value" from ordinary
practices, but we can also derive it solely from type theory
primitives by means of a notion of token repeatables, or maybe even
token reflexivity.  Token repeatables are always interchangeable.
Each token, qua symbol, "gives rise" to the type of its token
repeatables.  E.g. we have 3:Nat; but we also have the type of such
3-tokens (token repeatables), so every "occurence" of the symbol '3'
is a token of that type.  To make this explicit we probably need a new
type former that turns tokens into token-repeatable types.  Then we
can say e.g. 3:T(3), where T(3) names the type of all 3-tokens.  By
definition, every token of a T-type is interchangeable with every
other token of that type.

Normally we do not actually exchange such tokens, instead we just
write down fresh copies and count that as using the same token.  But
we could physically exchange them; we could cut the tokens out of our
piece of paper and shuffle them around.  So for a = a, we could
actually cut out those 'a' tokens and paste either of them in place
wherever we need an a-token in our proof.

No constants and no vars - the distinction presupposes denotational
semantics.  But in type theory we only have tokens and types.  When we
write \(x:A\) what we mean is that the inference from the particular
token \(x\) (which is token-repeatable) to type \(A\) is
\emph{assumed}, not that \(x\) is a variable of type \(A\) waiting for
assignment.

E.g. x:Nat, x = 1+1.  The latter does not mean ``assign the value of
1+1 to symbol x''; nor does it mean something like ``(it is true that)
the values of x and 1+1 are equal''.  It just means that we are
licensed to exchange x tokens and 2 tokens.



**************** old:


Goal is to explain type theory

    * in terms of the ordinary intuitions of lived experience

    * without relying on representational vocabulary like "refers", "denotes", etc.

    * without metaphysics or psychologism

The approach draws on Brandom:

    * normative pragmatics

    * inferential semantics

    * logical expressivism

Surprising results:

    *  the Unit type is incoherent

    * identity types (aka token-repeatable types) are primitive, all
       others derivative (in the order of explanation); this is
       because tokens are always repeatable, which gives rise to a
       type (token-repeatable type)

    * constructors are not functions; the former are primitive, the
       latter derivative

The starting point is the type-token distinction.  We show how this
relation can be explained in terms of practical norms instituting the
treatment of particulars as tokens of a type.

What is a type?  A type is an institution.  What is a token?  A token
is a particular in a functional role.


Token-particulars and perspectivism.  The only way a particular can
function as (play the role of) a token of a type is for us to treat it
as such.  We move from particular to token-particular - that is, from
particular to particular-as-token.  This is an inferential move, from
to particular to category.  It's inferential because categories
(concepts) are inferentialy articulated.  What Sellars called a
language-entry move.  But this move does not require language.  Or
rather categorization does not, though conceptualization does.  Even
primitive life forms categorize; in fact inanimate things categorize
(Brandom's example of a chunk of iron, rusting.)  But since we're
talking about human practices, it is proper to view the move from
particular to token-particular as an inferential move; let's call it a
token inferencing.

Token inferencing is a two-fold move: from particular to token of
type, i.e. inference to both type and functional role. (?)


A good "intuition pump" for illustrating the pragmatic basis of type
theory is the practice of tallying.  Before we can even begin to tally
- e.g. by dropping pebbles into a pouch, notching a stick, etc. - we
must have mastered the inferential practices involved in recognizing
particulars as tokens of a type.  We must have the practical ability
to treat distinct pebbles as indistinguishable tokens of a type - call
it Counter.  And this is a matter of practical norms - we treat
countings using pebbles as correct or incorrect, and that's about as
far as explanation can go.  It would be fruitless to try to explain
why or how we manage to recognize that a particular object is a
pebble; the best we can say is that treating it as a pebble just like
all the others is the correct thing to do.

So token inferencing is a primitive practice that precedes more
sophisticated token/type practices like tallying; the latter depends
on the former.  Neither depends on language; both are essentially
normative, practical skills.

The critical point here is that (the practice of) this sort of
inferential skill (token inferencing) is what institutes concepts.
Creatures can treat particular pebbles as tokens of a (nameless) type
even if they have no language; in fact, they do not even need an
antecendently available concept of type (or of "concept").  But once
they have instituted such normative practices they can say what they
otherwise can only do by expanding their vocabulary.  They can invent
words like ``that'' and ``pebble'', and go around point at things and
declaring "that pebble"!  Doing so makes explicit the normative
practice of treating things as pebbles; it makes the language-entrance
move explicit as a correct move.  The village idiot can go around
pointing at cows and exlaiming ``that pebble!'', but the other
villagers will treat him as linguistically incompetent.  ``That
pebble!'' as a kind of rule, which can be applied correctly or
incorrectly.

This is just what the standard type-theoretic notation "a : A" does.
Most expositions of type theory gloss this as "a has type A", or "a is
proof/witness/inhabitant of A", etc.  The proposal here is that we
should read this expression as exactly analogous to a propositional
implication like A -> B.  But only analogous; A -> B is a schema
licensing inference from proposition to proposition, whereas we treat
"a : A" as a schema licensing inference from particular to
token-particular.  We gloss it "from a infer A" or "from particular a
infer type A", or "infer that particular a is a token-particular of
type A".  It's also stipulative, not empirical; remember 'a' is a
token, not a symbol.

Note that token inferencing is perspectival; particulars only come to
play the role of tokens (i.e. are correctly treated as tokens) within
the context of particular purposes or "language games".  So it might
be better to gloss "a : A" as "treat particular a as a token of type
A, for the purpose at hand."  (Remembering that inferencing is
something we do that involves /treating/ things in correct ways, so to
treat a as a token of type A is to make a practical inference from
particular to token-particular.)

[Constants: 3:Nat is a rule licensing use of 3-tokens according to the
  rules of Nat.]


[FIXME: Now there is a (very) subtle but critical point here that is
  obscured by our very use of language.  We naturally tend to
  abstraction; most of us will treat the "a" in "a : A" not as a
  particular but as either a constant symbol that denotes a particular
  or a variable symbol that ranges over a collection of particulars,
  or maybe even an "unknown".  In any case, before it is any of that
  it is indeed a particular - a particular bit of ink on paper, or
  illumination on a screen.  And remember we have ruled out any appeal
  to denotation, so we cannot take the tokens in our expressions as
  refering to anything.]

But if "a : A" is an inference license (or: an inference that
institutes a license, i.e. a rule), don't we need to already know what
"a" is to proceed with the inference?  Only under denotational
semantics.  [Cmp. Platonic anamnesis] It can't mean "treat any old
thing as an A-token".  But if it just means "an A-token is an A-token"
then it does not involve inference.  The point is that a:A does not
express some relation or between or ``true'' fact about a and A that
we have discovered.  It makes no sense to say ``a:A'' is true, because
it is not a proposition, it is an inference (rule) that does not
depend no denotation .

The issue comes out more clearly if we go back to tallying practices
using pebbles.  It should be obvious that the pebbles involved in the
practice of tallying do not denote anything; in fact they are not even
symbols.  They are particulars; distinct entities located in space and
time.  Particulars are distinct by definition; to say that one two
particulars are "identical" is nonsensical, and to say that one
particular is identical to itself is vacous (Wittgenstein said
something like this, I believe.)  To move from particular to
token-particular means to ignore the particularities (the identity) of
the particular.  And it is by doing this that the notion of type
emerges (is instituted).  The Platonist might say that a particular is
a token of type A because it somehow (mysteriously) "participates" in
ideal A; the Aristotelean, that we somehow (mysteriously) derive the
abstract A from particulars.  The pragmatic perpective eliminates the
mystery: it is only by virtue of our treating particulars as A-tokens
that the type A and the role of A-token are instituted.

This has two subtle consequences that are not generally noticed in
type theories.  One is that the notion of a Unit type, with only a
single "element", is conceptually incoherent.  A type with only one
element would not be a generalization; the notion of type essentially
involves a plurality of tokens.  But can we not just stipulate that
type Unit has only one token?  We can try, but it won't work, since
that token is itself "repeatable".  In HoTT terms, if we have * :
Unit, we can create as many * tokens as we please.  The critical point
is just that these iterated tokenings are precisely *not* the same, or
identical, or equal: they are particulars.  We /treat/ them as the
same; but just think about it: that means we treat them as "identical
instances" of the one Tau type T(*).  And this illustrates the second
subtlety: it means that we are treating our one token itself as a
(proxy for a) type.  In fact every token of every type gives rise to
this sort of type, which we can call something like a token-repeatable
type.  People familiar with HoTT may recognize this as none other than
the Identity or Equality type of HoTT.

In other words, token-particulars have an inherently dual character.
On the one hand they are particulars; on the other hand, they are
/treated/ as indistinguishable tokens.

For example, we have 3 : Nat, glossed as "3 has type Nat".  But 3 also
is a type, namely the type of all 3-tokens.

Here's a more vivid example.  Using the standard inductive definition
of Nat, with Z and S as base case and successor, respectively:

    S S Z = S S Z

which says that 2 = 2.  The traditional way to explain this is in
denotational terms: the right and left hand sides of this equation
denote the same thing, which obviously must be the case since they
have the same form.

But the right and left sides of the equation are \emph{not} equal;
they are distinct particulars; they are not the same thing.  And since
we cannot appeal to denotation, we cannot say that they are the same
by virtue of denoting the same ``value''.  One way to explain the
meaning of this equation is to treat the two subexpressions as
distinct tokens of type T(2), which yields the following reading:

  "S S Z = S S Z" means that each "S S Z" token has type T(S S Z).

[BUT: what justifies this?  Only our exchange practices.  Go back to
pebbles, where each token is clearly a distinct physical particular,
and emphasize that every S and Z we write down is just as particular.
It is only our practice of treating such particulars (and their
accumulations) as interchangeable that institutes the idea of
equality.]

This is why it makes sense to say that a and b can be equal in more
than one way.  It just means that they are token repeatables.  And it
is always possible to ``create'' a new token, e.g. strike a new tally
mark or pick up a new pebble.

[FIXME: so we have two ideas we need to disentangle.  Once is
token-inferencing, the other exchangeability.  Plus the idea of a constructor.]


[We can treat token repeatables as ``the same'' effortlessly, without
  explicit rules; but this is precisely what machines cannot do, and
  that is why we need equality types in formal languages.]

In other words, the pragmatic perspective, emphasizing normative
inferential practices and particulars, leads directly the HoTT notion
of equality types.  But it explains that concept, not in terms of
equality of identification of tokens, but in terms of inference from
particular to token-particular - that is, to token-repeatable type.

And ultimately this is grounded in norms governing exchange or trade.

Concrete illustration: counting to two using two particular pebbles
can be done in two ways.

Also, treat e.g. S S Z as exactly like dropping pebbles - the S tokens
are particulars; they are every bit as particular as particular real
pebbles.

So every Nat "element" is a token-repeatable type.  E.g. any
particular 3 token has both type 3 and type Nat.

Constructors: in tallying our pebbles are constructors.  They are not
functions.  Ditto for S and Z: SSZ does not denote a number as result
of a function, it just /is/ a token from which we can infer Nat:

    SSZ : Nat

That is, SSZ is a particular (containing three particulars) that we
treat as a 3-token, which we treat as a Nat-token.  For this to be
intelligible we have no need of a function concept.  This is obvious
in the case of Z, which is analogous to our empty pouch.

More precisely: SSZ functions as (has the role of) a Nat in our game,
and S and Z function as tokens in that game; we can use tokens to
construct other tokens (e.g. adding an S token to SZ); but tokens are
not functions.  They don't do anything, although they /have/ a
function: they can be used (by rule) to construct new tokens.

Calling a ctor a function that constructs values of a type is like
calling a brick a function that builds walls.

Actions as types and tokens: concrete act of dropping a pebble as
token of type Tally (or TallyAction).  Then equality becomes
equivalence of action.

Reflexivity: a=a as a type is just the token-repeatability type of a,
and every a has this type; refl just means that each token of this type
is a token of this type.  Its a relation of token to type rather than
token to itself.

Tau types: every token is (naturally?) associated with a type, the
type of its token-repeatables.  The tau operator is analogous to
Church's lambda operator: just as lambda turns an open formula into a
name of its associated function, the tau operator turns a
token-particular into the name of its associated token-repeatable
type.  For example, T(2) names the type of all 2 tokens.

Now every token of a tau type is substitutable for any other token of
the same type; that's just what token-repeatable means.  (Particulars
are not repeatable, but token-particulars, that is tokens, are.)

Recall that "a : A" means that the inference from a to A is good.  So
e.g. a : Tau(2) means that the inference from a to Tau(2) is good.
What a : Tau(2) does not mean is that a is itself a 2-token.  So an
expression like 1+1 : Tau(2) is intelligible; it just means that from
the token 1+1 we can infer Tau(2).  And it makes sense in practice,
because we can (by the rules of the system) convert "1+1" to "2", and
we have 2 : Tau(2) by definition.  And notice that it works just as
well the other way around: 2 : Tau(1+1).  To make the inference, we
just rewrite 2 to obtain a "1+1" token.

Equality is closely related to tau types.  "a = b" means that a and b
are substitutable.  But we interpret substitution as an inference: "a
= b" means that the inference from any expression involving a to the
same expression with b replacing a is a good one.  So the equality
operator "=" is best thought of as an inference op.  To make this more
conspicuous we'll write "==>" instead of "=".

Caveat: a ==> b does not mean "from a infer b"; it means rather
something like "from any expression containing a, e.g. SEXP[a], infer
SEXP[b/a]".

How to relate tau types and this inferential interpretation of
equality?  How should we interpret e.g. "1+1 = 2"?  And what would
count as a "proof" of it?

First, a substitution inference rule is a schema; that means that the
rule a ==> b is not about the particulars embedded in the rule.
Rather, it relies on the implicit tau types T(a) and T(b); the
explicit formulation should be a:T(a) ==> b:T(a).

[NB: problem with the HoTT definition that given a:A and b:A, form
type \(a =_A b\).  This allows us to form e.g. 2 = 3.]

But (in HoTT at least) "a = b" just another way of expressing the Id
type: Id(a,b) (We're omitting the type sym A.), whose canonical
element is refl(a).  And this looks suspiciously like Tau(a).

The problem with the HoTT book is that it does not explain this, it
just glosses it with the very improbably "[t]he basic way to construct
an element of a = b is to know that a and be are the same".  There are
two problems with this gloss.  One is that "the way to construct... is
to know" makes no sense; constructing and knowing are not the same
thing.  The second and more basic problem is that the text offers no
explanation of what it is to know that a and b are "the same".  The
whole discussion is circular.  Equality, identity, sameness, and
related terms come out as unexplained explainers.

By focussing on normative practices of treating particulars as tokens,
treating token exchanges as correct, etc., we can offer a genuine and
non-mysterious explanation of the vocabulary of equality.  Terms like
equals, identification, etc. come out as explicitation devices that
allow us to say explicitly what we can otherwise only do in practice:
treat particulars as tokens, and treat one token as substitutable for
another for a particular purpose.  By foregrounding inference,
equality naturally emerges as a rule of (substitution) inference, and
since rules institute types, exchange practices institute equality
types, which the terminology of equality makes explicit.

Under this perspective the equality or identity types come out as
primitives - that is, primitive inference rules.

What about tau types?  There doesn't seem to be anything like a tau
type in the HoTT book.  But a tau type really just is the HoTT Id type
under a different perspective.  But there are differences.  In HoTT,
equality types are expressed in terms of two tokens of a single type:
given a:A and b:A, form the type \(Id_A(a,b)\) (equivalently, (\(a =_A b\)))
with canonical witness refl(a).  As noted above this does not rule out
things like "2 = 3" with witness refl(2).  Also, HoTT does not really
address the distinctions we've made between particulars,
token-particulars, token-repeatables, etc.  (I suspect that is because
the authors have a tendency to continue to think in
representationalist terms, which is not surprising since its a very
natural tendency).

So a question is: can we recover HoTT's Id types with their refl
witness just from our tau types and substitution-inference rules?


Token indiscernability: to treat a particular as a token-particular
(that is, as a particular that "has" some type) is just to treat it as
indiscernable from other token-particulars of the same type (for the
purpose at hand).  But "indiscernability" is an arcane philosophical
concept.  The practical basis for it is the normative practice of
substitutability or exchangeability.  It is only because we treat
particulars as tokens, and further because we treat tokens of the same
type as interchangeable, that we can speak of indiscernability.  We
don't need the philosophical apparatus of properties, satisfaction,
etc. to make sense of this.

(Compare chess pieces: distinct as particulars, but treated as
indiscernable - it doesn't matter which rook you put on the right or
left, though which rook you move does matter.  But even then you move
right or left rook, not this or that particular.  It's the position on
the board that makes them discernable not their properties as
individuals.)

So substitutability and token inferencing (or "tokenization"?) seem to
be equally primitive.  You cannot go treat a particular as a
token-particular unless you can also treat particulars as
substitutable "under the type".  Alternatively: it is not possible to
play the game of tokens and types if you cannot also play the
substitution game.  And vice-versa.

In more formal terms: tau types are primitive.  They are an
explicitation of the norms of both tokenization and substition
practices.

Natural types v. fiat types: the counter or pebble types in our
examples count as natural types.  They are instituted by our practices
but do not involve explicit rules.  The types of type theories like
HoTT are fiat types: they are instituted by explicitly articulated
rules.  Nevertheless the meaning of those rules is to be
explained in terms of normative practices.

Types as institutions, literally.

refl as token of type Tau(a).  What would count as a physical token of
equality in our tallying game?  In HoTT terms: to prove something for
all T(a) tokens, it suffices to prove it for ... T(a) itself?  Or any
T(a) token?  With the concept of T(a) we loose the notion of canonical
constructor.  Unless we want to treat Tau itself as the canonical
ctor.

But: tokens are not physical, particulars are.  Tokens (and types) are
(functional) roles, rules in the game.  So even our pebbles are not,
strictly speaking, physical tokens.  So a better question to ask is:
what plays the role of an equality token (i.e. a "proof" of equality)?
Answer: tau types?  A tau type T(a) is just the role that particulars
may play, as a-tokens.

Primary v. secondary identity.  Particulars have identities; this is
primary identity.  Token-particulars play roles: this is secondary
identity.  Each red checker piece has an identity as a particular; no
two pieces are identical in this sense.  But qua game pieces, they
play "the same" role; they are all "identical" in this derived sense.
And what is identical is not the pieces themselves, but the role they
play, since there is only one such role: red game piece.  Not the
pieces but their roles are "the same".  Every token-particular has two
identities in this sense, one primary (or primitive) and the other
secondary (or derived).  Derived from both primary identity and the
rules of the game.  The rules of the game institute the derived
identity, but the play that role, a game piece must (antecedently)
have primary identity, i.e. be a particular.

Inferential Type Semantics.  Or, Type-Inferential Semantics.

\section{Refl and Path Induction}

We don't need refl; it's an artifact of denotational semantics.  We
write ``a=a'' so we want to interpret this as a binary relation
between a thing and itself.  Which is (according to some,
e.g. Wittgenstein) nonsense.

Using Unit instead of Id, and ** (or ``exch'' or whatever) instead of
refl eliminates this conceptual wart from our system.  Then instead of
treating refl as the canonical constructor, and interpreting it to
mean that all a=b are ``freely generated'' by refl (nonsense), we
interpret our constructor \(**a: Unit_A a a\) as certification that
every token of type T[a] is exchangeable with every other such token.
We do not need a concept of reflexivity.

But: \(Unit_A a a\) and \(Unit_A a b\) are different types.  How can
one constructor work for both?  But are they really different types,
if a = b?  No; they only look like different types, since tokens 'a'
and 'b' look different.  But they are both tokens of the same Tau
type.  There is only one type \(Unit_A a \_\) for any give a.  Or,
there is only one type \(Unit_A \_ \_\) for every a:A - it doesn't
matter what the arguments to \(Unit_A\) are - fix either one and the
other must be equal to it for the type to be inhabited.

Why can't we say e.g. \(refl_2 : (2 = 3)\)?  In HoTT, refl is only
defined for (a=a), so we cannot write \(refl_a : (a = b)\).  But then
how can we prove a=b?  HoTT says that refl suffices for this, and the
reasoning is that paths a-b are ``freely generated'' by the constant
path at a.  Which implies that \(refl_a\) denotes the constant path at
a, although the HoTT book does not explicitly invoke denotation; it
just says ``We regard \(refl_a\) as being the constant path at the
point \(a\).'' (p. 48) The problem is that this principal of path
induction depends on a denotational interpretation; it can't be a mere
informal gloss.  Freely generated paths at \(a\) cannot count as
proofs of (a=b) unless we transfer this notion of free generation from
the semantic domain back to the type language; that is, we just
stipulate that any \(x:(a-b\) denotes one of the paths freely
generated by the constant path at a.  I.e. the proof tokens
\(x:(a=b)\) are ``freely generated'' by the canonical token
\(refl_a\), in the same way that paths a-b are freely generated by the
constant path at a.  Same concept, different domain.  The problem is
that ``free generation of tokens'' is not a concept of type theory,
since it is non-constructive.  Without (non-constructive) denotational
assignment we have no justification for treating \(refl_a\) as
sufficient for proving \((a=b)\), or for thinking that free generation
of paths justifies free generation of tokens (proofs).

(On the other hand: since we can freely create new tokens, it might
make sense to introduce an idea of ``free construction of tokens'' to
correspond to free generation of structures.)

So the HoTT book's account of path induction is understandable and
useful, but it is not genuinely constructive; it does not explain
equality types.

One possible reponse to this objection is that path induction
is actually defined by the Pi expressions in the book at page 49, not
by the ``informal'' explanation of equality in terms of freely
generated paths.  But what do those expressions really mean?  As far
as I can see, all they do is give two different ways of expressing
reflexivity, two ways of constructing a witness to a type built from
\(x, x, refl_x\).  Specifically, they build a new type from equality
types and refl, and provide two constructors (c and f) for that type;
that's all.  They specifically do not explain how refl can be counted
as sufficient for a=b.

In other words, the formal definition of path induction begs the
question; it does not explain how refl suffices, it just takes it for
granted that it does.

What's the heart of the problem?  Just that we can have multiple
``proofs'' of equality, i.e. a and b can be equal in a plurality of
ways.  The HoTT strategy for dealing with this is to privileged one
way of being equal (i.e. the constant path, the refl constructor), and
treat all the other ways as derivative (freely generated by the
privileged one).  We propose an alternative that does not privilege
any particular ``way of being equal'', in which reflexivity plays no
role.  Note that ``constant path at a'' presupposes that a path is a
function, which is not consistent with the notion that HoTT leaves out
the topology.  It would be more consistent to call it a null path, one
that goes nowhere, is not really a path.  On our approach, which
eschews the path interpretation, this corresponds to trivial notion
that every token has identity, which is a predicate rather than a
(reflexive) relation.

Better: free tokenization.  tau-types are ``freely tokenized''.

\section{Notes}

The HoTT book seems to treat its Id/= type as special.  But there's
nothing special about it; its one of infinitely many equality types
whose properties derive entirely from the structure of the type
theory, in particular the inductive structure of elements and
successor types.

``The key new idea of the homotopy interpretation is that the logical notion of identity a = b of two objects a, b : A of the same type A can be understood as the existence of a path... `` p. 3

`` In type theory, for every type A there is a (formerly somewhat
mysterious) type \(Id_A\) of identifications of two objects of A; ...''
p. 4 I think this is not right; there is no special type Id.  There
are families of relational types etc. and criteria for deciding when
to call something reflexive, symmetric, etc.  So there are lots of
equality types.

Just as every relation has reflexivity types \(a\circ_A a\) which may
or may not be inhabited, so they all have symmetry types \(p
\circ_{\circ_A} q\) which may or may not be inhabited.  No, because we
may not p and q.  We always have the types for which they are proofs,
however; maybe we should try relating those.

%% (p\ &\circ_{(x\circ_Ay)} p^{-1}){:}\,\Univ & \textit{---given \(p{:}(a\circ_A b),\  p^{-1}{:}(b\circ_A 

\subsection{Inductions}

Three kinds: mathematical, structural, transfinite.  What all have in
common is forward progress.  Which is missing from path induction.
Can we use successor induction to recover a notion of forward/upward
progress for path induction?  We need a way to construe \(p{:}x\circ_A
y\) as successor to \(\textsf{refl}_A(x){:}\,x\circ_A x\).

Both refl and p have successor types.  Need to prove conditional: if
tsucc(refl) then tsucc(p).  No way to do that, hence appeal to induction
principle.

Put it another way: if reflexivity types are inhabited, then so are
symmetry pairs.  So symmetry pairs together ``look like'' reflexivity
types.  Maybe their proofs are reflexive?

But how to avoid circularity?  If reflexivity types are inhabited,
then so are symmetry pairs -- but only for those symmetry pairs that
are inhabited, not all of them.  Not 2=3 and 3=2, for example, but 2=x
for all x equal to 2.

Note refl works for both canonical and non-canonical x:A.

\section{ST/FOL=}

``ST/FOL='' = Set Theory/First Order Logic with Equality

\subsection{Relations}

In ST/FOL=, a binary relation is a subset of \(A\times B\), where
\(A\) and \(B\) are sets.

Given set \(X\) and binary relation \(\circ\) on \(X\times X\):

\begin{itemize}
\item \textit{Reflexivity:}\quad \(\forall x\in X\ x\circ x\)
\item \textit{Symmetry:}\quad \(\forall x,y\in X\  (x\circ y)\leftrightarrow (y\circ x)\)

  E.g. both \(\neq\) and \(=\) are symmetric, but only the latter is also reflexive and transitive.
\item \textit{Transitivity:}\quad \(\forall x,y,z\in X\  (x\circ y)\land(y\circ z)\to (x\circ z)\)
\item \textit{Equality:}\quad a binary relation is an equality ``='' iff it satisfies reflexivity, symmetry, and transitivity
\item \textit{Anti-symmetry:}\quad \(\forall x,y\in X\  (x\circ y)\land(y\circ x)\to x=y\)

  E.g. \(\leq\) is anti-symmetric.
\end{itemize}

\subsection{Axioms of Set Theory}

\end{document}
