\section{Brandom on Assertion}
\label{sect:brandom}

\begin{remark}
  Relevance to type theory: type theory begins with an account of
  judgment, proposition, etc.  Robert Brandom offers a very
  sophisticated account of these concepts which IMO could be put to
  very good use in explaining the conceptual foundations of type
  theory.
\end{remark}

Brandom's ``deontic scorekeeping model of discursive practice'' is a
very sophisticated and ambitious philosophical project.  But the main
point of interest for us, his treatment of assertion, is relatively
easy to grasp.

First off, for Brandom logic is fundamentally \textit{expressive},
rather than epistemological.  ``Logic is for establishing the truth of
cerain kinds of claims, by \textit{proving} them.  But logic can also
be thought of in expressive terms, as a distinctive set of tools for
\textit{saying} something that cannot otherwise be made
explicit''.\citep[19]{brandom_articulating_2001} One of his favorite
examples is the inference from ``Pittsburgh is west of Princeton'' to
``Princeton is east of Pittburgh''.  We can endorse that inference as
a good material inference - material because it follows from the
meanings of the terms the sentences contain - even if our language
does not contain a conditional connstruction like ``if...then''.  But
once we extend our language by adding such a device, we can make that
endorsement explicit by saying ``If Pittsburgh is west of Princeton,
then Princeton is east of Pittsburgh''.  So ``if...then'' is an
expressive device, rather than an epistemological one.

Brandom's model of assertion involves both a
social aspect and a structure of commitment and entitlement.  The
basic metaphor is that in the game of giving and asking for reasons,
interlocutors maintain a deontic scorecard for each other and for
themselves, tracking commitments and entitlements.

``According to the model, to treat a performance as an assertion is to
treat it as the undertaking or acknowledgment of a certain kind of
\textit{commitment}---what will be called a 'doxastic' or 'assertional'
commitment.  To be doxastically commited is to have a certain social
status.  Doxastic commitments are normative, more specifically
\textit{deontic} statuses.  Such statuses are creatures of the
practical attitudes of the members of a linguistic community-they are
instituted by practices governing the taking and treating of
individuals \textit{as} committed.  Doxasitc commitments are
essentially a kind of deontic status for which the question of
\textit{entitlement} can arise.  Their inferential articulation, in
virtue of which they deserve to be understood as propsitionally
contentful, consists in consequential relations among the particular
doxastic commitments and entitlements---the ways in which one claim can
commit or entitle one to others (for which it accordingaly can serve
as a reason).''  \citep[142]{brandom_making_1998}

``Uttering a sentence with assertional force or significance is
putting it forward \textit{as} a potential reason.  Asserting is
givein reasons....The function of assertion is making sentences
available for use as premises in inferences.'' \citep[168]{brandom_making_1998}

``The basic model of inferential practices that institute assertional
significance...is defined by a structure that must be understood in
terms of the interaction of three different dimensions.  First, there
are two different sorts of deontic status involved:
\textit{commitments}, and \textit{entitlements} to commitments...The
second dimension ... turns on the distinction between the
\textit{concomitant} and the \textit{communicative} inheritance of
deontic statuses.  This is the \textit{social} difference between
\textit{intra}personal and \textit{inter}personal uses of a claim as a
premise...The third dimension of broadly inferential articulation
that is crucial to understanding assertional practice is that in which
discursive \textit{authority} is linked to and dependent upon a
corresponding \textit{responsibility}.... In asserting a claim, one
not only authorizes further assertions (for oneself and for others),
but undertakes a responsibility, for one commits oneself to being able
to vindicate the original claim by showing that one is entitled to
make it.''\citep[168-171]{brandom_making_1998}

``At the core of assertional practice lie three fundamental ways in
which one can demonstrate one's entitlement to a clam and thereby
fulfill the responsibility associated with making that
claim... First... one can demonstrate one's entitlement to a claim by
\textit{justifying} it, that is, by giving reasons for it.  Giving
reasons for a claim always consists in making more claims: asserting
premises from which the original claim follows as a conclusion... The
second way of vindicating a commitment by demonstrating entitlement to
it is to appeal to the authority of another asserter.  The
\textit{communcative} function of assertions is to license others who
hear the claim to reassert it.  The significance of this license is
that it makes available to those who rely on it and rassert the
original claim a special way of ischargin thheir responsibiity to
demonstrate their entitlement to it.''\citep[174; the third way
involves invoking one's own authority as a reliable noninferential
reporter, which is discussed later in MIE.]{brandom_making_1998}

%%%%%%%%
\subsection{Propositional Content}
\label{subs:}

Note that Brandom's notion of what it is to understand a proposition
or proof looks very different from Martin-L\"{o}f's.  But they
converge on the essential point, which involves grasping the
inferential relations among concepts and reasons.  For ML,
understanding a proposition means grasping what counts as a proof (or
something like that); for Brandom, it involves grasping the
``inferential articulation'' of the concept - the network of
propositions and inferences relating them that consitutes the concept
itself.  This is more or less just like ML's idea: to understand a
proposition is to grasp what counts as a reason for the proposition,
or---what is the same thing---entitlement to commitment to the
proposition.

%%%%%%%%
\subsection{Applying Brandom's Model}
\label{subs:bapply}

Let's look at what mathematical assertions and judgments look like
from a Brandomian perspective.

To start: we have a propositional content, which we can write as
``unasserted P''.  We have commitment, entitlement, and justification
(proof).

Uttering---or, usually, writing down---a proposition P makes explicit
one's commitment to the content of P, and makes one liable to
demonstrate entitlement to that commitment.  Hearing---reading---a
proposition P entitles one (by ``deontic inheritance'') to undertake a
commitment to P if one is willing to ascribe entitlement to the
utterer/author.  Otherwise, it authorizes one to demand a reason.  One
can also record (on one's ``deontic scorecard'') the speaker's
commitment to P while declining to undertake the commitment oneself.

{\todo MLTT analyzes the structure of (mathematical) assertion interms
  of proposition, judgment, truth, etc.  Map this structure to
  Brandom's structure.  Brandom's account should turn out better since
  it is more finely articulated, and distinguishes explicitly between
  commitment and entitlement.}

%%%%%%%%
\subsection{Understanding Propositions as Types}
\label{subs:brandomunderstanding}

There is a natural bridge from Brandom's account of assertion (of
propositions) to the type-theoretic notion of what it means to
understand a type.

In HoTT, to understand a type is to understand what counts as a
witness to or proof of that type.  Here ``witness'' is surely the
better word, since witnesses are always required to appear in court
and offer testimony in person.  ``Proof'' is too broad a term, since
(ordinarily) it encompasses classical non-constructive proof as well
as constructive proof.  A non-constructive proof does not produce a
witness; it can only offer hearsay testimony, so to speak.  Whatever
witnesses a non-constructive proof claim to have cannot be sworn in
and so cannot be interrogated, whereas a witness in the dock can be
directly challenged.  We can \textit{decide} whether or not to accept
the testimony of the witness---decide that the offered proof has a
flaw.

[On the other hand, the judicial metaphor breaks down at the crucial
  point: type theoretic witnesses do not really testify on behalf of
  the proposition; they \textit{produce} the proposition.  So maybe it
  would be better to drop ``witness'' in favor of some other term,
  like ``maker''.  (Actually we already have ``constructor''.)
  Ironically, the notion of ``truth-maker'' plays the central role in
  classic truth-conditional logic: the truth-conditions of a
  proposition are just those conditions that ``make'' it true.  Here,
  instead of truth-makers, we have type- or proposition-constructors;
  instead of truth-conditional logic, we have proof-conditional
  logic.]


For Brandom, understanding a proposition is to be construed in
inferential terms already.  The propositional content of an assertion
just \textit{is} its inferential articulation: the network of
inferences in which it plays the role of premise or conclusion.  So to
understand that propositional content is to grasp those inferences,
both upstream---inferences in which the proposition serves as
conclusion---and downstream---inferences in which it plays the role of
premise.

\marginnote{A caveat is in order here.  When Brandom talks about propositions and
inferences, he is not referring to the machinery of formal logic.  His
topic is ordinary discourse; for him, formal logic emerges as a set of
expressive devices whose functioning is parasitic on the normative
structure of discourse practices.  Nonetheless, the structure of his
philosophical account of discourse practices serves the needs of HoTT
just about perfectly.}

But an inference of which \(P\) is the conclusion is a \textit{reason}
for \(P\); it serves as justification that entitles one to a
commitment to \(P\).  It follows that understanding a proposition
means grasping what counts as its justification.  And this is just
what we need for HoTT.  The only difference is that HoTT imposes an
additional constraint: the only acceptable justifications are
constructive, ones that yield the proposition not only as the
conclusion of an inference but as the product of a procedure.  In
other words, constructive inferences.

Now: how do we get from this notion of what it takes to understand a
proposition to the notion of propositions-as-types?  First, notice
that Brandom's model imposes no restrictions on the \textit{kind} of
inferences involved in articulating propositional content.  That's
because his is the philosophical project of understanding language use
in general.\sidenote{On his account, assertion is fundamental; no
  practice can count as discursive practice unless it involves
  assertion.  His approach stands in contrast to Wittgensteinian
  approaches that treat language as a motley, with no ``downtown'', no
  one fundamental kind of speech act.  On the other hand, for
  mathematicians, logicians, and others working in similarly
  regimented idioms, assertion is usually not only the fundamental
  speech act, it is the \textit{only} speech act (within the idiom,
  that is).}  HoTT narrows this down by excluding from consideration
all but constructive inferences; in effect, it can be viewed as an
application of Brandom's model to a restricted subset of the universe
of discursive practices.  But in both cases, there is no antecedently
determinable restriction on the \textit{number} of distinct inferences
involved in the articulation of the conceptual content of a
proposition.\sidenote{It follows that the meaning of a mathematical
  statement like \(1+1=2\) is nothing more nor less than its proofs,
  which although distinct can be treated as equivalent---in the idiom
  of matheamtics, equivalent ``up to'' their common conclusion(up to
  iso-conclusionism?).  This gives us a single, univocal meaning for
  the proposition in spite of proof plurality.  Which is desireable,
  since we do not want to say that \(1+1=2\) has a variety of
  meanings.  Unity in diversity.  TODO: treat this separately.} In the
case of ordinary language, different people may have wildly different
reasons for undertaking a commitment to a particular proposition.  In
mathematics, as is well know, there many be many distinct proofs for a
given theorem.  This fundamental \textit{plurality} is what
underwrites the notion of propositions-as-types.  Since a proposition
\(P\) may play the role of conclusion in any number of distinct valid
inferences, we can generalize and say that \(P\) ``represents'' all
those inferences; or we can say, equivalently, that all those
inferences are the same \textit{kind} of inference, the kind whose
conclusion is \(P\).  Calling \(P\) the \textit{type} of those
inferences, and in turn calling the inferences \textit{witnesses} to
(or proofs of) \(P\) is then just a linguistic move---there is nothing
special about the word ``type'', and any similar word would have done
as well.  The critical point is the relation between the proposition
and its proofs.

To recapitulate: to say that a proposition is the type of its proofs
is just to say that those proofs form the category (or equivalence
class) of all proofs whose conclusion is the proposition; to use the
proposition as the designated symbol representing that class or
category; and to call the whole mess a ``type''.  By extension, we can
use the text of the proposition as the name of the type.  So for
example the proposition \(1>0\) is the (symbol of the) type of all
proofs whose conclusion is the proposition \(1>0\); or, the name
``\(1>0\)'' is the name of that type.

