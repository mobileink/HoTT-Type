%%%%%%%%%%%%%%%%%%%%%%%%%%%%%%%%
\section{Curry-Howard}
\label{sect:curry-howard}

\begin{ednote}
  Usually presented as ``propositions-as-types'', but this suggests an
  asymmetrical relationship; in fact the principle is that
  propositions \emph{are} types, and vice-versa.  This is a major move
  in type theory, introduced by \ML(?) based on work by Curry and
  Howard.  TODO: what exactly are the implications of this principle?
\end{ednote}

