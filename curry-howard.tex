%%%%%%%%%%%%%%%%%%%%%%%%%%%%%%%%
\chapter{Curry-Howard}
\label{sect:curry-howard}

\section{Two kinds of proof}

Proof in logic and math: \textit{discursive} proof.

Proof in ordinary language: evidence, demonstration, etc.

Empirical v. logical proof.

Example: to prove to you that there is a coin in this purse I can open
the purse and display the coin.

The ``proof'' of the proposition is thus a performance of a certain
kind: a proof-performance.  But \textit{kinds} are abstract; the
specific performance should thus be construed as a \textit{token} of
the kind.  In this particular case (displaying a coin in a purse), the
performance can be repeated.  Of course, each performance will differ
from all the others in its fine detail, but insofar as each repeat
performance counts as proof \textit{of the same kind}, each counts as
a \textit{proof token} of the same type.

But it is not merely a proof-token; a proof-token of the type that
proves a particular proposition.  Every specific proof-token is a
proof of a particular proposition P.  So a given performance of this
sort - a repeatable proof-token - is classifiable as a proof-of-P
token (=performance).  The notion of \enquote{proof token} is a
generalization over proof-of-P tokens for all P.

In the case of ordinary provings like the example given above -
consider the sort of ``proving'' that goes on in schoolyards, where
proving means showing - proof is non-discursive: it does not involve
explicit reasoning.  Caveat: this may count as a kind of empirical
proof, but is not to be confused with inductive reasoning.  What makes
a given performance count as a proof is a deep question we won't go
into here, but we all know that e.g. displaying a coin in a purse
counts as proof of the proposition that the purse contains a coin.

NB: the original proposition ends up as the conclusion of a piece of
practical reasoning: I see a coin in the purse; therefore there is a
coin in the purse.

In the case of mathematical and logical reasoning, proof involves
discursive performance.

Written proofs as traces of discursive performances.

The proposition to be proved ends up as the conclusion of the proof.

The type of a proposition (statement, etc.) is \textit{not} the type
of its proofs.  The type of a proof of P is exactly
\textit{proof-of-P}, not P.  More exactly,
proof-whose-\textit{conclusion}-is-P.  That's the kind of thing such a
proof is: it's the sort of thing that counts as a proof of (proves) P.

So what is the type of a proposition?  The question is malformed; what
we really want to know is, what is the type of a proposition
\textit{token}.  The answer is obvious but hard to articulate clearly
in English, due to the inherent circularity of the type/token
distinction.  The type of a token is just its type; the tokens of a
type are, well, its tokens.  This page has many tokens of type
\enquote{the}.

\subsection{Tokens, Terms, Types}
\label{subsec:tokens-terms-types}

To communicate clearly about these issues, we need special notation.
Quote marks are insufficient; they turn an expression into a name of
the expression.  The famous example (Tarski's convention T) is:

\enquote{Snow is white} if and only if snow is white.

This sentence contains two tokens of type `snow is white'; the first
is mentioned, the second used.  Technically the quoted version
functions as a name (thus mention) of the sentence, while the unquoted
version is just the sentence (used).  The quoted version does
\textit{not} denote or indicate the type of the sentence.  For that we
can use a designated notation such as \(\ulcorner \urcorner\), so that
\(\ulcorner\)3\(\urcorner\) refers to the type of tokens of the form
`3'.  Call these token-type quote marks.  When we need to explicitly
refer to some symbol \textit{qua} token, we use \(\llcorner \lrcorner\)
and write \(\llcorner 3\lrcorner\).  So \(\llcorner 3\lrcorner\) is a
token of type \(\ulcorner 3\urcorner\).

\subsection{Structure of Proofs}
\label{subsec:structproofs}

So: we have a proposition P, and we have a discursive proof of P.
What \textit{kinds} of things are involved here?  What is the
structure of the kinds?  Kinds rather than types, because we want to
reserve the notion of type for syntactic duty: a type system is a kind
formal notation that combines the notions of syntactic calculus and
kindedness.

A proof of P is a proof token whose conclusion is a token of type P
(and type P is in turn a token of type Proposition.)

The \textit{written} form of the proof token as a trace of a proof
process or computation.  So conclusion of a (static) written proof ~
end result of a (dynamic) computation or construction.

Compare proof of an int and proof of a proposition.  An integer symbol
like \(23\) is a formula, just like a propositional formula.  It
denotes a device that computes a result.  This is true even of
``simple'' symbols like \(3\): in contrast to the denotational
perspective, under which \(3\) simply denotes the integer, under the
constructive perspective the symbol \(3\) denotes a device that
computes the integer.  The type/token distinction applies here just
like it applies to propositions and proofs: a computation (proof) of
\(3\) is a process/proof/computation/whatever whose conclusion is a
token of type 3.  That type in turn is a token in the type Z
(integer).  Similarly, the type of a proposition token is a token of
type Proposition (or we might call it type Provable, or Proven or even
True or the like).

Remark: token and term.  Same thing?  Not really.  Term contrasts with
type, just like token, but at a different level of abstraction.  By
example: 3 (on the page) is a token of type \(\ulcorner\)3\(\urcorner\),
which in turn is a term of type Z (here ``type Z'' means type of
values, rather than \(\ulcorner Z\urcorner\), the type of `Z' tokens).

Now how does this related to Curry-Howard?  In particular proof checking etc.?

We can interpret the usual formulation \enquote{a proposition is the
  type of its proofs} to be an abbreviated way of saying that the type
of a proposition serves to categorize proofs whose conclusions are
tokens whose type is the proposition.

Key concept: token-repeatability.  In the example of pulling a coin
from my pocket in order to prove the proposition that there is a coin
in my pocket, once I have performed the proof, I cannot repeat it,
since the coin is no longer in my pocket. (See: linear logic.)  But if
the proposition is that there is a coin in purse, I can prove it by
opening the purse and displaying the coin.  Since the coin stays in
the purse, I can repeat this proof as often as I like: produce as many
proof-tokens of this kind as I wish.

In the case of formal logic and computation, proofs are repeatable.

\section{Proofs and Propositions}

The usual formulation is something along the lines of \enquote{a
  proposition is the type of its proofs}.  But this obviously cannot
be quite right: propositions and proofs are distinct \textit{kinds} of
things, so how can a proof be a kind of proposition?.  We would never
say that a building is the type of its blueprint; why say that a
proposition is the type of its proofs?

The problem is that the standard terminology ``forgets'' about
computation.  They type of a compound expression is by definition the
type of its \enquote{result}.  In the case of mathematical
expressions, the result (of a computation) is a value of a certain
type; in the case of propositions, the result (of a proof) is either a
proposition or a truth value, depending on your preferred perspctive.
In both cases, it would be more accurate to talk of both the type of a
computation and the type of the result of a computation.



\section{Misc. notes}
\begin{ednote}
  Usually presented as ``propositions-as-types'', but this suggests an
  asymmetrical relationship; in fact the principle is that
  propositions \emph{are} types, and vice-versa.  This is a major move
  in type theory, introduced by \ML(?) based on work by Curry and
  Howard.  TODO: what exactly are the implications of this principle?
\end{ednote}

\begin{ednote}
  The critical point is that we go minimalist: start with the minimal
  logical language, which means combinatory logic.  It is the
  isomorphism between the logical constants and the combinators
  (Curry) that motivates Curry-Howard.  Once you see the connection at
  this minimalist level, it is easy to see it at any level, since the
  logical constants are the basic building blocks from which all
  propositions are constructed.
\end{ednote}


\begin{ednote}
  Start with Schoenfinkel and Curry, and the goal of finding the
  absolute minimum, which means eliminating variables.  Then the basic
  combinatorys, then the isomorphism to the logical constants.

  Equivalence of combinatory logics (no vars) and lambda calculus (vars)
\end{ednote}


Analogies.  Proof/proposition, term/type: ``There is also a one-to-one
correspondence between proofs of a certain proposition in constructive
predicate logic and terms of the corresponding type.'' (Dependent Types at Work)

%%%%%%%%%%%%%%%%%%%%%%%%%%%%%%%%
\section{bhk}
\label{sect:bhk}


\begin{ednote}
  Importance of metaphors.  See ML on BHK: proof as task to be
  accomplished, problem to be solved.  Add another metaphor:
  destination to be reached.
\end{ednote}
