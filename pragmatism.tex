%%%%%%%%%%%%%%%%%%%%%%%%%%%%%%%%
\section{The Pragmatist Enlightenment}
\label{sect:enlightenment}

%%%%%%%%
\subsection{Liberation}
\label{subs:liberation}



%%%%%%%%
\subsection{Pluralism}
\label{subs:pluralism}

\begin{ednote}
  Not just propositions-as-types, but types-as-propositions.  Example:
  the type \N can be viewed as a proposition ``there exists a natural
  number''.  This means that there is no authoritative definition of
  what a type is, which means that pluralism is an essential aspect of
  type theory.  Is this a sharp contrast with traditional mathematics?
  For pre-modern mathematics, number was unequivocally quantity or
  magnitude - no pluralism there.  Modern mathematics discarded
  quantitative interpretations of number in favor of structural
  notions.  The issue of pluralism is not so clearly decided there.
  Once you have isomorphisms, you can't really say that one structure
  is \emph{the} structure for a given class.  Groups, for example.  So
  isn't modern math essentially pluralistic?  Well let's look at
  foundations - set theory doesn't seem to be very pluralistic; a set
  is a set is a set, and not something else.  You can come up with
  distinct set \emph{theories}, but they all depend on the primitive
  notion of set, or maybe set membership.  Type theory, by contrast,
  seems to be different.  It doesn't have this kind of unity.  In fact
  there are many distinct type theories, so we should probably always
  use the plural.  The primitive seems to be ``type''; but the concept
  of type is not primitive in all type theories---\HoTT{} being a case
  in point.  ``In fact, no type former is 'primitive' to the game of
  type theory in this sense: you can very well have a type theory with
  no type formers! But it won't be very interesting...'' (M. Shulman,
  \href{https://groups.google.com/d/msg/hott-amateurs/U1X0m4r6G-A/K5eeMSPXE5YJ})
\end{ednote}

``Type theory, formal or informal, is a collection of rules for
manipulating types and their elements.  But when writing mathematics
informally in natural language, we generally use familiar words,
particularly logical connectives such as “and” and “or”, and logical
quantifiers such as “for all” and “there exists”. In contrast to set
theory, type theory offers us more than one way to regard these
English phrases as operations on types. This potential ambiguity needs
to be resolved, by setting out local or global conventions, by
introducing new annotations to informal mathematics, or both.''\HoTTB, p. 101

%%%%%%%%
\subsection{Normative Pragmatics}
\label{subs:normprag}

  Chapter 1 of \cite{brandom_mie}

%%%%%%%%
\subsection{Inferential Semantics}
\label{subs:inferentialism}

  Chapter 2 of \cite{brandom_mie}


%%%%%%%%
\subsection{Expressivism}
\label{subs:expressivism}

See \cite{price_expressivism_2013}
