%%%%%%%%%%%%%%%%%%%%%%%%%%%%%%%%
\section{Assertion and Judgment}
\label{sect:assertionjudgment}

%%%%%%%%
\subsection{notes}
\label{subs:notes}

This section needs some serious revision.  Here's the straight dope,
in a nutshell.  In his paper ``Truth of a proposition, evidence of a
judgement, validity of a proof''\citep{martin-lof_truth_1987}, which
is specifically about the philosophical basis of \ITT{}, \ML{}
attempts to explain the concepts proposition, truth, evidence,
proof, and validity.  The first part of the paper, which gives some
historical and conceptual background, is just right for the most
part.  He points out, for example, that for the intuitionist proof
comes before truth.  But he makes a major blunder when he claims
that the classic truth-conditional account of the logical
connectives, an account that is based on truth table semantics, and
the BHK accounts, which treat a proposition as an expectation or
task etc., are just different ways of saying the same thing.
``Façons de parler'', as the saying goes.  But I think this is flat
out wrong.  Classic and intutionistic logic may use the same
formulas, but they could not be more different conceptually.
Classic truth-conditional logic presupposes something like what
Wittgenstein called (in his Tractatus days, at least) a picture
theory of meaning. (I may not be getting the exact wording right
here, but the idea should be clear enough.)  Proof in that kind of
logic has nothing to do with construction; it's all about
correspondence, a representational relation between language and
objective reality.  Obviously there's much more to be said about
this, but suffice it to say that \ML's claim that classic logic and
intuitionistic BHK logic are in the same line of business strikes me
as not only wrong but a little bit shocking.  So wrong that I have
to wonder why he made that sort of claim.  Maybe he was unfamiliar
with the pragmatist literature.  And we'll just proceed on the
assumption that I am not wrong, if you don't mind.  I'll provide a
more detailed justification of this claim later.

Another thing that looks wrong to me is his account of the BHK
interpretation of propositions; but in this case, he has an excuse:
R. Brandom's more refined account of assertion and proposition was
not yet available.  Brandom's account makes the problems with \ML's
account quite clear.  The latter follows BHK in treating a
proposition as a problem to be solved, a task to be accomplished, or
an expectation of a proof, etc.  The problem is that propositions
are clearly exactly \emph{not} these things.  Obviously we may
\emph{treat} a proposition as e.g. a task to be accomplished; but
that does not determine what a proposition \emph{is}.  Or to put it
differently \ML{} seems to ignore the significance of \emph{force},
which is distinct from conceptual content.

For now I don't have time to explicate the point in detail, so
here's the short version: Brandom divides assertion into commitment
and entitlement.  And what makes the proposition primitive is that
it is the minimal unit of \emph{responsibiity} - Brandom traces this
notion to Kant.  To assert a proposition is to undertake a
commitment to it, and also to license others to challenge ones
entitlement to that commitment.  Thus it inescapably involves a kind
of responsibiity: the responsibility to justify (``vindicate'', as
Brandom says) one's commitment.  One way to do this is to
demonstrate the entitlement by giving \emph{reasons} for it.

If you're familiar with the \ML{} paper mentioned above, the
connection should be fairly obvious.  Commitment and entitlement are
deontic attitudes, which institute deontic statuses (e.g. being
correct or incorrect).  They are emphatically \emph{not} properties
of propositions.  So it is just a mistake to think that propositions
are or express tasks, problems, or expectations.  On the other hand,
\emph{assertion} of a proposition does give rise to a responsibility
to vindicate commitment.  Talk of ``expectation of a proof'' is
entirely intelligible as a way of saying that assertion licenses
others to challenge one's commitment---to expect that one can or
will prove it.  Talk of task or problem is really a way of talking
of the justification or vindication that one is responsible
for.\sidenote{And note that this is not a mere matter of voluntary
  acceptance of responsibility; it arises because assertion licenses
  others to \emph{hold} one respondible, treat one \emph{as}
  responsible, and therefore sanction speakers who fail to vindicate
  their commitments.}

So in the end, \ML{} is speaking more or less the right vocabulary,
but his explanation is off, and his characterization of propositions
as involving something in addition to propositional content
(e.g. expectation, task, etc.) is not defensible, at least from a
Brandomian perspective.  \HoTT{}, unfortunately, duplicates his
error in its account of judgment.

The remedy is close at hand, though.  All we need do is recognize that
what \HoTTB calls ``judgments'', like ``a : A'' and ``a := b'', are
really \emph{stipulations}, and what it calls propositions,
\emph{assertions}.  A stipulation, unlike an assertion, does not
require justification.  A stipulator does not license listeners to
demand reasons for the stipulation.  Of course they can make such
demands, but almost by definition we can stipulate \emph{ad libitum}.
Listeners who don't like our stipulations can go elsewhere; but no
\emph{rational} challenge can be mounted against a stipulation.  The
reason it makes no sense to ask for a proof of ``a : A'' is not
because it is a ``judgment'' but because it is a stipulation and
therefore needs no justification.  By contrast, ``propositional
equality'' is just a fancy (and rather unfortunate) term for
``asserted equality''.

Another way to look at it: \HoTT, following \ML{}, attributes
special properties to propositions.  But that's hard to defend,
philosophically, and doesn't amount to genuine explanation; a better
way is to explain assertion in terms of deontic attitudes and
responsibilities.

%%%%%%%%
\subsection{Judgment}
\label{subs:judgment}


The account of judgment offered in the HoTT Book doesn't really work.
Ditto for Martin-L\"{o}f's account.  For example, it makes sense to
say ``P is a proposition'', but it doesn't make sense to say ``P is a
judgment''.  That's because judgment is a act, something one does.

On the other hand, ``judgment'', like ``proposition'', can be treated
as a verbal noun or as a ``plain'' noun.  Saying ``P is a
proposition'' is usually taken to mean that P refers to what has been
proposed.  There is no obvious reason not to treat ``P is a judgment''
in a similar manner: P refers to what has been judged.

However, there is a difference.  Judging a proposition (what was
proposed) amounts to \textit{evaluating} what was proposed, as good or
bad, true or false, or whatever.  By contrast, proposing a proposition
amounts to merely exhibiting it for consideration.  This arguably
involves an implicit evaluation - to propose a proposition is to
implicitly claim that it is good, or true, etc.  But proposing does
not involve offering an evaluation that is distinct from what is
proposed, whereas judgment does.  The two are distinct kinds of speech
act, and refering to the content of a speech act is not the same as
referring to the speech act itself.

Furthermore, it is not correct to treat the nominal sense of
``judgment'' as being the content, what has been judged.  The nominal
sense of ``judgment'' refers to the act of judgment itself, and not
the proposition judged.

Actually, by the same reasoning it is not correct to say that the
nominal sense of ``proposition'' is what-is-proposed; rather, it is
the act proposing, nominalized.  This makes perfect sense when you
consider that ``proposing'' can also be nominalized; ``the proposing''
is another way of saying ``the proposition''.

The same goes for all -tion words: suggestion, opposition, etc.  In
each case, the word can refer to the doing, or to what is done, and
what is done is always the act of doing itself -- not the subject or
object of the doing.

This suggests we should make a distinction between, for example, the
content of a proposition and ``proposition''.  But this term seems to
be a special case; it has the usual plain noun sense of
what-was-proposed, the usual verbal sense of ``proposing'', but also
the nominalized verbal sense of ``act of proposing''.

(But then the same considerations apply to ``judgment''.  The
difference must go back to semantics.)

\begin{remark}
  The Arabic grammatical tradition captures this distinction
  beautifully, mainly because the structure of the language makes it
  simple to do so.
\end{remark}

Or put it this way: when we judge a proposition like ``2+2=4'' to be
true, the what-was-judged is not ``2+2=4'' but the truth of ``2+2=4''.

\begin{remark}
  But how is this different from ordinary predication, like ``The
  triangle is red'' as a proposition?  Should we say that what is
  proposed is not that the triangle is red, but the redness of the
  triangle?  No, since we're treating it as a propostion, and the
  whole thing is proposed (exhibited).  If we judge it to be true,
  then again the judgment 
\end{remark}

So saying ``P is a judgment'' is incoherent if P is taken to refer to
nothing more than what is proposed.  If P refers to a claim of the
form ``X is true'' (or good, etc.), then ``P is a judgment'' seems to
make more sense; but it doesn't, really.  P still refers to an
unasserted content; to make sense, we would have to say something like
``P is a judgment when asserted''.  More explicitly, ``'X is true' is
a judgment'' (or better, ``'X is true' expresses a judgment'') only
\textit{exhibits} ``X is true'', which is a proposition, not a
judgment.  As a proposition it expresses a judgment; but when embedded
(equivalently, quoted) it does not express anything.

\begin{remark}
  Compare: ``Snow is white'' iff snow is white.  The quoted bit is a
  name of the sentence; it counts as a \textit{mention} of the
  sentence, which has no force.  The unquoted version of same is the
  sentence itself; it counts as a \textit{use} of the sentence, which
  has assertional force.  Obviously, the occurances of ``P'' in ``P is
  a proposition'' and ``P is a judgment are names of a proposition and
  thus mentions.  So they have no force.
\end{remark}

The key point is Frege's point: the content of a proposition is
distinct from the force of the utterance.  That means that P in ``P is
a proposition'' is unasserted, just as it is when embedded, as in ``If
P then Q''.  The truth of ``P is a proposition'' is independent of the
truth of P.

So even if we take the act of declaring ``P'' to be an act of
judgment, it does not follow that a reference to P is a reference to
the act of judging that P.  Hence there is no way to make ``P is a
judgment'' work.  If we take P to refer to what was judged, that again
is a proposition (or propositional content), so ``P is a judgment'' is
incoherent.

\begin{remark}
  We can assert that P, and we can assert P.  We can judge that P, but
  we cannot judge P.  I don't think this is a mere grammatical
  distintion; I think it reflects a genuine semantic difference.
\end{remark}

