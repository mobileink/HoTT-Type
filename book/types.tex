%%%%%%%%%%%%%%%%%%%%%%%%%%%%%%%%
\chapter{Types}
\label{sect:type}

\HoTTB page 27 describes a ``general pattern for introduction of a new
kind of type''.  Martin-L\"{o}f does this too, somewhere.  In \HoTTB,
the list is

\begin{description}
\item [Formation Rules]
\item [Introduction Rules]  or constructors
\item [Elimination Rules] or eliminators
\item [Computation Rules]  ``which express how an eliminator acts on a constructor''
\item [Uniqueness Principle] which ``expresses uniqueness of maps into
  or out of that type.  Optional.
\end{description}


The question is where to place this stuff in the description of \HoTT.
Are these things primitives?  Do they form essential aspects of a
type?  Or in other words, can we have (think of) types without these rules?

\HoTTB introduces them almost as an afterthought, as a Remark in the
third major construction defined in Chapter 1.  But I suspect this is
a mistake or oversight; it looks to me like these rules are indeed
fundamental, essential to the concept of type.  In that case, they
should be presented along with the introduction of the type concept,
rather than in the middle of a description of a particular type.

%%%%%%%%%%%%%%%%%%%%%%%%%%%%%%%%
\section{Terms}
\label{sect:terms}

\begin{ednote}
  ``Terms'' is Awodey's terminology.  More common terminology include:
  witness; inhabitant.  Also proof.
\end{ednote}

``Under the Curry-Howard cor- respondence, one identifies types
with propositions, and terms with proofs...''\cite{awodey_tth}

%%%%%%%%
\section{Witness}
\label{subs:witness}

\begin{ednote}
  In what sense is a proof a witness to a type, or an ``inhabitant''
  of a type?  Intuitively this language does not work very well; we
  don't intuitively think of a proposition as a type ``inhabited'' by
  proofs.  The notion of proof as ``witness'' to a type is a
  substantive epistemological notion; it not only says that the proof
  is related to the type, but also it says something about the nature
  of that relationship.

  The trick is to see it from the perspective of the machine.  A
  proposition like \(1+1=2\) is just a form to the machine.  We can
  see that it is true just by looking, due to some mysterious
  epistemic capability.  But machines do not have epistemic abilities;
  a form is a form is a form to a machine.  Hammer, nail.  So in order
  for the machine to treat \(1+1=2\) as a \emph{true} proposition, we
  have to give it something more: a proof.  But ``proof'', again, is
  an substantive epistemic notion; the machine analog must be purely
  formal.  From the machine perspective, a proof is just another form,
  or rather, collection of forms (including inference rules as complex
  forms): to give the machine a proof of P we must provide it with a
  form or forms that ``lead to'' (produce, result in) P.  To prove a
  proposition to a machine, we give it forms and reduction rules such
  that the formal use of those forms and rules results in the form of
  the proposition to be proved.  (FIXME: a more accurately way of
  putting this would involve reduction of formulae to normal form,
  confluence, etc.)

  So we can think of a proof as a kind of device---just another
  machine (or machine description), but one whose sole output is the
  proposition to be proved.  Since for any given proposition there may
  be many ways of building such a proving device, we can treat these
  devices as forming a kind of equivalence class, which we can
  identify by taking (the form of) the proposition as a symbol
  referring to the class.  Now the connection to types and witness
  becomes clear: the equivalence class of such proving devices forms a
  type, the type of the devices (proofs), and each device (proof)
  ``inhabits'' (or as we would prefer, expresses) the type.
\end{ednote}
