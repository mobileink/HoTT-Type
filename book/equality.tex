%%%%%%%%%%%%%%%%%%%%%%%%%%%%%%%%
\chapter{Equality: What's the Big Deal?}
\label{chap:equality}

\begin{ednote}
  Equality is arguably the most important concept of \HoTT{}, as far
  as I can tell, because of the ``Univalence Axiom''.
\end{ednote}

\begin{ednote}
  Intuitively, \(2+2=4\) is trivially obvious.  But it is also
  ``obvious'' that no two distinct things in the world can be
  ``equal''; that is, they cannot be the same thing.  If that is the
  case, then what ought we to make of \(2+2=4\)?  This is Frege's
  problem: what does it mean to say that \(a=b\)?  If \(a\) and \(b\)
  are the same thing, then we learn nothing from \(a=b\).  If they are
  not the same thing, what does it mean to say that they are equal?

  In the real world we often treat distinct things as equal; for
  example, two distinct apples are not the same thing but they are
  ``equally'' apples.  This suggests we should add a clause to
  equality statements: distinct things may be equal \emph{under a
    description}.  This is code for saying that although they are not
  the same thing--not in fact equal--they are nonetheless
  \emph{equivalent} from a particular perspective: equivalent under a
  description.

  So we can adopt the same (pun intended) perpsective with respect to
  mathematical equalities.  We can treat \(2+2\) and \(4\) as distinct
  things that are equivalent under a description--in this case, the
  ``description'' is the lambda calculus: computation.  The former
  expression reduces to the latter, which is taken as a cannonical
  form.

  Notice that this perspective is metaphysically conservative; it does
  not say anything about the denotations of \(2+2\) and \(4\).  It
  does not make any claim as to what the expressions ``really'' mean
  (denote).  In other words, this is an intensional perspective.  This
  makes intuitive sense, since \(2+2\) obviously ``means'' something
  more than just \(4\); in particular, its meaning includes the notion
  of addition.

  The upshot is that equality turns out to be code for equivalence.
  The Univalence Axiom says that the relation between equality and
  equivalence is itself and equivalence: equality is equivalent to
  equivalent.
\end{ednote}

\begin{ednote}
  Note the importance of equality/equivalence in uniqueness proofs
  (induction and coinduction).  See Jacobs and Rutten Tutorial on
  (Co)Algebras and (Co)Induction.
\end{ednote}

\begin{ednote}
  Illustration: the Myth of Purity.  Take \(2+2=4\) as an example.
  The Myth of Purity says that (in a ``purely'' functional language)
  the meaning of \(2+2=4\) is exhausted by the denotational semantics
  of the language -- each symbol denotes a value, and a (syntactically
  correct) combination of symbols of this kind (a sentence or
  proposition) denotes a truth value, these denotations exhaust the
  meaning of the expression.  In other words, \emph{no side-effects}.
  This corresponds to a ``classical'' (set-theoretic, extensional,
  representational, etc.) perspective, in which things have meanings
  (denotations) independent not only of what we know to be the case,
  but of what we \emph{do}.  But from a computational (intuitionist,
  constructivist, pragmatist, etc.) perspective, the meaning of
  \(2+2\) involves (represents, encodes, -- pick your terminology)
  computation, and \emph{computation always has side-effects}.

  Computation is something that we (or our machines) \emph{do}; it is
  not merely something we \emph{know}.  Computation always consumes
  space, time, and energy.  If you choose to think of \(2+2=4\) in
  denotational, extensional terms (which is fine, by the way), you
  effectively ``lose the phenomenon'', as the ethnomethodologists say.
  If you actually ``execute'' such a statement on a real-world
  computer, it will have side effects---its ``meaning'' will include
  consumption of memory, cpu cycles, and energy.  Such side effects
  are of course orthogonal to the part of meaning in which we are most
  interested, namely the value of \(2+2\), the correctness of
  \(2+2=4\), and so forth; but it is nonetheless an inalienable
  component or aspect of the meaning of the expressions.

  [Another way to illustrate: lambda expressions can be viewed as
    representations of their normal forms, or of the reduction
    ``process'' that computes normal form, or both.  The former view
    corresponds to the classic denotational perspective, the latter to
    a constructivist perspective.]

  This is also true from a theoretical perspective, but it appears
  under a different, theoretical description: complexity.  Even if we
  abstract from real world silicon and electrical grids, computation
  cannot be separated from complexity.

  [TODO: link mathematical representations of computational complexity
    with real-world correlates time, space, and energy.  It actually becomes more clear ]

  The moral of the story is that \emph{there are no pure languages},
  functional or otherwise.  At least, not in the sense of ``pure''
  that means something like ``purely denotational'' or the like.

  This does not mean that there are no pure \emph{perspectives}, or
  that such perspectives are silly or irrational.  Quite the contrary;
  it is perfectly reasonable--rational--to adopt a purely denotational
  perspective on computation, if you find it \emph{useful} to do so.
  And there are many good uses to which such a perspective can be put.
  But what is not reasonable is to make the further claim that this
  perspective is the One True Picture of the Way Things Really Are.
  That is a metaphysical claim, not a mathematical or logical or
  computability claim.  In other words, it is \emph{not} legitimate to
  claim that the classic perspective is primitive, privileged above
  other perspectives.  By the same token, neither is it legitimate to
  claim that a computational (constructivist, etc.) perspective is
  Mother Nature's Own Perspective.  The byword here is
  \emph{fundamental pluralism}---recognition of the legitimacy of a
  variety of perspectives (vocabularies, languages, etc.), no one of
  which can lay claim to foundational status prior to all the others.

  To return to the example we started with: \(2+2=4\) can be viewed
  from a variety of perspectives, none of which can claim to offer the
  whole truth and nothing but the truth about its meaning.  It is
  sometimes \emph{useful} to talk of a ``purely functional''
  interpretation of such expressions, but such talk should not be
  mistaken for talk about their ``true nature''.
\end{ednote}

\begin{ednote}
  This sort of ``pragmatist'' perspective is evident even in some of
  the higher reaches of contemporar mathematics.  Here's Awodey on
  Category Theory: ``The idea is that objects and arrow are determined
  by the role the play in the category via their relations to other
  objects and arrows, that is, by their position in a structure and
  not by what they `are' or `are made of' in some absolute sense.''
  (p. 29, 2nd ed.)  There is a striking resemblance between this claim
  about Category Theory and the sorts of claims (neo-)pragmatist
  philosophers make about language and meaning.
\end{ednote}

``In the intensional version of the theory, with which we are
concerned here, one thus has two different notions of equality:
propositional equality is the notion represented by the identity
types, in that two terms are propositionally equal just if their
identity type IdA(a,b) is inhabited by a term. By contrast,
definitional equality is a primitive relation on terms and is not
represented by a type; it behaves much like equality between terms in
the simply-typed lambda-calculus, or any conventional equational
theory.

If the terms a and b are definitionally equal, then (since they can be
freely substituted for each other) they are also propositionally
equal; but the converse is generally not true in the intensional
version of the theory''\cite{awodey_tth}

``The constructive character, computational tractability, and proof-
theoretic clarity of the type theory are owed in part to this rather
subtle treatment of equality between terms, which itself is
expressible within the theory using the identity types IdA(a, b).''\cite{awodey_tth}

%%%%%%%%
\subsection{Substitution}
\label{subs:substitution}

As the quote from Awodey above suggests, the concept of
substitutability plays a basic role.

\begin{ednote}
  Compare substitution in lambda calculus, and in Brandom's model.
  Maybe something about combinatory logic and the elimination of
  variables?
\end{ednote}

%%%%%%%%
\subsection{Bisimilarity}
\label{subs:bisimilarity}

\begin{ednote}
  See Jacobs and Rutten, p. 8; talking of ``observable'' states of a
  system X whose internal states are hidden, they say that the notion
  of bisimilarity of states ``expresses of two states that we cannot
  distinguish them via the operations that are at our disposal,
  i.e. that they are `equal as far as we can see'.  But this does not
  mean that these states are also identical as elements of X.
  Bisimilarity is an important, and typically coalgebraic, concept.''
  Here ``equal as far as we can see'' sounds a lot like ``equal under
  a description''.  J\&R prefer the notion of \emph{observation} for
  coalgebras, in contrast to \emph{construction} for algebras.  But
  that doesn't seem very symmetric.  We're not merely (passively)
  observing; we're \emph{treating} something in a certain way.  Making
  v. taking.  Construction v. description?  Experimentation?

  In a sense you can say that our observations are themselves
  constructed.  In J\&R's examples, we have buttons etc. for acting on
  the system even if we do not have knowledge of its internal states.
  So ``observation'' means interacting with the system.  So we can
  take algebraic construction as construction of something not already
  there, ``starting from nothing'', and coalgebraic
  ``co-construction'' as construction of a description of something
  already there.

  Algebraic construction as bottom-up; coalgebraic coconstruction as
  top-down?
\end{ednote}
