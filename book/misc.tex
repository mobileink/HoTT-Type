\chapter{Misc}
%%%%%%%%%%%%%%%%%%%%%%%%%%%%%%%%
\section{Expressivity}
\label{sect:expressivity}

Instead of ``P is a proposition'' etc. we should say ``P expresses a proposition''.

%%%%%%%%%%%%%%%%%%%%%%%%%%%%%%%%
\section{Determinism}
\label{sect:determinism}

Hypothesis: classical math with LEM and AC is inherently
non-deterministic.  Constructive math(s) and logic(s) that discard LEM
and AC are deterministic.

%%%%%%%%%%%%%%%%%%%%%%%%%%%%%%%%
\section{Modality}
\label{sect:modality}

Classic proofs (that use LEM) are modal.  Consider the way a classic
LEM-dependent proof works.  You start by stating the hypothesis: P is
true.  You assume that P is not true; then you derive a contradiction.
The conclusion is not merely that P is true, however; it is that P
\textit{must} be true.

Constructive proofs, by contrast, are not modal.  They do not say what
must be the case, they say what \textit{is} the case.  Or rather, they
\textit{show} what is the case.  (I leave aside the question of
whether what is, is necessary.)

%%%%%%%%%%%%%%%%%%%%%%%%%%%%%%%%
\section{Habeus Corpus Logics}
\label{sect:habeus}

The principle of ``Habeus Corpus'', from the Latin ``(You shall) have
the body'', was once enshrined as a fundamental principle of
Anglo-American law.  It was used to force the State to present a
detainee in person before the court, back in the days when we
occasionally had the temerity to question the wisdom of the State when
it tried to disappear people.

Type theory, and constructive logics generally, operate under a writ
of habeus corpus that is permanently in effect.  Except that this writ
requires the production not of a detainee, but of a witness.  If you
claim to have a proof, you must produce a witness who is competent to
testify to that fact.  So its actually more like a law of evidence,
but I'm too lazy to come up with a clever legalism to express that
idea.




%%%%%%%%%%%%%%%%%%%%%%%%%%%%%%%%
\section{Frege}
\label{sect:frege}

{\todo The Frege Point; force v. content, etc.}


%%%%%%%%%%%%%%%%%%%%%%%%%%%%%%%%
\section{Martin-L\"{o}f}
\label{sect:ml}

{\todo Summarize ML's remarks on assertion, proposition, etc.}

\section{Brandom on Assertion}

\begin{remark}
  Relevance to type theory: type theory begins with an account of
  judgment, proposition, etc.  Robert Brandom offers a very
  sophisticated account of these concepts which IMO could be put to
  very good use in explaining the conceptual foundations of type
  theory.
\end{remark}

Brandom's ``deontic scorekeeping model of discursive practice'' is a
very sophisticated and ambitious philosophical project.  But the main
point of interest for us, his treatment of assertion, is relatively
easy to grasp.

First off, for Brandom logic is fundamentally \textit{expressive},
rather than epistemological.  ``Logic is for establishing the truth of
cerain kinds of claims, by \textit{proving} them.  But logic can also
be thought of in expressive terms, as a distinctive set of tools for
\textit{saying} something that cannot otherwise be made
explicit''. (AR p. 19) One of his favorite examples is the inference
from ``Pittsburgh is west of Princeton'' to ``Princeton is east of
Pittburgh''.  We can endorse that inference as a good material
inference - material because it follows from the meanings of the terms
the sentences contain - even if our language does not contain a
conditional connstruction like ``if...then''.  But once we extend our
language by adding such a device, we can make that endorsement
explicit by saying ``If Pittsburgh is west of Princeton, then
Princeton is east of Pittsburgh''.  So ``if...then'' is an expressive
device, rather than an epistemological one.

Brandom's model of assertion involves both a
social aspect and a structure of commitment and entitlement.  The
basic metaphor is that in the game of giving and asking for reasons,
interlocutors maintain a deontic scorecard for each other and for
themselves, tracking commitments and entitlements.

``According to the model, to treat a performance as an assertion is to
treat it as the undertaking or acknowledgment of a certain kind of
\textit{commitment}---what will be called a 'doxastic' or 'assertional'
commitment.  To be doxastically commited is to have a certain social
status.  Doxastic commitments are normative, more specifically
\textit{deontic} statuses.  Such statuses are creatures of the
practical attitudes of the members of a linguistic community-they are
instituted by practices governing the taking and treating of
individuals \textit{as} committed.  Doxasitc commitments are
essentially a kind of deontic status for which the question of
\textit{entitlement} can arise.  Their inferential articulation, in
virtue of which they deserve to be understood as propsitionally
contentful, consists in consequential relations among the particular
doxastic commitments and entitlements---the ways in which one claim can
commit or entitle one to others (for which it accordingaly can serve
as a reason).''  (MIE p. 142)

``Uttering a sentence with assertional force or significance is
putting it forward \textit{as} a potential reason.  Asserting is
givein reasons....The function of assertion is making sentences
available for use as premises in inferences.'' (MIE p. 168)

``The basic model of inferential practices that institute assertional
significance...is defined by a structure that must be understood in
terms of the interaction of three different dimensions.  First, there
are two different sorts of deontic status involved:
\textit{commitments}, and \textit{entitlements} to commitments...The
second dimension ... turns on the distinction between the
\textit{concomitant} and the \textit{communicative} inheritance of
deontic statuses.  This is the \textit{social} difference between
\textit{intra}personal and \textit{inter}personal uses of a claim as a
premise...The third dimension of broadly inferential articulation
that is crucial to understanding assertional practice is that in which
discursive \textit{authority} is linked to and dependent upon a
corresponding \textit{responsibility}.... In asserting a claim, one
not only authorizes further assertions (for oneself and for others),
but undertakes a responsibility, for one commits oneself to being able
to vindicate the original claim by showing that one is entitled to
make it.'' (MIE p. 168-171)

``At the core of assertional practice lie three fundamental ways in
which one can demonstrate one's entitlement to a clam and thereby
fulfill the responsibility associated with making that
claim... First... one can demonstrate one's entitlement to a claim by
\textit{justifying} it, that is, by giving reasons for it.  Giving
reasons for a claim always consists in making more claims: asserting
premises from which the original claim follows as a conclusion... The
second way of vindicating a commitment by demonstrating entitlement to
it is to appeal to the authority of another asserter.  The
\textit{communcative} function of assertions is to license others who
hear the claim to reassert it.  The significance of this license is
that it makes available to those who rely on it and rassert the
original claim a special way of ischargin thheir responsibiity to
demonstrate their entitlement to it.'' (MIE p. 174; the third way
involves invoking one's own authority as a reliable noninferential
reporter, which is discussed later in MIE.)

%%%%%%%%
\subsection{Propositional Content}
\label{subs:}

Note that Brandom's notion of what it is to understand a proposition
or proof looks very different from Martin-L\"{o}f's.  But they
converge on the essential point, which involves grasping the
inferential relations among concepts and reasons.  For ML,
understanding a proposition means grasping what counts as a proof (or
something like that); for Brandom, it involves grasping the
``inferential articulation'' of the concept - the network of
propositions and inferences relating them that consitutes the concept
itself.  This is more or less just like ML's idea: to understand a
proposition is to grasp what counts as a reason for the proposition,
or---what is the same thing---entitlement to commitment to the
proposition.

%%%%%%%%
\subsection{Applying Brandom's Model}
\label{subs:bapply}

Let's look at what mathematical assertions and judgments look like
from a Brandomian perspective.

To start: we have a propositional content, which we can write as
``unasserted P''.  We have commitment, entitlement, and justification
(proof).

Uttering---or, usually, writing down---a proposition P makes explicit
one's commitment to the content of P, and makes one liable to
demonstrate entitlement to that commitment.  Hearing---reading---a
proposition P entitles one (by ``deontic inheritance'') to undertake a
commitment to P if one is willing to ascribe entitlement to the
utterer/author.  Otherwise, it authorizes one to demand a reason.  One
can also record (on one's ``deontic scorecard'') the speaker's
commitment to P while declining to undertake the commitment oneself.

{\todo MLTT analyzes the structure of (mathematical) assertion interms
  of proposition, judgment, truth, etc.  Map this structure to
  Brandom's structure.  Brandom's account should turn out better since
  it is more finely articulated, and distinguishes explicitly between
  commitment and entitlement.}


%%%%%%%%%%%%%%%%%%%%%%%%%%%%%%%%
\section{From Truth to Testimony}
\label{sect:truth}

We have propositions as types, and we have non-propositional types
like $\nat$.  There is an obvious conflict of intuitions here.
Propositions, like \(1>0\), have truth conditions; names like \(\nat\)
do not.  How can they be the same kind of thing?

I think the way out of this embarrassment is recognition that the
classic concept of truth is not relevant to type theory; or, in a more
positive vein, that only a deflationary or minimalist notion of truth
should be used in type theory.  In type theory one does not say that a
proposition \textit{is} true or false; instead one says that a
(propositional) type is proven or disproven, or that either it or its
negation has a witness\sidenote{Or, a ``maker'' or ``constructor''.}.  Instead of a concept of truth we have a
concept of testimony.  Of course, ordinarily witnesses testify as to
the truth of some proposition; but the witnesses of type theory do
more than that---or rather they do something else, namely they produce
or ``make'' the proposition.

%%%%%%%%
\subsection{Proof, Witness, Constructor}
\label{subs:pwc}

Type theory seems to have settled on an idiom; one says, for example,
that types have or do not have proofs or witnesses.  But there are
problems with both of these terms.  The former covers too much ground
since it includes non-constructive proofs.  The latter invokes a
misleading metaphor, since a witness testifies to the truth, whereas a
type-theoretic witness to a type constructs (makes, produces, etc.)
something.  In the case of propositional types, constructors ``make''
the proposition (in the sense that they are inferences that terminate
in the proposition); in the case of non-propositional types,
constructors make ``elements'' of the type, which serve as proxies(?) for
the type.

\begin{remark}
  Problem: here again propositional and non-propositional types behave
  differently.  Every proof of a proposition has the proposition as
  its conclusion; they are all ``the same'' because they all have that
  element in the same structural position.  But the proof of
  e.g. \(\nat\) is different.  For example, \(2\) is a witness for
  \(\nat\).  Or rather, anything that constructs \(2\) is such a
  witness.  What all such constructions have in common is \(2\), not
  \(\nat\).  So they are all clearly proofs of \(2\), but we want them
  to be proofs of \(\nat\).  How do we get there?
\end{remark}

\section{Misc. Niceties}

\begin{ednote}
  tait: ``objects are given or constructed as object of a given
  type''.  The expression ``a : A'' expresses the idea that we are
  given a of type A.  It does this by stipulation rather than
  assertion.  Assertions are challengable and must be justified on
  demand; stipulations are not and need not.
\end{ednote}

ML Type Theory is centered (more or less) on one of the major
logico-philosophical topics of the 20th century, namely the nature of
assertion and its relation to propositions and inferences.

You don't have to understand the arcana of this debate in order to
understand type theory (or HoTT), but some familiarity with the main
outline is very helpful.  Actually, I think it's essential, if you
want to understand the HoTT Book's account of \textit{judgement},
presented in HoTT Chapter 1 (reproduced below).  Fortunately the
presentation is relatively straightforward.

\begin{remark}
  Stress: this is largely a philosophical issue, or perhaps an issue
  in Philosophy of Language.  It's really about how our utterances
  come to have the significances they do.
\end{remark}

Outline:

\begin{itemize}
\item Frege's elevation of \textit{force} as essential
\item Dealing with embedded (and therefore forceless) propositions
\item Wittgenstein
\item Dummett
\item etc.
\item Brandom's recent innovation: decompose ``assertion'' into ``commitment'' and ``entitlement''
\end{itemize}

What the HoTT Book refers to as judgment (following ML) could also be
called assertion.  Brandom's account of the ``fine structure'' of
assertion is very helpful here.  Among other things, it provides a
very simple explanation of how embedded propositions work.  Embedded
propositions are unasserted; the problem is how to reconcile this with
the fact that they are function as assertions if unembedded.  On
Brandom's account, [todo...]

In other words, we can have commitment with or without entitlement,
and vice-versa.

A set membership statement can be explained in terms of commitments
and entitlements.  A free occurance of e.g. \(a\in A\) is ordinarily
taken as an assertion (judgment).  We can follow Frege and make this
\textit{force} explicit: \(\vdash a\in A\).  The problem with this,
however, is that, in contemporary usage, this would make \(a\in A\)
\textit{logically} true, which is not what we want.  Instead we want a
representation of committment to the proposition, as at least
ordinarily true, without regard to its logical truth.\marginnote{TODO:
  logical v. ordinary truth is pretty hairy for non-logicians so the
  distinction should be explicated.}

In sum:  the implicity sense of \(a\in A\) is something like: 
\[\exists \Gamma, a, A | \Gamma\vdash a\in A\]

Informally: there exists a set of propositions \(\Gamma\), a value (or
object) \(a\), and a set \(A\) such that the propostion \(a\in A\) is
deducible from \(\Gamma\).

So the meaning of \(a\in A\) essentially involves existential
quantification.  It is a statement about the world, that it contains
the relavant entities, not about the entities themselves.

\begin{remark}
  Not quite; \(a\in A\) is surely a statement about \(a\), maybe also
  about \(A\), no?  But still there must be an implicity existential
  quantification over the propositions that entail the statement.
\end{remark}

There is a logical subtlety here.  \(a\in A\) seems to be about a
determinate \(a\) and a determinate \(A\), but it isn't, not if we
take it to be an existentially quantified statement.  That's because
\(\exists a, A | a\in A\) does not pick out determinate individuals;
it just says that \textit{some} such individuals exist in the domain
of interpretation.  True, \(a\) and \(A\) are said to be bound by
\(\exists\), but that's not entirely accurate; quantified variables
are not bound in the way that constant symbols like \(\pi\) or \(0\)
are bound.  Whatever we go on to say about \(a\) and \(A\) --
e.g. \(a\in A\) -- remains within the scope of the quantifier, so it
does not count as a statement about determinate individuals.  It's a
statement about the world, that it contains entities that satisfy the
predicate.

On the other hand, the same seems to be true of \(a : A\): though
these symbols be bound, we don't know what they are bound to.  They
are not bound by an implicit existential quantifier; \(a : A\) does
\textit{not} mean \(\exists a, A | a : A\).

\begin{remark}
  Plus, quantifiers have to be used with a predicate; strictly
  speaking, \(\exists a, A\) is not a complete statement.
\end{remark}

By contrast, the Type Theoretic analogue \(a : A\) is a statement
about a specific value and a specific type, without any
quantification.  It is not directly a statement about the world, but
about part of the world.  Or: it expresses both commitment and
entitlement.  That's why it cannot be embedded in e.g. ``if \(a : A\)
then it is not the case that \(b : B\)''.  Embedded propositions
cannot carry force, but \(a : A\) always carries force, intrinsically,
as it were.

%%%%%%%%
\subsection{a : A}
\label{subs:aA}

Forms go from symbols to terms to sentences; from \(a\) to \(a+b\) to
\(a+b=c\).

The ``judgment'' \(a : A\) is clearly a compound term, so it cannot
merely name something.  But is it a sentence?  Does it denote a
proposition?  Or is it analogous to terms like \(a+b\) which are names
of a sort but involve some additional meaning beyond mere reference.

It seems it must involve a proposition, or let's say propositional
content.  We take \(a : A\) as a statement of fact, rather than a mere
reference to some part of the world.  Then how is it distinct from
\(a\in A\)?

The HoTT Book says it is ``analogous'' to the set-theoretic statement
\(a\in A\), but essentially different, since \(a\in A\) is a
proposition but \(a : A\) is a judgment.  It says that, \textit{when
  working internally in type theory}, \(a : A\) cannot be embedded, as
in `` if \(a : A\) then it is not the case that \(b : B\)'', nor can
the judgment \(a : A\) be disproved.

So let's look closely at what this means.  Earlier, HoTT says that
(some) judgments involving A ``exist at a different level from the
\textit{proposition} \(A\) itself, which is an internal statement of
the theory.''  (p. 18) There's a bit of circularity there; what is an
``internal statement''?

{\todo The nature of ``proposition'' has been a topic of
  considerable debate.  Review some of the alternative accounts on
  offer.}


The basic idea seems to be based on the well-known concept that
propositions by themselves are devoid of force, and must be asserted.
HoTT seems to imply that judgments are asserted propositions -- or
more correctly, assertings of propositions.

This seems a little bit off.  Assertion is something only people do.
An inked form on a page cannot really be construed as an assertion.
So we need to work out the mechanics of how a written form like \(a :
A\) can be viewed as a ``judgment'' in this sense.  I think Brandom's
model of assertion works.  It would say, I think, that \(a : A\)
counts as a judgment (assertion) because by convention we agree to
treat it that way, whereas we treat \(a\in A\) slightly differently,
because of the conventions elaborated by 20th century logic.

When HoTTB refers to ``working internally in type theory'', it seems,
the idea is to consider propositions in isolation from their
assertion.  Assertion, on this view, is something that comes from
outside of the world of propositions.  This is perfectly in tune with
the idea that asserting is something people do, but that what gets
assert\textit{ed} -- the \textit{content} of an assertion -- is
distinct from the assert\textit{ing}.

\begin{remark}
  Sellars called this the notorious -ing/-ed distinction.
\end{remark}

This would seem to make \(a : A\) an assert\textit{ing}.

We can think of \atypeA{} as a \textit{given} proposition: one that,
while unasserted, has the same force as a propositional assertion.  Or
another way to put it would be to say that use of \atypeA{} is
inalienably performative.

In fact \atypeA{} corresponds nicely to a common linguistic practice,
namely combining a proper name and a description, as in ``Joan of
Arc'', ``King George'', or ``Slick Willy''.  Or, more colloquially,
``poor Tom'', ``angry Joe'', or ``Gimpel the fool''.  And the
primitive nature of types can be clearly illustrated by analogy with
the military.  In type theory, every object has a type, just as
everybody in the military has a rank.  You cannot be in the military
unless you have a rank.  Within the military, the proper way refer to
someone in the miliary is to combine rank and name: General Custer,
Sergeant York, Private Bilko.  So the difference between \atypeA{} and
\(a\in A\) is like the difference between ``This is General Custer''
and ``This is Custer; he is a General''.

On the other hand, ``This is General Custer'' doesn't look much like a
\textit{judgment}, although it does look like a \textit{claim}.  But
not that it is not a claim about the meaning of ``General Custer'';
rather it is a claim about the relation between ``This'' and ``General
Custer''.  You could be wrong about the name or the rank of whomever
you mean by ``This'', but you cannot be wrong about ``General
Custer''; that's just a qualified name.  Being wrong in this sense
about ``This is General Custer'' is an empirical matter; in type
theory, the question of whether \atypeA (``this is a-of-A'') is
correct or not never even arises.  It doesn't make an empirical
assertion, it states a \textit{given}.  Or we might say it gives a
fact.  By contrast, \(a\in A\), as a proposition, may be either true
or false; when we say ``let \(a\in A\)'', we implicitly stipulate that
\(a\in A\) is to be \textit{assumed} to be true, but it is not
\textit{given} as true.  In other words, we can gloss it as ``\(a\in
A\) has a truth-value like any proposition, so it could be false, but
please assume that it is true.''

Another critical distinction: in standard set theory and logic,
judgments come from the outside, as it were.  But in HoTT, judgments
of the form \atypeA are internal.  They may be derivable inside the
system (by production of a proof or witness.)  In other words,
inference in set theory comes from outside of the world of sets, but
inference in HoTT is built in to the structure of types.  Inference
(construction) is part of the intrinsic meaning of types.

%%%%%%%%
\subsection{Justification}
\label{subs:just}

The HoTT Book's account of judgments in Chapter 1 section 1 seems to
conflate the distinction\sidenote{This is only to be
expected, since Brandom is the first (so far as I know) to see that
assertion (judgment) has an internal structure involving commitment
and entitlement (and some other stuff like a social dimension.)} Brandom makes between commitment and
entitlement.  ``Informally, a deductive system is a collection of
rules for deriving things called judgments.''

But derivation (proof) starts and ends in propositions; commitment is
something else.  The derivation or proof provide warrant for
entitlement to the commitment - justification of the conclusion.  So
how would Brandom parse ``judgment'' as HoTT uses the term?

